\chapter{Potential formulation}
\section{Scalar and vector potentaial}
Maxwell's equations are\\
(i) $\boldsymbol{\nabla} \cdot \mathbf{E}=\frac{1}{\epsilon_{0}} \rho$,\\
(iii) $\boldsymbol{\nabla} \times \mathbf{E}=-\frac{\partial \mathbf{B}}{\partial t}$,\\
(ii) $\boldsymbol{\nabla} \cdot \mathbf{B}=0$,\\
(iv) $\boldsymbol{\nabla} \times \mathbf{B}=\mu_{0} \mathbf{J}+\mu_{0} \epsilon_{0} \frac{\partial \mathbf{E}}{\partial t}$.\\
If $\rho(r,t)$ and $J(r,t)$ are bnown electric and magnetic field can be find out using Gauss's law and Bio-savart law.It is difficult to find E ans B if they are time dependant.To solve this problem first we are going to represents the fields in terms of potentials electric potential V and magnetic potential B.\\
In the static case $\nabla \times E=0$ So electric field cane written as a negative gradient of some scalar quandity called electric potential V\\
$$E=-\nabla V$$(not possible in elctrodynamics)\\
But $\nabla \cdot B=0$ always.Which gives \\
$$B=\nabla\times A$$
 Where A is the magnetic vector potential.Putting this value in 
$$\boldsymbol{\nabla} \times \mathbf{E}=-\frac{\partial \mathbf{B}}{\partial t}$$
We will get \\
$$\nabla \times E=\frac{-\partial }{\partial t}(\nabla \times A)$$
$$\nabla \times \left( E+\frac{\partial A}{\partial t}\right) =0$$
Again the curl of something become zero. so it can be written as negative gradient of potential
$$E+\frac{\partial A}{\partial t}=-\nabla V$$
$$E=-\nabla V-\frac{\partial A}{\partial t}$$
\begin{center}
	\framebox{
		\parbox[t][3cm]{3cm}{
			
			\addvspace{0.2cm} \centering
			
			\begin{align*}
			\begin{array}{lll}
			$$B=\nabla\times A$$\\
			$$E=-\nabla V-\frac{\partial A}{\partial t}$$
			\end{array}
			\end{align*}} }
\end{center}
If A and V are known we can find electric and magnetic field with these two equations.

Putting this equation $$E=-\nabla V-\frac{\partial A}{\partial t}$$  in Gauss's law
we will get,\\
$$\nabla ^2V+\frac{\partial }{\partial t}(\nabla \cdot A)=\frac{-\rho}{\epsilon_{0}}$$
Putting $B=\nabla\times A$  equation in Ampere/Maxwell's law and rearranging we will get,\\
$$\left( \nabla^2A-\mu_{0}\epsilon_{0}\frac{\partial^2 A}{\partial t^2}\right) -\nabla\left( \nabla \cdot A+\mu_{0}\epsilon_{0}\frac{\partial V}{\partial t}\right) =-\mu_{0} J$$ 

These two equations contain all information in the Maxwell's equations.\\
However we have succeeded in reducing six problems to find E and B ,down to four(V one component,A three component.),these equations are lengthy and difficult to find solution we have to abandon this potential formulation altogether.\\
\paragraph{Gauge transformation}
To avoid this problem we are transforming the potential equations by adding one extra term to A and V.This is called gauge tranformation.Consider the transformation occuring in the same fields E and B.Then\\
$$A^{\prime}=A+\alpha$$ and
$$V^{\prime}=V+\beta$$
Taking curl on each side of $A^{\prime}=A+\alpha$
we will get 
$$\nabla \times A^{\prime}=\nabla \times A+\nabla \times \alpha$$
Since B's are same we can written as($\nabla \times B=0,A^{\prime}=A$) \\
$$\nabla \times \alpha =0$$
Again curl of $\alpha$ is zero.then\\
$$\alpha=\nabla \lambda$$\\
The two potentials also gives the same E.so,\\
$$\nabla \beta +\frac{\partial \alpha }{\partial t}=0$$
$$\nabla\left( \beta+\frac{\partial \lambda}{\partial t}\right) =0$$
$$\beta= -\frac{\partial \lambda}{\partial t}$$
Now transformations become,\\

\begin{center}
	\framebox{
		\parbox[t][3cm]{3cm}{
			
			\addvspace{0.2cm} \centering
			
			\begin{align*}
			\begin{array}{lll}
			 $$A^{\prime}=A+\nabla \lambda$$\\
			 $$V^{\prime}=V-\frac{\partial \lambda}{\partial t}$$
			\end{array}
			\end{align*}} }
\end{center}
\subsection{Coulomb Gauge and Lorentz gauge}
\paragraph{Coluomb gauge}
$$\nabla \cdot A=0$$ is called Coulomb gauge.\\
\paragraph{importance}
$$\nabla \cdot A=0$$ Then the equation 
$$\nabla ^2V+\frac{\partial }{\partial t}(\nabla \cdot A)=\frac{-\rho}{\epsilon_{0}}$$ become,\\
$$\nabla^2 V=\frac{-\rho}{\epsilon_{0}}$$
This is poisson's equation.From this equation V can be findout by using the formula\\
$$V(r,t)=\frac{1}{4 \pi \epsilon_0}\int \frac{\rho(r^{\prime},t)}{r}d\tau^{\prime}$$
V is easy to get but to find E we need  A also($E=-\nabla V-\frac{\partial A}{\partial t}$) which is difficult.
\paragraph{Advantage of the coulomb gauge is that the scalar potential is simply to calculate .The disadvatage is that  A is particularly difficult to calculate. }
After applying coulomb gauge to the equation 
$$\left( \nabla^2A-\mu_{0}\epsilon_{0}\frac{\partial^2 A}{\partial t^2}\right) -\nabla\left( \nabla \cdot A+\mu_{0}\epsilon_{0}\frac{\partial V}{\partial t}\right) =-\mu_{0} J$$ 
we get\\
$$\left( \nabla^2A-\mu_{0}\epsilon_{0}\frac{\partial^2 A}{\partial t^2}\right)=-\mu_{0}J+\mu_{0} \epsilon_{0}\nabla \left( \frac{\partial V}{\partial t}\right) $$
\paragraph{Lorentz gauge}
$$\nabla \cdot A =-\mu_{0} \epsilon_{0} \frac{\partial}{\partial t}V$$

 is called the lorentz gauge.\\
 This designes to eliminate the middle term of the equation
$$\left( \nabla^2A-\mu_{0}\epsilon_{0}\frac{\partial^2 A}{\partial t^2}\right) -\nabla\left( \nabla \cdot A+\mu_{0}\epsilon_{0}\frac{\partial V}{\partial t}\right) =-\mu_{0} J$$
With lorentz gauge this equation become
$$\nabla^2A-\mu_{0}\epsilon_{0}\frac{\partial^2 A}{\partial t^2}=-\mu_{0} J$$
With lorentz gauge this equation 
$\nabla ^2V+\frac{\partial }{\partial t}(\nabla \cdot A)=\frac{-\rho}{\epsilon_{0}}$
becomes
$$\nabla^2A-\mu_{0}\epsilon_{0}\frac{\partial^2 A}{\partial t^2}=\frac{-\rho}{\epsilon_{0}}$$
The virtue of the lorentz gauge is that it treats V and A on the same differential operator
$$\nabla^2-\mu_{0} \epsilon_{0}\frac{\partial ^2}{\partial t^2}\equiv \mathcal{D}^2  $$
 is called d'Alembertian\\
Then both equation become\\
$$\mathcal{D}^2V=\frac{-\rho}{\epsilon_{0}}$$
$$\mathcal{D}^2A=-\mu_{0} J$$
\section{Lorentz invarience of Maxwell's equations}
The MFE's in frame $S$ may be expressed as
\begin{align}
\nabla \cdot \mathbf{E}&=\rho / \varepsilon_{0}\\
\nabla \times \mathbf{B}&=\partial \mathbf{E} / c^{2} \partial t+\mu_{0} \mathbf{J}\\
\nabla \cdot \mathbf{B}&=0\\
\nabla \times \mathbf{E}&=-\partial \mathbf{B} / \partial t
\end{align}
  Now by applying the relativity principle to Equations, they will preserve their form in frame $S^{\prime}$ moving with velocity $\mathbf{u}$ parallel to their common $x$ axis, and are expressed as follows:
\begin{align}
\nabla^{\prime} \cdot \mathbf{E}^{\prime}&=\rho^{\prime} / \varepsilon_{0}\\
\nabla^{\prime} \times \mathbf{B}^{\prime}&=\partial \mathbf{E}^{\prime} / c^{2} \partial t^{\prime}+\mu_{0} \mathbf{J}^{\prime}\\
\nabla^{\prime} \cdot \mathbf{B}^{\prime}&=0\\
\nabla^{\prime} \times \mathbf{E}^{\prime}&=-\partial \mathbf{B}^{\prime} / \partial t^{\prime}
\end{align}
where $\rho, \mathbf{J}, \rho^{\prime}, \mathbf{J}^{\prime}$ are the relativistic charge and current density in frames $S$ and $S^{\prime}$, respectively. It is obvious that in frame $S^{\prime}$ the field vectors $\mathbf{E}^{\prime}$ and $\mathbf{B}^{\prime}$ are propagated in free-space with the speed $c^{\prime}=1 / \sqrt{\mu_{0}^{\prime} \varepsilon_{0}^{\prime}}$. But we know that $\varepsilon_{0}$ and $\mu_{0}$ have the
same value in all frames, i.e. $\varepsilon_{0}$ and $\mu_{0}$ are universal constants, so $c^{\prime}$ must equal $c$.\\ 
 Taking the $x$-component of Eq. (2.2) and writing Eq. (2.1) in terms of Cartesian components, we have
\begin{align}
&\partial B_{z} / \partial y-\partial B_{y} / \partial z=\partial E_{x} / c^{2} \partial t+\mu_{0} J_{x} \\
&\partial E_{x} / \partial x+\partial E_{y} / \partial y+\partial E_{z} / \partial z=\rho / \varepsilon_{0}
\end{align}
Multiplying Eq. (2.9) by an arbitrary scalar factor $\gamma$ and (2.10) by $u \gamma / c^{2}$, and then subtracting, we get.
\begin{align*}
\frac{\partial}{\partial y} \gamma\left(B_{z}-u E_{y} / c^{2}\right)- \frac{\partial}{\partial z} \gamma\left(B_{y}+u E_{z} / c^{2}\right)
=\gamma\left(\frac{\partial}{\partial t}+u \frac{\partial}{\partial x}\right) E_{x} / c^{2}+\mu_{0} \gamma\left[J_{x}-u \rho\right]
\end{align*}
Comparing the last relation with the $x$-component of Eq. (2.6), we have
\begin{align}
&E_{x}^{\prime}=E_{x} \quad B_{y}^{\prime}=\gamma\left(B_{y}+u E_{z} / c^{2}\right) \quad B_{z}^{\prime}=\gamma\left(B_{z}-u E_{y} / c^{2}\right)\\
&\partial / \partial t^{\prime}=\gamma[\partial / \partial t+u \partial / \partial x] \quad \partial / \partial y^{\prime}=\partial / \partial y, \quad \partial / \partial z^{\prime}=\partial / \partial z\\
&J_{x}^{\prime}=\gamma\left(J_{x}-u \rho\right)
\end{align}
Now multiplying Eq. (2.9) by $u \gamma$ and Eq. (2.10) by $\gamma$, and then subtracting, we have
\begin{align*}
\gamma\left(\frac{\partial}{\partial x}+\frac{u}{c^{2}} \frac{\partial}{\partial t}\right) E_{x}+\frac{\partial}{\partial y} \gamma\left(E_{y}-u B_{z}\right)+\frac{\partial}{\partial z}  \gamma\left(E_{z}+u B_{y}\right) 
=\frac{\gamma}{\varepsilon_{0}}\left(\rho-u J_{x} / c^{2}\right)
\end{align*}
 Comparing the last relation with Eq. (2.5), we have 
 \begin{align}
 &E_{y}^{\prime}=\gamma\left(E_{y}-u B_{z}\right) \quad  E_{z}^{\prime}=\gamma\left(E_{z}+u B_{y}\right)\\
 &\partial / \partial x^{\prime}=\gamma\left(\partial / \partial x+u \partial / c^{2}\right)\\
 &\rho^{\prime}=\gamma\left(\rho-u J_{x} / c^{2}\right)
 \end{align}
The Eqs. (2.12) and (2.15) are the differential LT.\\
 The scalar factor $\gamma$ can be fixed by applying the relativity principle on Eq. (2.11). So Eq. (2.11) could be written in frame $S^{\prime}$ as
\begin{equation}
B_{z}=\gamma\left(B_{z}^{\prime}+u E_{y}^{\prime} / c^{2}\right)
\end{equation}
Hence we can substitute from Eqs. (2.11) and (2.14) into Eq. (2.17) and obtain
\begin{align*}
B_{z}=\gamma\left[\gamma\left(B_{z}-u E_{y} / c^{2}\right)+u \gamma\left(E_{y}-u B_{z}\right) / c^{2}\right]=\gamma^{2}\left(1-u^{2} / c^{2}\right) B_{z}
\end{align*}
 This equation is an algebraic identity; i.e., 
\begin{align}
\gamma^{2}\left(1-u^{2} / c^{2}\right)=1\\
\gamma=1 / \sqrt{1-u^{2} / c^{2}}
\end{align}
Starting now from the $y$-component of Eq. (2.2) and using Eq. (2.18), we get
\begin{align*}
\frac{\partial B_{x}}{\partial z}-\gamma^{2}\left(1-u^{2} / c^{2}\right) \frac{\partial B_{z}}{\partial x}=\gamma^{2}\left(1-u^{2} / c^{2}\right) \frac{\partial E_{y}}{c^{2} \partial t}+\mu_{0} J_{y}
\end{align*}
Adding and subtracting the two terms
$$
u \gamma^{2} \partial E_{y} / c^{2} \partial x, \quad u \gamma^{2} \partial B_{z} / c^{2} \partial t
$$
We obtain
\begin{align*}
\frac{\partial B_{x}}{\partial z}-\gamma\left(\frac{\partial}{\partial x}+\frac{u}{c^{2}} \frac{\partial}{\partial t}\right) \gamma \left(B_{z}-u E_{y} / c^{2}\right) 
=\frac{\gamma}{c^{2}}\left(\frac{\partial}{\partial t}+u \frac{\partial}{\partial x}\right) \gamma\left(E_{y}-u B_{z}\right)+\mu_{0} J_{y}
\end{align*}
Comparing the last relation with the $y$-component of Eq. (2.6), we deduce
\begin{align}
B_{x}^{\prime}=B_{x}\\
J_{y}^{\prime}=J_{y}
\end{align}
In a similar way, starting from the z-component of Eq. (2.2), we obtain
\begin{equation}
J_{z}^{\prime}=J_{z}
\end{equation}
The electromagnetic charge density $\rho$ for a particle of charge $q$ is $\rho=q \delta^{3}\left(x-x^{\prime}\right)$. It implies the conventional definition of current density $\mathbf{J}=q \mathbf{v} \delta^{3}\left(x-x^{\prime}\right)=\rho \mathbf{v}$. Starting from the definition of $\mathbf{J}$ in frames $S$ and $S^{\prime}$ as $\mathbf{J}=\rho \mathbf{v}, \mathbf{J}^{\prime}=\rho^{\prime} \mathbf{v}^{\prime}$, and using Eqs. (2.13)(2.16)(2.21)(2.22), one gets the relativistic velocity transformations, i.e.
\begin{equation}
v_{x}^{\prime}=\frac{v_{x}-u}{1-u v_{x} / c^{2}} \quad  v_{y}^{\prime}=\frac{v_{y} / \gamma}{\left(1-u v_{x} / c^{2}\right)} \quad v_{z}^{\prime}=\frac{v_{z} / \gamma}{\left(1-u v_{x} / c^{2}\right)}
\end{equation}
 Depending on Eqs. (2.23) and Eqs. (2.13)(2.16)(2.21)(2.22), one can derive the relativistic charge density in frame $S$ and $S^{\prime}$ respectively:
\begin{equation}
\rho=\rho_{0} / \sqrt{1-v^{2} / c^{2}} \quad \quad \rho^{\prime}=\rho_{0} / \sqrt{1-v^{\prime 2} / c^{2}}
\end{equation}
where $\rho_{0}$ is the electrostatic charge density measured when the charge is at rest relative to the observer.\\
 Here, then, is the complete set of transformation rules:
 \begin{center}
 	\framebox{
 		\parbox[t][2cm]{3.5cm}{
 			
 			\addvspace{0.2cm} \centering
 			
 			\begin{align*}
 			\begin{array}{lll}
 		  $$\begin{array}{lll}
 		  {E}_{x}^{\prime}=E_{x}, & {E}_{y}^{\prime}=\gamma\left(E_{y}-v B_{z}\right), & {E}_{z}^{\prime}=\gamma\left(E_{z}+v B_{y}\right) \\
 		  {B}_{x}^{\prime}=B_{x}, & {B}_{y}^{\prime}=\gamma\left(B_{y}+\frac{v}{c^{2}} E_{z}\right), & {B}_{z}^{\prime}=\gamma\left(B_{z}-\frac{v}{c^{2}} E_{y}\right)
 		  \end{array}$$
 			\end{array}
 			\end{align*}} }
 \end{center}
 we begin with the Maxwell equations, and by applying the relativity principle we get the Lorentz contracted charge density containing $\rho$, defined as Eqs. (2.24), without using the hypothesis proposed by Lorantz-Fitzgerlad as well as without using length contraction.















\newpage
\begin{abox}
	Previous year solutions
	\end{abox}
\begin{enumerate}
	\begin{minipage}{\textwidth}
		\item  The electric and the magnetic field $\vec{E}(z, t)$ and $\vec{B}(z, t)$, respectively corresponding to the scalar potential $\phi(z, t)=0$ and vector potential $\vec{A}(z, t)=\hat{i} t z$ are
		\exyear{GATE 2012}
	\end{minipage}
	\begin{tasks}(2)
		\task[\textbf{A.}] $\vec{E}=\hat{i} z$ and $\vec{B}=-\hat{j} t$
		\task[\textbf{B.}]$\vec{E}=\hat{i} z$ and $\vec{B}=\hat{j t}$
		\task[\textbf{C.}]$\vec{E}=-\hat{i} z$ and $\vec{B}=-\hat{j t}$
		\task[\textbf{D.}]$\vec{E}=-\hat{i} z$ and $\vec{B}=-\hat{j} \mathrm{t}$
	\end{tasks}
\begin{answer}
	$\vec{E}=-\vec{\nabla} \phi-\frac{\partial \vec{A}}{\partial t}=-\frac{\partial \vec{A}}{\partial t}=-\hat{i} z, \vec{B}=\vec{\nabla} \times \vec{A}=+\hat{j} t$\\
The correct option is \textbf{(d)}
\end{answer}
\begin{minipage}{\textwidth}
	\item If the vector potential $\vec{A}=\alpha x \hat{x}+2 y \hat{y}-3 z \hat{z}$, satisfies the Coulomb gauge, the value of the constant $\alpha$ is
	\exyear{GATE 2015}
\end{minipage}
\begin{answer}
$$\text { Coulomb gauge condition } \vec{\nabla} \cdot \vec{A}=0 \Rightarrow \alpha+2-3=0 \Rightarrow \alpha=1$$	
\end{answer}
\begin{minipage}{\textwidth}
	\item Consider magnetic vector potential $\tilde{A}$ and scalar potential $\Phi$ which define the magnetic field $\vec{B}$ and electric field $\vec{E}$. If one adds $\vec{\nabla} \lambda$ to $\vec{A}$ for a well-defined $\lambda$, then what should be added to $\Phi$ so that $\vec{E}$ remains unchanged up to an arbitrary function of time, $f(t)$ ?
	\exyear{JEST 2017}
\end{minipage}
\begin{tasks}(2)
	\task[\textbf{A.}] $\frac{\partial \lambda}{\partial t}$
	\task[\textbf{B.}]$-\frac{\partial \lambda}{\partial t}$
	\task[\textbf{C.}]$\frac{1}{2} \frac{\partial \lambda}{\partial t}$
	\task[\textbf{D.}]$-\frac{1}{2} \frac{\partial \lambda}{\partial t}$
\end{tasks}
\begin{answer}
Consider Gauge Transformation
$$
\vec{A}^{\prime}=\vec{A}-\vec{\nabla} \lambda=\vec{A}+\vec{V}(-\lambda) \quad \text { and } \quad \Phi^{\prime}=\Phi-\frac{\partial(-\lambda)}{\partial t}=\Phi+\frac{\partial \lambda}{\partial t}
$$
The correct option is \textbf{(a)}	
\end{answer}
\end{enumerate}





























\newpage
\section{Lagrangian and hamiltonian of a charged particle in a Electromagnetic field}
The force experienced by a particle of charge q at rest in an electric field of intensity E is given by
$$F_1=qE$$
The force experienced by a moving charge q in a magnetic field B is given by\\
$$F_2=q(v\times B)$$
Where V  is the velocity of the particle.The direction of $F_2$ is perpendicular to both v and B.\\
Total force on a uniformly moving charged particle of charge q is the sum of $F_1$ and $F_2$\\
$$F=F_1+F_2=qE+q(v\times B)$$
The above equation is known as lorentz formula.\\
Maxwell's equation in empty space is given by\\
$\nabla \cdot E=\frac{\rho}{\epsilon_0}$\\
$\nabla \cdot B=0$\\
$\nabla \times E=-\frac{\partial B}{\partial t}$\\
$\nabla \times B=\mu_0(J+\epsilon_0\frac{\partial E}{\partial t})$\\
We know that if $\nabla \cdot B=0$ ,B can be expressed as a curl of some vector A\\
$$B=\nabla \times A$$
Substitute this equation in $\nabla \times E=-\frac{\partial B}{\partial t}$\\
$\nabla \times E=-\frac{\partial B}{\partial t}=-\frac{\partial}{\partial t}(\nabla \times A)=-\nabla \times \frac{\partial A}{\partial t}$\\
$\therefore E=-\frac{\partial A}{\partial t}-\nabla s$ ,since curl of gradient is always zero and s is  scalar function\\
The lorentz force in terms of scalar potential s and vector potential A is given by\\
$$F=q\left[ E+v\times B\right] $$
$$F=q\left[ -\frac{\partial A}{\partial t}-\nabla s+(v\times \nabla \times A)\right] $$
Let us consider the last part $v\times(\nabla \times A)$\\
$$v\times(\nabla \times A)=\nabla (A\cdot v)-(\nabla \cdot v)A$$
Also $A=A(x,y,z,t)$\\
Therfore total time derivative of A is given by\\
$\frac{dA}{dt}=\frac{\partial A}{\partial x} \dot{x}+\frac{\partial A}{\partial y} \dot{y}+\frac{\partial A}{\partial z} \dot{z}+\frac{\partial A}{\partial t}=\frac{\partial A}{\partial x} v_x+\frac{\partial A}{\partial y} v_y+\frac{\partial A}{\partial z} v_z+\frac{\partial A}{\partial t}$\\
$=(\hat{i}v_x+\hat{j}v_y+\hat{k}v_z)\cdot \left( \hat{i}\frac{\partial A}{\partial x}+\hat{j}\frac{\partial A}{\partial y}+\hat{k}\frac{\partial A}{\partial z}\right) +\frac{\partial A}{\partial t}=v \cdot \nabla \mathbf{A}+\frac{\partial \mathbf{A}}{d t}$ \\
	$$v \cdot\nabla \mathbf{A}=\frac{d \mathbf{A}}{d t}-\frac{\partial \mathbf{A}}{\partial t}$$
	Substituting this value in 
$$v\times(\nabla \times A)=\nabla (A\cdot v)-(\nabla \cdot v)A$$
we will get \\
$$\mathrm{v} \times(\nabla \times \mathbf{A})=\nabla(\mathbf{A} \cdot \mathrm{v})-\frac{d \mathbf{A}}{d t}+\frac{\partial \mathbf{A}}{\partial t}$$
Then F becomes\\
$$F=q\left[ -\frac{\partial A}{\partial t}-\nabla s+(v\times \nabla \times A)\right] $$
$$\begin{aligned}
	\mathbf{F} &=q\left[-\frac{\partial \mathbf{A}}{d t}-\nabla s+\left\{\nabla(\mathbf{A} \cdot \mathrm{v})-\frac{d \mathbf{A}}{d t}+\frac{\partial \mathbf{A}}{\partial t}\right\}\right] \\
	&=q\left[-\nabla s+\left\{\nabla(\mathbf{A} \cdot \mathrm{v})-\frac{d \mathbf{A}}{d t}\right\}\right] \\
	&=q\left[-\nabla(s-\mathbf{A} \cdot \mathrm{v})-\frac{d \mathbf{A}}{d t}\right]
\end{aligned}$$
The $x$-component of the force is given by
$$
\begin{aligned}
F_{x} &=q\left[\{-\nabla(s-\mathbf{A} \cdot \mathrm{v})\}_{x}-\frac{d \mathbf{A}_{x}}{d t}\right] \\
&=q\left[-\frac{\partial}{\partial x}(s-\mathbf{A} \cdot \mathrm{v})-\frac{d \mathbf{A}_{x}}{d t}\right] \\
\text { or } \quad F_{x} &=q\left[-\frac{\partial}{\partial x}(s-\mathbf{A} \cdot \mathrm{v})-\frac{d}{d t}\left\{\frac{\partial}{\partial \mathrm{v}_{x}}(\mathbf{A} \cdot \mathrm{v}\}\right]\right.
\end{aligned}
$$
$q s \text { is independent of velocity } \mathrm{v}_{x} \text { i.e. } \frac{\partial(q s)}{\partial \mathrm{v}_{x}}=0$
So we can add that term to the above equation, it will not make any change.\\
After rewritting\\
$$\begin{aligned}
	&F_{x}=-\frac{\partial U}{\partial x}-\frac{d}{d t} \frac{\partial U}{\partial \mathrm{v}_{\mathrm{x}}} \\
	&U=q s-q(\mathbf{A} \cdot \mathbf{v})
\end{aligned}$$
FRom these it is clear that  U is a function of x and v ie $q_k$ and $\dot{q_k}$
U is called generalized potential or velocity dependent potential.\\
\paragraph{Lagrangian }
$L=T-U=T-\{q s-q(\mathbf{A} \cdot \mathbf{v}) .\}=T-q s+q(\mathbf{A} \cdot \mathbf{v})$\\
T=kinetic energy.$T=\frac{1}{2}mv^2$
$$L=\frac{1}{2}mv^2-q s+q(\mathbf{A} \cdot \mathbf{v})$$
\textbf{Momentum} $P_K=\frac{\partial L}{\partial \dot{q_k}}=\frac{\partial L}{\partial v_k}=mv_k+qA=mv+qA$\\
\paragraph{Hamiltonian}
$$\begin{aligned}
	H &=\sum_{k} p_{k} \dot{q}_{k}-L \\
	&=\sum_{k} p_{k} \dot{r}_{k}-L \\
	&=\sum_{k} p_{k} v_{k}-L \\
	&=\sum_{k}\left(m v_{k}+q A_{k}\right) v_{k}-\left[\frac{1}{2} m v^{2}-q s+q(\mathbf{v} \cdot \mathbf{A})\right] \\
	&=\sum_{k}\left(m v_{k}^{2}+q A_{k} v_{k}\right)-\frac{1}{2} m v^{2}+q s-q(\mathbf{v} \cdot \mathbf{A}) \\
	&=m v^{2}+q(\mathbf{A} \cdot \mathbf{v})-\frac{1}{2} m v^{2}+q s-q(\mathbf{v} \cdot \mathbf{A}) \\
	&=\frac{1}{2} m v^{2}+q s
\end{aligned}$$

