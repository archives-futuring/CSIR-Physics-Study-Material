\chapter{Potential Formulation and Radiation}
\section{Scalar and vector potentaial}
Maxwell's equations are
\begin{enumerate}[label=(\roman*)]
	\item $\boldsymbol{\nabla} \cdot \mathbf{E}=\frac{1}{\epsilon_{0}} \rho$
	\item $\boldsymbol{\nabla} \times \mathbf{E}=-\frac{\partial \mathbf{B}}{\partial t}$
	\item $\boldsymbol{\nabla} \cdot \mathbf{B}=0$
	\item $\boldsymbol{\nabla} \times \mathbf{B}=\mu_{0} \mathbf{J}+\mu_{0} \epsilon_{0} \frac{\partial \mathbf{E}}{\partial t}$
\end{enumerate}
If $\rho(r,t)$ and $J(r,t)$ are bnown electric and magnetic field can be find out using Gauss's law and Bio-savart law.It is difficult to find E ans B if they are time dependant.To solve this problem first we are going to represents the fields in terms of potentials electric potential V and magnetic potential B.\\
In the static case $\nabla \times \mathbf{E}=0$ So electric field cane written as a negative gradient of some scalar quandity called electric potential V\\
$$E=-\nabla V$$(not possible in elctrodynamics)\\
But $\nabla \cdot B=0$ always.Which gives \\
$$B=\nabla\times A$$
 Where A is the magnetic vector potential.Putting this value in 
$$\boldsymbol{\nabla} \times \mathbf{E}=-\frac{\partial \mathbf{B}}{\partial t}$$
We will get \\
$$\nabla \times \mathbf{E}=\frac{-\partial }{\partial t}(\nabla \times A)$$
$$\nabla \times \left( \mathbf{E}+\frac{\partial A}{\partial t}\right) =0$$
Again the curl of something become zero. so it can be written as negative gradient of potential
$$\mathbf{E}+\frac{\partial A}{\partial t}=-\nabla V$$
$$\mathbf{E}=-\nabla V-\frac{\partial A}{\partial t}$$
\begin{center}
	\framebox{
		\parbox[t][3cm]{3cm}{
			
			\addvspace{0.2cm} \centering
			
			\begin{align*}
			\begin{array}{lll}
			$$\mathbf{B}=\nabla\times A$$\\
			$$\mathbf{E}=-\nabla V-\frac{\partial A}{\partial t}$$
			\end{array}
			\end{align*}} }
\end{center}
If A and V are known we can find electric and magnetic field with these two equations.

Putting this equation $$\mathbf{E}=-\nabla V-\frac{\partial A}{\partial t}$$  in Gauss's law
we will get,\\
$$\nabla ^2V+\frac{\partial }{\partial t}(\nabla \cdot A)=\frac{-\rho}{\epsilon_{0}}$$
Putting $\mathbf{B}=\nabla\times A$  equation in Ampere/Maxwell's law and rearranging we will get,\\
$$\left( \nabla^2A-\mu_{0}\epsilon_{0}\frac{\partial^2 A}{\partial t^2}\right) -\nabla\left( \nabla \cdot A+\mu_{0}\epsilon_{0}\frac{\partial V}{\partial t}\right) =-\mu_{0} J$$ 

These two equations contain all information in the Maxwell's equations.\\
However we have succeeded in reducing six problems to find E and B ,down to four(V one component,A three component.),these equations are lengthy and difficult to find solution we have to abandon this potential formulation altogether.\\
\paragraph{Gauge transformation}
To avoid this problem we are transforming the potential equations by adding one extra term to A and V.This is called gauge tranformation.Consider the transformation occuring in the same fields E and B.Then\\
$$A^{\prime}=A+\alpha$$ and
$$V^{\prime}=V+\beta$$
Taking curl on each side of $A^{\prime}=A+\alpha$
we will get 
$$\nabla \times A^{\prime}=\nabla \times A+\nabla \times \alpha$$
Since B's are same we can written as($\nabla \times B=0,A^{\prime}=A$) \\
$$\nabla \times \alpha =0$$
Again curl of $\alpha$ is zero.then\\
$$\alpha=\nabla \lambda$$\\
The two potentials also gives the same E.so,\\
$$\nabla \beta +\frac{\partial \alpha }{\partial t}=0$$
$$\nabla\left( \beta+\frac{\partial \lambda}{\partial t}\right) =0$$
$$\beta= -\frac{\partial \lambda}{\partial t}$$
Now transformations become,\\

\begin{center}
	\framebox{
		\parbox[t][3cm]{3cm}{
			
			\addvspace{0.2cm} \centering
			
			\begin{align*}
			\begin{array}{lll}
			 $$A^{\prime}=A+\nabla \lambda$$\\
			 $$V^{\prime}=V-\frac{\partial \lambda}{\partial t}$$
			\end{array}
			\end{align*}} }
\end{center}
\subsection{Coulomb Gauge and Lorentz gauge}
\paragraph{Coluomb gauge}
$$\nabla \cdot A=0$$ is called Coulomb gauge.\\
\paragraph{importance}
$$\nabla \cdot A=0$$ Then the equation 
$$\nabla ^2V+\frac{\partial }{\partial t}(\nabla \cdot A)=\frac{-\rho}{\epsilon_{0}}$$ become,\\
$$\nabla^2 V=\frac{-\rho}{\epsilon_{0}}$$
This is poisson's equation.From this equation V can be findout by using the formula\\
$$V(r,t)=\frac{1}{4 \pi \epsilon_0}\int \frac{\rho(r^{\prime},t)}{r}d\tau^{\prime}$$
V is easy to get but to find E we need  A also($E=-\nabla V-\frac{\partial A}{\partial t}$) which is difficult.
\paragraph{Advantage of the coulomb gauge is that the scalar potential is simply to calculate .The disadvatage is that  A is particularly difficult to calculate. }
After applying coulomb gauge to the equation 
$$\left( \nabla^2A-\mu_{0}\epsilon_{0}\frac{\partial^2 A}{\partial t^2}\right) -\nabla\left( \nabla \cdot A+\mu_{0}\epsilon_{0}\frac{\partial V}{\partial t}\right) =-\mu_{0} J$$ 
we get\\
$$\left( \nabla^2A-\mu_{0}\epsilon_{0}\frac{\partial^2 A}{\partial t^2}\right)=-\mu_{0}J+\mu_{0} \epsilon_{0}\nabla \left( \frac{\partial V}{\partial t}\right) $$
\paragraph{Lorentz gauge}
$$\nabla \cdot A =-\mu_{0} \epsilon_{0} \frac{\partial}{\partial t}V$$

 is called the lorentz gauge.\\
 This designes to eliminate the middle term of the equation
$$\left( \nabla^2A-\mu_{0}\epsilon_{0}\frac{\partial^2 A}{\partial t^2}\right) -\nabla\left( \nabla \cdot A+\mu_{0}\epsilon_{0}\frac{\partial V}{\partial t}\right) =-\mu_{0} J$$
With lorentz gauge this equation become
$$\nabla^2A-\mu_{0}\epsilon_{0}\frac{\partial^2 A}{\partial t^2}=-\mu_{0} J$$
With lorentz gauge this equation 
$\nabla ^2V+\frac{\partial }{\partial t}(\nabla \cdot A)=\frac{-\rho}{\epsilon_{0}}$
becomes
$$\nabla^2A-\mu_{0}\epsilon_{0}\frac{\partial^2 A}{\partial t^2}=\frac{-\rho}{\epsilon_{0}}$$
The virtue of the lorentz gauge is that it treats V and A on the same differential operator
$$\nabla^2-\mu_{0} \epsilon_{0}\frac{\partial ^2}{\partial t^2}\equiv \square^2  $$
 is called d'Alembertian\\
Then both equation become\\
$$\square^2V=\frac{-\rho}{\epsilon_{0}}$$
$$\square^2A=-\mu_{0} J$$
\section{Retarded Potentials}
$$\square^{2} V=-\frac{1}{\epsilon_{0}} \rho, \quad \square^{2} \mathbf{A}=-\mu_{0} \mathbf{J}$$
In static case these equation reduces to poisson's equations
$$\nabla^{2} V=-\frac{1}{\epsilon_{0}} \rho, \quad \nabla^{2} \mathbf{A}=-\mu_{0} \mathbf{J}$$
with solutions
$$V(\mathbf{r})=\frac{1}{4 \pi \epsilon_{0}} \int \frac{\rho\left(\mathbf{r}^{\prime}\right)}{r} d \tau^{\prime}, \quad \mathbf{A}(\mathbf{r})=\frac{\mu_{0}}{4 \pi} \int \frac{\mathbf{J}\left(\mathbf{r}^{\prime}\right)}{r} d \tau^{\prime}$$
$r \rightarrow$ distance from source point $\vec{r}$ to the field point $r$.\\
Imagine a electromagnetic news travels at the speed of light. In non-static case therefore, it's not the status of source right now that matters but rather its condition at some earlier time $t_r$ (retarded time) when the message left.\\
Since the message must travel a distance $r$ the delay is $r/c$
$$t_{r}=t-r / c$$
$\therefore$ Potentials become
$$V(\mathbf{r})=\frac{1}{4 \pi \epsilon_{0}} \int \frac{\rho\left(\mathbf{r}^{\prime}\right)}{r} d \tau^{\prime}, \quad \mathbf{A}(\mathbf{r})=\frac{\mu_{0}}{4 \pi} \int \frac{\mathbf{J}\left(\mathbf{r}^{\prime}\right)}{r} d \tau^{\prime}$$
$P(r^\prime,t_r)$: charge density prevailed at point $r^\prime$ at the retarded time $t_r$. Because the integrants are evaluated at the retarded time these are called retarded potentials.\\
The more distant parts of the charge distribution have earlier retarded times than nearby ones.\\
In calculating the Laplacian of $V(r,t)$\\
The integrand depends on $\vec{r}$ in two places explicitly, in the denomenator $(r=|r-r^\prime|)$ and implicitly through $t_r-t-r/c$ in neumerator 
\begin{align*}
\nabla V&=\frac{1}{4 \pi \epsilon_{0}} \int\left[(\nabla \rho) \frac{1}{r}+\rho \nabla\left(\frac{1}{2}\right)\right] d \tau^{\prime}\\
\nabla \rho&=\dot{\rho} \nabla t_{r}=-\frac{1}{c} \dot{\rho} \nabla r\\
\nabla r&=\hat{r} \text{ and } \nabla \left( \frac{1}{r}\right) =\frac{-\hat{r}}{r^2}\\
\nabla V&=\frac{1}{4 \pi \epsilon_{0}} \int\left[-\frac{\rho}{c} \frac{\hat{r}}{r}-\rho \frac{\hat{\varepsilon}}{r^{2}}\right] d \tau^{\prime}
\intertext{Taking the divergence}
\nabla^{2} V&=\frac{1}{4 \pi \epsilon_{0}} \int\left\{-\frac{1}{c}\left[\frac{\hat{\imath}}{r} \cdot(\nabla \dot{\rho})+\dot{\rho} \nabla \cdot\left(\frac{\hat{r}}{r}\right)\right]\right.
-\left.-\left[\frac{\hat{z}}{r^{2}} \cdot(\nabla \rho)+\rho \nabla \cdot\left(\frac{\hat{\varepsilon}}{r^{2}}\right)\right]\right\} d \tau^{\prime} .\\
\nabla \dot{\rho}&=-\frac{1}{c} \ddot{\rho} \nabla r=-\frac{1}{c} \ddot{\rho} \hat{r},\quad 
\nabla \cdot\left(\frac{\hat{r}}{r}\right)=\frac{1}{r^{2}} 
\intertext{whereas}
&\nabla \cdot\left(\frac{\hat{r}}{r^{2}}\right)=4 \pi \delta^{3}(r)
\intertext{So}
\nabla^{2} V&=\frac{1}{4 \pi \epsilon_{0}} \int\left[\frac{1}{c^{2}} \frac{\ddot{\rho}}{r}-4 \pi \rho \delta^{3}(r)\right] d \tau^{\prime}=\frac{1}{c^{2}} \frac{\partial^{2} V}{\partial t^{2}}-\frac{1}{\epsilon_{0}} \rho(\mathbf{r}, t)\\
\nabla^{2} V&=\frac{1}{c^{2}} \frac{\partial^{2} V}{\partial t^{2}}-\frac{1}{\epsilon_{0}} \rho(\mathbf{r}, t)
\intertext{retarded potential satisfies the inhomogeneous wave equation}
\end{align*}
\section{Jefimentro's Equations}
\begin{align*}
V(r, t)&=\frac{1}{4 \pi \varepsilon_{0}} \int \frac{f\left(r^{\prime}, t_{1}\right)}{r} d z^{\prime},\quad A(r, t)=\frac{\mu_{0}}{4 \pi} \int \frac{J\left(r^{\prime}, t_{r}\right)}{r} d r !
\intertext{It is in principle, a straight forward matter to determine the fields}
E&=-\nabla V-\frac{\partial A}{\partial t}, B=\nabla \times A \\\\
E(r, t)&=\frac{1}{4 \pi \varepsilon_{0}} \int\left[\frac{\rho\left(r^{\prime}, t_{1}\right)}{x^{2}} \hat{x}+\right.\left.\frac{\dot{p}\left(r^{\prime}, t_{1}\right)}{c r} \hat{x}-\frac{\ddot{j}\left(r^{\prime}, t_{1}\right)}{c^{2} r}\right] d \tau^{\prime}
\intertext{This is the time dependent generalization of couloumb's law. In static case the second term and third term drops out and the first term losed its dependence on $t_r$}
B(r, t)&=\frac{\mu_{0}}{4 \pi}\int\left[\frac{J\left(r^{\prime}, f_{r}\right)}{\lambda^{2}}+\frac{J\left(r^{\prime}, t_{r}\right)}{c \lambda}\right]x r^2d z
\intertext{This is the time-denpendent generalzation of the Biot-savart law, to which it reduces in the static case.}
\end{align*}
\section{Poynting Theorem}
If there exist a continuous distribution of charge and current, the total rate of doing work by the fields in a finite volume $v$ is
$$\frac{d w}{d t}=\int_{v} \vec{\j} \cdot \vec{E} d^{3} x$$
This power represent a convertion of electromagnetic energy to mechanical or thermal energy. It must be balanced by corresponding rate  of decrease of energy in the electromagnetic field within the volume $V$. We can use maxwell equations to express the above equation in other terms.
\begin{align*}
\int_{V} J \cdot E d ^3 x&=\int_{V}\left[E \cdot(\nabla x H)-E \cdot \frac{\partial D}{\partial F}\right] d{ }^{3} x\\
\text{using the vector identity}&\\
\nabla \cdot(E \times H)&=H \cdot(\nabla \times \mathbf{E})-E \cdot(\nabla \times H)\\
\because \quad \int_{V} \vec{J} \cdot \vec{E} d^{3} x&=-\int_{V} \bar{V} \cdot(E \times M)+E \cdot \frac{\partial D}{\partial t}+H \cdot \frac{\partial B}{\partial t} \\
\int_{V} \vec{J} \cdot \vec{E} d^{3} x&=-\int_{V}\left[\nabla \cdot(E \times H)+E \cdot \frac{\partial D}{\partial t}+H \cdot \frac{\partial B}{\partial t}\right] d^{3} x
\end{align*}
\textbf{Assumptions}\\
\begin{enumerate}
	\item The mactoscopic medium is lenear in its electric and magnetic properties, with negligible dispersion or losses.
	\item The sum of work necessary to assemble  a static charge distribution and work required to get currents going represent the total deelectromagnetic energy density.
\end{enumerate}
\begin{align*}
u&=\frac{1}{2}(E \cdot D+B \cdot H)\\
-\int_{v} \vec{J} \cdot t^{2} d^{3} x&=\int_{v}\left[\frac{\partial u}{\partial t}+\nabla \cdot(E \times H)\right] d^{3} x
\intertext{since the volume $V$ is arbitrary, this can be cast into the form of a differential contunuity equation or conservation law,}
\frac{\partial u}{\partial t}+\nabla \cdot s&=-j \cdot E\\
\text{The vector $S$ representing }&\text{energy flow is called the poynting vector}\\
S&=E \times H\\
\text{It has diamensions of }&\frac{\text{energy}}{\text{area}\times\text{time}}
\end{align*}
\subsection{Poynting Theorem for Microscopic Fields}
Matter is ultimately composed of charged particles, we can think of this rate of conversion of electromagnetic energy to machanical energy as the rate of increase of energy of charged particles per unit volume\\
We can interpret pointing's theorem for the microscopic fields (E.B) as a statement of conservation of energy of the combined system of particles and fields.\\
Let $E_{mech}$ be the total energy of the particles within the volume $V$ assume no particles move out of the volume.
$$\frac{d E_{\text {mech }}}{d t}=\int_{v} \vec{J} \cdot \vec{E} d^{3} x$$
Poynting theorem express the conservation of energy for the combined system as
$$\frac{d E}{d t}=\frac{d}{d t}\left(E_{\text {mech }}+E_{\text {field }}\right)=-\oint_{s} \vec{n} \cdot \vec{s} d a$$
when the total field energy within $V$ is
$$E_{\text {field } }=\int_{V} u d^{3} x=\frac{\varepsilon_{0}}{2} \int_{V}\left(E^{2}+c^{2} B^{2}\right) d^{3} x$$

\newpage
\begin{abox}
	Practise Set-1
\end{abox}
\begin{enumerate}
	\item  For constant uniform electric and magnetic field $\vec{E}=\vec{E}_{0}$ and $\vec{B}=\vec{B}_{0}$, it is possible to choose a gauge such that the scalar potential $\phi$ and vector potential $\vec{A}$ are given by
	{\exyear{NET/JRF(JUNE-2011)}}
	\begin{tasks}(1)
		\task[\textbf{A.}] $\phi=0$ and $\vec{A}=\frac{1}{2}\left(\vec{B}_{0} \times \vec{r}\right)$
		\task[\textbf{B.}] $\phi=-\vec{E}_{0} \cdot \vec{r}$ and $\vec{A}=\frac{1}{2}\left(\vec{B}_{0} \times \vec{r}\right)$
		\task[\textbf{C.}]  $\phi=-\vec{E}_{0} \cdot \vec{r}$ and $\vec{A}=0$
		\task[\textbf{D.}] $\phi=0$ and $\vec{A}=-\vec{E}_{0} t$
	\end{tasks}
	
	\item	A constant electric current $I$ in an infinitely long straight wire is suddenly switched on at $t=0$. The vector potential at a perpendicular distance $r$ from the wire is given by $\vec{A}=\frac{\hat{k} \mu_{0} I}{2 \pi} \ln \left[\frac{1}{r}\left(c t+\sqrt{c^{2} t^{2}-r^{2}}\right)\right]$. The electric field at a distance $r(<c t)$ is
	{\exyear{NET/JRF(DEC-2011)}}
	\begin{tasks}(2)
		\task[\textbf{A.}] 0
		\task[\textbf{B.}] $\frac{\mu_{0} I}{2 \pi t} \frac{1}{\sqrt{2}}(\hat{i}-\hat{j})$
		\task[\textbf{C.}] $\frac{c \mu_{0} I}{2 \pi \sqrt{c^{2} t^{2}-r^{2}}} \frac{1}{\sqrt{2}}(\hat{i}+\hat{j})$
		\task[\textbf{D.}] $-\frac{c \mu_{0} I}{2 \pi \sqrt{c^{2} t^{2}-r^{2}}} \hat{k}$
	\end{tasks}
	\item Consider an infinite conducting sheet in the $x y$-plane with a time dependent current density $K t \hat{i}$, where $K$ is a constant. The vector potential at $(x, y, z)$ is given by $\vec{A}=\frac{\mu_{0} K}{4 c}(c t-z)^{2} \hat{i}$. The magnetic field $\vec{B}$ is
	{	\exyear{NET/JRF(DEC-2012)}}
	\begin{tasks}(2)
		\task[\textbf{A.}] $\frac{\mu_{0} K t}{2} \hat{j}$
		\task[\textbf{B.}] $-\frac{\mu_{0} K z}{2 c} \hat{j}$
		\task[\textbf{C.}] $-\frac{\mu_{0} K}{2 c}(c t-z) \hat{i}$
		\task[\textbf{D.}] $-\frac{\mu_{0} K}{2 c}(c t-z) \hat{j}$
	\end{tasks}
	\item A current $I$ is created by a narrow beam of protons moving in vacuum with constant velocity $\vec{u}$. The direction and magnitude, respectively of the Poynting vector $\vec{S}$ outside the beam at a radial distance $r$ (much larger than the width of the beam) from the axis, are
	{	\exyear{NET/JRF(JUNE-2013)}}
	\begin{tasks}(2)
		\task[\textbf{A.}] $\vec{S} \perp \vec{u}$ and $|\vec{S}|=\frac{I^{2}}{4 \pi^{2} \varepsilon_{0}|\vec{u}| r^{2}}$
		\task[\textbf{B.}] $\vec{S} \|(-\vec{u})$ and $|\vec{S}|=\frac{I^{2}}{4 \pi^{2} \varepsilon_{0}|\vec{u}| r^{4}}$
		\task[\textbf{C.}] $\vec{S} \| \vec{u}$ and $|\vec{S}|=\frac{I^{2}}{4 \pi^{2} \varepsilon_{0}|\vec{u}| r^{2}}$
		\task[\textbf{D.}] $\vec{S} \| \vec{u}$ and $|\vec{S}|=\frac{I^{2}}{4 \pi^{2} \varepsilon_{0}|\vec{u}| r^{4}}$
	\end{tasks}
	\item
	If the electric and magnetic fields are unchanged when the potential $\vec{A}$ changes (in suitable units) according to $\vec{A} \rightarrow \vec{A}+\hat{r}$, where $\vec{r}=r(t) \hat{r}$, then the scalar potential $\Phi$ must simultaneously change to
	{	\exyear{NET/JRF(JUNE-2013)}}
	\begin{tasks}(4)
		\task[\textbf{A.}] $\Phi-r$
		\task[\textbf{B.}] $\Phi+r$
		\task[\textbf{C.}] $\Phi-\partial \mathrm{r} / \partial t$
		\task[\textbf{D.}] $\Phi+\partial \mathrm{r} / \partial t$
	\end{tasks}
	\item
	Let $(V, \vec{A})$ and $\left(V^{\prime}, \overrightarrow{A^{\prime}}\right)$ denote two sets of scalar and vector potentials, and $\psi$ is a scalar function. Which of the following transformations leave the electric and magnetic fields (and hence Maxwell's equations) unchanged?
	{\exyear{NET/JRF(DEC-2013)}}
	\begin{tasks}(2)
		\task[\textbf{A.}] $\overrightarrow{A^{\prime}}=\vec{A}+\nabla \psi$ and $V^{\prime}=V-\frac{\partial \psi}{\partial t}$
		\task[\textbf{B.}] $\overrightarrow{A^{\prime}}=\vec{A}-\nabla \psi$ and $V^{\prime}=V+2 \frac{\partial \psi}{\partial t}$
		\task[\textbf{C.}] $\overrightarrow{A^{\prime}}=\vec{A}+\nabla \psi$ and $V^{\prime}=V+\frac{\partial \psi}{\partial t}$
		\task[\textbf{D.}] $\overrightarrow{A^{\prime}}=\vec{A}-\nabla \psi$ and $V^{\prime}=V-\frac{\partial \psi}{\partial t}$
	\end{tasks}
	\item
	A time-dependent current $\vec{I}(t)=K t \hat{z}$ (where $K$ is a constant) is switched on at $t=0$ in an infinite current-carrying wire. The magnetic vector potential at a perpendicular distance $a$ from the wire is given (for time $t>a / c$ ) by
	{	\exyear{NET/JRF(JUNE-2014)}}
	\begin{tasks}(2)
		\task[\textbf{A.}]  $\hat{z} \frac{\mu_{0} K}{4 \pi c} \int_{-\sqrt{c^{2} t^{2}-a^{2}}}^{\sqrt{c^{2} t^{2}-a^{2}}} d z \frac{c t-\sqrt{a^{2}+z^{2}}}{\left(a^{2}+z^{2}\right)^{1 / 2}}$
		\task[\textbf{B.}]  $\hat{z} \frac{\mu_{0} K}{4 \pi} \int_{-c t}^{c t} d z \frac{t}{\left(a^{2}+z^{2}\right)^{1 / 2}}$
		\task[\textbf{C.}] $\hat{z} \frac{\mu_{0} K}{4 \pi c} \int_{-c t}^{c t} d z \frac{c t-\sqrt{a^{2}+z^{2}}}{\left(a^{2}+z^{2}\right)^{1 / 2}}$
		\task[\textbf{D.}] $\hat{z} \frac{\mu_{0} K}{4 \pi} \int_{-\sqrt{c^{2} t^{2}-a^{2}}}^{\sqrt{c^{2} t^{2}-a^{2}}} d z \frac{t}{\left(a^{2}+z^{2}\right)^{1 / 2}}$
	\end{tasks}
	
	
	\item The vector potential $\vec{A}=k e^{-a t} r \hat{r}$ (where $a$ and $k$ are constants) corresponding to an electromagnetic field is changed to $\overrightarrow{A^{\prime}}=-k e^{-a t} r \hat{r}$. This will be a gauge transformation if the corresponding change $\phi^{\prime}-\phi$ in the scalar potential is
	{\exyear{NET/JRF(JUNE-2017)}}
	\begin{tasks}(4)
		\task[\textbf{A.}] $a k r^{2} e^{-a t}$
		\task[\textbf{B.}] $2 a k r^{2} e^{-a t}$
		\task[\textbf{C.}] $-a k r^{2} e^{-a t}$
		\task[\textbf{D.}] $-2 a k r^{2} e^{-a t}$
	\end{tasks}
	\item The charge distribution inside a material of conductivity $\sigma$ and permittivity $\in$ at initial time $t=0$ is $\rho(r, 0)=\rho_{0}$, a constant. At subsequent times $\rho(r, t)$ is given by
	{\exyear{NET/JRF(JUNE-2017)}}
	\begin{tasks}(2)
		\task[\textbf{A.}]  $\rho_{0} \exp \left(-\frac{\sigma t}{\epsilon}\right)$
		\task[\textbf{B.}] $\frac{1}{2} \rho_{0}\left[1+\exp \left(\frac{\sigma t}{\in}\right)\right]$
		\task[\textbf{C.}]  $\frac{\rho_{0}}{\left[1-\exp \left(\frac{\sigma t}{\epsilon}\right)\right]}$
		\task[\textbf{D.}] $\rho_{0} \cosh \frac{\sigma t}{\in}$
	\end{tasks}
	\item  The electric field $\vec{E}$ and the magnetic field $\vec{B}$ corresponding to the scalar and vector potentials, $V(x, y, z, t)=0$ and $\vec{A}(x, y, z, t)=\frac{1}{2} \hat{k} \mu_{0} A_{0}(c t-x)$, where $A_{0}$ is a constant, are 
	{\exyear{NET/JRF(JUNE-2018)}}
	\begin{tasks}(2)
		\task[\textbf{A.}] (a) $\vec{E}=0$ and $\vec{B}=\frac{1}{2} \hat{j} \mu_{0} A_{0}$
		\task[\textbf{B.}] $\vec{E}=-\frac{1}{2} \hat{k} \mu_{0} A_{0} c$ and $\vec{B}=\frac{1}{2} \hat{j} \mu_{0} A_{0}$
		\task[\textbf{C.}]  $\vec{E}=0$ and $\vec{B}=-\frac{1}{2} \hat{i} \mu_{0} A_{0}$
		\task[\textbf{D.}] $\vec{E}=\frac{1}{2} \hat{k} \mu_{0} A_{0} c$ and $\vec{B}=-\frac{1}{2} \hat{i} \mu_{0} A_{0}$
	\end{tasks}
\item Which of the following transformations $(V, \vec{A}) \rightarrow\left(V^{\prime}, \overrightarrow{A^{\prime}}\right)$ of the electrostatic potential $V$ and the vector potential $\vec{A}$ is a gauge transformation?
{\exyear{ NET/JRF-(JUNE-2015)}}
 \begin{tasks}(2)
	\task[\textbf{a.}]$\left(V^{\prime}=V+a x, \vec{A}^{\prime}=\vec{A}+a t \hat{k}\right)$
	\task[\textbf{b.}]$\left(V^{\prime}=V+a x, \vec{A}^{\prime}=\vec{A}-a t \hat{k}\right)$
	\task[\textbf{c.}]$\left(V^{\prime}=V+a x, \vec{A}^{\prime}=\vec{A}+a t \hat{i}\right)$
	\task[\textbf{d.}] $\left(V^{\prime}=V+a x, \vec{A}^{\prime}=\vec{A}-a t \hat{i}\right)$
\end{tasks}
\end{enumerate}



\newpage
\begin{abox}
	Practise Set-2
\end{abox}
\begin{enumerate}
	\begin{minipage}{\textwidth}
		\item  The electric and the magnetic field $\vec{E}(z, t)$ and $\vec{B}(z, t)$, respectively corresponding to the scalar potential $\phi(z, t)=0$ and vector potential $\vec{A}(z, t)=\hat{i} t z$ are
		\exyear{GATE 2012}
	\end{minipage}
	\begin{tasks}(2)
		\task[\textbf{A.}] $\vec{E}=\hat{i} z$ and $\vec{B}=-\hat{j} t$
		\task[\textbf{B.}]$\vec{E}=\hat{i} z$ and $\vec{B}=\hat{j t}$
		\task[\textbf{C.}]$\vec{E}=-\hat{i} z$ and $\vec{B}=-\hat{j t}$
		\task[\textbf{D.}]$\vec{E}=-\hat{i} z$ and $\vec{B}=-\hat{j} \mathrm{t}$
	\end{tasks}
	\begin{minipage}{\textwidth}
		\item If the vector potential $\vec{A}=\alpha x \hat{x}+2 y \hat{y}-3 z \hat{z}$, satisfies the Coulomb gauge, the value of the constant $\alpha$ is
		\exyear{GATE 2015}
	\end{minipage}
	\begin{minipage}{\textwidth}
		\item Consider magnetic vector potential $\tilde{A}$ and scalar potential $\Phi$ which define the magnetic field $\vec{B}$ and electric field $\vec{E}$. If one adds $\vec{\nabla} \lambda$ to $\vec{A}$ for a well-defined $\lambda$, then what should be added to $\Phi$ so that $\vec{E}$ remains unchanged up to an arbitrary function of time, $f(t)$ ?
		\exyear{JEST 2017}
	\end{minipage}
	\begin{tasks}(2)
		\task[\textbf{A.}] $\frac{\partial \lambda}{\partial t}$
		\task[\textbf{B.}]$-\frac{\partial \lambda}{\partial t}$
		\task[\textbf{C.}]$\frac{1}{2} \frac{\partial \lambda}{\partial t}$
		\task[\textbf{D.}]$-\frac{1}{2} \frac{\partial \lambda}{\partial t}$
	\end{tasks}
	\item A long solenoid is embedded in a conducting medium and is insulated from the medium. If the current through the solenoid is increased at a constant rate, the induced current in the medium as a function of the radial distance $r$ from the axis of the solenoid is proportional to
	{\exyear{GATE 2015}}
	\begin{tasks}(1)
		\task[\textbf{A.}] $r^{2}$ inside the solenoid and $\frac{1}{r}$ outside $\quad$  
		\task[\textbf{B.}]$r$ inside the solenoid and $\frac{1}{r^{2}}$ outside
		\task[\textbf{C.}] $r^{2}$ inside the solenoid and $\frac{1}{r^{2}}$ outside
		\task[\textbf{D.}]$r$ inside the solenoid and $\frac{1}{r}$ outside
	\end{tasks}
	\item An infinitely long straight wire is carrying a steady current $I$. The ratio of magnetic energy density at distance $r_{1}$ to that at $r_{2}\left(=2 r_{1}\right)$ from the wire is
	{	\exyear{GATE 2018}}
	\item Consider an infinitely long solenoid with $N$ turns per unit length, radius $R$ and carrying a current $I(t)=\alpha \cos \omega t$, where $\alpha$ is a constant and $\omega$ is the angular frequency. The magnitude of electric field at the surface of the solenoid is
	{	\exyear{GATE 2018}}
	\begin{tasks}(2)
		\task[\textbf{A.}] $\frac{1}{2} \mu_{0} N R \omega \alpha \sin \omega t$
		\task[\textbf{B.}]$\frac{1}{2} \mu_{0} \omega N R \cos \omega t$
		\task[\textbf{C.}]$\mu_{0} N R \omega \alpha \sin \omega t$
		\task[\textbf{D.}]$\mu_{0} \omega N R \cos \omega t$
	\end{tasks}
	
	\item A long straight wire, having radius $a$ and resistance per unit length $r$, carries a current $I$. The magnitude and direction of the Poynting vector on the surface of the wire is
	{\exyear{GATE 2018}}
	
	\begin{tasks}(1)
		\task[\textbf{A.}] $I^{2} r / 2 \pi a$, perpendicular to axis of the wire and pointing inwards
		\task[\textbf{B.}]$I^{2} r / 2 \pi a$, perpendicular to axis of the wire and pointing outwards
		\task[\textbf{C.}]$I^{2} r / \pi a$, perpendicular to axis of the wire and pointing inwards
		\task[\textbf{D.}]$I^{2} r / \pi a$, perpendicular to axis of the wire and pointing outwards
	\end{tasks}
\end{enumerate}
