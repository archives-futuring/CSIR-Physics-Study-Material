
\chapter{Introduction to Thermodynamics}
\section{State of a System}
In thermodynamics a different and much simpler concept of the state of a system is introduced other than that in Mechanics. Indeed, to use the dynamical definition of state would be inconvenient, because all the systems which are dealt with in thermodynamics contain a very large number of mass-points (the atoms or molecules). In order to explain the thermodynamic concept of the state of a system, we shall first discuss a simple example. A system composed of a chemically defined homogeneous fluid.\\ We can make the following measurements on such a system: the temperature $t$, the volume $V$, and the pressure $p$. For a given amount of the substance contained in the system, the temperature, volume, and pressure are not independent quantities; they are connected by a relationship of the general form:
\begin{equation}
f(p, V, t)=0
\end{equation}\label{Therm1}
Which is called the equation of state. Its form depends on the special properties of the substance. Any one of the three variables in the above relationship can be expressed as a function of the other two by solving equation\ref{Therm1} with respect to the given variable. Therefore, the state of the system is completely determined by any two of the three quantities, $p, V$, and $t$.