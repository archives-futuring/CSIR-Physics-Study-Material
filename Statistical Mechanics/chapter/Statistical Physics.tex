\chapter{Statistical Physics}
\par In statistical mechanics we try to find the answers to questions like how are the molecules distributed in space, how they are distributed in velocity.
\section{Degrees of freedom}
The degrees of freedom of any dynamical system is defined as the total number of independent coordinates necessary to specify the position and configuration of the system completely\\
For a particle in translatory motion three coordinates $(x, y, z)$ are needed to define its position. To describe energy of the particle three momentum coordinates $\left(p_{x}, p_{y}, p_{z}\right)$ are required. Hence to specify position and energy, 6 -coordinates are needed. We say the particle has 6 degrees of freedom.
\section{Phase Space}
To specify the position and energy we require three position (space) coordinates $\mathrm{x}, \mathrm{y}, \mathrm{z}$ and three momentum coordinates $\mathrm{p}_{\mathrm{x}}, \mathrm{p}_{\mathrm{y}}$ and $\mathrm{p}_{\mathrm{z}}$. We imagine a six dimensional space in which coordinates are $x, y, z, p_{x}, p_{y}$ and $\mathrm{p}_{z}$. It is an imaginary space and is a combination of physical space and momentum space. This six dimensional space for a single particle is called phase space or $\mu$-space. The instantaneous state of a particle in the phase space is denoted by a point known as plase point. An element of volume $d x d y d z d p_{x} d p_{y} d p_{z}$ in six dimensional space is called a cell. A phase space can be divided into a large number of cells. A cell contains a large number of phase points. The dimensions of phase space depend upon the degrees of freedom.
\section{Statistical Probability}
Probability of a particular event is the ratio of number of cases in which the event occurs to the total number of possible events. The theory of probability is a method for making better guesses. We need this in statistics. Because, you know in statistical mechanics we deal with a very large number of particles. So nothing can be said about a particular particle with definiteness. We can only make a guess about its behaviour.\\
Let us toss a coin to find whether we get the head or tail. In tossing of a coin, the total number possibilities or total number of events is two. So the chance of getting a head $=1 / 2$. This chance is called probability in statistics. This means you toss a coin 100 times. Then also this chance of getting a head will be 50 out of 100 . i.e., probability is $1 / 2$.\\
\textbf{The probability of an event is equal to the ratio of the number of favourable events to the total number of equally likely ways of happening of that event.}\\
$$\text{Probability of an event }=\frac{\text{Number of favourable events}}{\text{Total number of equally likely events}}$$
\subsection{Probability Theorems}
\subsection{Additional Theorem}
First let us consider an example. An opaque box contains 3 red balls, 4 blue balls and 6 black balls. If a ball is drawn from the bag what is the probability that it is either blue or black?
\begin{align*}
\renewcommand*{\arraystretch}{1.5}
\begin{tabular}{p{6cm}p{2cm}}
\text{Total number of balls}&=3+4+6=13\\
\text{Probability of blue balls $p_1$}&=4/13\\
\text{Probability of black balls $p_2$}&=6/13\\
\text{Total probability of white or black ball}&=4/13+6/13\\
 &10/13
\end{tabular}
\end{align*}
\subsection{Addition law of probability}
If $P_{1}, P_{2}, \ldots . P_{n}$ be the separate probability of mutually exclusive events, then probability, $P$ that any of these events will happen is $P=$ $P_{1}+P_{2}+\ldots+P_{n}$.\\
\subsection{Multiplication Law of Probability}
If the probabilities of occurrence of two independent events are $P_{1}$ and $P_{2}$, the probability of occurrence of the two events simultaneously is the product of $P_{1}$ and $P_{2}$.
$$\text { i.e. } P=P_{1} \times P_{2}$$
\textbf{Solved Examples}\\
You are given four particles $a, b, c$ and $d$. What are the different ways in which they can be distributed in two identical halves of a box? Also calculate the probabilities of different distributions? What is the frequency with which these distribution occur?\\
$\begin{array}{ll}\text{Distribution}&\text{ Probability }\\(4,0) & 1 / 16 \\ (3,1) & 4 / 16 \\ (2,2) & 6 / 16 \\ (1,3) & 4 / 16 \\ (0,4) & 1 / 16\end{array}$\\
	\begin{table}[ht]
	\caption{Multi-row table}
	\begin{center}
		\begin{tabular}{|c|c|c|c|}
			\hline
			$\text{Left half}$&$\text{Right half}$&$\text{Distribution}$&$\text{Freq.}$\\\hline
			a,b,c,d&---&4,0&1\\\hline
			b,c,d&a&\multirow{4}{*}{(3,1)}&\multirow{4}{*}{4}\\\cline{1-2}
			a,c,d&b & &\\\cline{1-2}
			a,b,d&c& &\\\cline{1-2}
			a,b,c&d& &\\\cline{1-2}\hline
			a,b&c,d&\multirow{6}{*}{(2,2)}&\multirow{6}{*}{(6)}\\\cline{1-2}
			a,c&bd& & \\\cline{1-2}
			a,d& b,c& & \\\cline{1-2}
			c,d&a,b& & \\\cline{1-2}
			b,c& a,d& & \\\cline{1-2}
			b,d&a,c& & \\\cline{1-2}\hline
			a&b,c,d&\multirow{4}{*}{(1,3)}&\multirow{4}{*}{(4)}\\\cline{1-2} 
			b&a,c,d& & \\\cline{1-2}
			c&a,b,d & &\\\cline{1-2}
			d&a,b,c & &\\\hline
			---&a,b,c,d &(0,4) &1 \\\hline
		\end{tabular}
	\end{center}
	\label{tab:multicol}
\end{table}
\section{Macroscopic and Microscopic Coordinates}
A system whose size is of the order of atomic dimensions or smaller $(<10 \AA)$ is called a microscopic system. Example, a molecule, an atom. If the size of the system is very large compared to size of an atom, say greater than one micron it is called a macroscopic system. A macroscopic system contains a large number of particles.\\
\par Let us consider a system, say a cylinder fitted with a frictionless weightless piston containing a gas at a pressure $\mathrm{P}$ and temperature $\mathrm{T}$. The state of the gas can be specified by $P, T$ and volume $V$ of the gas. These coordinates are called macroscopic coordinates.\\
\par For the microscopic desvription of the state of the gas we need the position and momentum coordinates of all the $N$ coordinates say $\left(x_{1}, y_{1}, z_{1}\right), \quad\left(x_{2}, y_{2}, z_{2}\right) \ldots \ldots .\left(x_{n}, y_{n}, z_{n}\right)$ and $\left(p_{x 1}, p_{y 1}, p_{z 1}\right),\left(p_{x 2}, p_{y 2}, p_{y 3}\right) \ldots \ldots \ldots$\\$\left(p_{x N}, \bar{p}_{y N}, p_{x N}\right)$ are called micro-coordinates.
\subsection{Macrostates and Microstates}
\par Consider a system consisting of $\mathrm{N}$ structureless identical weakly interacting particles, occupying, a fixed volume $V$. Let the internal energy of the system be U. The state of the system is completely determined by $6 \mathrm{~N}$ coordinates; $3 \mathrm{~N}$ position coordinates and $3 \mathrm{~N}$ momentum coordinates. Since there are $6 \mathrm{~N}$ coordinates this phase space is $6 \mathrm{~N}$ dimensional. Let the phase space be divided into allowed discrete energy levels and all the $\mathrm{N}$ particles be distributed among these energy levels. Let there be $n_{1}$ particles in the energy state $E_{1}, n_{2}$ in $E_{2}$ and so on. Then
$$n_{1}+n_{2}+n_{3} \ldots \ldots=\sum_{i} n_{i}=N$$
$$\mathrm{n}_{1} \mathrm{E}_{1}+\mathrm{n}_{2} \mathrm{E}_{2}+\ldots \ldots \ldots=\sum_{\mathrm{i}} \mathrm{n}_{\mathrm{i}} \mathrm{E}_{\mathrm{i}}=\mathrm{U}$$
\par The set of numbers, $\mathrm{n}_{1}, \mathrm{n}_{2}, \mathrm{n}_{3}$ of the particles represent one set of distribution of $\mathrm{N}$ particles corresponding to the same macrostate of the system. We can have different sets of numbers of particles in the energy levels satisfying equations (1) and (2). These different sets of number of particles give different distributions corresponding to the same macrostate. Thus a macrostate is specifled by just giving the number of particles in each energy state. Macrostate is a state of a system which is represented by the macro properties like pressure, volume, temperature, density etc. in the equilibrium state.\\\\
For defining a microstate we should specify to which energy state each particle of the system belongs at a particular instant. For example in solved problem $5.2$ each arrangement like $(a b c, d)$ or (acd, b) is a microstate. There are five macrostates in the example 5.1. The number of microstates corresponding to the given macrostate of the system is called thermodynamic probability or thermodynamic frequency.
\par For indistinguishable partieles, each distribution of the particles among the energy levels, corresponding to the same macrostate of the system is called a microstate, or the state of a system represented by the instant positions and momenta of all the particles is called the microstate. The microstate may change continuously with time. So corresponding to a macrostate there will be a large number of microstates.
\subsection{Constraints and Accessible State}
Physical laws impose certain restrictions on the distribution of the molecules among various energy states in phase space and are called constraints of the system. The total number of molecules in a system remains a constant. This is a constraint. i.e. $\sum \mathrm{n}_{\mathrm{i}}=\mathrm{N}=\mathrm{a}$ constant. Another constraint is that the total energy of the system is a constant. $\sum n_{i} E_{i}=U=a$ costant\\
\par Those microstates into which the molecules are permitted to go under the constraints imposed on the system are called accessible microstates. States which are not permitted by constraints are called non-accessible states., \\
\subsection{More About Phase Space}
Phase space is an imaginary space which combines the physical space and the momentum space. It is used to represent the state (phase) of each śystem.\\
\par To specify the position of one particle in space we need three position coordinates $(x, y, z)$. Suppose the momentum of the particle is $p$. Then it can be resolved into three components along $x, y$ and $z$ axes as $p_{x}, p_{y}$ and $\mathrm{p}_{z}$ respectively. These components are called momentum coordinates. Hence the dynamical state of a particle can be represented by six coordinates - 3 position coordinates and 3 momentum coordinates.\\
\par A space where the three position coordinates can be represented along three 'mutually perpendicular axes is called the physical space [Fig. $4.1$ (a)]. Volume of a small element in this space is $d V=d x d y d z$.
\begin{figure}[H]
	\centering
	\includegraphics[height=3.5cm,width=8cm]{SP-01}
\end{figure}
\par A space where the three momentum coordinates are represented along the three mutually perpendicular directions is called the momentum space. In Fig. 4.1 (b) the point of a small element in this space $=d p_{x} d p_{y} d p_{z}$\\
\par Fig. 4.1(a) gives only the instantaneous position of a particle anc Fig. (b) gives only the instantaneous momentum. Suppose we want to specify both this position and momentum at a particular instant in a single space. Such a space is called phase space. In phase space a point $P$ has 6 coordinates, $\left(x, y, z, p_{x}, p_{y}\right.$ and $\left.p_{z}\right)$. So it is a 6 dimensional space. Any point $\mathrm{P}$ in this space is called a phase point. The phase point gives a complete description of the dynamical state of the particle at any instant.\\
\par The volume of an infinitesimal element in phase space $=d \tau$
$$
d \tau=d x d y d z d p_{x} d p_{y} d p_{z}
$$
Let us now consider a system consisting of N-particles. For one particle there are 6 coordinates. So for N-particle there are $6 \mathrm{~N}$ coordinates. This means to specify the state of the system completely we need $3 \mathrm{~N}$ position coordinates and $3 \mathrm{~N}$ momentum coordinates. The instantaneous position of a system(i.e., N-particles) can be represented by a single point in the $6 \mathrm{~N}$-dimensional phase space.
\subsection{$\mu$.Space and $\Gamma$-space}
\par The 6-dimensional phase space for a single particle is called the h-space. The $6 \mathrm{~N}$-dimensional phase space for $\mathbf{N}$ particles is termed as -space ( $\gamma$-space). Thus gamma space is built up as the product of $N, \mu$ spaces.\\
\par Suppose a particle has $f$ degrees of freedom. Then to specify its state, f position coordinates and $\mathrm{f}$ momentum coordinates are required.
If there are $N$ particles in this system then to represent the dynamical state of the system at any instant the total number of coordinates required are $N f$ position coordinates and $N f$ momentum coordinates The phase space of the system will be a $2 \mathrm{~N} f$ dimensional space. We can use generalised coordinates to represent a system. Let $q_{i}$ represent the generalised position coordinate and $\mathrm{p}_{\mathrm{j}}$ represent the generalised momentum coordinates. For $N$ independent particles, in the 6 dimensional phase space, there will be $3 \mathrm{~N}$ position coordinates
$q_{1}, q_{2}, \ldots \ldots . q_{N}$ and $3 N$ momentum coordinates $p_{1}, p_{2}, \ldots \ldots \ldots \ldots p_{N}$
\subsection{Phase Cell}
\begin{figure}[H]
	\centering
	\includegraphics[height=2cm,width=2cm]{SP-02}
\end{figure}
\par Consider a particle in one dimension. Its state can be specified using the position coordinates $p$ and the momentum coordinates $q$ as shown in Fig.4.2 The two coordinates change with time and the representative point moves through phase space and the line traced is called phase line or phase trajectory. Each point on the phase line represents one possible microstate.\\
\textbf{Example}\\
Consider one dimensional harmonic oscillator of mass $m$ and spring constant $\mathrm{k}$. The total energy of the oscillator is $\mathrm{E}=\frac{\mathrm{p}^{2}}{2 \mathrm{~m}}+\frac{1}{2} \mathrm{kq}^{2}$, where $p$ is its momentum and $q$ its position coordinate. If $E$ is constant then this equation represents an ellipse in the phase space.
\begin{figure}[H]
	\centering
	\includegraphics[height=2.5cm,width=4cm]{SP-03}
\end{figure}
$$\frac{q^{2}}{(2 E / k)}+\frac{p^{2}}{2 m E}=1$$
The semi major axis of the ellipse [see Fig. 4.3] is $\sqrt{2 \mathrm{E} / \mathrm{k}}$ and the semi minor axis is $\sqrt{2 \mathrm{mE}}$.\\
\par We can divide the phase space into small volumes called phase cell. In the $\mu$-space of a particle, the volume element $d \tau=d x\  d y \ d z\  p_{x}\  p_y \ p_z$ . $d\tau$  is the volume of a six dimensional cell having sides ${d x, d y, d z}$ $d p_{x}, d p_{y}, d p_{z}$. Such a cell of minimum volume is called a unit cell in the $\mu$-space.
$$(d \tau)_{\min }=\left(d x d y d z d p_{x} d p_{y} d p_{z}\right)_{\min }$$
$$=\left(\delta x \delta p_{x}\right)_{\min }\left(\delta y \delta p_{y}\right)_{\min }\left(\delta z \delta p_{z}\right)_{\min }$$
According to classical mechanics the volume d$\Gamma$
 l can take any minimum value. But in quantum mechanics, according to Heisenberg's uncertainty principle the minimum value of the products is approximately equal to the Planck's constant $h$.\\
$$\text{ So }(d \tau)_{\min }=\mathrm{h} \times \mathrm{h} \times \mathrm{h}=\mathrm{h}^{3}$$
 The minimum value of volume element of $\mu$-space is $h^{3}$, where 3 is the number of degrees of freedom.\\
 \par Consider a system consisting of $\mathrm{N}$ particles having $f$ degrees of freedom each. Then the phase space will be $2 \mathrm{fN}$ dimensional and the volume of the cell be $\mathrm{h}^{\mathrm{Nf}}$.
 \begin{align*}
  \text{The volume}&\text{ of one element in the $\Gamma$ space}\\
  d\Gamma&=\left(\mathrm{dq}_{1} \mathrm{dq}_{2} \mathrm{dq}_{3} \ldots \ldots . d \mathrm{q}_{\mathrm{fN}}\right)\left(\mathrm{dp}_{1} \mathrm{dp}_{2} \mathrm{dp}_{3} \ldots \ldots \mathrm{dp}_{\mathrm{fN}}\right)\\
  \text{The number }&\text{of cells in this volume element is d $\Omega$.}
 \\
 d \Omega&=\frac{\left(d q_{1} d q_{2} \ldots \ldots . d q_{f N}\right)\left(d p_{1} d p_{2} \ldots . d p_{f N}\right)}{h^{N f}}
 \end{align*}
\subsection{ Thermodynamic probability}
 The thermodynamic probability is the number of microstates corresponding to the given macrostate of the system.
 
 As an example consider the distribution of 4 particles in the two halves of a box. There are five macrostates viz., $(4,0),(3,1),(2,2),(1,$, 3) and $(0,4)$ are possible. The thermodynamic probability corresponding to the distribution $(2,2)$ is 6 .\\
 The redistribution or permutation of indistinguishable particles in a cell is meaningless because it does not produce any new microstate and so it is neglected.
 
 Consider the distribution of $\mathrm{n}$ similar particles in $\mathrm{k}$ similar boxes such that $n_{1}, n_{2}, n_{3} \ldots . n_{k}$ particles are in the boxes $1,2,3, \ldots . k$ respectively. Then $n_{1}+n_{2}+n_{3}+\ldots \ldots n_{k}=n$.
 
It can be shown that the number of microstates corresponding to this distribution or its thermodynamic probability $\Omega=\frac{n !}{\pi n_{i} !}$\\
\textbf{Proof}
\begin{align*}
\text{The number of ways of chosing }&n_{1} \text{particles for the first box}\\
&=\frac{n !}{n_{1} !\left(n-n_{1}\right) !}\\
\text{From the remaining particle }&\left(n-n_{1}\right)\text{ we have to chose $n_{2}$ particles }\\
\text{for the box 2. This can be done in }&\frac{\left(n-n_{1}\right) !}{n_{2} !\left(n-n_{1}-n_{2}\right) !}\text{ ways. For the box}\\
\text{$3, \mathrm{n}_{3}$ particles can be chosen in }& \frac{\left(\mathrm{n}-\mathrm{n}_{1}-\mathrm{n}_{2}\right) !}{\mathrm{n}_{3} !\left(\mathrm{n}-\mathrm{n}_{1}-\mathrm{n}_{2}-\mathrm{n}_{3}\right) !}\text{ ways and so}
\intertext{So the total number of meaningful ways of arranging $n_{1}, n_{2}, n_{3} \ldots . n_{1}$ particles in boxes $1,2,3 \ldots . \mathrm n{k}$ respectively is,}
\Omega=\frac{n !}{n_{1} !\left(n-n_{1}\right) !} &\times \frac{\left(n-n_{1}\right) !}{n_{2} !\left(n-n_{1}-n_{2}\right) !} \times \frac{\left(n-n_{1}-n_{2}\right) !}{n_{3} !\left(n-n_{1}-n_{2}-n_{3}\right) !} \times \ldots, k\text{ terms}\\
\Omega&=\frac{n !}{n_{1} ! n_{2} ! n_{3} ! \ldots \ldots n_{k} !}\\
\text{For the }i^{\text {th }}\text{ distribution then the probability}\Omega_{\mathbf{i}}&=\frac{\mathbf{n} !}{\pi \mathbf{n}_{\mathbf{i}} !}
\end{align*}
\subsection{Statistical Ensemble}
A collection of particles is called a system. A number of systems constitute an ensemble. Sometimes it is difficult to study the behaviour of a complex system. But a suitably created ensemble help us to compute the statistical behaviour of the system
\begin{figure}[H]
	\centering
	\includegraphics[height=3cm,width=5cm]{SP-04}
\end{figure}
The phase line of a single system is as shown in Fig. 4.4. Each phase point on the phase line emerges from the previous point, with passage of time, in accordance with the laws of mechanics. Suppose we want to compute a physical quantity of the system. Usually what we do is we find its time average over a certain interval of time. But Gibbs reduced this time dependent picture by a static picture. According to this model, the entire phase line shown by dotted line, in Fig.4.4 exists at one time. Then each phase point represents a separate system, with the same macroscopic properties $(N, V, E)$ but a different microscopic state where $\mathrm{N}$ is the total number of particles in the system, $V$ its volume and $E$ its total energy. i.e., we just imagine a large number, (almost infinity) of systems, which are similar in structure to the system under consideration, but placed at random in the accessible, unobservable microscopic states. Now we don't take the time average but take an average over this artificially created group of systems created simultaneously one time. This collection of replicas of similar, independent noninteracting systems is calle an ensemble. Gibbs assumed that the time average of some property of a system in equilibrium is same as the instantaneous ensemble average. This is called the ergodic hypothesis.

All the members of the ensemble which are identical in $\mathrm{N}, \mathrm{V}, \mathrm{E}$ are called elements.
\begin{figure}[H]
	\centering
	\includegraphics[height=3.3cm,width=4cm]{SP-05}
\end{figure}
In Fig. $4.5 \mathrm{~N}, \mathrm{~V}, \mathrm{E}$ represents one system. A collection of such system constitute an ensemble.
\subsection{Characteristics of an ensemble}
\begin{enumerate}
	\item  The elements of an ensemble are identical in structure. i.e., they have the same macroscopic state.
\item  The elements differ in their miccroscopic state. i.e., they diffet from each other in the position coordinates and momentum coordinates.
\item  The various elements do not interact each other.
\item  Each element of the ensemble behaviour independently obeying the laws of mechanics.
\item  The elements of the ensemble are the mental copies of the system under consideration. But the system is a physical object, whose properties are to be evaluated. The elements help us to use the probability theory.
\end{enumerate}
\subsection{Ensemble average}
The average value found out at a fixed time over all the elements in an ensemble is called an ensemble average. The ensemble average will be equal to the time average when the system contains a large mumber of particles (molecules) and the ensemble contains a large namber systems (both almost infinite in number).
\subsection{Liouville's Theorem}
The condition of an ensemble at any time can be specified by the density $\rho$ with which the phase points are distributed over the phase apace.  It is called the density of distribution and is a function of the position coordinates (q) and momentum coordinates (p).

The dynamical state of a system at any instant of time can be represented by a point in the phase space.  This point will not be stationary, but will move along a certain trajectory. The trajectory is determined by the equatio of motion,
$$\dot{q}_{i}=\frac{\partial \mathrm{H}}{\partial \mathrm{p}_{i}}\text{ and}$$
$$\dot{p}_{i}=-\frac{\partial H}{\partial \dot{q}_{i}}$$
where $H=H\left(q_{i}, p_{i}\right)$ is the Hamiltonian of the system. Due to this motion, the density $\rho$ of the system in phase space changes with time.

Lioville's theorem gives the value of $\frac{\partial \rho}{\partial t}$ at a given point in phase space.
Liouville's theorem states that the density of systems in the neighbourhood of some given system in phase space remains constant in time.
\begin{equation}
\frac{d \rho}{d t}=\frac{\partial \rho}{\partial t}+\sum_{i}\left(\frac{\partial \rho}{\partial q_{i}} \dot{q}_{i}+\frac{\partial \rho}{\partial p_{i}} \dot{p}_{i}\right)=0
\end{equation}
The sum of the two terms give the total change in density with time, Liouville's theorem states that the variation of density with time is zero.i.e., the total rate of change of density $d\rho/d\tau$ near any selected phase point of a system, as it moves through the phase space is zero.

This is known as the principle of the conservation of the density in phase.  This implies that the distribution of phase points move in phase space like an incompressible fluid.\\
\subsection{Different Types of Ensembles}
Depending on the behaviour of the constituents of an ensemble with respect to the surroundings, ensembles are classified into mainly three types (i) Microcanonical ensemble (ii) Canonical ensemble and (iii) Grand canonical ensemble.\\
\textbf{Uniform ensemble}\\
- If the density in phase space is constant it is called uniform ensemble.
\subsection{Microcanonical Ensemble}
Microcanonical ensemble is a collection of independent systems of constant volume separated from the neighbours with rigid impermeable adiabatic walls. $N, V$ and $E$ remains constant, $N$ is the total number of particles in the system. $V$ and $E$ are the volume and energy of the system [See Fig. 4.6] respectively.\\
Consider a system for which the total energy $H(q, p)=E$ is conserved. i.e., $E\left[q_{1} \ldots \ldots . q_{f}, p_{1} \ldots \ldots p_{f}\right]=a$ constant\\
The locus of all the phase points having the same value for the energies in the phase space is called an energy surface or ergodic surface. A number of such energy surfaces can be constructed in the phase space. Each energy surface divides the phase space into two parts, one of lower energy and the other of higher energy. Since the two are of different energies they do not intersect each other.
\begin{figure}[H]
	\centering
	\includegraphics[height=3.3cm,width=4cm]{SP-10}
\end{figure}
Let $\mathrm{E}$ and $\mathrm{E}+\delta \mathrm{E}$ be two neighbouring ergodic surfaces. The phase volume in between the two surfaces encloses a certain number of phase points and will be a constant. Let us assume the density to be zero for all values of energy except in a narrow range of energy between $E$ and $\mathrm{E}+\delta \mathrm{E}$ [See Fig. 4.7]. Then the ensemble specified in terms of $\rho$ as, is called a micro canonical ensemble.

$\rho$ is a function of energy. Here $\rho$ is constant and hence $E$ is also a constant. So the ensemble is in statistical equilibrium. Since $\rho$ is a constant within the energy shell, the distribution of phase points is also uniform, by Lioville's theorem. As the ensemble is in statistical equilibrium, the average properties predicted will not change with time [See Fig. 4.7].\\
\begin{figure}[H]
	\centering
	\includegraphics[height=2.4cm,width=2.7cm]{SP-07}
\end{figure}
A microcanonical ensemble can be obtained from a uniform ensemble by neglecting those systems whose phase points do not lie within the phase space corresponding to the energy range between $\mathrm{E}$ and $\mathrm{E}+\delta \mathrm{E}$.

A microcanonical ensemble is an idealised concept and hence cannot exist in practice because the systems which we come across always interact with their surroundings, either thermally or mechanically.
In microcanonical ensemble there is exchange of neither energy nor particles (molecules) among the systems.
\section{Canonical Ensemble}
Canonical ensemble is a collection of independent systems of constant volume separated from the neighbours by rigid, impermeable, diathermic walls so that the systems are in thermal equilibrium. The particles can exchange energy and hence all the systems will attain the same temperature. The systems have the same temperature $\mathrm{T}$ and constant volume $\mathrm{V}$. The number of particles $\mathrm{N}$ is also a constant. In canonical ensemble systems can exchange energy and not particles. [See Fig. 4.8]
\begin{figure}[H]
	\centering
	\includegraphics[height=3cm,width=3.5cm]{SP-08}
\end{figure}
\subsection{Grand Canonical Ensemble}
Grand Canonical ensemble is a collection of independent systems of constant volume but open and separated from its neighbours by diathermic permeable membrane so that both material and energy can be exchanged between the neighbours. Since there is exchange of energy, the energy of each system $E$ is not a constant. Also there is exchange of particles, so the total number of particles in each system do not remain constant. The temperature T, volume $\mathrm{V}$ and chemical potential $\mu$ of each system remain constant [See Fig. 4.9]. 
\begin{figure}[H]
	\centering
	\includegraphics[height=3cm,width=4cm]{SP-09}
\end{figure}
In grand Canonical ensemble there is exchange of both energy and particles (molecules) among the systems.
\subsection{Statistical Equilibrium}
A system of particles which does not interact with any other system so that its total energy remains constant is called an isolated system. A small part of the isolated system which is still macroscopic is called a subsystem. Subsystem interacts with other parts of the system and hence is not isolated.\\
If in any macroscopic subsystem of an isolated system, the average number of particles per unit volume and the average energy per particle are equal to their mean values, then the isolated system is said to be in statistical equilibrium i.e, when the mean values of macroscopic parameters like pmessure, volume, ternperature etc. are independent of time, the systern is in equilibrinarn. If the systern is foumd with equal probability in cach of its possible quantum states then the isolated system is in statistical or thenmodynamic equilibrinm. In tenms of density p, the condition for an ensemble to be in statistical equitibrinm is $\left(\frac{\partial \rho}{\partial t}\right)_{q, p}=0$, i.e, $\rho$ is independent of time at all points in the phase space for the ensemble. For this p must be a function of some property\\
of the ensemble which does not depend on time.
\subsection{Postulate of Equal a priori probability}
Consider an isolated 'many particle system' in equilibrium. We want to know the probability for any one particle to be in a given specified state. The answer to this is given by the a priori postulate.\\
"All accessible microstates corresponding to possible macrostates are equally probable". This is the most fundamental postulate of statistical mechanics. This means that the probability of finding the particle in any one region is identical with that for any other region of equal volume, under similar conditions.
\section{ Classical Distribution Law }
\textbf{Assessible microstate:}\\ Microstates corresponding to a macrostate.\\
\textbf{Thermodynamic Probability:} \\Number of microstates corresponding to a macrostate.\\
$$\text{Probability }=\frac{\text{Number of microstates for accessible macrostate}}{\text{Total number of microstates}}$$
\section{ Probability Distibution: }
\textbf{Coin Related Probelm: }\\
\textbf{a.\quad  Tossing a coin:}\\
\begin{align*}
\text{Total events }&=H+T\\
\text{Total no of events }&=2\\
\text{Total number of microstate }&=2\\
\text{equally likely probability event }&=1 / 2
\end{align*}
\textbf{b.\quad  Two coin are tossed together:}\\
\begin{align*}
\text{Possible events: }&H H, H T, T H, T T\\
\text{ equally likely Probability }&=1 / 4\\
\rightarrow\text{ Total number of microstates }&=(2)^{n}\\
\rightarrow\text{ equally likely Probability }=(1 / 2)^{n}\\
\text { Probability of general event }& 
\Rightarrow{ }^{n} C_{r}(p)^{n-r}(1-p)^{r} \\
&\Rightarrow{ }^{n} C_{r}\left(\frac{1}{2}\right)^{n}
\end{align*}
Q. If to similar coins are tossed together what is the probability of gerring 3 Head?
\begin{answer}
	\begin{align*}
	P(3 H) &={ }^{10} C_{3}\left(\frac{1}{2}\right)^{10-3}\left(\frac{1}{2}\right)^{3} \\
	&={ }^{10} C_{3}\left(\frac{1}{2}\right)^{10} \Rightarrow \frac{10 !}{3 ! 7 !}\left(\frac{1}{2}\right)^{10}\\
	&=\frac{10 \times 9 \times 8}{3 \times 2} \times \frac{1}{1024} \Rightarrow \frac{15}{123}
	\end{align*}
\end{answer}
\textbf{c.\quad  Number of microstates in spin system:}\\
$\Omega=(2 s+1)^{N}$
\begin{note}
	entropy: $S=k \ln \Omega=k \ln(2)^N\Rightarrow Nk\ln 2$
\end{note}
Q. A system of 5 identical but distinguishable particles having energy $3\epsilon$. The single particle states are available at energies $0,\epsilon,2\epsilon,3\epsilon$ find the probablity of each macro state?\\\\
\renewcommand*{\arraystretch}{2}
\begin{tabular}{|p{1cm}|p{1cm}|p{1cm}|p{1cm}|p{1cm}|p{3cm}|}
	\hline
	0&$\epsilon$&2$\epsilon$&3$\epsilon$&$E_{\text{Total}}$&No.of Microstates\\\hline
	4&-&-&1&3$\epsilon$&$\Omega$=$\frac{5!}{4!0!0!}=5$\\\hline
	3&1&1&-&3$\epsilon$&$\Omega$=$\frac{5!}{3!1!0!}=20$\\\hline
	2&3&-&-&3$\epsilon$&$\Omega$=$\frac{5!}{2!3!0!0!}=10$\\\hline
\end{tabular}
\subsection{Random Walk}
\textbf{1-Dimensional Random Walk:}\\
\begin{align*}
\text{let }N &\rightarrow\text{ steps of equal length along a line.}\\
p &\rightarrow\text{ Probability of taking a step right}\\
q &\rightarrow\text{ probability of taking a step left}\\
n_{1} &\rightarrow \text{no. of sleps taken to right }\\
n_{2} &\rightarrow\text{ no. of steps taken to left.}\\
&p+q=1\qquad n_{1}+n_{2}=N
\end{align*}
Q. A one dimensional random walker takes 10 steps in left and right with equal probability.  Find out the probability that the random walker starting from origine is back to origine after taking 10 steps?
\begin{answer}
	\begin{align*}
	P(5,5) &={ }^{10} C_{5}\left(\frac{1}{2}\right)^{10} \\
	&=\frac{10 !}{5 ! 5 !} \times \frac{1}{1024}
	\end{align*}
\end{answer}
$\rightarrow$ \section{Ensemble:} collection of large number of systems that are macroscopically identical but microscopically different.\\\\
\begin{tabular}{|p{4cm}|p{4cm}|p{4cm}|}
	Microcanonical&canonical&Grand Canonical\\\hline
	E, V, N&T, V, N&$T, V, \mu \rightarrow$fixed values \\\hline
	insulating, rigid and impermeable&conducting, rigid and impermeanle&conducting, rigid and permeable \\\hline
	isolated system &thermal equilibrium &  thermal +chemical equilibrium \\\hline
	universe, thermal flask etc.&-&-\\\hline
\end{tabular}\\
\subsection{Microcanonical Ensemble:}
\textbf{For Classical Ideal Gas }\\
a). Non Relativistic Case
\begin{align*}
E&=\frac{p^{2}}{2 m} \Rightarrow p=\sqrt{2 m E}\\
\text{For }N-\text{ particle : DOF }&=3 N\\
\text{volume of position space }&=V^{N}\\
\text{volume of momentum space }&=\frac{\pi^{3 N / 2} p^{3 N}}{(3 N / 2) !}\\
\rightarrow \Omega&=\frac{V^{N} \cdot \pi^{3 N / 2}(\sqrt{2 m E})^{3 N}}{h^{3 N}(3 N / 2) !}\\
\rightarrow E&=U=\frac{3}{2} N K T=\frac{3}{2} n R T \\
\text{If }E\propto P^s \text{ and }d \rightarrow\text{dimension}
\rightarrow P&=\frac{s}{d}\left(\frac{E}{V}\right)=\frac{2}{3}\left(\frac{E}{V}\right) \\
\rightarrow P V^{5 / 3}&=\text { constant } \\
\rightarrow S&=K \ln \Omega\\
S&=N_{i} k \ln v_{i}+\frac{3}{2} N_{i} k \ln \left(\frac{4 \pi m_{i} E}{3 N_{i} h^{2}}\right)+\frac{3}{2} N_{i} k
\end{align*}\\
b). Relativistic case\\
\begin{align*}
E&=pc\\
\Omega&=\frac{U^N\pi^\frac{3N}{2}}{\frac{3N}{2}!}\left(\frac{E}{C h}\right)^{3 N}\\
P&=\frac{S}{d}\left(\frac{U}{V} \right) \\
P&=\frac{1}{3}\left(\frac{U}{V}\right)\\
P V^{4 / 3}&=\text { constant }
\end{align*}
\subsection{Canonical Ensemble:}
$P_{i}=\frac{g_{i} e^{-\beta \varepsilon_{i}}}{\sum g_{i} e^{-\beta E_{i}}}$\hspace{2cm}$Z=\Sigma g_{i} e^{-\beta \varepsilon_{i}}$
\quad$Q_1 = Z$\\\\
 For N particles:
\begin{align*}
 Q_{N}(T, V) &=\left[Q_{1}(T, V)\right]^{N} \rightarrow \text { distinguishable } \\ &=\frac{\left[Q_{1}(T, V)\right]^{N}}{N !} \rightarrow \text { indistinguishable. } 
 \end{align*}
\textbf{Avarage value of Different Quantities}\\
1. Avarage Energy:\\
\begin{align*}
\left\langle E_{i}\right\rangle&=\sum E_{i} P_{i} \Rightarrow \frac{\sum \varepsilon_{i} g_{i} e^{-\beta \xi_{i}}}{\sum g_{i} e^{-\beta \varepsilon_{i}}}\\
\left\langle E_{i}\right\rangle&=-\frac{\partial}{\partial \beta} \ln z=+k t^{2} \frac{\partial}{\partial T} \ln z
\end{align*}\\
2. Average Pressure\\
$\langle P\rangle=\sum p \frac{g e^{-\beta \varepsilon}}{z}=\frac{1}{\beta} \frac{\partial}{\partial v} \ln z$\\\\
3. Average Chemical Potential\\
$\langle\mu\rangle=\Sigma \frac{\mu g e^{-\beta \varepsilon_{i}}}{z}=-\frac{1}{\beta} \frac{\partial}{\partial N} \ln z$\\\\
4. Average entropy\\
$\langle S\rangle=K_{B} \ln z+\frac{\langle E\rangle}{T} \Rightarrow K_{B} \ln z-\frac{1}{T} \frac{\partial}{\partial B} \ln z$\\\\
5. Average Free Energy\\
$\langle A\rangle=\langle E\rangle-\langle TS\rangle \Rightarrow-K_{B} T \ln z$




\newpage
Boltzman entrophy relation: $S=K \ln \Omega$\\
striling approximation: $\ln n !=n \ln n-n$\\
Number of microstates: 
$\Omega_{M B}=\underset{i}{\pi}\left(g_{i}\right)^{n_{i}}$\\
$\Omega_{B E}=\underset{i}{\pi}\frac{\left(n_{i}+g_{i}-1\right) !}{n_{i} !(g i-1) !}$\\
${ }^{\Omega_{F D}}=\underset{i}{\pi}{g i }\  C_{n_{i}}$\\
$g_{i} \rightarrow$ degeneracy\\
$n_{i} \rightarrow$ identical Particles\\
Energy distribution function :$f(E)=\frac{n_{i}}{g_{i}}$
Q.The number of ways in which $N$ identical bosons can distributed in two energy levels is 
 \begin{tasks}(4)
	\task[\textbf{a.}] $N+1$
	\task[\textbf{b.}]$N(N-1) \mid 2$
	\task[\textbf{c.}]$N(N+1) \mid 2$
	\task[\textbf{d.}]$N$ 
\end{tasks}
\begin{answer}
	\begin{align*}
	\Omega&=N_ {C_n}=\frac{N !}{n !(N-n) !}\hspace{2cm}\underset{\text{-------------- }e}{n}\\
	\text { entropy }(s)&=k \ln \Omega=k \ln \frac{N !}{n !(N-n) !}\hspace{1.1cm}\underset{\text{-------------- } 0}{(N-n)}\\
	\operatorname{s\alpha } &\ln \frac{N !}{n !(N-n) !}
	\end{align*}
		So the correct answer is \textbf{Option (d)}
\end{answer}
\begin{note}
	Probability will be maximum for equal distribution.
\end{note}
Q. Let $N_{M B}, N_{B E}, N_{F D}$ denote the number of ways in which two particles can be distributed in two energy states according to maxwell-Boltzmann, Bose-Einstein \& fermi-Dirac statistics respectively. Then $N_{M B}: N_{B E}: N_{F D}$ is:
 \begin{tasks}(4)
	\task[\textbf{a.}]$4: 3: 1$
	\task[\textbf{b.}] $4: 2: 3$
	\task[\textbf{c.}]$4: 3: 3$
	\task[\textbf{d.}] $4: 3: 2$
\end{tasks}
\begin{answer}
	\begin{align*}
	\Omega_{M B}&=\left(g_{i}\right)^{n_{i}}=(2)^{2} \Rightarrow 4 \\
	\Omega_{B E}&=\frac{\left(n_{i}+g_{i}-1\right) !}{n_{i} !((-i-1) !}=\frac{(2+2-1) !}{2 !(2-1) !}=\frac{3 !}{2 !}=3\\
	\Omega_{F D}&=g i C_{n_{i}}=\frac{2 !}{2 ! 1 !}=1\\
	&4:3:1
	\end{align*}
	So the correct answer is \textbf{Option (a)}
\end{answer}
Relative Probability: \\
$$r=\frac{P_1}{P_2}=\frac{\text{Probability of finding particle in same state}}{\text{Probability of finding particle in different state.}}$$
$r_{B E}: r_{M B}: r_{F \cdot D}=1: 1 / 2: 0$\\
$\Rightarrow U_{B E} \leqslant U_{M B} \leqslant U_{F D} \Rightarrow P_{B E} \leqslant P_{M B} \leqslant P_{F D}$ \\
Q If $N$ particle are to be distributed into two groups such that group i contain $n_1$ particles and group two contains $n_2$ particles, and $N=n_1+n_2$ Find the number of ways of distribution?
\begin{answer}
	\begin{align*}
	\Omega&={ }^{N} C_{n_{1}} \text { or }{ }^{N} C_{n_{2}} 4 N=n_{1}+n_{2}\\
	N_{C_{n_{i}}}&=\frac{N !}{n_{1} !(N-n) !}=\frac{N !}{n_{1} ! n_{2} !}\\
	&\text{similaarly, for $ni$ particles}\\
	\Omega&=\frac{N!}{n_1!n_2!.......n_i!}
	\end{align*}
\end{answer}
Q. A system has energy levels $E_{0}, 2 E_{0}, 3 E_{0}, \ldots .$ where the excited states are triphy generate four non-interacting bosons are placed in this system. If the total energy og these bosons is $5E_0$ the number of microstates is:
 \begin{tasks}(4)
	\task[\textbf{a.}]2
	\task[\textbf{b.}]3
	\task[\textbf{c.}]4
	\task[\textbf{d.}] 5
\end{tasks}
\begin{answer}
	\begin{figure}[H]
		\centering
		\includegraphics[height=4cm,width=6cm]{SP-12}
	\end{figure}
	So the correct answer is \textbf{Option (b)}
\end{answer}
\textbf{Phase Space}:(Position + momentum) space\\
volume of single phase cell: $d \tau=h^{f} \quad f \rightarrow D O F$\\
allowed vol. of phace space: $\int d^{3} q \int d^{3} p$\\
$$
V_{p} \Rightarrow V \times \frac{4}{3} \pi p^{3}\text{( for 3-Dim).}
$$
\textbf{Non-Relativistic Free Particle: }$E=p^2/2m$\\
\renewcommand*{\arraystretch}{1.8}
 \begin{tabular}{|p{4cm}|p{4cm}|p{4cm}|}
 	\hline
 3-Dimensional&2-Dimensional&1-Dimensional\\\hline
 $d \tau=h^{3}$&$d \tau=h^{2}$&$d \tau=h$\\\hline
 $V_{p}=V \times \frac{4}{3} \pi p^{3}$&$V_{p}=A \times \pi p^{2}$&$V_p=L\times2p$\\\hline
 $\Omega=\frac{V \cdot \frac{4}{3} \pi p^{3}}{\hbar^{3}}$&$\Omega(p)=\frac{\pi \times \pi p^{2}}{h^{2}}$&$\Omega(p)=\frac{L \times 2 p}{h}$\\\hline
 $\Omega(E)=\frac{\mu\pi V(2mE)^3}{3h^3}$&$\Omega(E)=\frac{\pi A(2mE)}{h^2}$&$\Omega(E)=\frac{2L\sqrt{2mE}}{h}$\\\hline
 $\Omega^{\prime}(E)=\frac{2 \pi V(2 m)^{3 / 2} E^{1 / 2}}{h^2}dE$&$\Omega^{\prime}(E)=\frac{2 \pi mA}{h^3}dE$&$\Omega^{\prime}(E)=\frac{L(2m)^{1/2}E^{-1/2}}{h}dE$\\\hline
 \end{tabular}
\begin{figure}[H]
	\includegraphics[height=5.5cm,width=13cm]{SP-13}
\end{figure}
Q.In a classical microcanonical ensemble for a system of $N$ non-interacting particles, the fundamental volume of phase space which is regarded as "equivalent to one-microstate" is 
 \begin{tasks}(4)
	\task[\textbf{a.}] $h^{3 N}$
	\task[\textbf{b.}]$h^{2 N}$
	\task[\textbf{c.}]$h^{N}$
	\task[\textbf{d.}]$h$
\end{tasks}
\begin{answer}
	\begin{align*}
	\text{For $N$ non-interacting particles }\text { DOF }&=3 N=f\\
	\text{volume of phase space }&=h^{f}\\
	&=R^{3 N}
	\end{align*}
	So the correct answer is \textbf{Option (a)}
\end{answer}
\textbf{Extremely Relativistic Free Particle: }(E=pc)\\
\renewcommand*{\arraystretch}{1.8}
\begin{tabular}{|p{4cm}|p{4cm}|p{4cm}|}
	\hline
	3-Dimensions & 2-Dimensions & 1-Dimensions\\\hline
	$\Omega(p)=\frac{v \cdot \frac{4}{3} \pi p^{3}}{h^{3}}$&$\Omega(p)=\frac{\pi A p^{2}}{h^{2}}$&$\Omega(p)=\frac{L \times 2 p}{h}$\\
	$\Omega^{\prime}(p)=\frac{4 \pi V p^{2} d p}{h^{3}}$&$\Omega^{\prime}(p)=\frac{2 \pi A p d p}{h^{2}}$&$\Omega^{\prime}(p)=\frac{2 L d p}{h}$\\\hline
	$\Omega(E)=\frac{4 \pi V E^{3}}{3 c^{3} h^{3}}$&$\Omega(E)=\frac{A \pi E^{2}}{C^{2} n^{2}}$&$\Omega(E)=\frac{2 L E}{\hbar c}$\\
	$\Omega(E)=2 \times \frac{4 \pi v\left(E^{2}\right)}{c^{3} h^{3}} d E$&$\Omega^{\prime}(E)=\frac{2 \pi A E}{C^{2} h^{2}}$&$\Omega^{\prime}(E)=\frac{2 L}{R C} d E$\\\hline
\end{tabular}
Q. Find density of states at energy $E$ of the quantized radiation (photon) is:
 \begin{tasks}(4)
	\task[\textbf{a.}]$\frac{8 \pi v}{h^{3} c^{3}} E^{2}$
	\task[\textbf{b.}]$\frac{8 \pi V}{h^{3} c^{3}} E^{3 / 2}$
	\task[\textbf{c.}]$\frac{8 \pi V}{h^{3} c^{3}} E$
	\task[\textbf{d.}] $\frac{8 \pi V}{h^{3} c^{3}} E^{3 / 2}$
\end{tasks}
\begin{answer}
	\begin{align*}
	\Omega(p)&=\frac{V \times \frac{4}{3} \pi p^{3}}{h^{3}} \Rightarrow \Omega(E)=\frac{4 \pi V E^{3}}{3 c^{3} h^{3}}\\
	\Omega^{\prime}(E)&=\left(\frac{4 \pi V E^{2}}{c^{3} h^{3}} d E\right) \times 2\\
	\Omega^{\prime}(E)&=g(E) d E\\
	g(E)&=\frac{8 \pi V}{c^{3} h^{3}} E^{2}
	\end{align*}
	So the correct answer is \textbf{Option (a)}
\end{answer}
Q. \textbf{For free particle (with $E=ap^s$) with d-DOF: }\\
\begin{align*}
\text{Volume of sphere of radius (R) }V_{d}&=\frac{\pi^{d / 2} R^{d}}{\left(\frac{d}{2}\right) !}\\
\text{area of sphere of radius (R) }s_{d}&=\frac{2 \pi^{d / 2} R^{d-1}}{\left(\frac{d}{2}-1\right) !}\\
\text{ volume of single phase cell }d \tau&=h^{d}\\
g(E)&=\frac{g i V_{d} \pi^{d / 2}\left(\frac{1}{a}\right)^{d / s}\left(\frac{d}{s}\right) E^{(d / s-1)}}{R^{d}(d / 2) !}\\
&g(E) \propto E^{(d / s-1)}\hspace{3cm}d=DOF\\
s=1 \text{(marsless)},\quad
2 \text{(massive)}
\end{align*}
Q.For an energy state $E$ of a photon gas, the density of states is proportional to 
 \begin{tasks}(4)
	\task[\textbf{a.}] $\sqrt{E}$
	\task[\textbf{b.}] $E $
	\task[\textbf{c.}]$E^{3 / 2}$
	\task[\textbf{d.}] $E^{2}$
\end{tasks}
\begin{answer}
	\begin{align*}
	D(E) &\propto E^{\left(\frac{d}{s}-1\right)}\\
	\text { For proton: } d&=3 \text { and } s=1\\
	D(E) \propto E^{3-1} &\Rightarrow D(E) \alpha E^{2}
	\end{align*}
		So the correct answer is \textbf{Option (d)}
\end{answer}
Q. The number of states for a system of $N$ indentical free particles in a three dimensional space having total energy between $E$ and $E+\delta E(\delta E<<E),$ is proportional to 
 \begin{tasks}(4)
	\task[\textbf{a.}]$E^{\left(\frac{3 N}{2}-1\right)} \delta E$
	\task[\textbf{b.}]$E^{\frac{N}{2}} \delta E$
	\task[\textbf{c.}]$N E^{\frac{1}{2}} \delta E$
	\task[\textbf{d.}]$N^{N} \delta E$ 
\end{tasks}
\begin{answer}
	\begin{align*}
	D(E) &\propto E^{(d / s-1)}\\
	\text { Here } d&=3 \mathrm{~N} \text { and } s=2 \text { (massive) }\\
	D(E) &\propto E^{\left(\frac{3 N}{2}-1\right)}
	\end{align*}
		So the correct answer is \textbf{Option (a)}
\end{answer}
Q. Draw phase space trajectory of linear (1-dimension) Harmonic oscillator and calculate number of microstate if particle has:\\
a). sharp value of energy\\
b). Energy lies between $E$ to $E+d E$
\begin{answer}
	\begin{align*}
	E&=\frac{p^{2}}{2 m}+\frac{1}{2} m \omega^2 x^{2}\\
	1&=\frac{p^{2}}{\sqrt{2 m E}}+\frac{x^{2}}{\left(\sqrt{\frac{2 E}{m \omega^2}}\right)^{2}}\\
	\text{area of ellipse }&=\pi a b\\
	\text{volume of single phase cell }&(1-D)=h\\
	\Omega(E)=\frac{\pi a b}{h}&=\frac{\pi \times \sqrt{\frac{2 E}{m \omega^2}}  \times \sqrt{2 m E}}{h}\\
	\Omega(E)&=\frac{2 \pi E}{h \omega}=\frac{E}{h \omega}\\
	\Omega^{\prime}(E)&=\frac{d E}{\hbar \omega}
	\end{align*}
\end{answer}
\begin{note}
	\begin{align*}
	\text{	maxinum valle of }&x \\
\text{	on }x-\text{axis: }p&=0 \hspace{3cm}x_{0}=\sqrt{\frac{2 E_{0}}{m \omega^{2}}}\\
	E_{0}&=0+\frac{1}{2} m u^{2} x_{0}^{2}\hspace{2cm}x_{0} \alpha \sqrt{E_{0}}\\
	\end{align*}
\end{note}





\newpage
\begin{abox}
	Practise set-01
\end{abox}
\begin{enumerate}
	\item A particle is confined to the region $x \geq 0$ by a potential which increases linearly as $u(x)=u_{0} x$. The mean position of the particle at temperature $T$ is
	 \begin{tasks}(2)
		\task[\textbf{a.}] $\frac{k_{B} T}{u_{0}}$
		\task[\textbf{b.}]$\left(k_{B} T\right)^{2} / u_{0}$
		\task[\textbf{c.}]$\sqrt{\frac{k_{B} T}{u_{0}}}$
		\task[\textbf{d.}]  $u_{0} k_{B} T$
	\end{tasks}
\begin{answer}
	\begin{align*}
	\text{Partition function }Z&=\frac{1}{h} \int e^{-\frac{p^{2}}{2 m k_{B} T}} d p \int e^{-\frac{u_{0} x}{k_{B} T}} d x\text{ and }\langle x\rangle=\int x p(x) d x d p_{x}\\
	\Rightarrow\langle x\rangle&=\frac{\iint x e^{-\frac{p^{2}}{2 m k_{B} T}} d p \cdot e^{\frac{u_{0} x}{k_{B} T}} d x}{\iint e^{\frac{p^{2}}{2 m k_{B} T}} d p \cdot e^{\frac{u_{0} x}{k_{B} T}} d x}=\frac{\int_{0}^{\infty} x e^{\frac{\mu_{0} x}{k_{B} T}} d x}{\int_{0}^{\infty} e^{-\frac{\mu_{0} x}{k_{B} T}} d x}=\frac{\left(\frac{k_{B} T}{u_{0}}\right)^{2}}{\left(\frac{k_{B} T}{u_{0}}\right)} \frac{\int_{0}^{\infty} t e^{-t} d t}{\int_{0}^{\infty} e^{-t} d t}=\frac{k_{B} T}{u_{0}}
	\end{align*}
	So the correct answer is \textbf{Option (a)}
\end{answer}
\item 	Consider a system of $N$ non-interacting spins, each of which has classical magnetic moment of magnitude $\mu$. The Hamiltonian of this system in an external magnetic field $\vec{H}$ is $\sum_{i=1}^{N} \vec{\mu}_{i} \cdot \vec{H}$, where $\vec{\mu}_{i}$ is the magnetic moment of the $i^{\text {th }}$ spin. The magnetization per spin at temperature $T$ is
	 \begin{tasks}(2)
		\task[\textbf{a.}]$\frac{\mu^{2} H}{k_{B} T}$
		\task[\textbf{b.}]$\mu\left[\operatorname{coth}\left(\frac{\mu H}{k_{B} T}\right)-\frac{k_{B} T}{\mu H}\right]$
		\task[\textbf{c.}] $\mu \sinh \left(\frac{\mu H}{k_{B} T}\right)$
		\task[\textbf{d.}] $\mu \tanh \left(\frac{\mu H}{k_{B} T}\right)$
	\end{tasks}
\begin{answer}
	\begin{align*}
	\text { For classical limit } M&=\frac{\int_{0}^{2 \pi} \int_{0}^{\pi} \mu \cos \theta \exp \frac{\mu H \cos \theta}{k T} \sin \theta d \theta d \phi}{\iint \exp \frac{\mu H \cos \theta}{k_{B} T} \sin \theta d \theta d \phi} \\
	M&=\mu\left[\operatorname{coth}\left(\frac{\mu H}{k_{B} T}\right)-\frac{k_{B} T}{\mu H}\right]
	\end{align*}
		So the correct answer is \textbf{Option (b)}
\end{answer}
\item 	The internal energy $E$ of a system is given by $E=\frac{b S^{3}}{V N}$, where $b$ is a constant and other symbols have their usual meaning. The temperature of this system is equal to
	 \begin{tasks}(2)
		\task[\textbf{a.}]$\frac{b S^{2}}{V N}$
		\task[\textbf{b.}]$\frac{3 b S^{2}}{V N}$
		\task[\textbf{c.}]$\frac{b S^{3}}{V^{2} N}$
		\task[\textbf{d.}] $\left(\frac{S}{N}\right)^{2}$
	\end{tasks}
\begin{answer}
	\begin{align*}
	T d S=d E+P d V \Rightarrow d E=T d S-P d V \Rightarrow\left(\frac{\partial E}{\partial S}\right)_{V}=T \Rightarrow T=\frac{3 b S^{2}}{V N}
	\end{align*}
	So the correct answer is \textbf{Option (b)}
\end{answer}
\item A gas of $N$ non-interacting particles is in thermal equilibrium at temperature $T$. Each particle can be in any of the possible non-degenerate states of energy $0,2 \varepsilon$ and $4 \varepsilon$. The average energy per particle of the gas, when $\beta \varepsilon<<1$, is
 \begin{tasks}(2)
	\task[\textbf{a.}]$2 \varepsilon$
	\task[\textbf{b.}] $3 \varepsilon$
	\task[\textbf{c.}]$2 \varepsilon / 3$
	\task[\textbf{d.}]  $\varepsilon$
\end{tasks}	
\begin{answer}
	\begin{align*}
	E_{1}&=0, E_{2}=2 \varepsilon, E_{3}=4 \varepsilon, Z=e^{-0 \beta}+e^{-2 \varepsilon \beta}+e^{-4 \varepsilon \beta} \Rightarrow\langle E\rangle=\frac{0 \cdot e^{-o \beta}+2 \varepsilon e^{-2 \varepsilon \beta}+4 \varepsilon e^{-4 \varepsilon \beta}}{e^{-0 \beta}+e^{-2 \varepsilon \beta}+e^{-4 \varepsilon 1 \beta}} \\
	\Rightarrow\langle E\rangle&=\frac{2 \varepsilon e^{-2 \varepsilon \beta}+4 \varepsilon e^{-4 \varepsilon \beta}}{1+e^{-2 \varepsilon \beta}+e^{-4 \varepsilon \beta}}=\frac{2 \varepsilon(1-2 \varepsilon \beta \ldots .)+4 \varepsilon(1-(4 \varepsilon \beta) \ldots .)}{1+(1-2 \varepsilon \beta \ldots .)+(1-4 \varepsilon \beta \ldots . .)}=\frac{2 \varepsilon+4 \varepsilon}{1+1+1}=\frac{6 \varepsilon}{3}=2 \varepsilon\\
&\text{	where }\beta \varepsilon<<1.
	\end{align*}
		So the correct answer is \textbf{Option (a)}
\end{answer}
\item 	Gas molecules of mass $m$ are confined in a cylinder of radius $R$ and height $L$ (with $R>L$ ) kept vertically in the Earth's gravitational field. The average energy of the gas at low temperatures (such that $m g L \gg k_{B} T$ ) is given by
	 \begin{tasks}(2)
		\task[\textbf{a.}]$N k_{B} T / 2$
		\task[\textbf{b.}]$3 N k_{B} T / 2$
		\task[\textbf{c.}]$2 N k_{B} T$
		\task[\textbf{d.}] $5 N k_{B} T / 2$
	\end{tasks}
\begin{answer}
	\begin{align*}
	Z&=\frac{1}{h^{3}} \int e^{-\beta H} d p_{x} d p_{y} d p_{z} d x d y d z\\
	Z&=\int_{-\infty}^{\infty} e^{\frac{-p_{x}^{2}}{2 m k_{B} T}} d p_{x} \int_{-\infty}^{\infty} e^{\frac{-p_{y}^{2}}{2 m k_{B} T}} d p_{y} \int_{-\infty}^{\infty} e^{\frac{-p_{z}^{2}}{2 m k_{B} T}} d p_{z} \int d x d y \int_{0}^{L} e^{-\frac{m g z}{k_{B} T}} d z\\
	Z&=\pi R^{2}\left(\frac{m k_{B} T}{2 \pi \hbar^{2}}\right)^{\frac{3}{2}} \int_{0}^{L} e^{-\frac{m g z}{k_{B} T}} d z \Rightarrow Z=\pi R^{2}\left(\frac{m k_{B} T}{2 \pi \hbar^{2}}\right)^{\frac{3}{2}}\left(\frac{1-e^{-\frac{m g L}{k_{B} T}}}{\frac{m g}{k_{B} T}}\right)\\
	Z_{N}&=Z^{N}\\
	\Rightarrow\langle E\rangle&=k_{B} T^{2} \frac{\partial \ln z}{\partial T}=\frac{5 N k_{B} T}{2} \text {, since } m g L>>k_{B} T
	\end{align*}
			So the correct answer is \textbf{Option (d)}
\end{answer}
\item 	The free energy of the gas of $N$ particles in a volume $V$ and at a temperature $T$ is $F=N k_{B} T \ln \left[a_{0} V\left(k_{B} T\right)^{5 / 2} / N\right]$, where $a_{0}$ is a constant and $k_{B}$ denotes the Boltzmann constant. The internal energy of the gas is
 \begin{tasks}(2)
	\task[\textbf{a.}]$\frac{3}{2} N k_{B} T$
	\task[\textbf{b.}]$\frac{5}{2} N k_{B} T$
	\task[\textbf{c.}]$N k_{B} T \ln \left[a_{0} V\left(k_{B} T\right)^{5 / 2} / N\right]-\frac{3}{2} N k_{B} T$
	\task[\textbf{d.}]  $N k_{B} T \ln \left[a_{0} V /\left(k_{B} T\right)^{5 / 2}\right]$
\end{tasks}	
\begin{answer}
	\begin{align*}
    F&=N k_{B} T \ln \left[a_{0} V\left(k_{B} T\right)^{5 / 2} / N\right], F=U-T S, U=F+T S\\
	d F&=-S d T-P d V \Rightarrow\left(\frac{\partial F}{\partial T}\right)_{V}=-S \text { or } S=-\left(\frac{\partial F}{\partial T}\right)_{V} \Rightarrow U=F-T\left(\frac{\partial F}{\partial T}\right)_{V} \\
	F&=N k_{B} T \ln \left(C T^{5 / 2}\right) \text { where } C=\frac{a_{0} V k_{B}^{5 / 2}}{N} \\
	\left(\frac{\partial F}{\partial T}\right)_{V}&=N k_{B} \ln \left(C T^{5 / 2}\right)+N k_{B} T \frac{C}{C T^{5 / 2}} \frac{5}{2} T^{3 / 2} \Rightarrow T\left(\frac{\partial F}{\partial T}\right)_{V}=N k_{B} T \ln \left(C T^{5 / 2}\right)+\frac{5}{2} N k_{B} T \\
	T\left(\frac{\partial F}{\partial T}\right)_{V}&=F+\frac{5}{2} N k_{B} T \Rightarrow U=F-T\left(\frac{\partial F}{\partial T}\right)_{V}=-\frac{5}{2} N k_{B} T .\\
	T\left(\frac{\partial F}{\partial T}\right)_{V}&=F+\frac{5}{2} N k_{B} T \Rightarrow U=F-T\left(\frac{\partial F}{\partial T}\right)_{V}=-\frac{5}{2} N k_{B} T .
	\end{align*}
	So the correct answer is \textbf{Option (b)}
\end{answer}
	\item A system has two normal modes of vibration, with frequencies $\omega_{1}$ and $\omega_{2}=2 \omega_{1}$. What is the probability that at temperature $T$, the system has an energy less than $4 \hbar \omega_{1}$ ?
	[In the following $x=e^{-\beta \hbar \omega_{1}}$ and $Z$ is the partition function of the system.]
	 \begin{tasks}(2)
		\task[\textbf{a.}]$x^{3 / 2}\left(x+2 x^{2}\right) / Z$
		\task[\textbf{b.}]$x^{3 / 2}\left(1+x+x^{2}\right) / Z$
		\task[\textbf{c.}]$x^{3 / 2}\left(1+2 x^{2}\right) / Z$
		\task[\textbf{d.}] $x^{3 / 2}\left(1+x+2 x^{2}\right) / Z$
	\end{tasks}
\begin{answer}
	\begin{align*}
	 \intertext{There is two normal mode so there is two degree of freedom.}
\intertext{	Energy of harmonic oscillator is $E=\left(n_{1}+\frac{1}{2}\right) \hbar \omega_{1}+\left(n_{2}+\frac{1}{2}\right) \hbar \omega_{2}$.} E&=\left(n_{1}+\frac{1}{2}\right) \hbar \omega_{1}+\left(n_{2}+\frac{1}{2}\right) \hbar 2 \omega_{1}\text{ where } n_{1}=0,1,2,3 \ldots .
\text{ and }n_{2}=0,1,2,3 \ldots .\\
\intertext{	Ground state energy $E=\frac{3 \hbar \omega_{1}}{2}$, first excited state energy $E=\frac{5 \hbar \omega_{1}}{2}$. Second excited state energy $E=\frac{7 \hbar \omega_{1}}{2}$ which is doubly degenerate state so $g=2$, other state have more energy than $4 \hbar \omega_{1}$.}
	P\left(E<4 \hbar \omega_{1}\right)&=\frac{e^{-\frac{3 \beta \hbar \omega_{1}}{2}}+e^{-\frac{5 \beta \hbar \omega_{1}}{2}}+2 e^{-\frac{7 \beta \hbar \omega_{1}}{2}}}{Z}=\frac{x^{3 / 2}\left(1+x+2 x^{2}\right)}{Z}\text{ where }x=e^{-\beta \hbar \omega_{1}} .
	\end{align*}
	So the correct answer is \textbf{Option (d)}
\end{answer}
\item 	The entropy of a system, $(S)$, is related to the accessible phase space volume $\Gamma$ by $S=k_{B} \ln \Gamma(E, N, V)$ where $E, N$ and $V$ are the energy, number of particles and volume respectively. From this one can conclude that $\Gamma$
	 \begin{tasks}(2)
		\task[\textbf{a.}]does not change during evolution to equilibrium
		\task[\textbf{b.}]oscillates during evolution to equilibrium
		\task[\textbf{c.}]is a maximum at equilibrium
		\task[\textbf{d.}]is a minimum at equilibrium 
	\end{tasks}
\begin{answer}
	Entropy is maximum at equilibrium.\\
		So the correct answer is \textbf{Option (c)}
\end{answer}
\item 	Consider a one-dimensional Ising model with $N$ spins, at very low temperatures when almost all spins are aligned parallel to each other. There will be a few spin flips with each flip costing an energy $2 J$. In a configuration with $r$ spin flips, the energy of the system is $E=-N J+2 r J$ and the number of configuration is ${ }^{N} C_{r} ; r$ varies from 0 to $N$. The partition function is
	 \begin{tasks}(2)
		\task[\textbf{a.}] $\left(\frac{J}{k_{B} T}\right)^{N}$
		\task[\textbf{b.}]$e^{-N J / k_{B} T}$
		\task[\textbf{c.}] $\left(\sinh \frac{J}{k_{B} T}\right)^{N}$
		\task[\textbf{d.}] $\left(\cosh \frac{J}{k_{B} T}\right)^{N}$
	\end{tasks}
\begin{answer}
	\begin{align*}
	\text { Let us}&\text{ consider only three energy levels, } E_{r}=-2 J+2 r J \text { i.e. } E_{0}=-2 J, E_{1}=0 \text { and }\\
	E_{2}&=2 J.\\
	Q_{2}&=\frac{\left({ }^{2} C_{0} e^{-\beta E_{0}}+{ }^{2} C_{1} e^{-\beta E_{1}}+{ }^{2} C_{2} e^{-\beta E_{2}}\right)}{\sum_{r=0}^{2}{ }^{2} C_{r}}=\frac{\left(e^{\beta 2 J}+2 e^{0}+e^{\beta 2 J}\right)}{4}=\frac{\left(e^{\beta J}+e^{\beta J}\right)^{2}}{4} \\
	Q_{2}&=\left(\frac{e^{\beta J}+e^{\beta J}}{2}\right)^{2}=(\cosh \beta J)^{2} \Rightarrow(\cosh \beta J)^{2} \Rightarrow Q_{N}=(\cosh \beta J)^{N}
	\end{align*}
		So the correct answer is \textbf{Option (d)}
\end{answer}
	\item Consider a system of three spins $S_{1}, S_{2}$ and $S_{3}$ each of which can take values $+1$ and $-1$. The energy of the system is given by $E=-J\left[S_{1} S_{2}+S_{2} S_{3}+S_{3} S_{1}\right]$ where $J$ is a positive constant. The minimum energy and the corresponding number of spin configuration are, respectively,
	 \begin{tasks}(2)
		\task[\textbf{a.}]$J$ and 1
		\task[\textbf{b.}]$-3 J$ and 1
		\task[\textbf{c.}]$-3 J$ and 2
		\task[\textbf{d.}]  $-6 J$ and 2
	\end{tasks}
\begin{answer}
	\begin{align*}
\text{If we take }S_{1}&=S_{2}=S_{3}=+1\text{ i.e. }\hspace{3cm}\stackrel{\uparrow}{s_1}\qquad \stackrel{\uparrow}{s_2}\qquad\stackrel{\uparrow}{s_3}\\
\text { Then energy, } E&=-J[1 \times 1+1 \times 1+1 \times 1]=-3 J\\
\text { Again } S_{1}&=S_{2}=S_{3}=-1 \text {, then }\hspace{2.8cm}\downarrow\qquad \ \downarrow\qquad \ \downarrow\\
\text{Energy }(E)&=-3 J
\intertext{So, minimum energy is $(-3 J)$ and there are two spin configuration.}\\
\text { If we take }&\stackrel{\uparrow}{s_1}\qquad \stackrel{\downarrow}{s_2}\qquad\stackrel{\uparrow}{s_3}\\
\text { Then we get Maximum energy } E&=J \text {. }
	\end{align*}
	So the correct answer is \textbf{Option (c)}
\end{answer}
\item 	Consider a system of two Ising spins $S_{1}$ and $S_{2}$ taking values $\pm 1$ with interaction energy given by $\varepsilon=-J S_{1} S_{2}$, when it is in thermal equilibrium at temperature $T$. For large $T$, the average energy of the system varies as $C / k_{B} T$, with $C$ given by
	 \begin{tasks}(2)
		\task[\textbf{a.}]$-2 J^{2}$
		\task[\textbf{b.}] $-J^{2}$
		\task[\textbf{c.}]$J^{2}$
		\task[\textbf{d.}] $4 J$ 
	\end{tasks}
\begin{answer}
	\begin{align*}
	\intertext{The interaction energy is given by $E=-J S_{1} S_{2}$ where $S_{1}$ and $S_{2}$ taking values $\pm 1$.}
	\text { Possible }&\text{values of the Energy of the system are }\\
	E_{1}&=-J 1 \cdot 1=-J, \quad E_{2}=-J(-1) \cdot(1)=+J\\
	E_{3}&=-J(1) \cdot(-1)=+J, \quad E_{4}=-J(-1) \cdot(-1)=-J\\
	\langle U\rangle&=\frac{\sum_{r} E_{r} g_{r} e^{-\frac{E_{r}}{k T}}}{\sum_{r} g_{r} e^{\frac{E_{r}}{k T}}}=\frac{-2 J e^{\frac{J}{k T}}+2 J e^{\frac{J}{k T}}}{2 e^{\frac{J}{k T}}+2 e^{-\frac{J}{k T}}}=-J\left(\frac{e^{\frac{J}{k T}}-e^{-\frac{J}{k T}}}{e^{\frac{J}{k T}}+e^{\frac{J}{k T}}}\right)=-J \frac{\left(1+\frac{J}{k T}-\left(1-\frac{J}{k T}\right)\right)}{1+\left(\frac{J}{k T}\right)+1-\left(\frac{J}{k T}\right)}\\
	\Rightarrow\langle U\rangle&=-\frac{J^{2}}{k T} \Rightarrow C=-J^{2} \quad \text { (For large } T, \frac{J}{k T}<<1 \text { ) }
	\end{align*}
		So the correct answer is \textbf{Option (b)}
\end{answer}
\item A collection $N$ of non-interacting spins $S_{i}, i=1,2, \ldots \ldots, N,\left(S_{i}=\pm 1\right)$ is kept in an external magnetic field $B$ at a temperature $T$. The Hamiltonian of the system is $H=-\mu B \Sigma_{i} S_{i}$. What should be the minimum value of $\frac{\mu B}{k_{B} T}$ for which the mean value $\left\langle S_{i}\right\rangle \geq \frac{1}{3}$ ?
 \begin{tasks}(2)
	\task[\textbf{a.}]$\frac{1}{2} N \ln 2$
	\task[\textbf{b.}] $2 \ln 2$
	\task[\textbf{c.}]$\frac{1}{2} \ln 2$
	\task[\textbf{d.}] $N \ln 2$	
\end{tasks}	
\begin{answer}
	\begin{align*}
	 P\left(S_{i}=+1\right)&=\frac{e^{-\frac{\mu B}{k T}}}{e^{-\frac{\mu B}{k T}}+e^{\frac{\mu B}{k T}}}, P\left(S_{i}=-1\right)=\frac{e^{\frac{\mu B}{k T}}}{e^{-\frac{\mu B}{k T}}+e^{\frac{\mu B}{k T}}}\\
	\left\langle S_{i}\right\rangle&=\frac{+1 e^{-\frac{\mu B}{k T}}-e^{+\frac{\mu B}{k T}}}{e^{-\frac{\mu B}{k T}}+e^{\frac{\mu B}{k T}}} \Rightarrow\left\langle S_{i}\right\rangle=-\left(\tanh \frac{\mu B}{k T}\right)\\
	\text{For $N$ particle }&\left\langle S_{i}\right\rangle=-N \tanh \frac{\mu B}{k T}\\
\text{	According to question, }&\frac{\left\langle S_{i}\right\rangle}{N} \geq \frac{1}{3} \Rightarrow-\tanh \left(\frac{\mu B}{k T}\right)=\frac{1}{3} \Rightarrow \frac{\mu B}{k T}=\frac{1}{2} \ln 2
	\end{align*}
		So the correct answer is \textbf{Option (c)}
\end{answer}
	\item A system of $N$ distinguishable particles, each of which can be in one of the two energy levels 0 and $\in$, has a total energy $n \in$, where $n$ is an integer. The entropy of the system is proportional to
	 \begin{tasks}(2)
		\task[\textbf{a.}]$N \ln n$
		\task[\textbf{b.}]$n \ln N$
		\task[\textbf{c.}] $\ln \left(\frac{N !}{n !}\right)$
		\task[\textbf{d.}] $\ln \left(\frac{N !}{n !(N-n) !}\right)$
	\end{tasks}
\begin{answer}
	\begin{align*}
	??
	\end{align*}
	So the correct answer is \textbf{Option (d)}
\end{answer}
\item Consider three Ising spins at the vertices of a triangle which interact with each other with a ferromagnetic Ising interaction of strength $J$. The partition function of the system at temperature $T$ is given by $\left(\beta=\frac{1}{k_{B} T}\right)$ :
 \begin{tasks}(2)
	\task[\textbf{a.}] $2 e^{3 \beta J}+6 e^{-\beta J}$
	\task[\textbf{b.}] $2 e^{-3 \beta J}+6 e^{\beta J}$
	\task[\textbf{c.}]$2 e^{3 \beta J}+6 e^{-3 \beta J}+3 e^{\beta J}+3 e^{-\beta J}$
	\task[\textbf{d.}] $(2 \cosh \beta J)^{3}$
\end{tasks}
\begin{answer}
	\begin{align*}
	&H=J\left(S_{1} S_{2}+S_{1} S_{3}+S_{2} S_{3}\right)\\
	&\begin{array}{llll}
	S_{1} & S_{2} & S_{3} & E
	\end{array}\\
	&\begin{array}{llll}
	1 & 1 & 1 & 3 J
	\end{array}\\
	&\left.\begin{array}{ccc}
	1 & 1 & -1 \\
	1 & -1 & 1 \\
	-1 & 1 & 1
	\end{array}\right\}-J\\
	&\begin{array}{lll}
	-1 & -1 & 1
	\end{array}\\
	&\left.\begin{array}{ccc}
	-1 & 1 & -1 \\
	1 & -1 & -1
	\end{array}\right\}-J\\
	&\left.\begin{array}{lll}
	-1 & -1 & -1
	\end{array}\right\} 3 J\\
	\Rightarrow&z=2 e^{-3 \beta J}+6 e^{\beta J}
	\end{align*}
	So the correct answer is \textbf{Option (b)}
\end{answer}
\item The partition function of a system of $N$ Ising spins is $Z=\lambda_{1}^{N}+\lambda_{2}^{N}$ where $\lambda_{1}$ and $\lambda_{2}$ are functions of temperature, but are independent of $N$. If $\lambda_{1}>\lambda_{2}$, the free energy per spin in the limit $N \rightarrow \infty$ is
 \begin{tasks}(2)
	\task[\textbf{a.}]$-k_{B} T \ln \left(\frac{\lambda_{1}}{\lambda_{2}}\right)$
	\task[\textbf{b.}] $-k_{B} T \ln \lambda_{2}$
	\task[\textbf{c.}]$-k_{B} T \ln \left(\lambda_{1} \lambda_{2}\right)$
	\task[\textbf{d.}] $-k_{B} T \ln \lambda_{1}$
\end{tasks}
\begin{answer}
	\begin{align*}
Z&=\lambda_{1}^{N}+\lambda_{2}^{N}, F=-k T \ln \left(\lambda_{1}^{N}+\lambda_{2}^{N}\right)\text{, it is given } \lambda_{1}>>\lambda_{2}\\
	\Rightarrow F&=-k T \ln \left[\lambda_{1}^{N}\left(1+\left(\frac{\lambda_{2}}{\lambda_{1}}\right)^{N}\right], \quad \frac{\lambda_{2}}{\lambda_{1}} \approx 0\right. \\
	F&=-k T \ln \lambda_{1}^{N}=-N k T \ln \lambda_{1} \Rightarrow \frac{F}{N}=-k T \ln \lambda_{1}
	\end{align*}
	So the correct answer is \textbf{Option (d)}
\end{answer}
\item The Hamiltonian of a system of $N$ non interacting spin $\frac{1}{2}$ particles is $H=-\mu_{0} B \sum_{i} S_{i}^{z}$, where $S_{i}^{z}=\pm 1$ are components of $i^{\text {th }}$ spin along an external magnetic field $B$. At a temperature $T$ such that $e^{\frac{\mu_{0} B}{k_{B} T}}=2$. the specific heat per particle is
 \begin{tasks}(2)
	\task[\textbf{a.}]$\frac{16}{25} k_{B}$
	\task[\textbf{b.}] $\frac{8}{25} k_{B} \ln 2$
	\task[\textbf{c.}]$k_{B}(\ln 2)^{2}$
	\task[\textbf{d.}]  $\frac{16}{25} k_{B}(\ln 2)^{2}$
\end{tasks}
\begin{answer}
	\begin{align*}
	\text {  For the given system } E&=-\mu_{0} B \tanh \frac{\mu_{0} B}{k T}\\
	C_{V}&=\left(\frac{\mu_{0} B}{k T}\right)^{2} N k \sec ^{2} h \frac{\mu_{0} B}{k T}\\
	\frac{C_{V}}{N}&=\left(\frac{\mu_{0} B}{k T}\right)^{2} k \frac{4}{\left(e^{\frac{\mu_{0} B}{k T}}+e^{-\frac{\mu_{0} B}{k T}}\right)^{2}} \quad \because e^{\frac{\mu_{0} B}{k T}}=2 \Rightarrow \frac{\mu_{0} B}{k T}=\ln 2\\
	\frac{C_{V}}{N}&=\left(\frac{\mu_{0} B}{k T}\right)^{2} k \frac{4}{\left(2+\frac{1}{2}\right)^{2}}=\left(\frac{\mu_{0} B}{k T}\right)^{2} k \frac{16}{25}=(\ln 2)^{2} \frac{16 k}{25}=\frac{16}{25} k(\ln 2)^{2}
	\end{align*}
	So the correct answer is \textbf{Option (d)}
\end{answer}
\item A gas of non-relativistic classical particles in one dimension is subjected to a potential $V(x)=\alpha|x|$ (where $\alpha$ is a constant). The partition function is $\left(\beta=\frac{1}{k_{B} T}\right)$
 \begin{tasks}(2)
	\task[\textbf{a.}] $\sqrt{\frac{4 m \pi}{\beta^{3} \alpha^{2} h^{2}}}$
	\task[\textbf{b.}] $\sqrt{\frac{2 m \pi}{\beta^{3} \alpha^{2} h^{2}}}$
	\task[\textbf{c.}]$\sqrt{\frac{8 m \pi}{\beta^{3} \alpha^{2} h^{2}}}$
	\task[\textbf{d.}] $\sqrt{\frac{3 m \pi}{\beta^{3} \alpha^{2} h^{2}}}$
\end{tasks}
\begin{answer}
	\begin{align*}
	z&=\frac{1}{h} \int_{-\infty}^{\infty} e^{-\frac{p_{x}^{2}}{2 m k T}} d p_{x} \int_{-\infty}^{\infty} e^{-\frac{\alpha|x|}{k T}} d x=\frac{1}{h}(2 \pi m k T)^{1 / 2} \int_{-\infty}^{\infty} e^{\frac{\alpha|x|}{k T}} d x\\
	\Rightarrow z&=\left(\frac{2 \pi m k T}{h^{2}}\right)^{1 / 2} \int_{-\infty}^{\infty} e^{-\frac{\alpha|x|}{k T}} d x\\
	\because \int_{-\infty}^{\infty} e^{-\frac{\alpha|x|}{k T}} d x&=\int_{-\infty}^{0} e^{+\frac{\alpha x}{k T}} d x+\int_{0}^{\infty} e^{-\frac{\alpha x}{k T}} d x=\frac{k T}{\alpha}+\frac{k T}{\alpha}=\frac{2 k T}{\alpha}\\
	z&=\left(\frac{2 \pi m k T}{h^{2}}\right)^{1 / 2}\left(\frac{2 k T}{\alpha}\right)=\left(\frac{8 \pi m}{h^{2} \beta^{3} \alpha^{2}}\right)^{\frac{1}{2}}\qquad
\text{	put }\beta=\frac{1}{k T}
	\end{align*}
		So the correct answer is \textbf{Option (c)}
\end{answer}
\item Consider a random walk on an infinite two-dimensional triangular lattice, a part of which is shown in the figure below.
\begin{figure}[H]
	\centering
	\includegraphics[height=3cm,width=4cm]{SP-11}
\end{figure}
If the probabilities of moving to any of the nearest neighbour sites are equal, what is the probability that the walker returns to the starting position at the end of exactly three steps?
 \begin{tasks}(2)
	\task[\textbf{a.}]$\frac{1}{36}$
	\task[\textbf{b.}]$\frac{1}{216}$
	\task[\textbf{c.}] $\frac{1}{18}$
	\task[\textbf{d.}] $\frac{1}{12}$
\end{tasks}
\begin{answer}
Solution: For walker to return to starting position it must move along an equivalent triangle in three steps.\\
For steps one any movement can result in equilateral triangle.\\
For step two, two out of six options will form equilateral triangle.\\
For step three, only one out of six options will form equilateral\\ triangle Total probability $=\frac{6}{6} \times \frac{2}{6} \times \frac{1}{6}=\frac{1}{18}$\\
	So the correct answer is \textbf{Option (c)}
\end{answer}
\item 	An atom has a non-degenerate ground-state and a doubly-degenerate excited state. The energy difference between the two states is $\varepsilon$. The specific heat at very low temperatures $(\beta \varepsilon \gg 1)$ is given by
	 \begin{tasks}(2)
		\task[\textbf{a.}]$k_{B}(\beta \varepsilon)$
		\task[\textbf{b.}]$k_{B} e^{-\beta \varepsilon}$
		\task[\textbf{c.}] $2 k_{B}(\beta \varepsilon)^{2} e^{-\beta \varepsilon}$
		\task[\textbf{d.}]  $k_{B}$
	\end{tasks}
\begin{answer}
	\begin{align*}
	\intertext{Assume energy at ground state is 0 and energy at first excited state is $\in$. The partition function is $Z=1+2 e^{-\beta \epsilon}$}
\text{	Energy }&=\frac{2 \in e^{-\beta \epsilon}}{\left(1+2 e^{-\beta \epsilon}\right)}\\
	\text{Specific heat, }C_{V}&=\left(\frac{\partial U}{\partial T}\right)_{V}=\frac{2 \in e^{-\frac{\epsilon}{k T}}(-\in) \frac{-1}{k T^{2}}}{\left(1+2 e^{\frac{-\epsilon}{k T}}\right)}+\frac{2 \in e^{\frac{-\epsilon}{k T}} \in \in \frac{2}{k T^{2}}}{\left(1+2 e^{\frac{-\epsilon}{k T}}\right)^{2}}\\
	&=2 k\left(\frac{\epsilon}{k T}\right)^{2} e^{\frac{-\epsilon}{k T}} \frac{\left(1+2 e^{\frac{-\epsilon}{k T}}\right)}{\left(1+2 e^{\frac{-\epsilon}{k T}}\right)^{2}}=2 k(\beta \in)^{2} e^{-\beta \in} \frac{\left(1+2 e^{-\beta \epsilon}\right)}{\left(1+2 e^{-\beta \epsilon}\right)^{2}}\\
	C_{V} &\simeq 2 k(\beta \in)^{2} e^{-\beta \epsilon}, \quad \beta \in \rightarrow \infty
	\end{align*}
		So the correct answer is \textbf{Option (c)}
\end{answer}
\item In a thermodynamic system in equilibrium, each molecule can exist in three possible states with probabilities $1 / 2,1 / 3$ and $1 / 6$ respectively. The entropy per molecule is
 \begin{tasks}(2)
	\task[\textbf{a.}] $k_{B} \ln 3$
	\task[\textbf{b.}]$\frac{1}{2} k_{B} \ln 2+\frac{2}{3} k_{B} \ln 3$
	\task[\textbf{c.}]$\frac{2}{3} k_{B} \ln 2+\frac{1}{2} k_{B} \ln 3$
	\task[\textbf{d.}] $\frac{1}{2} k_{B} \ln 2+\frac{1}{6} k_{B} \ln 3$
\end{tasks}
\begin{answer}
	\begin{align*}
	S&=-k_{B} \sum_{i} P i \ln P i\\
	P_{1}&=\frac{1}{2}, P_{2}=1 / 3 \text { and } P_{3}=1 / 6 \\
	S&=-k_{B}\left(\frac{1}{2} \ln 1 / 2+1 / 3 \ln 1 / 3+1 / 6 \ln 1 / 6\right) . \\
	&=-k_{b}\left(\frac{1}{2}(\ln 1-\ln 2)+\frac{1}{3}(\ln 1-\ln 3)+\frac{1}{6}(\ln 1-\ln 6)\right. \\
	&=k_{B}\left[\frac{1}{2} \ln 2+\frac{1}{3} \ln 3+\frac{1}{6} \ln 2+\frac{1}{6} \ln 3\right]=k_{B}\left[\frac{1}{2} \ln 2+\frac{1}{6} \ln 2+\frac{1}{3} \ln 3+\frac{1}{6} \ln 3\right] \\
	S&=k_{B}\left[\frac{3 \ln 2+\ln 2}{6}+\frac{2 \ln 3+\ln 3}{6}\right]=k_{B}\left(\frac{4 \ln 2}{6}+\frac{3 \ln 3}{6}\right)=k_{B}\left[\frac{2}{3} \ln 2+\frac{1}{2} \ln 3\right]
	\end{align*}
		So the correct answer is \textbf{Option (c)}
\end{answer}
\end{enumerate}
\newpage
\begin{abox}
	Practise set-02
\end{abox}
\begin{enumerate}
	\item The total number of accessible states of $N$ non interacting particles of spin $1/2$ is:
	 \begin{tasks}(2)
		\task[\textbf{a.}]$2^{N}$
		\task[\textbf{b.}]$N^{2}$
		\task[\textbf{c.}]$2 N^{N / 2}$
		\task[\textbf{d.}]  $N$
	\end{tasks}
	\begin{answer}
		\begin{align*}
		\Omega=\left(2 \times \frac{1}{Q}+1\right)^{N}=2^{N}
		\end{align*}
		So the correct answer is \textbf{Option (a)}
	\end{answer}
	\item The number of distinct ways of placing four indistinguishable balls into 5 distinguishable boxes is :
	 \begin{tasks}(2)
		\task[\textbf{a.}]20
		\task[\textbf{b.}]40
		\task[\textbf{c.}]50
		\task[\textbf{d.}] 70
	\end{tasks}
	\begin{answer}
		\begin{align*}
		{ }^{n+r-1} C_{r}=\frac{8 !}{4 ! 41}&=\frac{8 \times 7 \times 6 \times 5}{4 \times 3 \times 2 \times 1}\\
		&\Rightarrow70
		\end{align*}
			So the correct answer is \textbf{Option (d)}
	\end{answer}
	\item Q.A one dimensional random walker takes steps to left with equal probability.  The probability that the random walker starting from origine is back to origine after $N$ even number of steps:
	 \begin{tasks}(2)
		\task[\textbf{a.}]$\frac{N !}{\left(\frac{N}{2}\right) !\left(\frac{N}{2}\right) !}\left(\frac{1}{2}\right)^{n}$
		\task[\textbf{b.}]$\frac{N !}{\left(\frac{N}{2}\right) !\left(\frac{N}{2}\right) !}$
		\task[\textbf{c.}] $2N!\left(\frac{1}{2}\right)^{2 \mu}$
		\task[\textbf{d.}] $N !\left(\frac{1}{2}\right)^{N}$
	\end{tasks}
	\begin{answer}
		\begin{align*}
		\text{Number of steps }&=N\\
		P(b)&=P(q)=1/2 \\
		\Rightarrow n_{1}&=N / 2 \text { and } n_{2}=N / 2\\
		\text { Probability  }&={ }^{N} C_{N / 2}\left(\frac{1}{2}\right)^{N / 2}\left(\frac{1}{2}\right)^{N / 2}\\
		&=\frac{N !}{\left(\frac{N}{2}\right) !\left(\frac{N}{2}\right) !}\left(\frac{1}{2}\right)^{N}
		\end{align*}
		So the correct answer is \textbf{Option (a)}
	\end{answer}
	\item A random walker takes a step of unit length in +ve direction with probability $2/3$ and a step of unit length in -ve direction with probability $1/3$. The net displacement of random walker after $n$ steps is?
	 \begin{tasks}(4)
		\task[\textbf{a.}]$n$
		\task[\textbf{b.}]$(1 / 3)^{n}$
		\task[\textbf{c.}]$(2 / 3) n$
		\task[\textbf{d.}]$2 n$
	\end{tasks}
	\begin{answer}
		\begin{align*}
		\text { Displacement } &=n_{f_{L}}-n_{f_{L}} \\
		&=\frac{2}{3} n-\frac{1}{3} n \Rightarrow \frac{1}{3} n
		\end{align*}
		So the correct answer is \textbf{Option (b)}
	\end{answer}
	\item In a one dimension random walker takes a step with equal probability to the left or right. What is the probability that the walker returns to starting point after four steps?
	 \begin{tasks}(2)
		\task[\textbf{a.}]$1 / 2$
		\task[\textbf{b.}]$3 / 4$
		\task[\textbf{c.}]$3 / 8$
		\task[\textbf{d.}] $3 / 16$ 
	\end{tasks}
	\begin{answer}
		\begin{align*}
		P(2,2) &={ }^{4} \mathrm{c}_{2}\left(\frac{1}{2}\right)^{2}\left(\frac{1}{2}\right)^{2} \\
		&=\frac{4 !}{2 ! 2 !}\left(\frac{1}{2}\right)^{4}=\frac{3}{8}
		\end{align*}
		So the correct answer is \textbf{Option (c)}
	\end{answer}
	\item Consider a random walker on a squate lattice. At each step the walker moves to a nearest neighbour site with equal probability for each of the four sites.  The walker starts to the origine and takes. The probability that during this walk no site is visited more than once is:
	 \begin{tasks}(2)
		\task[\textbf{a.}]12/27
		\task[\textbf{b.}]$27 / 64$
		\task[\textbf{c.}]$3 / 8$
		\task[\textbf{d.}]$9 / 16$ 
	\end{tasks}
	\begin{answer}
		\begin{align*}
		\text{Total ways }&=(4)^{3} \Rightarrow 64\\
		\text{favourable }&=4 \times 3 \times 3\\
		&=36\\
		\text { Prob. }&=\frac{36}{64}=\frac{9}{16}
		\end{align*}
		So the correct answer is \textbf{Option (d)}
	\end{answer}
	\item A child makes a random walk on a square latiice of lattice costent $a$ taking a step in the north, east, south, or west direction with probabilities $0.255,0.255,0.245$ respectively. After a large number of steps $N$ the expected position of child w.r.t starting point is at a distance
	 \begin{tasks}(2)
		\task[\textbf{a.}]$\sqrt{2} \times 10^{-2} \mathrm{Na}$ in north-east direction.
		\task[\textbf{b.}]$\sqrt{2 N} \times 10^{-2} a$  in north east direction.
		\task[\textbf{c.}]$2 \sqrt{2} \times 10^{-2} \mathrm{Na}$ in south east direction
		\task[\textbf{d.}] 0
	\end{tasks}
	\begin{answer}$\left. \right. $
		\begin{figure}[H]
			\centering
			\includegraphics[height=3.4cm,width=6cm]{SP-14}
		\end{figure}
		\begin{align*}
\langle x\rangle_{N}&=0.255 \mathrm{Na}\hat{i}
+0.255 \mathrm{Na} \hat{j}-0.245 \mathrm{Na} \hat{i}
-0.245 \mathrm{Na\hat{j}}\\
\langle x\rangle_{N}&=10^{-2}(Na \hat{i}+Na \hat { j })\\
&=10^{-2} \sqrt{(1)^{2}+(1)^{2}} \mathrm{Na}\\
\langle x\rangle N&=\sqrt{2} \times 10^{-2} \mathrm{Na} \text { in } N-E
		\end{align*}
		So the correct answer is \textbf{Option (a)}
	\end{answer}
	\item The entropy of a system $S$ is related to the accessible phase space volume $\Gamma$ by 
	$$S=K_{B} \ln \Gamma(E, V, N)$$
	where $E,N \text{ and }V$ are energy number of particles and volume respectively. From this one can conclude that $\Gamma$.
	 \begin{tasks}(2)
		\task[\textbf{a.}]Does not change during evolution to equilibrium.
		\task[\textbf{b.}]Oscillated during evolution to equilibrium .
		\task[\textbf{c.}]Is a maximum in equilibrium .
		\task[\textbf{d.}] Is a minimum in equilibrium .
	\end{tasks}
	\begin{answer}
		\begin{align*}
		E, U, N \rightarrow& \text{ micro canonical ensemble}\\
		S&=K \ln T(E, V, N)\\
		\text{maximum when }&\text{$\Gamma$ is maximum}\\
		\text{In equilibrium } &s \rightarrow\text{ maximum}\\
		\Rightarrow &r \rightarrow \text { maximum }
		\end{align*}
			So the correct answer is \textbf{Option (c)}
	\end{answer}
	\item In a canonical ensemble description of a system, which of the following quantities remain fixed?
	 \begin{tasks}(2)
		\task[\textbf{a.}]Energy of system 
		\task[\textbf{b.}]Square of energy
		\task[\textbf{c.}]Volume of system
		\task[\textbf{d.}] Number of particles
	\end{tasks}
	\begin{answer}
		\begin{align*}
		\text{Canonical ensemble:
		fixed values are }T, V, N
		\end{align*}
		So the correct answers are \textbf{Option (c) and (d)}
	\end{answer}
	\item A microcanonical ensemble represents
	 \begin{tasks}(2)
		\task[\textbf{a.}]A system in contact with a heat reservoir.
		\task[\textbf{b.}]An isolated system in equilibrium.
		\task[\textbf{c.}]A system can exchange particle with surroundings
		\task[\textbf{d.}]A system under constant external pressure. 
	\end{tasks}
	\begin{answer}
		microcanonical: $E, V, N$ fixed isolated system in equilibrium\\
			So the correct answer is \textbf{Option (b)}
	\end{answer}
	\item The pressure for a non interacting fermi gas with internal energy V at temperature T is 
	 \begin{tasks}(2)
		\task[\textbf{a.}]$P=\frac{3}{2}\left(\frac{u}{v}\right)$
		\task[\textbf{b.}]$p=\frac{2}{3} \frac{u}{V}$
		\task[\textbf{c.}]$ P=\frac{3}{5} \frac{U}{V}$
		\task[\textbf{d.}]  none
	\end{tasks}
\begin{answer}
	\begin{align*}
	\text{Non interacting fermi gas=}&\text{ ideal classical gas}\\
	U&=\frac{3}{2} N K+=\frac{3}{2} P V\\
	P&=\frac{2}{3}\left(\frac{U}{V}\right)
	\end{align*}
	So the correct answer is \textbf{Option (b)}
\end{answer}
\item The partitian function of two base particles each of which can occupy any of the two energy levels of $\epsilon$ is.
 \begin{tasks}(2)
	\task[\textbf{a.}] $1+e^{-2 \varepsilon \mid K T}+2 e^{-\varepsilon \mid K T}$
	\task[\textbf{b.}] $1+e^{-2 \varepsilon \mid k t}+e^{-\varepsilon \mid k-}$
	\task[\textbf{c.}]$2+e^{-2 \varepsilon \mid k t}+e^{-\varepsilon \mid k f}$
	\task[\textbf{d.}] $e^{-2 \varepsilon \mid k T}+e^{-\varepsilon \mid k \rho \text {. }}$
\end{tasks}  
\begin{answer}$\left. \right. $\\
\begin{tabular}{|p{1cm}|p{1cm}|p{2cm}|}
	\hline
	0 & $\epsilon$ & $E_{Total}$\\\hline
	aa&-&0\\\hline
	a&a&$\epsilon$\\\hline
	-&aa&2$\epsilon$\\\hline
\end{tabular}\\
\begin{align*}
	z&=\sum g_{i} e^{-\beta \varepsilon i} \\
	z&=1 e^{0}+e^{-\beta \varepsilon}+e^{-2 \beta \varepsilon} \\
	z&=1+e^{-\beta \varepsilon}+e^{-2 \beta \varepsilon}\\
	g_{i}&=1
\end{align*}
	So the correct answer is \textbf{Option (b)}
\end{answer}
\item Consider a system of two non-interacting classical particles which can occupy any of the three energy levels with energy values $E=0,\epsilon \text{ and }2\epsilon$ havinh degeneracies $g(E)=1,2 \text{ and } 4$ respectively. The mean energy of the system is.
 \begin{tasks}(2)
	\task[\textbf{a.}] $\varepsilon\left(\frac{4 \exp \left(-\varepsilon \mid k_{B} T\right)+8 \exp \left(-2 \varepsilon \mid K_{B} T\right)}{1+2 \exp \left(-\varepsilon \mid k_{B} T\right)+4 \exp \left(-2 \varepsilon \mid k_{B} T\right)}\right)$
	\task[\textbf{b.}]$\varepsilon\left[\frac{2 \exp \left(-\varepsilon \mid K_{B} T\right)+8 \exp \left(-2 \varepsilon \mid K_{B} T\right)}{1+2 \exp \left(-\varepsilon \mid K_{B} T\right)+4 \exp \left(-2 \varepsilon||_{B} T\right)}\right)$
	\task[\textbf{c.}]$\varepsilon\left[\frac{2 \exp \left(-\varepsilon \mid k_{B} T\right)+4 \exp \left(-2 \varepsilon \mid k_{B} T\right)}{1+2 \exp \left(-\varepsilon \mid K_{B} T\right)+4 \exp \left(-2 \varepsilon \mid k_{B} T\right)}\right]^{2}$
	\task[\textbf{d.}] $\varepsilon\left[\frac{\exp \left(-\varepsilon \mid k_{B} T\right)+2 \exp \left(-2 \varepsilon \mid k_{B} T\right)}{1+\exp \left(-\varepsilon \mid k_{B} T\right)+\exp \left(-2 \varepsilon \mid k_{B} T\right)}\right]^{2}$
\end{tasks}
\begin{answer}
	\begin{align*}
	\left\langle E_{i}\right\rangle&=\sum E_{i} P_{i}\\
	&=\frac{\sum \varepsilon_{i} g_{i} e^{-\beta \varepsilon_{i}}}{\sum g_{i} e^{-\beta \varepsilon i}} \\
	&=\frac{0+2 \varepsilon e^{-\beta \varepsilon}+g \varepsilon^{-2 \beta \varepsilon}}{1 e^{0}+2 e^{-\beta \varepsilon}+4 e^{-2 \beta \varepsilon}} \\
	&=\varepsilon\left(\frac{2 e^{-\beta \varepsilon}+8 e^{-2 \beta \varepsilon}}{1+2 e^{-\beta \varepsilon}+4 e^{-2 \beta \varepsilon}}\right)
	\end{align*}
\end{answer}
\item The free energy of a gas, $N$ particles in a volume $V$ and at a temperature $T$ is
$$F=N K_{B} T \ln \left(\frac{a_{0} V\left(H_{B} T\right)^{512}}{N}\right)$$
where $a_0$ is a constant and $k_B$ denotes the Boltzmann constant. The internal energy of a gas is 
 \begin{tasks}(2)
	\task[\textbf{a.}]$\frac{3}{2} N_{B} T$
	\task[\textbf{b.}]$\frac{5}{2} N K_{B} T$
	\task[\textbf{c.}] $N K_{B} T \ln \left[\frac{a_{0} V\left(K_{B} T\right)^{5 / 22}}{N}\right]-\frac{3}{2} N K_{B} T$
	\task[\textbf{d.}] $N K_{B} T \ln \left[\frac{a_{O} V}{\left(K_{B} T\right)^{5 / 2}}\right]$
\end{tasks}
\begin{answer}
	\begin{align*}
	F&=K_{B} T \ln \left[\frac{a_{O} V\left(K_{B} T\right)^{5 / 2}}{N}\right]^{N}\\
	F&= K_{B}+\ln z
	\intertext{compare 1 and 2, we get }
	\ln z&=-\ln \left[\frac{a_{0} V\left(K_{B} T\right)^{5 / 2}}{N}\right]^{N} \\
	\langle u\rangle&=-\frac{\partial}{\partial \beta} \ln z=+K T^{2} \frac{\partial}{\partial T} \ln z\\
	\langle u\rangle&=N K_{B} T^{2} \frac{\partial}{\partial T}\left(-\ln \left(\frac{a_{O} V\left(K_{B} T\right)^{5 / 2}}{N}\right)\right)\\
	&=N K_{B} T^{2} \frac{\partial}{\partial T}\left[-\ln T^{5 / 2}-\ln \left(\frac{a_{0} \cup K_{B}}{N}\right)\right]\\
	&=-\frac{5}{2} N K_{B} T^{2} \times \frac{1}{T} \\
	&=-\frac{5}{2} N K_{B} T
	\end{align*}
	So the correct answer is \textbf{Option (b)}
\end{answer}
\item A system has two energy levels with energies $\epsilon \text{ and }2\epsilon$. The lower level is 4-fold generate white the upper level is double degenerate. If there are $N$ non-interacting classical particles in the system, which is in thermodynamic equilibrium at a temperature $T$, the fraction of particles in the upper level is 
 \begin{tasks}(2)
	\task[\textbf{a.}] $\frac{1}{1+e^{-\left.\varepsilon\right|_{B} T}}$
	\task[\textbf{b.}] $\frac{1}{1+2 e^{\varepsilon \mid} k_{B} \mid}$
	\task[\textbf{c.}]$\frac{1}{2 e^{\varepsilon \mid K_{B} T}+4 e^{2 \varepsilon \mid K_{B} T}}$
	\task[\textbf{d.}] $\frac{1}{2 e^{\varepsilon \mid \forall_{B} T}-4 e^{2 \varepsilon \mid K_{B} T}}$
\end{tasks}
\begin{answer}
	\begin{align*}
	\begin{array}{ll}
	E_{1}=\varepsilon & g_{1}=4 \\
	E_{2}=2 \varepsilon & g_{2}=2
	\end{array}
	\text{fraction of }&\text{particles in upper level}\\
	P_{2}&=\frac{N_{2}}{N}=\frac{2 e^{-2 \beta \varepsilon}}{4 e^{-\beta \varepsilon}+2 e^{-\beta \varepsilon}}\\
	P_{\theta}&=\frac{1}{2 e^{\beta \varepsilon}+1}
	\end{align*}
	So the correct answer is \textbf{Option (b)}
\end{answer}
\item A one dimension chain consists of a set of $N$ rods each of length $a$. When stretched by a load, each rod can align either parallel of antiparallel to length. The energy of a rod is $-\epsilon$ when aligned parallel to length of the chain is $+\epsilon$ when perpendicular to it. When the chain is in thermal equilibrium at temperature $T$ its average length is,
 \begin{tasks}(2)
	\task[\textbf{a.}]$\frac{\mathrm{Na}}{2}$
	\task[\textbf{b.}]$\mathrm{Na}$
	\task[\textbf{c.}]$\frac{N a}{1+e^{2 \varepsilon} k_{B} T}$
	\task[\textbf{d.}] $\frac{N_{a}}{1+e^{-\varepsilon \mid K_{B} T}}$
\end{tasks}
 \begin{answer}
 	\begin{align*}
 	\langle L\rangle&=\sum_{i} L_{i} P_{i}\\
 	&=\frac{N_{a} P_{11}+0 \cdot P_{\perp a r}}{P_{11}+P_{\perp}}=\frac{N a e^{\beta \varepsilon}}{e^{\beta \varepsilon}+e^{-\beta \varepsilon}}\\
 	&=\frac{N a}{1+e^{2 \beta \varepsilon}}
 	\end{align*}
 	So the correct answer is \textbf{Option (c)}
 \end{answer}
\item An atom has a non-degenerate ground state and a double degenerate excited state. The energy difference between the two states is $\epsilon$. The specific heat at very low temperature $(\beta \varepsilon \gg) \mid$ is given by
 \begin{tasks}(2)
	\task[\textbf{a.}]$K_{B}(\beta \varepsilon)$
	\task[\textbf{b.}]$k_{B} e^{-\beta \varepsilon}$
	\task[\textbf{c.}]$2 k_{B}(\beta \varepsilon)^{2} e^{-\beta \varepsilon}$
	\task[\textbf{d.}] $K_{B}$
\end{tasks}
\begin{answer}$\left. \right. $\\
	\begin{figure}[H]
		\centering
		\includegraphics[height=2cm,width=4cm]{SP-15}
	\end{figure}
	\begin{align*}
	z &=1 e^{o \beta}+2 e^{-\beta \varepsilon} \\
	&=1+2 e^{-\beta \varepsilon}\\
	\langle E\rangle&=\frac{0+2 \varepsilon e^{-\beta \varepsilon}}{1+2 e^{-\beta \varepsilon}}=\frac{2 \varepsilon e^{-\beta \varepsilon}}{1+2 e^{-\beta \varepsilon}}\\
	C_{V}&=\left(\frac{\partial E}{\partial T}\right)_{V}\hspace{7.3cm}\beta=\frac{1}{K_{B} T}\\
	&=\frac{-1}{K_{B} T^{2}}\left(\frac{\partial E}{\partial \beta}\right)_{V}\hspace{6cm}\partial \beta=-\frac{\partial T}{K_{B} T^{2}}\\
	C_{V}&=-K_{B} \beta^{2}\left(\frac{\partial E}{\partial \beta}\right)_{V}\hspace{5.7cm}\frac{1}{\partial T}=-\frac{1}{B_{B} T}2\frac{1}{\partial \beta}\\
	c_{V}&=-k_{B} \beta^{2} \frac{\left(1+2 e^{-\beta \varepsilon}\right)\left(-2 \varepsilon^{2} e^{-\beta \varepsilon}\right)-\left(2 \varepsilon e^{-\beta \varepsilon}\right)\left(-2 \varepsilon e^{-\beta \varepsilon}\right)}{\left(1+2 e^{-\beta \varepsilon}\right)^{2}}\\
	c_{V}&=-k_{B} \beta^{2}\left(-\frac{2 \varepsilon^{2} e^{-\beta \varepsilon}}{\left(1+2 e^{-\beta \varepsilon}\right)^{2}}\right)\\
	C_{v}&=2 k_{B}(\beta \varepsilon)^{2} e^{\beta \varepsilon} \qquad(\therefore \beta \varepsilon>>1)
	\end{align*}
	So the correct answer is \textbf{Option (c)}
\end{answer}
\item Consider a system maintained at temperature $T$, with two available energy states $E_1$\&$E_2$ each with degeneracies $g_1$\&$g_2$. If $p_1$\&$p_2$ are probabilities of occupancy of the energy states, what is the entropy of the system.
 \begin{tasks}(2)
	\task[\textbf{a.}]$S=-K_{B}\left[P_{1} \ln \left(\frac{P_{1}}{g_{1}}\right)+P_{2} \ln \left(\frac{P_{2}}{g_{2}}\right)\right]$
	\task[\textbf{b.}]$S=-K_{B}\left[P_{1} \ln \left(P_{1} g_{1}\right)+P_{2} \ln \left(P_{2} g_{2}\right)\right]$
	\task[\textbf{c.}]$S=-K_{B}\left[P_{1} \ln \left(p_{1}^{g_{1}}\right)+P_{2} \ln (P_2 g_2)\right]$
	\task[\textbf{d.}] $s=-K_{B}\left[\left(\frac{1}{P_{1}}\right) \ln \left(\frac{P_{1}}{g_{1}}\right)+\left(\frac{1}{P_{2}}\right) \ln \left(\frac{P_{2}}{g_{2}}\right)\right]$
\end{tasks}
\begin{answer}
	\begin{align*}
	S&=-K \sum_{i} P_{i} \ln \left(\frac{P_{i}}{g_{i}}\right)\\
	s&=-k_{B}\left[P_{1} \ln \left(\frac{p_{1}}{g_{1}}\right)+P_{2} \ln \left(\frac{p_{2}}{g_{2}}\right)\right]
	\end{align*}
		So the correct answer is \textbf{Option (a)}
\end{answer}
\end{enumerate}
\newpage
\begin{abox}
	Practise set-03
\end{abox}
\begin{enumerate}
	\item A system of $\mathrm{N}$ non-interacting classical point particles is constrained to move on the twodimensional surface of a sphere. The internal energy of the system is
	{\exyear{GATE 2010}}
\begin{tasks}(4)
\task[\textbf{A.}] $\frac{3}{2} N k_{B} T$
\task[\textbf{B.}] $\frac{1}{2} N k_{B} T$
\task[\textbf{C.}] $N k_{B} T$
\task[\textbf{D.}] $\frac{5}{2} N k_{B} T$
\end{tasks}
\begin{answer}
\begin{align*}
\text{There are $2 \mathrm{~N}$ degree of freedom.}\\
\text{The internal energy of the system is}\frac{N k_{B} T}{2}+\frac{N k_{B} T}{2}&=N k_{B} T
\end{align*}
So the correct answer is \textbf{Option (C)}
\end{answer}
\item Partition function for a gas of photons is given as, $\ln Z=\frac{\pi^{2} V\left(k_{B} T\right)^{3}}{45 \hbar^{3} C^{3}}$. The specific heat of the photon gas varies with temperature as
{\exyear{GATE 2010}}
\begin{tasks}(2)
\task[\textbf{A.}] \begin{figure}[H]
	\centering
	\includegraphics[height=3cm,width=4.5cm]{diagram-20210910(14)-crop}
\end{figure}
\task[\textbf{B.}] \begin{figure}[H]
	\centering
	\includegraphics[height=3cm,width=4.5cm]{diagram-20210910(15)-crop}
\end{figure}
\task[\textbf{C.}] \begin{figure}[H]
	\centering
	\includegraphics[height=3cm,width=4.5cm]{diagram-20210910(16)-crop}
\end{figure}
\task[\textbf{D.}] \begin{figure}[H]
	\centering
	\includegraphics[height=3cm,width=4.5cm]{diagram-20210910(17)-crop}
\end{figure}
\end{tasks}
\begin{answer}
\begin{align*}
\mathrm{U}=\mathrm{K}_{\mathrm{B}} \mathrm{T}^{2} \frac{\partial \ln \mathrm{z}}{\partial \mathrm{T}}, \quad \mathrm{C}_{\mathrm{v}}=\left(\frac{\partial \mathrm{U}}{\partial \mathrm{T}}\right)_{\mathrm{v}} \Rightarrow \mathrm{C}_{\mathrm{v}} \propto \mathrm{T}^{3}
\end{align*}
So the correct answer is \textbf{Option (A)}
\end{answer}
\item From Q. no. 2, the pressure of the photon gas is
{\exyear{GATE 2010}}
\begin{tasks}(4)
\task[\textbf{A.}] $\frac{\pi^{2}\left(k_{B} T\right)^{3}}{15 \hbar^{3} C^{3}}$
\task[\textbf{B.}]  $\frac{\pi^{2}\left(k_{B} T\right)^{4}}{8 \hbar^{3} C^{3}}$
\task[\textbf{C.}] $\frac{\pi^{2}\left(k_{B} T\right)^{4}}{45 \hbar^{3} C^{3}}$
\task[\textbf{D.}] $\frac{\pi^{2}\left(k_{B} T\right)^{3 / 2}}{45 \hbar^{3} C^{3}}$
\end{tasks}
\begin{answer}
\begin{align*}
\text{Since, }P&=-\frac{\partial F}{\partial V} \Rightarrow P=K T\left(\frac{\partial \ln z}{\partial V}\right)_{T}\\&=\frac{\pi^{2}\left(k_{0} T\right)^{4}}{45 \hbar^{3} C^{3}}
\end{align*}
So the correct answer is \textbf{Option (C)}
\end{answer}
\item A system of $N$ non-interacting and distinguishable particle of spin 1 is in thermodynamic equilibrium. The entropy of the system is
{\exyear{GATE 2011}}
\begin{tasks}(4)
\task[\textbf{A.}] $2 k_{B} \ln N$
\task[\textbf{B.}] $3 k_{B} \ln N$
\task[\textbf{C.}] $N k_{B} \ln 2$
\task[\textbf{D.}] $N k_{B} \ln 3$
\end{tasks}
\begin{answer}
\begin{align*}
\mathrm{S}=\mathrm{k}_{\mathrm{B}} \sum_{\mathrm{i}} \ln \Omega, \Omega&=3 \text{is number of microstate. }\mathrm{S}=1 ; \quad \mathrm{S}_{\mathrm{z}}=-1, \quad 0, \quad 1\\
\text{	The entropy of the system is }& N k_{B} \ln 3
\end{align*}
So the correct answer is \textbf{Option (D)}
\end{answer}	
\item A system has two energy levels with energies $\varepsilon$ and $2 \varepsilon .$ The lower level is 4 -fold degenerate while the upper level is doubly degenerate. If there are $N$ non-interacting classical particles in the system, which is in thermodynamic equilibrium at a temperature $T$, the fraction of particles in the upper level is
{\exyear{GATE 2011}}
\begin{tasks}(4)
\task[\textbf{A.}] $\frac{1}{1+e^{\varepsilon / k_{B} T}}$
\task[\textbf{B.}] $\frac{1}{1+2 e^{\varepsilon / k_{B} T}}$
\task[\textbf{C.}] $\frac{1}{2 e^{\varepsilon / k_{B} T}+4 e^{2 \varepsilon / k_{B} T}}$
\task[\textbf{D.}] $\frac{1}{2 e^{\varepsilon / k_{B} T}-4 e^{2 \varepsilon / k_{B} T}}$
\end{tasks}
\begin{answer}
\begin{align*}
\text{Partition function }Z&=4 e^{-\epsilon / k T}+2 e^{-\epsilon / k T} \Rightarrow P(2 \varepsilon)\\&=\frac{2 e^{-2 \in / k T}}{4 e^{-\epsilon / k T}+2 e^{-2 \varepsilon / k T}}=\frac{1}{1+2 e^{\epsilon / k T}}
\end{align*}
So the correct answer is \textbf{Option (B)}
\end{answer}	
\item Consider a system whose three energy levels are given by $0, \varepsilon$ and $2 \varepsilon$. The energy level $\varepsilon$ is two-fold degenerate and the other two are non-degenerate. The partition function of the system with $\beta=\frac{1}{k_{B} T}$ is given by
{\exyear{GATE 2012}}
\begin{tasks}(4)
\task[\textbf{A.}] $1+2 e^{-\beta \varepsilon}$
\task[\textbf{B.}] $2 e^{-\beta \varepsilon}+e^{-2 \beta \varepsilon}$
\task[\textbf{C.}] $\left(1+e^{-\beta \varepsilon}\right)^{2}$
\task[\textbf{D.}] $1+e^{-\beta \varepsilon}+e^{-2 \beta \varepsilon}$
\end{tasks}
\begin{answer}
\begin{align*}
E_{1}&=0, E_{2}=\varepsilon, E_{3}=2 \varepsilon ; g_{1}=1, g_{2}=2, g_{3}=1\\&\text{ where} g_{1}, g_{2}\text{ and }g_{3}\text{ are degeneracy}\\
\text{	The partition function }Z&=g_{1} e^{-\beta \cdot E_{1}}+g_{2} e^{-\beta \cdot E_{2}}+g_{3} e^{-\beta \cdot E_{3}}\\&=1+2 e^{-\beta c}+e^{-\beta 2 \varepsilon}=\left(1+e^{-\beta c}\right)^{2}
\end{align*}
\end{answer}	
\item Consider a linear collection of $N$ independent spin $1 / 2$ particles, each at a fixed location. The entropy of this system is $(k$ is the Boltzmann constant)
{\exyear{GATE 2013}}
\begin{tasks}(4)
\task[\textbf{A.}] Zero
\task[\textbf{B.}]  $N k$
\task[\textbf{C.}]  $\frac{1}{2} N k$
\task[\textbf{D.}] $N k \ln (2)$
\end{tasks}
\begin{answer}
There are two microstates possible for spin $\frac{1}{2}$ particle, so entropy is given by $N k \ln (2)$\\\\
So the correct answer is \textbf{Option (D)}
\end{answer}	
\item Consider a system of $N$ non-interacting spin $-\frac{1}{2}$ particles, each having a magnetic moment $\mu$, is in a magnetic field $\vec{B}=B \hat{z} .$ If $E$ is the total energy of the system, then number of accessible microstates $\Omega$ is given by
{\exyear{GATE 2015}}
\begin{tasks}(2)
\task[\textbf{A.}] $\Omega=\frac{N !}{\frac{1}{2}\left(N-\frac{E}{\mu B}\right) ! \frac{1}{2}\left(N+\frac{E}{\mu B}\right) !}$
\task[\textbf{B.}] $\Omega=\frac{\left(N-\frac{E}{\mu B}\right) !}{\left(N+\frac{E}{\mu B}\right) !}$
\task[\textbf{C.}] $\Omega=\frac{1}{2}\left(N-\frac{E}{\mu B}\right) ! \frac{1}{2}\left(N+\frac{E}{\mu B}\right) !$
\task[\textbf{D.}] $\Omega=\frac{N !}{\left(N+\frac{E}{\mu B}\right) !}$
\end{tasks}
\begin{answer}
\begin{align*}
\intertext{Number of microstate is ${ }^{N} C_{n_{1}}$, where $n_{1}$ is number of particle in $+\frac{1}{2}$ state and}
n_{2}&=\left(N-n_{1}\right)\text{ is number of state in }-\frac{1}{2}\text{ state.}\\
\text{	where }n_{1}&=\frac{1}{2}\left(N-\frac{E}{\mu B}\right), n_{2}=\frac{1}{2}\left(N+\frac{E}{\mu B}\right)\\
\text{So, number of microstate}=&\frac{\lfloor N}{\frac{1}{2}\left(N-\frac{E}{\mu B}\right) \mid \frac{1}{2}\left(N+\frac{E}{\mu B}\right)}
\end{align*}
\end{answer}	
\item The average energy $U$ of a one dimensional quantum oscillator of frequency $\omega$ and in contact with a heat bath at temperature $T$ is given by
{\exyear{GATE 2015}}
\begin{tasks}(2)
\task[\textbf{A.}] $U=\frac{1}{2} \hbar \omega \operatorname{coth}\left(\frac{1}{2} \beta \hbar \omega\right)$
\task[\textbf{B.}] $U=\frac{1}{2} \hbar \omega \sinh \left(\frac{1}{2} \beta \hbar \omega\right)$
\task[\textbf{C.}] $U=\frac{1}{2} \hbar \omega \tanh \left(\frac{1}{2} \beta \hbar \omega\right)$
\task[\textbf{D.}] $U=\frac{1}{2} \hbar \omega \cosh \left(\frac{1}{2} \beta \hbar \omega\right)$
\end{tasks}
\begin{answer}
\begin{align*}
\because Z&=\sum e^{-\beta E_{i}}=\sum_{i=0}^{\infty} e^{-\beta\left(n+\frac{1}{2}\right) \hbar \omega}\text{ where }E\\&=\left(n+\frac{1}{2}\right) \hbar \omega \Rightarrow Z=\frac{1}{2 \sinh \left(\frac{\beta \hbar \omega}{2}\right)}\\
\because U&=\frac{-\partial}{\partial \beta} \ln Z=-\frac{\partial}{\partial \beta} \ln \left[\frac{1}{2 \sinh \left(\frac{\beta \hbar \omega}{2}\right)}\right]\\&=\frac{\hbar \omega}{2} \operatorname{coth}\left(\frac{\beta \hbar \omega}{2}\right)
\end{align*}
So the correct answer is \textbf{Option (A)}
\end{answer}	
\item  The entropy of a gas containing $N$ particles enclosed in a volume $V$ is given by $S=N k_{B} \ln \left(\frac{a V E^{3 / 2}}{N^{5 / 2}}\right)$, where $E$ is the total energy, $a$ is a constant and $k_{B}$ is the Boltzmann constant. The chemical potential $\mu$ of the system at a temperature $T$ is given by
{\exyear{GATE 2015}}
\begin{tasks}(2)
\task[\textbf{A.}] $\mu=-k_{B} T\left[\ln \left(\frac{a V E^{3 / 2}}{N^{5 / 2}}\right)-\frac{5}{2}\right]$
\task[\textbf{B.}] $\mu=-k_{B} T\left[\ln \left(\frac{a V E^{3 / 2}}{N^{5 / 2}}\right)-\frac{3}{2}\right]$
\task[\textbf{C.}] $\mu=-k_{B} T\left[\ln \left(\frac{a V E^{3 / 2}}{N^{3 / 2}}\right)-\frac{5}{2}\right]$
\task[\textbf{D.}] $\mu=-k_{B} T\left[\ln \left(\frac{a V E^{3 / 2}}{N^{3 / 2}}\right)-\frac{3}{2}\right]$
\end{tasks}
\begin{answer}
\begin{align*}
\left(\frac{\partial G}{\partial T}\right)_{P}&=-S=-N k_{B} \ln \left(\frac{a V E^{3 / 2}}{N^{5 / 2}}\right)\\ \because S&=N k_{B} \ln \left(\frac{a V E^{3 / 2}}{N^{5 / 2}}\right)\\
\Rightarrow G&=-N k_{B} T \ln \left(\frac{a V E^{3 / 2}}{N^{5 / 2}}\right)+\ln A\\
\Rightarrow \mu&=\left(\frac{\partial G}{\partial N}\right)=-\left[k_{B} T \ln \left(\frac{a V E^{3 / 2}}{N^{5 / 2}}\right)+N k_{B} T \frac{N^{5 / 2}}{a V E^{3 / 2}} \cdot \frac{(-5 / 2)}{N^{7 / 2}} a V E^{3 / 2}\right]\\&=-k_{B} T\left[\ln \left(\frac{a V E^{3 / 2}}{N^{\frac{5}{2}}}\right)-\frac{5}{2}\right]
\end{align*}
So the correct answer is \textbf{Option (A)}
\end{answer}	
\item entropy $S$ of a system of $N$ spins, which may align either in the upward or in the downward direction, is given by $S=-k_{B} N[p \ln p+(1-p) \operatorname{In}(1-p)] .$ Here $k_{B}$ is the Boltzmann constant. The probability of alignment in the upward direction is $p$. The value of $p$, at which the entropy is maximum, is---------- (Give your answer upto one decimal place)
{\exyear{GATE 2016}}
\begin{answer}
\begin{align*}
S&=-k_{B} N[p \ln p+(1-p) \operatorname{In}(1-p)]\\
\text{	For maximum entropy, }\frac{d S}{d p}&=0 \Rightarrow \ln p+p \times \frac{1}{p}-\ln (1-p)+(1-p) \times \frac{1}{1-p}(-1)=0\\
\ln p+1-\ln (1-p)-1&=0 \Rightarrow \ln \left(\frac{p}{1-p}\right)\\&=0 \Rightarrow p=1-p \Rightarrow p=0.5
\end{align*}
\end{answer}
\item 	$N$ atoms of an ideal gas are enclosed in a container of volume $V$. The volume of the container is changed to $4 V$, while keeping the total energy constant. The change in the entropy of the gas, in units of $N k_{B} \ln 2$, is------ where $k_{B}$ is the Boltzmann constant.
{\exyear{GATE 2016}}
\begin{answer}
\begin{align*}
S_{1}&=-N k_{B} \ln 1, S_{2}=-N k_{B} \ln \frac{1}{4} \Rightarrow \Delta S\\&=S_{2}-S_{1}=N k_{B} \ln 4=2 N k_{B} \ln 2
\end{align*}
\end{answer}
\item A two-level system has energies zero and $E$. The level with zero energy is nondegenerate, while the level with energy $E$ is triply degenerate. The mean energy of a classical particle in this system at a temperature $T$ is
{\exyear{GATE 2016}}
\begin{tasks}(4)
\task[\textbf{A.}] (a) $\frac{E e^{\frac{-E}{k_{B} T}}}{1+3 e^{\frac{-E}{k_{B} T}}}$
\task[\textbf{B.}] $\frac{E e^{\frac{-E}{k_{B} T}}}{1+e^{\frac{-E}{k_{s} T}}}$
\task[\textbf{C.}] $\frac{3 E e^{\frac{-E}{k_{B} T}}}{1+e^{\frac{-E}{k_{B} T}}}$
\task[\textbf{D.}] $\frac{3 E e^{\frac{-E}{k_{B} T}}}{1+3 e^{\frac{-E}{k_{s} T}}}$
\end{tasks}
\begin{answer}
\begin{align*}
\langle E\rangle&=\frac{\sum_{i} g_{i} E_{i} e^{-\frac{E_{i}}{k T}}}{\sum_{i} g_{i} e^{-\frac{E_{i}}{k T}}}=\frac{0 \times e^{-\frac{0}{k T}}+3 \times E \times e^{-\frac{E}{k T}}}{e^{-\frac{0}{k T}}+3 \times e^{-\frac{E}{k T}}}=\frac{3 E e^{\frac{-E}{k_{B} T}}}{1+3 e^{\frac{-E}{k_{B} T}}}
\end{align*}
So the correct answer is \textbf{Option (D)}
\end{answer}	
\item Consider a triatomic molecule of the shape shown in the figure in three dimensions. The heat capacity of this molecule at high temperature (temperature much higher than the vibrational and rotational energy scales of the molecule but lower than its bond dissociation energies) is:
{\exyear{GATE 2017}}
\begin{figure}[H]
\centering
\includegraphics[height=3cm,width=4.5cm]{diagram-20210911(13)-crop}
\end{figure}
\begin{tasks}(4)
\task[\textbf{A.}] $\frac{3}{2} k_{B}$
\task[\textbf{B.}] $3 k_{B}$
\task[\textbf{C.}] $\frac{9}{2} k_{B}$
\task[\textbf{D.}] $6 k_{B}$
\end{tasks}
\begin{answer}
If given molecules are at lower temperature i.e. atoms are attached to rigid rod then degree of freedom is 6 , so internal energy is $\frac{6 k_{B} T}{2}$, but at high temperature, vibration mode will active, so there are three extra vibration mode will active, so total energy
\begin{align*}
U&=3 k_{B} T+3 k_{B} T=6 k_{B} T\\
C_{V}&=\left(\frac{\partial U}{\partial T}\right)_{V}=6 k_{B}
\end{align*}
So the correct answer is \textbf{Option (D)}
\end{answer}	
\item Consider $N$ non- interacting, distinguishable particles in a two-level system at temperature $T$. The energies of the levels are 0 and $\varepsilon$, where $\varepsilon>0$. In the high temperature limit $\left(k_{B} T>\varepsilon\right)$, what is the population of particles in the level with energy $\varepsilon ?$
{\exyear{GATE 2017}}
\begin{tasks}(4)
\task[\textbf{A.}] $\frac{N}{2}$
\task[\textbf{B.}] $N$
\task[\textbf{C.}] $\frac{N}{4}$
\task[\textbf{D.}] $\frac{3 N}{4}$
\end{tasks}
\begin{answer}
\begin{align*}
P(\varepsilon)&=\frac{\exp -\frac{\varepsilon}{k T}}{1+\exp -\frac{\varepsilon}{k T}},\text{ population of particle in the level with energy $\varepsilon$ is}\\
N P(\varepsilon)&=N \frac{\exp -\frac{\varepsilon}{k T}}{1+\exp -\frac{\varepsilon}{k T}},\text{ for }\left(k_{B} T>\varepsilon\right), N P(\varepsilon)\\&=N \frac{\exp -\frac{\varepsilon}{k T}}{1+\exp -\frac{\varepsilon}{k T}}=N \frac{1}{1+1}=\frac{N}{2}
\end{align*}
So the correct answer is \textbf{Option (A)}
\end{answer}	
\item  A microcanonical ensemble consists of 12 atoms with each taking either energy 0 state, or energy $\in$ state. Both states are non-degenerate. If the total energy of this ensemble is $4 \in$, its entropy will be----------- $k_{B}$ (up to one decimal place), where $k_{B}$ is the Boltzmann constant.
{\exyear{GATE 2018}}
\begin{answer}
\begin{align*}
\intertext{The number of ways having total energy $4 \in$, out of 12 atom is}
&={ }^{12} C_{4}=\frac{12}{\lfloor 48}=\frac{12 \times 11 \times 10 \times 9}{4 \times 3 \times 2}=495\\
\text{Hence, entropy, }S&=k_{B} \ln w=k_{B} \ln (495)=k_{B}(6.204)=6.204 k_{B}
\end{align*}
\end{answer}	
\item  The partition function of an ensemble at a temperature $T$ is
$$
Z=\left(2 \cosh \frac{\varepsilon}{k_{B} T}\right)^{N}
$$
where $k_{B}$ is the Boltzmann constant. The heat capacity of this ensemble at $T=\frac{\varepsilon}{k_{B}}$ is $X N k_{B}$, where the value of $X$ is (up to two decimal places).
{\exyear{GATE 2018}}
\begin{answer}
\begin{align*}
\text{	The partition function, }z&=\left[2 \cosh \left(\frac{\varepsilon}{k_{B} T}\right)\right]^{N}\\
\text{The average energy, }\langle E\rangle&=k_{B} T^{2} \frac{\partial(\ln z)}{\partial T}\\
&=\frac{N k_{B} T^{2}\left[2 \sinh \left(\frac{\varepsilon}{k_{B} T}\right)\right]\left(\frac{-\varepsilon}{k_{B} T^{2}}\right)}{2 \cosh \left(\frac{\varepsilon}{k_{B} T}\right)}\\&=-N \varepsilon \tanh \left(\frac{\varepsilon}{k_{B} T}\right)\\
C&=\frac{d\langle E\rangle}{d T}=-N \varepsilon \sec h^{2}\left(\frac{\varepsilon}{k_{B} T}\right) \cdot\left(\frac{-\varepsilon}{k_{B} T^{2}}\right)\\
\text{At }T&=\frac{\varepsilon}{k}, C=\frac{N \varepsilon^{2}}{k \cdot\left(\varepsilon^{2} / k^{2}\right)} \sec h^{2}(1)\\&=N k \sec h^{2}(1)=0.42 N k_{B}
\end{align*}
\end{answer}	
\item A collection of $N$ two-level systems with energies 0 and $E>0$ is in thermal
equilibrium at temperature $T$. For $T \rightarrow \infty$, the specific heat approaches to,
{\exyear{JEST 2012}}
\begin{tasks}(4)
\task[\textbf{A.}] 0
\task[\textbf{B.}] $N k_{B}$
\task[\textbf{C.}] $\frac{3 N k_{B}}{2}$
\task[\textbf{D.}] $\infty$
\end{tasks}
\begin{answer}
\begin{align*}
Z&=\sum e^{-\beta E_{i}}=e^{-\beta \times 0}+e^{-\beta E_{i}} \Rightarrow Z\\&=1+e^{-\beta E} \Rightarrow \ln z\\&=\ln \left(1+e^{-\beta E}\right)\\
U&=\langle E\rangle=-\frac{\partial}{\partial \beta} \ln z\\&=-\frac{\partial}{\partial \beta} \ln \left(1+e^{-\beta E}\right)\\&=-\frac{1}{1+e^{-\beta E}} \times e^{-\beta E}(-E)\\&=\frac{E e^{-\beta E}}{1+e^{-\beta E}}\\
\text{Now},\left(\frac{\partial U}{\partial T}\right)_{V}&=C_{V}\\
&=\frac{\partial}{\partial T}\left(\frac{E e^{-\frac{E}{k T}}}{1+e^{-\frac{E}{k T}}}\right)\\
\Rightarrow C_{V}&=\frac{\left(\frac{E^{2}}{k T^{2}} e^{\frac{-E}{k T}}+\frac{E^{2}}{k T^{2}} e^{\frac{-2 E}{k T}}-\frac{E^{2}}{k T^{2}} e^{\frac{-2 E}{k T}}\right)}{\left(1+e^{\frac{-E}{k T}}\right)^{2}} \Rightarrow C_{V}\\
&=\left.\frac{\frac{E^{2}}{k T^{2}} e^{\frac{-E}{k T}}}{\left(1+e^{\frac{-E}{k T}}\right)^{2}} \Rightarrow C_{V}\right|_{T \rightarrow \infty}=0
\end{align*}
So the correct answer is \textbf{Option (A)}
\end{answer}	
\item  A monoatomic gas consists of atoms with two internal energy levels, ground state $E_{0}=0$ and an excited state $E_{1}=E$. The specific heat of the gas is given by
{\exyear{JEST 2014}}
\begin{tasks}(2)
\task[\textbf{A.}] $\frac{3}{2} k$
\task[\textbf{B.}] $\frac{E^{2} e^{E / k T}}{k T^{2}\left(1+e^{E / k T}\right)^{2}}$
\task[\textbf{C.}] $\frac{3}{2} k+\frac{E^{2} e^{E / k T}}{k T^{2}\left(1+e^{E / k T}\right)^{2}}$
\task[\textbf{D.}] $\frac{3}{2} k-\frac{E^{2} e^{E / k T}}{k T^{2}\left(1+e^{E / k T}\right)^{2}}$
\end{tasks}
\begin{answer}
\begin{align*}
E_{0}&=0, \quad E_{1}=E\\
\text{Then partition function is}\\
z&=\sum e^{-\beta E_{i}} \Rightarrow z\\&=e^{-\beta \times 0}+e^{-\beta E} \Rightarrow \ln z\\&=\ln \left(1+e^{-\beta E_{1}}\right)\\
U&=\langle E\rangle=\frac{-\partial}{\partial \beta} \ln z\\&=-\frac{\partial}{\partial \beta} \ln \left(1+e^{-\beta E}\right)\\&=-\frac{1}{\left(1+e^{-\beta E}\right)}(-E) e^{-\beta E}\\&=\frac{E e^{-\beta E}}{1+e^{-\beta E}} \quad\left[\because \beta=k_{B} T\right]\\
\left(\frac{\partial U}{\partial T}\right)_{v}=C_{V}&=\frac{\left(1+e^{-\frac{E}{k_{B} T}}\right) E . e^{-\frac{E}{k_{B} T}} \cdot\left(\frac{E}{k_{B} T^{2}}\right)-E e^{-\frac{E}{k_{B} T}} \cdot e^{-\frac{E}{k_{B} T}}\left(\frac{E}{k_{B} T^{2}}\right)}{\left(1+e^{-\frac{E}{k_{B} T}}\right)^{2}}\\
C_{V}&=\frac{\frac{E^{2}}{k_{B} T^{2}} e^{-\frac{E}{k_{\mathrm{B}} T}}+\frac{E^{2}}{k_{B} T^{2}} e^{-\frac{2 E}{k_{\mathrm{B}} T}}-\frac{E^{2}}{k_{B} T^{2}} e^{-\frac{2 E}{k_{\mathrm{B}} T}}}{\left(1+e^{-\frac{E}{k_{\mathrm{B}} T}}\right)^{2}}\\&=\frac{E^{2} e^{-\frac{E}{k_{\mathrm{B}} T}}}{k_{B} T^{2}\left(1+e^{-\frac{E}{k_{\mathrm{B}} T}}\right)^{2}}\\&=\frac{E^{2} e^{\frac{E}{k_{\mathrm{B}} T}}}{k_{B} T^{2}\left(1+e^{\frac{E}{k_{\mathrm{B}} T}}\right)^{2}}\\
\text{If gas will classically allowed, then }C_{V}&=\frac{3}{2} k_{B}\\
\text{	and quantum mechanically, }C_{V}&=\frac{E^{2} e^{\frac{E}{k_{B} T}}}{k_{B} T^{2}\left(1+e^{\frac{E}{k_{B} T}}\right)^{2}}\\
\therefore \quad C_{V}&=\frac{3}{2} k_{B}+\frac{E^{2} e^{E / k T}}{k T^{2}\left(1+e^{E / k T}\right)^{2}}
\end{align*}
So the correct answer is \textbf{Option (C)}
\end{answer}	
\item Consider a system of $2 N$ non-interacting spin $1 / 2$ particles each fixed in position and carrying a magnetic moment $\mu$. The system is immersed in a uniform magnetic field $B$. The number of spin up particles for which the entropy of the system will be maximum is
{\exyear{JEST 2014}}
\begin{tasks}(4)
\task[\textbf{A.}]  0
\task[\textbf{B.}] $N$
\task[\textbf{C.}] $2 N$
\task[\textbf{D.}] $N / 2$
\end{tasks}
\begin{answer}
\begin{align*}
\intertext{Let us consider $n$ number of spin out of $2 N$ particle have spin up remaining $2 N-n$ is down.}
\text{Number of ways, }\omega&=\left\{\begin{array}{ll}2^{N} C_{n} & \text { for spin } \frac{1}{2}(\text { up }) \\ 2^{N} C_{2 N-n} & \text { for spin } \frac{1}{2}(\text { down })\end{array}\right.\\
\text{Entropy, }S&=k \ln \omega \Rightarrow S\\&=k \ln { }^{2 N} C_{2 N-n}+k \ln { }^{2 N} C_{n}\\
S&=k\left\{\left[\ln \frac{2 N !}{(n !)(2 N-n) !}\right]+\left[\ln \frac{2 N !}{(n !)(2 N-n) !}\right]\right\}\\
S&=2 k[(\ln 2 N !-\ln n !-\ln (2 N-n) !)]\\
S&=2 k[2 N \ln 2 N-2 N-n \ln n+n-\{(2 N-n) \ln (2 N-n)-(2 N-n)\}]\\
[\because \ln N !&=N \ln N-N !]\\
S&=2 k[2 N \ln 2 N-2 N-n \ln n+n-2 N \ln (2 N-n)+n \ln (2 N-n)+(2 N-n)]\\
S&=2 k[2 N \ln 2 N-n \ln n-2 N \ln (2 N-n)+n \ln (2 N-n)]
\intertext{Now for maximum entropy at equilibrium for spin $\frac{1}{2}$ up particle,}
\frac{d S}{d n}&=0\\
\frac{d S}{d n}&=2 k\left[-\frac{n}{n} \cdot 1-\ln n-\frac{2 N}{2 N-n}(-1)+\frac{n}{2 N-n}(-1)+\ln (2 N-n)\right]\\
&=2 k\left[-1-\ln n+\frac{2 N}{2 N-n}-\frac{n}{2 N-n}+\ln (2 N-n)\right]\\
&=2 k\left[-1+\frac{2 N-n}{2 N-n}+\ln (2 N-n)-\ln n\right]\\& \Rightarrow 2 k\left[-1+1+\ln \frac{(2 N-n)}{n}\right]=0\\
\because \quad 2 k &\neq 0\\
\therefore \ln \left(\frac{2 N-n}{n}\right)&=0 \Rightarrow \frac{2 N-n}{n}\\&=1 \Rightarrow 2 N=2 n \\
\Rightarrow n&=N
\end{align*}
So the correct answer is \textbf{Option (B)}
\end{answer}	
\item For a system in thermal equilibrium with a heat bath at temperature $T$, which one of the following equalities is correct? $\left(\beta=\frac{1}{k_{B} T}\right)$
{\exyear{JEST 2015}}
\begin{tasks}(2)
\task[\textbf{A.}] $\frac{\partial}{\partial \beta}\langle E\rangle=\langle E\rangle^{2}-\left\langle E^{2}\right\rangle$
\task[\textbf{B.}] $\frac{\partial}{\partial \beta}\langle E\rangle=\left\langle E^{2}\right\rangle-\langle E\rangle^{2}$
\task[\textbf{C.}] $\frac{\partial}{\partial \beta}\langle E\rangle=\left\langle E^{2}\right\rangle+\langle E\rangle^{2}$
\task[\textbf{D.}] $\frac{\partial}{\partial \beta}\langle E\rangle=-\left(\left\langle E^{2}\right\rangle+\langle E\rangle^{2}\right)$
\end{tasks}
\begin{answer}
\begin{align*}
\because\langle E\rangle&=\frac{\sum_{i} E_{i} e^{-\beta E_{i}}}{\sum_{i} e^{-\beta E_{i}}}\\
\frac{\partial\langle E\rangle}{\partial \beta}&=-\frac{\sum_{i} E_{i}^{2} e^{-\beta E_{i}}}{\sum_{i} e^{-\beta E_{i}}}+\frac{\sum_{i} E_{i}^{2} e^{-\beta E_{i}} \cdot e^{-\beta E_{i}}}{\left(\sum_{i} e^{-\beta E_{i}}\right)^{2}}\\&=-\frac{\sum_{i} E_{i}^{2} e^{-\beta E_{i}}}{\sum_{i} e^{-\beta E_{i}}}+\frac{\sum_{i} E_{i}^{2} e^{-2 \beta E_{i}}}{\left(\sum_{i} e^{-\beta E_{i}}\right)^{2}}\\
\Rightarrow \frac{\partial\langle E\rangle}{\partial \beta}&=\langle E\rangle^{2}-\left\langle E^{2}\right\rangle
\end{align*}
So the correct answer is \textbf{Option (A)}
\end{answer}	
\item A particle in thermal equilibrium has only 3 possible states with energies $-\in, 0, \in .$ If the system is maintained at a temperature, $T>>\frac{\epsilon}{k_{B}}$, then the average energy of the particle can be approximated to,
{\exyear{JEST 2015}}
\begin{tasks}(4)
\task[\textbf{A.}] $\frac{2 \in^{2}}{3 k_{B} T}$
\task[\textbf{B.}] $\frac{-2 \in^{2}}{3 k_{B} T}$
\task[\textbf{C.}] $\frac{-\epsilon^{2}}{k_{B} T}$
\task[\textbf{D.}]  0
\end{tasks}
\begin{answer}
\begin{align*}
\langle E\rangle&=\frac{-\in e^{+\frac{\epsilon}{k T}}+0+\in e^{-\frac{\epsilon}{k T}}}{e^{\frac{\epsilon}{k T}}+1+e^{-\frac{\epsilon}{k T}}}\\&=\epsilon\left(\frac{e^{-\frac{\epsilon}{k T}}-e^{\frac{\epsilon}{k T}}}{1+e^{-\frac{\epsilon}{k T}}+e^{\frac{\epsilon}{k T}}}\right)\\
\Rightarrow\langle E\rangle&=\frac{\left[\left(1-\frac{\epsilon}{k T}\right)-\left(1+\frac{\epsilon}{k T}\right)\right]}{1+\left(1-\frac{\epsilon}{k T}\right)+\left(1+\frac{\epsilon}{k T}\right)}\\&=\frac{-2 \epsilon^{2}}{3 k T}
\end{align*}
So the correct answer is \textbf{Option (B)}
\end{answer}	
\item A collection of $N$ interacting magnetic moments, each of magnitude $\mu$, is subjected to a magnetic field $\mathrm{H}$ along the $\mathrm{z}$ direction. Each magnetic moment has a doubly degenerate level of energy zero and two non-degenerate levels of energies $-\mu H$ and $\mu H$ respectively.The collection is in thermal equilibrium at temperature $T$. The total energy $E(T, H)$ of the collection is
{\exyear{JEST 2018}}
\begin{tasks}(2)
\task[\textbf{A.}] $-\frac{\mu H N \sinh \left(\frac{\mu H}{k_{B} T}\right)}{1+\cosh \left(\frac{\mu H}{k_{b} T}\right)}$
\task[\textbf{B.}] $-\frac{\mu H N}{2\left(1+\cosh \left(\frac{\mu H}{k_{b} T}\right)\right)}$
\task[\textbf{C.}] $-\frac{\mu H N \cosh \left(\frac{\mu H}{k_{B} T}\right)}{1+\cosh \left(\frac{\mu H}{k_{b} T}\right)}$
\task[\textbf{D.}] $-\mu H N \frac{\sinh \left(\frac{\mu H}{k_{B} T}\right)}{\cosh \left(\frac{\mu H}{k_{b} T}\right)}$
\end{tasks}
\begin{answer}
\begin{align*}
Z_{1}&=\left(2 \times \exp \left(-\frac{0}{k_{B} T}\right)+\exp \left(-\frac{-\mu H}{k_{B} T}\right)+\exp \left(-\frac{\mu H}{k_{B} T}\right)\right) \\\Rightarrow Z_{1}&=\left(2+2 \cosh \frac{\mu H}{k_{B} T}\right)\\
Z_{N}&=\left(2+2 \cosh \frac{\mu H}{k_{B} T}\right)^{N}\\U&=k_{B} T^{2}\left(\frac{\partial \ln Z_{N}}{\partial T}\right)_{N, V}\\&=-\frac{N \mu H \sinh \left(\frac{\mu H}{k_{B} T}\right)}{1+\cos \frac{\mu H}{k_{B} T}}
\end{align*}
So the correct answer is \textbf{Option (A)}
\end{answer}
\item 	Consider a system of $N$ distinguishable particles with two energy levels for each particle, a ground state with energy zero and an excited state with energy $\varepsilon>0$. What is the average energy per particle as the system temperature $T \rightarrow \infty$ ?
{\exyear{JEST 2019}}
\begin{tasks}(4)
\task[\textbf{A.}] 0
\task[\textbf{B.}]  $\frac{\varepsilon}{2}$
\task[\textbf{C.}] $\varepsilon$
\task[\textbf{D.}] $\infty$
\end{tasks}
\begin{answer}
\begin{align*}
\langle E\rangle&=\sum_{i} P_{i} E_{i} \Rightarrow P_{i}=\frac{e^{\beta E_{i}}}{z}\\
\langle E\rangle&=0 \times \frac{01}{1+e^{-\beta \varepsilon}}+\varepsilon \times \frac{1}{1+e^{-\beta \varepsilon}}\\
&=\frac{\varepsilon}{1+e^{-\varepsilon / k_{B} T}}=\frac{\varepsilon}{2}\text{`}T \rightarrow \infty
\end{align*}
So the correct answer is \textbf{Option (B)}
\end{answer}
\item Consider a diatomic molecule with an infinite number of equally spaced non-degenerate energy levels. The spacing between any two adjacent levels is $\varepsilon$ and the ground state energy is zero. What is the single particle partition function $Z$ ?
{\exyear{JEST 2019}}
\begin{tasks}(2)
\task[\textbf{A.}] $Z=\frac{1}{1-\frac{\varepsilon}{k_{B} T}}$
\task[\textbf{B.}] $Z=\frac{1}{1-e^{\frac{\varepsilon}{k_{\mathrm{B}} T}}}$
\task[\textbf{C.}] $Z=\frac{1}{1-e^{\frac{2 \varepsilon}{k_{B} T}}}$
\task[\textbf{D.}] $Z=\frac{1-\frac{\varepsilon}{k_{B} T}}{1+\frac{\varepsilon}{k_{B} T}}$
\end{tasks}
\begin{answer}
\begin{align*}
Z&=\sum_{i} g_{i} e^{-\beta \varepsilon_{i}}\\
g_{i}&=1\\
Z&=1+e^{-\beta \varepsilon}+e^{-2 \beta \varepsilon}+\ldots \ldots\\
Z&=\frac{1}{1-e^{-\beta \varepsilon}}
\end{align*}
No option is matched
\end{answer}	
	
	
	
	
	
	
	
\end{enumerate}