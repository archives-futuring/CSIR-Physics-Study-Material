\chapter{SM Second-Solutions}
\begin{abox}
	Practise set-1
\end{abox}
\begin{enumerate}
	\item Consider an ideal Bose gas in three dimensions with the energy-momentum relation $\varepsilon \propto p^{s}$ with $s>0$. The range of $s$ for which this system may undergo a Bose-Einstein condensation at a non-zero temperature is
	{\exyear{NET/JRF(JUNE-2011)}}
	\begin{tasks}(4)
		\task[\textbf{a.}] $1<s<3$
		\task[\textbf{b.}] $0<s<2$
		\task[\textbf{c.}] $0<s<3$
		\task[\textbf{d.}] $0<s<\infty$
	\end{tasks}
\begin{answer}
	So the correct answer is \textbf{Option (a)}
\end{answer}
	\item The internal energy $E$ of a system is given by $E=\frac{b S^{3}}{V N}$, where $b$ is a constant and other symbols have their usual meaning. The temperature of this system is equal to
	{\exyear{NET/JRF(JUNE-2011)}}
	\begin{tasks}(4)
		\task[\textbf{a.}] $\frac{b S^{2}}{V N}$
		\task[\textbf{b.}] $\frac{3 b S^{2}}{V N}$
		\task[\textbf{c.}]  $\frac{b S^{3}}{V^{2} N}$
		\task[\textbf{d.}] $\left(\frac{S}{N}\right)^{2}$
	\end{tasks}
\begin{answer}
	\begin{align*}
	T d S&=d E+P d V \Rightarrow d E\\&=T d S-P d V \Rightarrow\left(\frac{\partial E}{\partial S}\right)_{V}=T \Rightarrow T=\frac{3 b S^{2}}{V N}
	\end{align*}
	So the correct answer is \textbf{Option (b)}
\end{answer}
	\item Consider a Maxwellian distribution of the velocity of the molecules of an ideal gas. Let $V_{m p}$ and $V_{r m s}$ denote the most probable velocity and the root mean square velocity, respectively. The magnitude of the ratio $V_{m p} / V_{r m s}$ is
	{\exyear{NET/JRF(DEC-2011)}}
	\begin{tasks}(4)
		\task[\textbf{a.}] (a) 1
		\task[\textbf{b.}] $2 / 3$
		\task[\textbf{c.}] $\sqrt{2 / 3}$
		\task[\textbf{d.}] $3 / 2$
	\end{tasks}
\begin{answer}
	\begin{align*}
	\text{For Maxwellian distribution }\mathrm{V}_{\mathrm{mp}}&=\sqrt{\frac{2 \mathrm{kT}}{\mathrm{m}}}, V_{\text {rms }}\\&=\sqrt{\frac{3 k T}{m}} \Rightarrow \frac{V_{m b}}{V_{\text {rms }}}=\sqrt{\frac{2}{3}}
	\end{align*}
	So the correct answer is \textbf{Option (c)}	
\end{answer}
	\item If the number density of a free electron gas in three dimensions is increased eight times, its Fermi temperature will
	{\exyear{NET/JRF(DEC-2011)}}
	\begin{tasks}(2)
		\task[\textbf{a.}] Increase by a factor of 4
		\task[\textbf{b.}] Decrease by a factor of 4
		\task[\textbf{c.}] Increase by a factor of 8
		\task[\textbf{d.}] Decrease by a factor of 8
	\end{tasks}
\begin{answer}
	\begin{align*}
	\text{Fermi energy }E_{F}&=\left(\frac{3 N}{4 \pi V g}\right)^{\frac{2}{3}} \frac{\hbar^{2}}{2 m},\\\text{ where }&\frac{N}{V}\text{ is number density and $g$ is degeneracy}\\
	E_{F} \propto T_{F} K &\Rightarrow T_{F} \propto\left(\frac{n}{V}\right)^{\frac{2}{3}} \Rightarrow T_{F} \propto(n)^{\frac{2}{3}} \Rightarrow \frac{T_{F_{1}}}{T_{F_{2}}}\\&=\left(\frac{n_{1}}{n_{2}}\right)^{\frac{2}{3}}=4\text{ since } \frac{n_{1}}{n_{2}}=8
	\end{align*}
	So the correct answer is \textbf{Option (a)}
\end{answer}
	\item A one-dimensional chain consists of a set of $N$ rods each of length $a$. When stretched by a load, each rod can align either parallel or perpendicular to the length of the chain. The energy of a rod is $-\varepsilon$ when perpendicular to it. When the chain is in thermal equilibrium at temperature $T$, its average length is
	{\exyear{NET/JRF(DEC-2011)}}
	\begin{tasks}(2)
		\task[\textbf{a.}] $\mathrm{Na} / 2$
		\task[\textbf{b.}] $\mathrm{Na}$
		\task[\textbf{c.}] $N a /\left(1+e^{-2 \varepsilon / k_{B} T}\right)$
		\task[\textbf{d.}] $N a\left(1+e^{-2 \varepsilon / k_{B} T}\right)$
	\end{tasks}
\begin{answer}
	\begin{align*}
	\intertext{Let $n_{1}$ no. of rods are parallel and $n_{2}$ no. of rods are perpendicular.}
	\intertext{Energy of rod when it is perpendicular $=-\varepsilon$}
	\intertext{Energy of rod when it is parallel is $\varepsilon$.}
	P(-\varepsilon)&=\frac{e^{-\beta(-\varepsilon)}}{e^{-\beta(-\varepsilon)}+e^{-\beta \varepsilon}}=\frac{\mathrm{e}^{\beta \varepsilon}}{\mathrm{e}^{\beta \varepsilon}+\mathrm{e}^{-\beta \varepsilon}}\text{ and }P(\varepsilon)=\frac{e^{-\beta \varepsilon}}{e^{\beta \varepsilon}+e^{-\beta \varepsilon}}\\
	\text{Average length }&=n_{1} a P(\varepsilon)+n_{2} a P(-\varepsilon)=\frac{n_{1} a e^{-\beta \varepsilon}+n_{2} a e^{\beta \varepsilon}}{e^{\beta \varepsilon}+e^{-\beta \varepsilon}}\\&=\frac{N a e^{\beta \varepsilon}}{e^{\beta \varepsilon}+e^{-\beta \varepsilon}}=\frac{N a}{1+e^{-2 \beta \varepsilon}}\\
	\text{Since }&P(-\varepsilon)>>P(\varepsilon)\text{ so }n_{2} \cong N, n_{1} \cong 0
	\end{align*}
	So the correct answer is \textbf{Option (c)}
\end{answer}
	\item The excitations of a three-dimensional solid are bosonic in nature with their frequency $\omega$ and wave-number $k$ are related by $\omega \propto k^{2}$ in the large wavelength limit. If the chemical potential is zero, the behavior of the specific heat of the system at low temperature is proportional to
	{\exyear{NET/JRF(DEC-2011)}}
	\begin{tasks}(4)
		\task[\textbf{a.}] $T^{1 / 2}$
		\task[\textbf{b.}] $T$
		\task[\textbf{c.}]  $T^{3 / 2}$
		\task[\textbf{d.}] $T^{3}$
	\end{tasks}
\begin{answer}
	\begin{align*}
	\text{If dispersion relation is $\omega \propto k^{s}$,}
	\text{If dispersion relation is $\omega \propto k^{s}$,}
	\end{align*}
	So the correct answer is \textbf{Option (c)}
\end{answer}
	\item Consider a system of non-interacting particles in $d$ dimensional obeying the dispersion relation $\varepsilon=A k^{s}$, where $\varepsilon$ is the energy, $k$ is the wave vector; $s$ is an integer and $A$ is constant. The density of states, $N(\varepsilon)$, is proportional to
	{\exyear{NET/JRF(JUNE-2012)}}
	\begin{tasks}(4)
		\task[\textbf{a.}] $\varepsilon^{\frac{s}{d}-1}$
		\task[\textbf{b.}] $\varepsilon^{\frac{d}{s}-1}$
		\task[\textbf{c.}] $\varepsilon^{\frac{d}{s}+1}$
		\task[\textbf{d.}] $\varepsilon^{\frac{s}{d}+1}$
	\end{tasks}
\begin{answer}
	\begin{align*}
	\text{We can solve this problem with intuition for example $\varepsilon=A k^{2}$}\\
	\text{Density of state in 3-dimensional $N(\varepsilon) \propto \varepsilon^{\frac{1}{2}}=\varepsilon^{\frac{3}{2}-1}$}\\
	\text{Density of state in 2-dimensional $N(\varepsilon) \propto \varepsilon^{0}=\varepsilon^{\frac{2}{2}-1}$}\\
	\text{Density of state in 1-dimensional $N(\varepsilon) \propto \varepsilon^{\frac{-1}{2}}=\varepsilon^{\frac{1}{2}-1}$}\\
	\text{	Density of state in d-dimensional, where $\varepsilon=A k^{s} \Rightarrow \mathrm{N}(\varepsilon) \propto \varepsilon^{\frac{\mathrm{d}}{5}-1}$}
	\end{align*}
	So the correct answer is \textbf{Option (b)}
\end{answer}
	\item The number of ways in which $\mathrm{N}$ identical bosons can be distributed in two energy levels, is
	{\exyear{NET/JRF(JUNE-2012)}}
	\begin{tasks}(4)
		\task[\textbf{a.}] $N+1$
		\task[\textbf{b.}] $\frac{N(N-1)}{2}$
		\task[\textbf{c.}] $\frac{N(N+1)}{2}$
		\task[\textbf{d.}] $N$
	\end{tasks}
\begin{answer}
	\begin{align*}
	\intertext{	Number of boson $=N$, Number of energy level $=g$}
	\intertext{So number of ways to distribute $N$ boson into $g$ level is, }
	W&={ }^{N+g-1} c_{N}=N+1\text{ since }g=2
	\end{align*}
	So the correct answer is \textbf{Option (a)}
\end{answer}
	\item The free energy of the gas of $N$ particles in a volume $V$ and at a temperature $T$ is $F=N k_{B} T \ln \left[a_{0} V\left(k_{B} T\right)^{5 / 2} / N\right]$, where $a_{0}$ is a constant and $k_{B}$ denotes the Boltzmann constant. The internal energy of the gas is
	{\exyear{NET/JRF(JUNE-2012)}}
	\begin{tasks}(2)
		\task[\textbf{a.}] $\frac{3}{2} N k_{B} T$
		\task[\textbf{b.}] $\frac{5}{2} N k_{B} T$
		\task[\textbf{c.}] $N k_{B} T \ln \left[a_{0} V\left(k_{B} T\right)^{5 / 2} / N\right]-\frac{3}{2} N k_{B} T$
		\task[\textbf{d.}] $N k_{B} T \ln \left[a_{0} V /\left(k_{B} T\right)^{5 / 2}\right]$
	\end{tasks}
\begin{answer}
	\begin{align*}
	F&=N k_{B} T \ln \left[a_{0} V\left(k_{B} T\right)^{5 / 2} / N\right], F\\&=U-T S, U=F+T S\\
	d F&=-S d T-P d V \Rightarrow\left(\frac{\partial F}{\partial T}\right)_{V}=-S \text{ or } \\S&=-\left(\frac{\partial F}{\partial T}\right)_{V} \Rightarrow U=F-T\left(\frac{\partial F}{\partial T}\right)_{V}\\
	F&=N k_{B} T \ln \left(C T^{5 / 2}\right)\text{ where }C=\frac{a_{0} V k_{B}^{5 / 2}}{N}\\
	\left(\frac{\partial F}{\partial T}\right)_{V}&=N k_{B} \ln \left(C T^{5 / 2}\right)+N k_{B} T \frac{C}{C T^{5 / 2}} \frac{5}{2} T^{3 / 2} \Rightarrow T\left(\frac{\partial F}{\partial T}\right)_{V}\\&=N k_{B} T \ln \left(C T^{5 / 2}\right)+\frac{5}{2} N k_{B} T\\
	T\left(\frac{\partial F}{\partial T}\right)_{V}&=F+\frac{5}{2} N k_{B} T \Rightarrow U\\&=F-T\left(\frac{\partial F}{\partial T}\right)_{V}=-\frac{5}{2} N k_{B} T
	\end{align*}
	So the correct answer is \textbf{Option (b)}
\end{answer}
	\item Bose condensation occurs in liquid $H e^{4}$ kept at ambient pressure at $2.17 K$. At which temperature will Bose condensation occur in $H e^{4}$ in gaseous state, the density of which is 1000 times smaller than that of liquid $H e^{4}$ ? (Assume that it is a perfect Bose gas.)
	{\exyear{NET/JRF(JUNE-2012)}}
	\begin{tasks}(4)
		\task[\textbf{a.}] $2.17 \mathrm{mK}$
		\task[\textbf{b.}] $21.7 \mathrm{mK}$
		\task[\textbf{c.}] $21.7 \mu K$
		\task[\textbf{d.}] $2.17 \mu K$
	\end{tasks}
\begin{answer}
	\begin{align*}
	\text{	For bosons }T \propto\left(\frac{N}{V}\right)^{\frac{2}{3}}
	\end{align*}
	So the correct answer is \textbf{Option (b)}
\end{answer}
	\item Consider black body radiation contained in a cavity whose walls are at temperature $T$. The radiation is in equilibrium with the walls of the cavity. If the temperature of the walls is increased to $2 T$ and the radiation is allowed to come to equilibrium at the new temperature, the entropy of the radiation increases by a factor of
	{\exyear{NET/JRF(JUNE-2012)}}
	\begin{tasks}(4)
		\task[\textbf{a.}] 2
		\task[\textbf{b.}] 4
		\task[\textbf{c.}] 8
		\task[\textbf{d.}] 16
	\end{tasks}
\begin{answer}
	\begin{align*}
	\text{For Black Body, energy is given by }F&=\frac{-8 \pi^{5} k_{B}^{4} T^{4}}{45 \hbar^{2} C^{3}} V, S\\&=-\left(\frac{\partial F}{\partial T}\right)_{V}=\left(\frac{32 \pi^{5} k_{B}^{4}}{45 \hbar^{3} C^{3}}\right) V T^{3}
	\intertext{$\Rightarrow S \propto T^{3}$, If temperate increase from $T$ to $2 T$ then entropy will incase $S$ to $8 S$.}
	\end{align*}
		So the correct answer is \textbf{Option (c)}
\end{answer}
	\newcommand*{\downuparrow}[1]{\ensuremath{\overset{\uparrow}{#1}}}
	\newcommand*{\downdownarrow}[1]{\ensuremath{\overset{\downarrow}{#1}}}
	\item A system of non-interacting spin-1/2 charged particles are placed in an extemal magnetic field. At low temperature $T$, the leading behavior of the excess energy above the ground state energy, depends on $T$ as: $(c$ is a constant)
	{\exyear{NET/JRF(JUNE-2013)}}
	\begin{tasks}(2)
		\task[\textbf{a.}] $c T$
		\task[\textbf{b.}]  $c T^{3}$
		\task[\textbf{c.}] $e^{-c / T}$
		\task[\textbf{d.}] $c$ (is independent of $T$ )
	\end{tasks}
\begin{answer}
	\begin{align*}
	U&=-\mu_{B} H \tanh \frac{\mu_{B} H}{k T}=-\mu_{B} H\left(\frac{e^{\frac{\mu_{B} H}{k T}}-e^{-\frac{\mu_{B} H}{k T}}}{e^{\frac{\mu_{B} H}{k T}}+e^{\frac{\mu_{B} H}{k T}}}\right)
	\intertext{Excess energy from the ground level}
	&=-\mu_{B} H\left(\frac{e^{\frac{\mu_{B} H}{k T}}-e^{-\frac{\mu_{B} H}{k T}}}{e^{\frac{\mu_{B} H}{k T}}+e^{-\frac{\mu_{B} H}{k T}}}\right)-\left(-\mu_{B} H\right)\\&=\mu_{B} H\left[1-\left(\frac{e^{\frac{\mu_{B} H}{k T}}-e^{-\frac{\mu_{B} H}{k T}}}{e^{\frac{\mu_{B} H}{k T}}+e^{-\frac{\mu_{B} H}{k T}}}\right)\right]=\mu_{B} H\left(\frac{2 e^{-\frac{\mu_{B} H}{k T}}}{e^{\frac{\mu_{B} H}{k T}}+e^{-\frac{\mu_{B} H}{k T}}}\right)
	\intertext{At low temperature, the lower value, $\Delta U \propto e^{-C / T}$, where $C=\mu_{B} H$.}
	\end{align*}
	So the correct answer is \textbf{Option (c)}
\end{answer}
	\item Consider two different systems each with three identical non-interacting particles. Both have single particle states with energies $\varepsilon_{0}, 3 \varepsilon_{0}$ and $5 \varepsilon_{0},\left(\varepsilon_{0}>0\right)$. One system is populated by spin $-\frac{1}{2}$ fermions and the other by bosons. What is the value of $E_{F}-E_{B}$ where $E_{F}$ and $E_{B}$ are the ground state energies of the fermionic and bosonic systems respectively?
	{\exyear{NET/JRF(JUNE-2013)}}
	\begin{tasks}(4)
		\task[\textbf{a.}] $6 \varepsilon_{0}$
		\task[\textbf{b.}] $2 \varepsilon_{0}$
		\task[\textbf{c.}] $4 \varepsilon_{0}$.
		\task[\textbf{d.}] $\varepsilon_{0}$
	\end{tasks}
\begin{answer}
	\begin{align*}
	\text{Energy of Fermion }&=2 \times 1 \varepsilon_{0}+3 \varepsilon_{0}=5 \varepsilon_{0}\\
	\text{Energy of boson }&=3 \times 1 \varepsilon_{0}=3 \varepsilon_{0}\\
	E_{F}-E_{B}&=5 \varepsilon_{0}-3 \varepsilon_{0}=2 \varepsilon_{0}
	\end{align*}
	So the correct answer is \textbf{Option (b)}
\end{answer}
	\item Three identical spin- $\frac{1}{2}$ fermions are to be distributed in two non-degenerate distinct energy levels. The number of ways this can be done is
	{\exyear{NET/JRF(DEC-2013)}}
	\begin{tasks}(4)
		\task[\textbf{a.}] 8
		\task[\textbf{b.}] 4
		\task[\textbf{c.}] 3
		\task[\textbf{d.}] 2
	\end{tasks}
\begin{answer}
	\begin{align*}
	\text{Total number of degeneracy}
	\intertext{$g=($ Number of energy state $(n)) \times($ Number of degeneracy due to spin $(2 s+1))$}
	n=2, s=\frac{1}{2}, \quad g=2 \times\left(2 \cdot \frac{1}{2}+1\right)&=4\\
	\text{Number of particle, $N=3 .$ So number of ways, }{ }^{g} c_{N}&={ }^{4} c_{3}=4
	\end{align*}
	So the correct answer is \textbf{Option (b)}
\end{answer}
	\item Which of the graphs below gives the correct qualitative behaviour of the energy density $E_{r}(\lambda)$ of blackbody radiation of wavelength $\lambda$ at two temperatures $T_{1}$ and $T_{2}\left(T_{1}<T_{2}\right) ?$
	{\exyear{NET/JRF(JUNE-2014)}}
	\begin{tasks}(2)
		\task[\textbf{a.}] \begin{figure}[H]
			\centering
			\includegraphics[height=4cm,width=5cm]{diagram-20210924(10)-crop}
		\end{figure}
		\task[\textbf{b.}] \begin{figure}[H]
			\centering
			\includegraphics[height=4cm,width=5cm]{diagram-20210924(11)-crop}
		\end{figure}
		\task[\textbf{c.}] \begin{figure}[H]
			\centering
			\includegraphics[height=4cm,width=5cm]{diagram-20210924(12)-crop}
		\end{figure}
		\task[\textbf{d.}] \begin{figure}[H]
			\centering
			\includegraphics[height=4cm,width=5cm]{diagram-20210924(13)-crop}
		\end{figure}
	\end{tasks}
\begin{answer}
	So the correct answer is \textbf{Option (c)}	
\end{answer}
	\item The pressure of a non-relativistic free Fermi gas in three-dimensions depends, at $T=0$, on the density of fermions $n$ as
	{\exyear{NET/JRF(JUNE-2014)}}
	\begin{tasks}(4)
		\task[\textbf{a.}] $n^{5 / 3}$
		\task[\textbf{b.}] $n^{1 / 3}$
		\task[\textbf{c.}]  $n^{2 / 3}$
		\task[\textbf{d.}] $n^{4 / 3}$
	\end{tasks}
\begin{answer}
	\begin{align*}
	\text{	Pressure }P&=\frac{2}{3} n E_{F}, E_{F} \propto n^{2 / 3},\text{ at }T=0\\
	P&=\frac{2}{3} n \times n^{2 / 3}=\frac{2}{3} n^{5 / 3} \Rightarrow P \propto n^{5 / 3}
	\end{align*}
	So the correct answer is \textbf{Option (a)}
\end{answer}
	\item An ideal Bose gas in $d$-dimensions obeys the dispersion relation $\in(\vec{k})=A k^{s}$, where $A$ and $s$ are constants. For Bose-Einstein condensation to occur, the occupancy of excited states
	$$
	N_{e}=c \int_{0}^{\infty} \frac{\epsilon^{\frac{(d-s)}{s}}}{\left(e^{\beta(\epsilon-\mu)}-1\right)} d \in
	$$
	where $c$ is a constant, should remain finite even for $\mu=0$. This can happen if
	{\exyear{NET/JRF(JUNE-2015)}}
	\begin{tasks}(4)
		\task[\textbf{a.}] $\frac{d}{s}<\frac{1}{4}$
		\task[\textbf{b.}] $\frac{1}{4}<\frac{d}{s}<\frac{1}{2}$
		\task[\textbf{c.}] $\frac{d}{s}>1$
		\task[\textbf{d.}] $\frac{1}{2}<\frac{d}{s}<1$
	\end{tasks}
\begin{answer}
	\begin{align*}
	N e&=c \int_{0}^{\infty} \frac{\epsilon^{\frac{(d-s)}{s}}}{e^{\beta(\epsilon-\mu)}-1} d \in\\
	\text{B.E. condensation is possible in $3-D$}\\
	\text{For materlistic particle }g(\in) \propto \epsilon^{\frac{1}{2}} \Rightarrow \frac{d-s}{s}&=\frac{1}{2} \Rightarrow \frac{d}{s}=\frac{3}{2}\\
	\text{For massless particle }g(\in) \propto \epsilon^{2} \Rightarrow \frac{d-s}{s}&=2 \Rightarrow \frac{d}{s}=3\\
	\text{In both cases }&\frac{d}{s}>1
	\end{align*}
	So the correct answer is \textbf{Option (c)}
\end{answer}
	\item For a system of independent non interacting one-dimensional oscillators, the value of the free energy per oscillator, in the limit $T \rightarrow 0$, is
	{\exyear{NET/JRF(DEC-2015)}}
	\begin{tasks}(4)
		\task[\textbf{a.}] $\frac{1}{2} \hbar \omega$
		\task[\textbf{b.}] $\hbar \omega$
		\task[\textbf{c.}] $\frac{3}{2} \hbar \omega$
		\task[\textbf{d.}] 0
	\end{tasks}
\begin{answer}
	\begin{align*}
	\text{For the given system }Z_{N}&=\left[2 \sinh \frac{\hbar \omega}{2 k T}\right]^{-N} \Rightarrow F\\&=-k T \ln Z_{N}=N k T \ln \left[2 \sinh \left(\frac{\hbar \omega}{2 k T}\right)\right]\\
	&=N k T \ln \left[\frac{2\left(e^{\frac{\hbar \omega}{2 k T}}-e^{-\frac{\hbar \omega}{2 k T}}\right)}{2}\right]=N k T \ln \left[e^{\frac{\hbar \omega}{2 k T}}\left(1-e^{-\frac{\hbar \omega}{k T}}\right)\right]\\&=N k T \ln e^{\frac{\hbar \omega}{2 k T}}+N k T \ln \left(1-e^{-\frac{\hbar \omega}{k T}}\right)\\
	\frac{F}{N}&=\frac{\hbar \omega}{2}+k T \ln \left(1-e^{\frac{\hbar \omega}{k T}}\right)\\&=\frac{\hbar \omega}{2}+0=\frac{\hbar \omega}{2}\quad (\because k T \rightarrow 0)
	\end{align*}
	So the correct answer is \textbf{Option (a)}
\end{answer}
	\item The specific heat per molecule of a gas of diatomic molecules at high temperatures is
	{\exyear{NET/JRF(JUNE-2016)}}
	\begin{tasks}(4)
		\task[\textbf{a.}] $8 k_{B}$
		\task[\textbf{b.}] $3.5 k_{B}$
		\task[\textbf{c.}] $4.5 k_{B}$
		\task[\textbf{d.}] $3 k_{B}$
	\end{tasks}
\begin{answer}
	\begin{align*}
	\intertext{For high temperature all number are excited so degree of freedom for diatomic molecule is 7 .}
	\text{Internal energy is }\frac{f k_{B} T}{2}, U&=\frac{7 k_{B} T}{2}, C_{V}=\left(\frac{\partial U}{\partial T}\right)_{V}=3.5 k_{B}
	\end{align*}
	So the correct answer is \textbf{Option (b)}
\end{answer}
	\item A gas of non-relativistic classical particles in one dimension is subjected to a potential $V(x)=\alpha|x|$ (where $\alpha$ is a constant). The partition function is $\left(\beta=\frac{1}{k_{B} T}\right)$
	{\exyear{NET/JRF(JUNE-2016)}}
	\begin{tasks}(4)
		\task[\textbf{a.}] $\sqrt{\frac{4 m \pi}{\beta^{3} \alpha^{2} h^{2}}}$
		\task[\textbf{b.}] $\sqrt{\frac{2 m \pi}{\beta^{3} \alpha^{2} h^{2}}}$
		\task[\textbf{c.}] $\sqrt{\frac{8 m \pi}{\beta^{3} \alpha^{2} h^{2}}}$
		\task[\textbf{d.}] $\sqrt{\frac{3 m \pi}{\beta^{3} \alpha^{2} h^{2}}}$
	\end{tasks}
\begin{answer}
	\begin{align*}
	z&=\frac{1}{h} \int_{-\infty}^{\infty} e^{-\frac{p_{x}^{2}}{2 m k T}} d p_{x} \int_{-\infty}^{\infty} e^{-\frac{\alpha|x|}{k T}} d x\\&=\frac{1}{h}(2 \pi m k T)^{1 / 2} \int_{-\infty}^{\infty} e^{-\frac{\alpha|x|}{k T}} d x\\
	\because \int_{-\infty}^{\infty} e^{-\frac{\alpha|x|}{k T}} d x&=\int_{-\infty}^{0} e^{+\frac{\alpha x}{k T}} d x+\int_{0}^{\infty} e^{-\frac{\alpha x}{k T}} d x\\&=\frac{k T}{\alpha}+\frac{k T}{\alpha}=\frac{2 k T}{\alpha}\\
	z&=\left(\frac{2 \pi m k T}{h^{2}}\right)^{1 / 2}\left(\frac{2 k T}{\alpha}\right)=\left(\frac{8 \pi m}{h^{2} \beta^{3} \alpha^{2}}\right)^{\frac{1}{2}}\\
	\text{ put }\beta&=\frac{1}{k T}
	\end{align*}
	So the correct answer is \textbf{Option (c)}
\end{answer}
	\item An atom has a non-degenerate ground-state and a doubly-degenerate excited state. The energy difference between the two states is $\varepsilon$. The specific heat at very low temperatures $(\beta \varepsilon \gg 1)$ is given by
	{\exyear{NET/JRF(DEC-2016)}}
	\begin{tasks}(4)
		\task[\textbf{a.}] $k_{B}(\beta \varepsilon)$
		\task[\textbf{b.}] $k_{B} e^{-\beta \varepsilon}$
		\task[\textbf{c.}] $2 k_{B}(\beta \varepsilon)^{2} e^{-\beta \varepsilon}$
		\task[\textbf{d.}] $k_{B}$
	\end{tasks}
\begin{answer}
	\begin{align*}
	\intertext{ Assume energy at ground state is 0 and energy at first excited state is $\in$. The partition function is $Z=1+2 e^{-\beta \in}$}\\
	\text{Energy }&=\frac{2 \in e^{-\beta \in}}{\left(1+2 e^{-\beta \in}\right)}\\
	\text{Specific heat, }C_{V}&=\left(\frac{\partial U}{\partial T}\right)_{V}\\&=\frac{2 \in e^{-\frac{\epsilon}{k T}}(-\epsilon) \frac{-1}{k T^{2}}}{\left(1+2 e^{\frac{-\epsilon}{k T}}\right)}+\frac{2 \in e^{\frac{-\epsilon}{k T}} \in \in \frac{2}{k T^{2}}}{\left(1+2 e^{\frac{-\epsilon}{k T}}\right)^{2}}\\
	&=2 k\left(\frac{\in}{k T}\right)^{2} e^{\frac{-\epsilon}{k T}} \frac{\left(1+2 e^{\frac{-\epsilon}{k T}}\right)}{\left(1+2 e^{\frac{-\epsilon}{k T}}\right)^{2}}=2 k(\beta \in)^{2} e^{-\beta \epsilon} \frac{\left(1+2 e^{-\beta \epsilon}\right)}{\left(1+2 e^{-\beta \epsilon}\right)^{2}}\\
	C_{V} &\simeq 2 k(\beta \in)^{2} e^{-\beta \in}, \quad \beta \in \rightarrow \infty
	\end{align*}
	So the correct answer is \textbf{Option (c)}
\end{answer}
	\item  A gas of photons inside a cavity of volume $V$ is in equilibrium at temperature $T$. If the temperature of the cavity is changed to $2 T$, the radiation pressure will change by a factor of
	{\exyear{NET/JRF(JUNE-2017)}}
	\begin{tasks}(4)
		\task[\textbf{a.}] 2
		\task[\textbf{b.}] 16
		\task[\textbf{c.}] 8
		\task[\textbf{d.}] 4
	\end{tasks}
\begin{answer}
	\begin{align*}
	\intertext{For 3 dimensional system $P \propto T^{4}$}
	\frac{P_{2}}{P_{1}}&=\left(\frac{2 T}{T}\right)^{4} \Rightarrow P_{2}=P_{1} 2^{4}=p_{1} \times 16=16 P_{1}
	\end{align*}
	So the correct answer is \textbf{Option (b)}
\end{answer}
	\item The single particle energy levels of a non-interacting three-dimensional isotropic system, labelled by momentum $k$, are proportional to $k^{3}$. The ratio $\bar{P} / \in$ of the average pressure $\bar{P}$ to the energy density $\in$ at a fixed temperature, is
	{\exyear{NET/JRF(JUNE-2017)}}
	\begin{tasks}(4)
		\task[\textbf{a.}] $1 / 3$
		\task[\textbf{b.}] $2 / 3$
		\task[\textbf{c.}] 1
		\task[\textbf{d.}] 3
	\end{tasks}
\begin{answer}
	\begin{align*}
	\intertext{$E \propto p^{s}$, where $p$ is momentum}
	P&=\frac{s}{3}\left(\frac{E}{V}\right),\text{ where $P$ is pressure}\\
	\frac{P}{E} &\propto \frac{s}{3}\\
	\text{In problem, }E \propto k^{3},\text{ so, }s&=3\\
	\text{	pressure }P&=\frac{3}{3}\left(\frac{E}{V}\right) \Rightarrow P \propto\left(\frac{E}{V}\right)\text{ at fixed $T$.}
	\end{align*}
	So the correct answer is \textbf{Option (c)}
\end{answer}
	\item The number of ways of distributing 11 indistinguishable bosons in 3 different energy levels is
	{\exyear{NET/JRF(JUNE-2018)}}
	\begin{tasks}(4)
		\task[\textbf{a.}]  $3^{11}$
		\task[\textbf{b.}] $11^{3}$
		\task[\textbf{c.}] $\frac{(13) !}{2 !(11) !}$
		\task[\textbf{d.}]  $\frac{(11) !}{3 ! 8 !}$
	\end{tasks}
\begin{answer}
	\begin{align*}
	n&=11 \quad g=3\\
	\color{red}\text{answer not completed}
	\end{align*}
	So the correct answer is \textbf{Option (c)}
\end{answer}
	\item The rotational energy levels of a molecule are $E_{\ell}=\frac{\hbar^{2}}{2 I_{0}} \ell(\ell+1)$, where $\ell=0,1,2, \ldots$ and $I_{0}$ is its moment of inertia. The contribution of the rotational motion to the Helmholtz free energy per molecule, at low temperatures in a dilute gas of these molecules, is approximately
	{\exyear{NET/JRF(DEC-2018)}}
	\begin{tasks}(2)
		\task[\textbf{a.}]  $-k_{B} T\left(1+\frac{\hbar^{2}}{I_{0} k_{B} T}\right)$
		\task[\textbf{b.}] $-k_{B} T e^{\frac{\hbar^{2}}{I_{0} k_{B} T}}$
		\task[\textbf{c.}] $-k_{B} T$
		\task[\textbf{d.}] $-3 k_{B} T e^{-\frac{\hbar^{2}}{I_{0} k_{B} T}}$
	\end{tasks} 
\begin{answer}
	\begin{align*}
	E_{\ell}&=\frac{\hbar^{2}}{2 I_{0}} \ell(\ell+1) \quad \ell=0,1,2, \ldots\\
	z&=\sum_{\ell=0}^{\infty}(2 \ell+1) e^{\frac{-\beta \hbar^{2} \ell(\ell+1)}{2 I_{0}}}\\
	z&=1+\sum_{\ell=0}^{\infty}(2 \ell+1) e^{\frac{-\hbar^{2} \ell(\ell+1)}{2 I_{0} k_{B} T}}\\
	F&=-k_{B} T \ln z=-k_{B} T \ln \left(1+\sum_{\ell=1}^{\infty}(2 \ell+1) e^{\frac{-\hbar^{2} \ell(\ell+1)}{2 I_{0} k_{B} T}}\right)\\
	\ln (1+x)&=x-\frac{x^{2}}{2}+\ldots
	\intertext{For low temperature, higher temperature can be neglected}
	F&=-k_{B} T \sum_{\ell=1}^{\infty}(2 \ell+1) e^{-\frac{-\hbar^{2} \ell(\ell+1)}{2 I_{0} k_{B} T}}\\&=-k_{B} T\left[3 e^{\frac{-\hbar^{2}}{I_{0} k_{B} T}}+\ldots\right]=-3 k_{B} T e^{-\frac{\hbar^{2}}{I_{0} k_{B} T}}
	\end{align*}
	So the correct answer is \textbf{Option (d)}
\end{answer} 
	\item Consider an ideal Fermi gas in a grand canonical ensemble at a constant chemical potential. The variance of the occupation number of the single particle energy level with mean occupation number $\bar{n}$ is
	{ \exyear{NET/JRF(DEC-2018)}}
	\begin{tasks}(4)
		\task[\textbf{a.}] $\bar{n}(1-\bar{n})$
		\task[\textbf{b.}]  $\sqrt{\bar{n}}$
		\task[\textbf{c.}] $\bar{n}$
		\task[\textbf{d.}] $\frac{1}{\sqrt{\bar{n}}}$
	\end{tasks}
\begin{answer}
	\begin{align*}
	\bar{n}&=k_{B} T \frac{1}{z}\left(\frac{\partial z}{\partial \mu}\right)_{V, T}=\frac{1}{e^{(\in-\mu) / k_{B} T}+1}\\
	\text{Variance }&=k_{B} T\left(\frac{d n}{d \mu}\right)_{V, T}=\bar{n}(1-\bar{n})
	\intertext{Note: This may also be divided using simple Bernoulli distribution.}
	\end{align*}
	So the correct answer is \textbf{Option (a)}
\end{answer}
	\item The Hamiltonian of a one-dimensional Ising model of $N$ spins ( $N$ large) is
	$$
	H=-J \sum_{i=1}^{N} \sigma_{i} \sigma_{i+1}
	$$
	where the spin $\sigma_{i}=\pm 1$ and $J$ is a positive constant. At inverse temperature $\beta=\frac{1}{k_{B} T}$, the correlation function between the nearest neighbor spins $\left(\sigma_{i} \sigma_{i+1}\right)$ is
	{\exyear{NET/JRF(DEC-2018)}}
	\begin{tasks}(4)
		\task[\textbf{a.}] $\frac{e^{-\beta J}}{\left(e^{\beta J}+e^{-\beta J}\right)}$
		\task[\textbf{b.}] $e^{-2 \beta J}$
		\task[\textbf{c.}] $\tanh (\beta J)$
		\task[\textbf{d.}] $\operatorname{coth}(\beta J)$
	\end{tasks}
\begin{answer}
	\begin{align*}
	\left\langle\sigma_{i} \cdot \sigma_{i+1}\right\rangle&=\frac{\sum \sigma_{i} \cdot \sigma_{i+1}}{N-1}=\frac{\sum \sigma_{i} \cdot \sigma_{i+1}}{N} N>>1=\frac{1-1}{-J N}\hspace{1cm}\text{(i)}\\
	\text{For such an Ising model for }&N>>1\\
	z&=(\cosh \beta J)^{N}\\
	\text{Average Energy }&=\frac{-\partial}{\partial \beta} \ln z\\
	&=-N \frac{1}{\cosh \beta J} \sinh \beta J \cdot J\\
	&=-N J \tanh \beta J\hspace{4cm}\text{(ii)}\\
	\left\langle\sigma_{i} \cdot \sigma_{i+1}\right\rangle&=\frac{-N j \tanh \beta j}{-j N}=\tanh \beta j
	\end{align*}
	So the correct answer is \textbf{Option (c)}
\end{answer}
	\item  The Hamiltonian of a system of 3 spins is $H=J\left(S_{1} S_{2}+S_{2} S_{3}\right)$, where $S_{i}=\pm 1$ for $i=1,2,3$. Its canonical partition function, at temperature $T$, is
	{ \exyear{NET/JRF(JUNE-2020)}}
	\begin{tasks}(2)
		\task[\textbf{a.}] $2\left(2 \sinh \frac{J}{k_{B} T}\right)^{2}$
		\task[\textbf{b.}]  $2\left(2 \cosh \frac{J}{k_{B} T}\right)^{2}$
		\task[\textbf{c.}]  $2\left(2 \cosh \frac{J}{k_{B} T}\right)$
		\task[\textbf{d.}] $2\left(2 \cosh \frac{J}{k_{B} T}\right)^{3}$
	\end{tasks}
\begin{answer}
	\begin{align*}
	\renewcommand*{\arraystretch}{1.5}
	\begin{tabular}{|c|c|c|c|}
	\hline$S_{1}$ & $S_{2}$ & $S_{3}$ & $H$ \\
	\hline 1 & 1 & 1 & $2 \mathrm{~J}$ \\
	\hline 1 & 1 & $-1$ & 0 \\
	\hline 1 & $-1$ & 1 & 0 \\
	\hline 1 & $-1$ & $-1$ & $-2 \mathrm{~J}$ \\
	\hline$-1$ & 1 & 1 & 0 \\
	\hline$-1$ & 1 & $-1$ & $-2 \mathrm{~J}$ \\
	\hline$-1$ & $-1$ & 1 & 0 \\
	\hline$-1$ & $-1$ & $-1$ & $2 \mathrm{~J}$ \\
	\hline
	\end{tabular}\\
	\text{Number of states }2^{3}&=8\\
	\end{align*}
	\begin{align*}
	H&=J\left(S_{1} S_{2}+S_{2} S_{3}\right)\\
	Z&=2 e^{-\beta 2 J}+2 e^{\beta 2 J}+4=2\left[e^{\beta 2 J}+e^{-\beta 2 J}\right]+4\\&=2\left(\left[e^{\beta J}+e^{-\beta J}\right]^{2}-2\right)+4\\
	\Rightarrow Z&=2\left(\frac{2\left(e^{\beta J}+e^{-\beta J}\right)}{2}\right)^{2}=2\left(2 \cosh \frac{J}{k_{B} T}\right)^{2}
	\end{align*}
	So the correct answer is \textbf{Option (b)}
\end{answer}
\end{enumerate}
\colorlet{ocre1}{ocre!70!}
\colorlet{ocrel}{ocre!30!}
\setlength\arrayrulewidth{1pt}
\begin{table}[H]
	\centering
	\arrayrulecolor{ocre}
	\begin{tabular}{|p{1.5cm}|p{1.5cm}||p{1.5cm}|p{1.5cm}|}
		\hline
		\multicolumn{4}{|c|}{\textbf{Answer key}}\\\hline\hline
		\rowcolor{ocrel}Q.No.&Answer&Q.No.&Answer\\\hline
		1&\textbf{a} &2&\textbf{b}\\\hline
		3&\textbf{c} &4&\textbf{a} \\\hline
		5&\textbf{c} &6&\textbf{c} \\\hline
		7&\textbf{b}&8&\textbf{a}\\\hline
		9&\textbf{b}&10&\textbf{b}\\\hline
		11&\textbf{c} &12&\textbf{c}\\\hline
		13&\textbf{b}&14&\textbf{b}\\\hline
		15&\textbf{c}&16 &\textbf{a}\\\hline
		17&\textbf{c}&18 &\textbf{a}\\\hline
		19&\textbf{b}&20&\textbf{c}\\\hline
		21&\textbf{c} &22&\textbf{b}\\\hline
		23&\textbf{c}&24&\textbf{c}\\\hline
		25&\textbf{d}&26 &\textbf{a}\\\hline
		27&\textbf{c}&28 &\textbf{b}\\\hline
	\end{tabular}
\end{table}
\newpage
\begin{abox}
	Practise set-2
\end{abox}
\begin{enumerate}
	\item Consider a system of two particles $A$ and $B$. Each particle can occupy one of three possible quantum states $|1\rangle,|2\rangle$ and $|3\rangle$. The ratio of the probability that the two particles
	are in the same state to the probability that the two particles are in different states is calculated for bosons and classical (Maxwell-Boltzmann) particles. They are respectively
	{\exyear{JEST 2013}}
	
	\begin{tasks}(4)
		\task[\textbf{a.}] 1,0
		\task[\textbf{b.}]  $\frac{1}{2}, 1$
		\task[\textbf{c.}] $1, \frac{1}{2}$
		\task[\textbf{d.}] $0, \frac{1}{2}$
	\end{tasks}
	\begin{answer}
		For two particle in same state:\\
		\begin{figure}[H]
			\centering
			\includegraphics[height=3cm,width=8cm]{diagram-20210816(1)-crop}
		\end{figure}
		\begin{align*}
		\text{Probability ratio: }&\frac{1 / 3}{1 / 3}=1
		\end{align*}
		For two particle in different state;
		\begin{figure}[H]
			\centering
			\includegraphics[height=3cm,width=11cm]{diagram-20210816-crop}
		\end{figure}
		\begin{align*}
		\text{Probability ratio: }&\frac{1 / 3}{2 / 3}=\frac{1}{2}
		\end{align*}
		So the correct answer is \textbf{Option (c)}
	\end{answer}
	\item Consider three situations of 4 particles in one dimensional box of width $L$ with hard walls. In case (i), the particles are fermions, in case (ii) they are bosons, and in case (iii) they are classical. If the total ground state energy of the four particles in these three cases are $E_{F}, E_{B}$ and $E_{c l}$ respectively, which of the following is true?
	{\exyear{JEST 2013}}
	\begin{tasks}(2)
		\task[\textbf{a.}] $E_{F}=E_{B}=E_{c l}$
		\task[\textbf{b.}] $E_{F}>E_{B}=E_{c l}$
		\task[\textbf{c.}] $E_{F}<E_{B}<E_{c l}$
		\task[\textbf{d.}] $E_{F}>E_{B}>E_{c l}$
	\end{tasks}
	\begin{answer}
		\begin{align*}
		\intertext{For fermions, in 1-D box of width $L$, the ground state energy for single particle is written as,}
		\frac{\pi^{2} \hbar^{2}}{2 m l^{2}}&=\epsilon_{0}\\
		\Rightarrow 1 \times \in_{0}+1 \times 4 \in_{0}+1 \times 9 \in_{0}+1 \times 16 \in_{0}&=30 \in_{0}\\
		\text{ For Boson }&=4 \times \epsilon_{0},\\\text{ For Maxwell }&=4 \times \epsilon_{0}\\
		E_{F}>E_{B}&=E_{c l}
		\end{align*}
		So the correct answer is \textbf{Option (b)}
	\end{answer}
	\item If the mean square fluctuations in energy of a system in equilibrium at temperature $T$ is proportional to $T^{\alpha}$, then the energy of the system is proportional to
	{\exyear{JEST 2017}}
	\begin{tasks}(4)
		\task[\textbf{a.}] $T^{\alpha-2}$
		\task[\textbf{b.}] $T^{\frac{\alpha}{2}}$
		\task[\textbf{c.}] $T^{\alpha-1}$
		\task[\textbf{d.}] $T^{\alpha}$
	\end{tasks}
\begin{answer}
	\begin{align*}
	(\Delta E)^{2}&=k T^{2} C_{V} \Rightarrow T^{\alpha-2} \propto C_{V} \\&\Rightarrow T^{\alpha-2} \propto\left(\frac{\partial U}{\partial T}\right)_{V} \Rightarrow U \propto T^{\alpha-1}
	\end{align*}
	So the correct answer is \textbf{Option (c)}
\end{answer}
	\item For a two-dimensional free electron gas, the electronic density $\mathrm{n}$, and the Fermi energy $\mathrm{E}_{\mathrm{F}}$, are related by
	{\exyear{GATE 2010}}
	\begin{tasks}(4)
		\task[\textbf{a.}] $n=\frac{\left(2 m E_{F}\right)^{3 / 2}}{3 \pi^{2} \hbar^{3}}$
		\task[\textbf{b.}] $n=\frac{m E_{F}}{\pi \hbar^{2}}$
		\task[\textbf{c.}] $n=\frac{m E_{F}}{2 \pi \hbar^{2}}$
		\task[\textbf{d.}] $n=\frac{\left(2 m E_{F}\right)^{3 / 2}}{\pi \hbar}$
	\end{tasks}
	\begin{answer}
		\begin{align*}
		n&=\int_{0}^{E_{F}} g(E) f(E) d E, \quad g(E) d E=\frac{2 m}{h^{2}} d E\\
		\text{At }T&=0, f(E)=\left\{\begin{array}{ll}1, & \text { if } E<E_{F} \\ 0, & \text { if } E>E_{F}\end{array} \Rightarrow n=\frac{2 m E_{F}}{h^{2}}=\frac{m E_{F}}{2 \pi^{2} \hbar^{2}}\right.
		\end{align*}
		So the correct answer is \textbf{Option (c)}
	\end{answer}
	\item For an ideal Fermi gas in three dimensions, the electron velocity $V_{F}$ at the Fermi surface is related to electron concentration $n$ as,
	{\exyear{GATE 2012}}
	\begin{tasks}(4)
		\task[\textbf{a.}] $V_{F} \propto n^{2 / 3}$
		\task[\textbf{b.}]  $V_{F} \propto n$
		\task[\textbf{c.}] $V_{F} \propto n^{1 / 2}$
		\task[\textbf{d.}] $V_{F} \propto n^{1 / 3}$
	\end{tasks}
	\begin{answer}
		\begin{align*}
		E_{F}&=\frac{1}{2} m V_{F}^{2} \quad \because E_{F} \propto n^{2 / 3} \Rightarrow V_{F}^{2} \propto n^{2 / 3} \Rightarrow V_{F} \propto n^{1 / 3}
		\end{align*}
		So the correct answer is \textbf{Option (d)}
	\end{answer}
	\item The total energy, $E$ of an ideal non-relativistic Fermi gas in three dimensions is given by $E \propto \frac{N^{5 / 3}}{V^{2 / 3}}$, where $N$ is the number of particles and $V$ is the volume of the gas. Identify the CORRECT equation of state ( $P$ being the pressure),
	{\exyear{GATE 2012}}
	\begin{tasks}(4)
		\task[\textbf{a.}] $P V=\frac{1}{3} E$
		\task[\textbf{b.}] $P V=\frac{2}{3} E$
		\task[\textbf{c.}] $P V=E$
		\task[\textbf{d.}] $P V=\frac{5}{3} E$
	\end{tasks}
	\begin{answer}
		\begin{align*}
		P&=-\left(\frac{\partial E}{\partial V}\right)_{N}=\frac{2}{3}\left(\frac{N}{V}\right)^{\frac{5}{3}} \Rightarrow P V\\&=\frac{2}{3} \frac{N^{5 / 3}}{V^{2 / 3}}=\frac{2}{3} E
		\end{align*}
		So the correct answer is \textbf{Option (b)}
	\end{answer}
	\item Consider a linear collection of $N$ independent spin $1 / 2$ particles, each at a fixed location. The entropy of this system is $(k$ is the Boltzmann constant)
	{\exyear{GATE 2013}}
	\begin{tasks}(4)
		\task[\textbf{a.}] Zero
		\task[\textbf{b.}]  $N k$
		\task[\textbf{b.}]  $\frac{1}{2} N k$
		\task[\textbf{c.}] $N k \ln (2)$
	\end{tasks}
	\begin{answer}
		There are two microstates possible for spin $\frac{1}{2}$ particle, so entropy is given by $N k \ln (2)$\\\\
		So the correct answer is \textbf{Option (d)}
	\end{answer}
	\item Two identical fermions
	{\exyear{GATE 2013}}
	\begin{tasks}(2)
		\task[\textbf{a.}] $e^{-2 \beta E}+e^{-4 \beta E}+e^{-6 \beta E}+e^{-8 \beta E}$
		\task[\textbf{b.}] $e^{-3 \beta E}+e^{-4 \beta E}+e^{-5 \beta E}+e^{-6 \beta E}+e^{-7 \beta E}$
		\task[\textbf{c.}] $\left(e^{-\beta E}+e^{-2 \beta E}+e^{-3 \beta E}+e^{-4 \beta E}\right)^{2}$
		\task[\textbf{d.}] $e^{-2 \beta E}-e^{-4 \beta E}+e^{-6 \beta E}-e^{-8 \beta E}$
	\end{tasks}
	\begin{answer}
		\begin{align*}
		\intertext{The possible value of Energy for two Fermions}
		E_{1}&=3 E, E_{2}=4 E, E_{3}=5 E, E_{4}=6 E, E_{5}=7 E
		\intertext{The partition function is $Z=e^{-3 \beta E}+e^{-4 \beta E}+2 e^{-5 \beta E}+e^{-6 \beta E}+e^{-7 \beta E}$, then the answer may be option (B).}
		\end{align*}
		So the correct answer is \textbf{Option (b)}
	\end{answer}
	\begin{minipage}{\textwidth}
		\item Two distinguishable particles
		{\exyear{GATE 2014}}
	\end{minipage}
	\begin{tasks}(1)
		\task[\textbf{a.}] $e^{-2 \beta E}+e^{-4 \beta E}+e^{-6 \beta E}+e^{-8 \beta E}$
		\task[\textbf{b.}] $e^{-3 \beta E}+e^{-4 \beta E}+e^{-5 \beta E}+e^{-6 \beta E}+e^{-7 \beta E}$
		\task[\textbf{c.}] $\left(e^{-\beta E}+e^{-2 \beta E}+e^{-3 \beta E}+e^{-4 \beta E}\right)^{2}$
		\task[\textbf{d.}] $e^{-2 \beta E}-e^{-4 \beta E}+e^{-6 \beta E}-e^{-8 \beta E}$
	\end{tasks}
	\begin{answer}
		\begin{align*}
		\intertext{ When two particles are distinguishable then minimum value of Energy is $2 E$ and maximum value is $8 E$.
			So from checking all four options $\left(Z=e^{-\beta E}+e^{-2 \beta E}+e^{-3 \beta E}+e^{-4 \beta E}\right)^{2}$}
		\end{align*}
		So the correct answer is \textbf{Option (c)}
	\end{answer}
	\item For a system of two bosons each of which can occupy any of the two energy levels 0 and $\varepsilon$. The mean energy of the system at temperature $T$ with $\beta=\frac{1}{k_{\beta} T}$ is given by
	{\exyear{GATE 2014}}
	
	\begin{tasks}(4)
		\task[\textbf{a.}] $\frac{\varepsilon e^{-\beta \varepsilon}+2 \varepsilon e^{-2 \beta \varepsilon}}{1+2 e^{-\beta \varepsilon}+e^{-2 \beta \varepsilon}}$
		\task[\textbf{b.}] $\frac{1+\varepsilon e^{-\beta \varepsilon}}{2 e^{-\beta \varepsilon}+e^{-2 \beta \varepsilon}}$
		\task[\textbf{c.}] $\frac{2 \varepsilon e^{-\beta \varepsilon}+\varepsilon e^{-2 \beta \varepsilon}}{2+e^{-\beta \varepsilon}+e^{-2 \beta \varepsilon}}$
		\task[\textbf{d.}] $\frac{\varepsilon e^{-\beta \varepsilon}+2 \varepsilon e^{-2 \beta \varepsilon}}{2+e^{-\beta \varepsilon}+e^{-2 \beta \varepsilon}}$
	\end{tasks}
	\begin{answer}
		\begin{align*}
		\intertext{Solution: If both particle will in ground state the energy will 0 , which is non-degenerate. If one particle is in ground state and other is in first excited state then energy is $\varepsilon$ and non degenerate. If both particles will in first excited state, then energy will $2 \varepsilon$, which is non-degenerate.}
		\text{	Then partition function is }Z&=1+\exp (-\beta \varepsilon)+\exp (-2 \beta
		\varepsilon)\\\text{Average value of energy }&=\frac{\exp -\beta \varepsilon+2 \varepsilon \exp -2 \beta \varepsilon}{1+\exp -\beta \varepsilon+\exp -2 \beta \varepsilon}\intertext{No one answer is correct, but answer may be (a).}
		\end{align*}
		None of the options are matched.
	\end{answer}
	\begin{minipage}{\textwidth}
		\item Consider a system of 3 fermions which can occupy any of the 4 available energy states with equal probability. The entropy of the system is
		{\exyear{GATE 2014}}
	\end{minipage}
	\begin{tasks}(4)
		\task[\textbf{a.}] $k_{B} \ln 2$
		\task[\textbf{b.}] $2 k_{B} \ln 2$
		\task[\textbf{c.}]  $2 k_{B} \ln 4$
		\task[\textbf{d.}]  $3 k_{B} \ln 4$
	\end{tasks}
\begin{answer}
	\begin{align*}
	\intertext{Solution: Number of ways that 3 fermions will adjust in 4 available energy is ${ }^{4} C_{3}=4$ so entropy is $k_{B} \ln 4=2 k_{B} \ln 2$}
	\end{align*}
	So the correct answer is \textbf{Option (b)}
\end{answer}
	\item In Boss-Einstein condensation, the particles
	{\exyear{GATE 2015}}
	\begin{tasks}(1)
		\task[\textbf{a.}] Have strong interparticle attraction
		\task[\textbf{b.}] Condense in real space
		\task[\textbf{c.}]  Have overlapping wavefunctions
		\task[\textbf{d.}] Have large and positive chemical potential
	\end{tasks}
	\begin{answer}
		\begin{align*}
		\intertext{In Bose- Einstein condensates, the particles have overlapping wave function.}
		\end{align*}
		So the correct answer is \textbf{Option (c)}
	\end{answer}
	\begin{minipage}{\textwidth}
		\item For a black body radiation in a cavity, photons are created and annihilated freely as a result of emission and absorption by the walls of the cavity. This is because
		{\exyear{GATE 2015}}
	\end{minipage}
	\begin{tasks}(1)
		\task[\textbf{a.}] The chemical potential of the photons is zero
		\task[\textbf{b.}] Photons obey Pauli exclusion principle
		\task[\textbf{c.}] Photons are spin-1 particles
		\task[\textbf{d.}] The entropy of the photons is very large
	\end{tasks}
	\begin{answer}
		\begin{align*}
		\text{The chemical potential of photon is zero}
		\end{align*}
		So the correct answer is \textbf{Option (a)}
	\end{answer}
	\item The total power emitted by a spherical black body of radius $R$ at a temperature $T$ is $P_{1}$. Let $P_{2}$ be the total power emitted by another spherical black body of radius $\frac{R}{2}$ kept at temperature $2 T$. The ratio, $\frac{P_{1}}{P_{2}}$ is----------- (Give your answer upto two decimal places)
	{\exyear{GATE 2016}}
	\begin{answer}
		\begin{align*}
		P \propto A T^{4} \Rightarrow \frac{P_{1}}{P_{2}}&=\frac{R_{1}^{2} T_{1}^{4}}{R_{2}^{2} T_{2}^{4}}=\frac{R^{2} T^{4}}{\left(\frac{R}{2}\right)^{2}(2 T)^{4}}\\&=\frac{4}{16}=\frac{1}{4}=0.25
		\end{align*}
		So the correct answer is \textbf{0.25}
	\end{answer}
	\item Consider a system having three energy levels with energies $0,2 \varepsilon$ and $3 \varepsilon$, with respective degeneracies of 2,2 and 3 . Four bosons of spin zero have to be accommodated in these levels such that the total energy of the system is $10 \varepsilon$. The number of ways in which it can be done is------------
	{\exyear{GATE 2016}}
	\begin{answer}
		\begin{align*}
		\intertext{The system have energy $10 \varepsilon$, if out of four boson two boson are in energy level $2 \varepsilon$ and two boson are in energy level $3 \varepsilon$ and}W&=\prod_{i} \frac{\mid n_{i}+g_{i}-1}{\left|n_{i}\right| g_{i}-1}, n_{1}=2, g_{1}\\&=2\text{ and }n_{2}=2, g_{2}=3\\
		W&=\frac{\lfloor 2+2-1}{\lfloor 2-1} \times \frac{\mid 2+3-1}{\lfloor 2 \mid 3-1}\\&=3 \times 6=18
		\end{align*}
		So the correct answer is \textbf{18}
	\end{answer}
	\item Consider a triatomic molecule of the shape shown in the figure in three dimensions. The heat capacity of this molecule at high temperature (temperature much higher than the vibrational and rotational energy scales of the molecule but lower than its bond dissociation energies) is:
	{\exyear{GATE 2017}}
	\begin{figure}[H]
		\centering
		\includegraphics[height=3cm,width=4.5cm]{diagram-20210911(13)-crop}
	\end{figure}
	\begin{tasks}(4)
		\task[\textbf{a.}] $\frac{3}{2} k_{B}$
		\task[\textbf{b.}] $3 k_{B}$
		\task[\textbf{c.}] $\frac{9}{2} k_{B}$
		\task[\textbf{d.}] $6 k_{B}$
	\end{tasks}
	\begin{answer}
		If given molecules are at lower temperature i.e. atoms are attached to rigid rod then degree of freedom is 6 , so internal energy is $\frac{6 k_{B} T}{2}$, but at high temperature, vibration mode will active, so there are three extra vibration mode will active, so total energy
		\begin{align*}
		U&=3 k_{B} T+3 k_{B} T=6 k_{B} T\\
		C_{V}&=\left(\frac{\partial U}{\partial T}\right)_{V}=6 k_{B}
		\end{align*}
		So the correct answer is \textbf{Option (d)}
	\end{answer}
	\item Consider two particles and two non-degenerate quantum levels 1 and $2 .$ Level 1 always contains a particle. Hence, what is the probability that level 2 also contains a particle for each of the two cases:\\
	(i) when the two particles are distinguishable and (ii) when the two particles are bosons?
	{\exyear{GATE 2017}}
	\begin{tasks}(2)
		\task[\textbf{a.}] (i) $\frac{1}{2}$ and (ii) $\frac{1}{3}$
		\task[\textbf{b.}] (i) $\frac{1}{2}$ and (ii) $\frac{1}{2}$
		\task[\textbf{c.}] (i) $\frac{2}{3}$ and (ii) $\frac{1}{2}$
		\task[\textbf{d.}] (i) 1 and (ii) 0
	\end{tasks}
\begin{answer}
	(I): For distinguishable particle:\\\begin{figure}[H]
		\centering
		\includegraphics[height=1.2cm,width=5cm]{diagram-20210911(14)-crop.pdf}
	\end{figure}
	(II): For indistinguishable particle (Bosons):\\
	\begin{figure}[H]
		\centering
		\includegraphics[height=1.2cm,width=4.2cm]{diagram-20210911(16)-crop}
	\end{figure}
	So the correct answer is \textbf{Option (c)}
\end{answer}
	\item A microcanonical ensemble consists of 12 atoms with each taking either energy 0 state, or energy $\in$ state. Both states are non-degenerate. If the total energy of this ensemble is $4 \in$, its entropy will be----------- $k_{B}$ (up to one decimal place), where $k_{B}$ is the Boltzmann constant.
	{\exyear{GATE 2018}}
	\begin{answer}
		\begin{align*}
		\intertext{The number of ways having total energy $4 \in$, out of 12 atom is}
		&={ }^{12} C_{4}=\frac{12}{\lfloor 48}=\frac{12 \times 11 \times 10 \times 9}{4 \times 3 \times 2}=495\\
		\text{Hence, entropy, }S&=k_{B} \ln w=k_{B} \ln (495)=k_{B}(6.204)=6.204 k_{B}
		\end{align*}
		So the correct answer is \textbf{6.204}
	\end{answer}
	\item Consider two system $A$ and $B$ each having two distinguishable particles. In both the systems, each particle can exist in states with energies $0,1,2$ and 3 units with equal probability. The total energy of the combined system is 5 units. Assuming that the system $A$ has energy 3 units and the system $B$ has energy 2 units, the entropy of the system is $k_{B} \ln \lambda$. The value of $\lambda$ is----------
	{\exyear{GATE 2019}}
	\begin{answer}$\left. \right. $
		\begin{figure}[H]
			\centering
			\includegraphics[height=4cm,width=8cm]{diagram-20210911(20)-crop}
		\end{figure}
		\begin{align*}
		\Omega&=4 \times 3=12\\
		S&=\ln \Omega=k_{B} \ln 12\\
		\lambda&=12
		\end{align*}
	\end{answer}
	\item The partition function of an ensemble at a temperature $T$ is
	$$
	Z=\left(2 \cosh \frac{\varepsilon}{k_{B} T}\right)^{N}
	$$
	where $k_{B}$ is the Boltzmann constant. The heat capacity of this ensemble at $T=\frac{\varepsilon}{k_{B}}$ is $X N k_{B}$, where the value of $X$ is (up to two decimal places).
	{\exyear{GATE 2018}}
\begin{answer}
	\begin{align*}
	\text{	The partition function, }z&=\left[2 \cosh \left(\frac{\varepsilon}{k_{B} T}\right)\right]^{N}\\
	\text{The average energy, }\langle E\rangle&=k_{B} T^{2} \frac{\partial(\ln z)}{\partial T}\\
	&=\frac{N k_{B} T^{2}\left[2 \sinh \left(\frac{\varepsilon}{k_{B} T}\right)\right]\left(\frac{-\varepsilon}{k_{B} T^{2}}\right)}{2 \cosh \left(\frac{\varepsilon}{k_{B} T}\right)}\\&=-N \varepsilon \tanh \left(\frac{\varepsilon}{k_{B} T}\right)\\
	C&=\frac{d\langle E\rangle}{d T}=-N \varepsilon \sec h^{2}\left(\frac{\varepsilon}{k_{B} T}\right) \cdot\left(\frac{-\varepsilon}{k_{B} T^{2}}\right)\\
	\text{At }T&=\frac{\varepsilon}{k}, C=\frac{N \varepsilon^{2}}{k \cdot\left(\varepsilon^{2} / k^{2}\right)} \sec h^{2}(1)\\&=N k \sec h^{2}(1)=0.42 N k_{B}
	\end{align*}
		So the correct answer is \textbf{0.42}
\end{answer}
\end{enumerate}	
\colorlet{ocre1}{ocre!70!}
\colorlet{ocrel}{ocre!30!}
\setlength\arrayrulewidth{1pt}
\begin{table}[H]
	\centering
	\arrayrulecolor{ocre}
	\begin{tabular}{|p{1.5cm}|p{1.5cm}||p{1.5cm}|p{1.5cm}|}
		\hline
		\multicolumn{4}{|c|}{\textbf{Answer key}}\\\hline\hline
		\rowcolor{ocrel}Q.No.&Answer&Q.No.&Answer\\\hline
		1&\textbf{c} &2&\textbf{b}\\\hline
		3&\textbf{c} &4&\textbf{d} \\\hline
		5&\textbf{d} &6&\textbf{b} \\\hline
		7&\textbf{d}&8&\textbf{b}\\\hline
		9&\textbf{c}&10&\textbf{a}\\\hline
		11&\textbf{b} &12&\textbf{c}\\\hline
		13&\textbf{a}&14&\textbf{0.25}\\\hline
		15&\textbf{18}&16 &\textbf{d}\\\hline
		17&\textbf{c}2&18&\textbf{6.204}\\\hline
		19&\textbf{12}&20&\textbf{0.42}\\\hline
		
	\end{tabular}
\end{table}	