\chapter{Statistical Mechanics Problems}
\begin{enumerate}
	\item A system of $\mathrm{N}$ particles has only two allowed state $\mathrm{A}$ and $\mathrm{B}$. The probability for $\mathrm{A}$ is $\mathrm{P}$ and for $\mathrm{B}$ is 1-P? What is the probability for the system to be in macrostate defined by the distribution of $(r, N-r) ?$
	\begin{answer}
		 \begin{align*}
		 	\text{The probability of finding }&\mathrm{r}\text{ particle in state } \mathrm{A}=\mathrm{P}^{r}\\
		 	\text{ The probability of finding }&N-r\text{ particle in state } B=(1-P)^{N-r}\\
		 \text{	The Total no. of-ways in which}&\text{ r particle can be choosen from }N \text{- particle is }{ }^{N} C_{r}=N ! / r !(N-r) !\\
		 \text{ The probability in which r particle }&\text{are in state $A$ and $N-r$ particle in state $B$ is }=\frac{N !}{r !(N-r) !} P^{r}(1-P)^{N-r}
		 \end{align*}
	\end{answer}
	\item A one dimensiional random walker takes step to left or right with equal probability. The probability that the random walker starting from orign is back to orign after $N$ even number of step,is 
	 \begin{tasks}(2)
		\task[\textbf{a.}]$\frac{N !}{\left(\frac{N}{2}\right) !\left(\frac{N}{2}\right) !}\left(\frac{1}{2}\right)^{N}$
		\task[\textbf{b.}]$\frac{N!}{(\frac{N}{2})!(\frac{N}{2})!}$
		\task[\textbf{c.}]$2N!(\frac{1}{2})^{2N}$
		\task[\textbf{d.}] $N!(\frac{1}{2})^{N}$
	\end{tasks}
	\begin{answer}
		\begin{align*}
		\text{Probability} \quad P&=\frac{N!}{r~(N-r)}P(1-P)^{N-r}\\
		&=\frac{N !}{\left(\frac{N}{2}\right) !\left(\frac{N}{2}\right) !}\left(\frac{1}{2}\right)^{\frac{N}{2}}\left(1-\frac{1}{2}\right)^{N-\frac{N}{2}}=\frac{N!}{(\frac{N}{2})!(\frac{N}{2})!}\left( \frac{1}{2}\right) ^N
		\end{align*}
		Option \textbf{(a)} is correct
	\end{answer}
	\item Calculate the no. of microstates for a configuration of a system of $N$ distinguishable particles in which there are $n_1$ particles in a particle state $1 \& n_2$ particle in state $2,n_3$ particle .....$n_1$ particle in the $i$ th state.
	\begin{answer}
		\begin{align*}
		\text{Total No. of particle }&=N\\
		\text{No: of microstate for state}-1&={{N_C}_{n1}}\\
	\text{	No. of microstate for state }-2&=\mathrm{N}-\mathrm{n}_{1}{ }_{\mathrm{C}_{\mathrm{n}} 2}\\
\text{	No. of microstate for state }i^{\text {th }}&=N-n_{1}-n_{2} \ldots . n_{i}-1_{c_{n_{i}}}
\intertext{So\quad total number of microstate is}
\mathrm{N}_{\mathrm{C}_{\mathrm{n}_{1}}} \times \mathrm{N}-\mathrm{n}_{1} \mathrm{C}_{\mathrm{n}_{2}} &\times \mathrm{N}-\mathrm{n}_{1}-\mathrm{n}_{2} \mathrm{C}_{\mathrm{n}_{3}} \times \ldots \ldots \ldots \mathrm{N}-\mathrm{n}_{1}-\mathrm{n}_{2}-\mathrm{n}_{3}-\mathrm{ni}-1_{\mathrm{n}_{\mathrm{i}}}\\
=\frac{N !}{n_{1} !\left(N-n_{1}\right) !} &\times \frac{(N-n) !}{n_{2} !\left(N-n_{1}-n_{2}\right) !} \times \ldots \ldots \cdot \frac{\left(N-n_{1}-n_{2} \ldots n_{i}\right) !}{n_{i} !\left(N-n_{1}-n_{2} \ldots n_{i}\right) !}\\
=\frac{N !}{n_{i} !\left(N-n_{1}-n_{2} \ldots . n_{i}\right) !}&
-\text{ For distinguishable particle}\\
=\frac{1}{n_{i} !\left(N-n_{1}-n_{2} \ldots n_{i} !\right) !}& \rightarrow
\text{For indistinguishable particle}
		\end{align*}
	\end{answer}
	\item Four distinguishable coins are tossed a large no. of time write down the different microstate which may be observed \& the macrostate into which they would fall. Give the probability of the most probable macrostate.\\
	\begin{center}
	\begin{tabular}{|p{2.5cm}|p{2.7cm}|p{2.5cm}|p{2.7cm}|p{2.5cm}|}
		\hline Macrostate & Microstate coins having head up & Microstate coins having tail up & No.of microstate & Probability \\
		\hline $\mathrm{n}_{1}=4, \mathrm{n}_{2}=0$ & $\mathrm{a} \mathrm{b} \mathrm{c} \mathrm{d}$ & $-$ & 1 & $\frac{1}{16}$ \\
		\hline & $\mathrm{abc}$ & $\mathrm{d}$ & & \\
		$\mathrm{n}_{1}=3, \mathrm{n}_{2}=1$ & $\mathrm{bcd}$ & $\mathrm{a}$ & 4 & $\frac{4}{16}$ \\
		& $\mathrm{cda}$ & $\mathrm{b}$ & & \\
		& $\mathrm{dab}$ & $\mathrm{c}$ & & \\
		\hline
		 &$\mathrm{ab}$&$\mathrm{cd}$ & &\\
		 &$\mathrm{ac}$&$\mathrm{bd}$ & &\\
		 $n_1=2,n_2=3$&$\mathrm{ad}$&$\mathrm{bc}$ &6&$\frac{6}{16}$\\
		 &$\mathrm{bc}$&$\mathrm{ad}$ & &\\
		 &$\mathrm{bd}$&$\mathrm{ac}$ & &\\
		 &$\mathrm{cd}$&$\mathrm{ab}$ & &\\
		 \hline
		  &$\mathrm{a}$&$\mathrm{bcd}$ & &\\
		$n_1=1,n_2=3$&$\mathrm{b}$&$\mathrm{acd}$ &4 &$\frac{4}{16}$\\
		  &$\mathrm{c}$&$\mathrm{abd}$ & &\\
		 &$\mathrm{d}$&$\mathrm{abc}$ & &\\
		  \hline
		  	$n_1=0,n_2=1$&$-$&$\mathrm{abcd}$ &1 &$\frac{1}{16}$\\
		  	 \hline
	\end{tabular}
\end{center}
\begin{answer}
	\begin{align*}
	\text{Total no. of microstate }&=6\\
	\text{The probability of most probable state }&=6/16
	\end{align*}
\end{answer}
	\item An isolated system consist of two non-interacting spin $-1/2$ particles a \& b fixed in space \& kept in magnetic field B. Find out the total number of microstates allowed in the system.

	\begin{answer}
		$\left. \right. $\\
		\begin{table}[H]
	\begin{tabular}{|p{2.5cm}|p{2.5cm}|p{2.5cm}|p{2.5cm}|p{2.5cm}|}
		\hline
		Systemstate or macrostate  &Particlestate or Microstate & Magnetic moment &Energy  &No.of microstate  \\ \hline
		1& U \quad U&$2\mu_0$ & $-2\mu_0 B$& 1   \\ \hline
	 \multirow{ 2}{*}{1} & U \quad D & 0 & 0 & \multirow{ 2}{*}{2}\\ 
		 & U \quad D & 0 & 0 &  \\ \hline
		 3&D\quad D&$-2\mu_0$& $+-2\mu_0B$&1\\
		 \hline
	\end{tabular}
\end{table}
\begin{align*}
\intertext{Total no. of microstate $=4$ for $\operatorname{Spin}-\frac{1}{2}$ two particle system no. of accessible microstate corresponding to energy.}
E=0\text{ is }2&
\intertext{The no. of microstate for $\mathrm{N}$ no. of Spin S particle}
(2 S+1)^{N}&
\intertext{The probability of getting a macrostate in which there are r particle out of N Spin $1 / 2$ particle in spin up state}
\mathrm{N}_{\mathrm{C}_{\mathrm{r}}} \times \frac{1}{2^{\mathrm{N}}}
\end{align*}
	\end{answer}
\item We have equal amount of two identical ideal gases at the same temperature $T$ but at different pressure $P_1$ and $P_2$ in two containers of volume $V_1$ and $V_2$ respectively. The containers are connected. Find the change in entropy (The temperature after mixing also ramains same)
\begin{answer}
	\begin{align*}
	S&=Nk_B \ \ell n\left( \frac{V}{N}\right)+\frac{3}{2}Nk_B \left[\frac{5}{3}+\ell n \left(\frac{2\pi mk_BT}{h^2} \right) \right] \\
	\sum\limits_{i=1}^{2}Si&=N_1k_B\ \ell n\frac{V_i}{N}+\frac{3}{2}N_1k_B \left[\frac{5}{3}+\ell n \left(\frac{2\pi mk_BT}{h^2} \right) \right] \\
	&=Nk_B\ \ell n\frac{V_1}{N}+\frac{3}{2}Nk_B \left[\frac{5}{3}+\ell n \left(\frac{2\pi mk_BT}{h^2} \right) \right]+\\
		&=Nk_B\ \ell n\frac{V_2}{N}+\frac{3}{2}Nk_B \left[\frac{5}{3}+\ell n \left(\frac{2\pi mk_BT}{h^2} \right) \right]+\\
		&\left\{\begin{array}{l}P_{1} V_{1}=N k_{B} T \Rightarrow \frac{V_1}{N}=\frac{k_B T}{P_1}\\ P_{2} V_{2}=N k_{B} T \Rightarrow \frac{V_2}{N}=\frac{k_B T}{P_2} \end{array}\right.\\
		\sum\limits_{i=1}^{2}Si&=Nk_B\ \ell n\frac{k_B T}{P_1}+\frac{3}{2}Nk_B \left[\frac{5}{3}+\ell n \left(\frac{2\pi mk_BT}{h^2} \right) \right] +\\
		&=Nk_B\ \ell n\frac{k_B T}{P_2}+\frac{3}{2}Nk_B \left[\frac{5}{3}+\ell n \left(\frac{2\pi mk_BT}{h^2} \right) \right]
		\intertext{\textbf{Total entropy:}}
		S_T&=2Nk_B\ \ell n\frac{V_1+V_2}{2N}+\frac{3}{2}\times 2Nk_B \left[\frac{5}{3}+\ell n \left(\frac{2\pi mk_BT}{h^2} \right) \right]
		\intertext{Change in entropy}
		\Delta S&=S_T-\sum\limits_{i=1}^{2}Si=2Nk_B\ \ell n\frac{V_1+V_2}{2N}-Nk_B \ \ell n\frac{k_B T}{P_1}-NK \ \ell n\frac{k_B T}{P_2}
		\intertext{Let after mixing pressure $=P$ and after mixing total No of particle $=N+N=2 N$}
		\end{align*}
		\begin{align*}
		&\mathrm{P}\left(\mathrm{V}_{1}+\mathrm{V}_{2}\right)=2 \mathrm{Nk}_{\mathrm{B}} \mathrm{T}\hspace{6cm}\mathrm{P}\left(\mathrm{V}_{1}+\mathrm{V}_{2}\right)=2 \mathrm{Nk}_{\mathrm{B}} \mathrm{T}\\
		&\frac{\left(\mathrm{V}_{1}+\mathrm{V}_{2}\right)}{2 \mathrm{~N}}=\frac{\mathrm{k}_{\mathrm{B}} \mathrm{T}}{\mathrm{P}}\hspace{6.5cm}\frac{1}{\mathrm{P}}=\frac{\mathrm{V}_{1}+\mathrm{V}_{2}}{2 \mathrm{Nk}_{\mathrm{B}} \mathrm{T}}=\frac{\mathrm{V}_{1}}{2 \mathrm{Nk}_{\mathrm{B}} \mathrm{T}}+\frac{\mathrm{V}_{2}}{2 \mathrm{Nk}_{\mathrm{B}} \mathrm{T}}\\
		&\Delta \mathrm{S}=2 \mathrm{Nk}_{\mathrm{B}} \ \ell n \frac{\mathrm{k}_{\mathrm{B}} \mathrm{T}}{\mathrm{P}}-\mathrm{Nk}_{\mathrm{B}} \ell \mathrm{n} \frac{\mathrm{k}_{\mathrm{B}} \mathrm{T}}{\mathrm{P}_{1}}-\mathrm{Nk}_{\mathrm{B}}  \ \ell n \frac{\mathrm{k}_{\mathrm{B}} \mathrm{T}}{\mathrm{P}_{2}}\hspace{2cm}\frac{1}{\mathrm{P}}=\frac{1}{2}\left(\frac{1}{\mathrm{P}_{1}}+\frac{1}{\mathrm{P}_{2}}\right)\\
		&=N k_{\mathrm{B}}  \ \ell n \left(\frac{\mathrm{k}_{\mathrm{B}} \mathrm{T}}{\mathrm{P}}\right)^{2}-\mathrm{Nk}_{\mathrm{B}}  \ \ell n \frac{\mathrm{k}_{\mathrm{B}} \mathrm{T}}{\mathrm{P}_{1}}-\mathrm{Nk}_{\mathrm{B}}  \ \ell n \frac{\mathrm{k}_{\mathrm{B}} \mathrm{T}}{\mathrm{P}_{2}}\hspace{2cm}P=\frac{2 P_{1} P_{2}}{P_{1}+P_{2}}\\
		&=N k_{B}  \ \ell n \frac{\left(\frac{k_{B} T}{P}\right)^{2}}{\left(\frac{k_{B} T}{P_{1}}\right)\left(\frac{k_{B} T}{P_{2}}\right)}\\
		&=N k_{B}  \ \ell n \left[\frac{\left(k_{B} T\right)^{2}\left(P_{1}+P_{2}\right)^{2}}{\left(2 P_{1} P_{2}\right)^{2}} \times \frac{P_{1} P_{2}}{\left(k_{B} T\right)^{2}}\right]\\
		\Delta \mathrm{S}&=\mathrm{Nk}_{\mathrm{B}} \ell \mathrm{n} \frac{\left(\mathrm{P}_{1}+\mathrm{P}_{2}\right)^{2}}{4 \mathrm{P}_{1} \mathrm{P}_{2}}\\
		\Delta \mathrm{S}&=\mathrm{Nk}_{\mathrm{B}}  \ \ell n \frac{\left(\mathrm{P}_{1}+\mathrm{P}_{2}\right)^{2}}{4 \mathrm{P}_{1} \mathrm{P}_{2}}\quad\text{ Here we get }\Delta S>0\\
		\intertext{(We are considering identical particle. So it is a reversible case only when the particle density is same otherwise for identical gas also we will get $\Delta S>0$)}
	\end{align*}
\end{answer}
	\item Calculate the number of microstate for a free particle in 3-dimension that have momentum $p$.
	\begin{answer}
		\begin{align*}
		\mathrm{d} \tau&=\mathrm{dx} d y d z d p_{x} d p_{y} d p_{z}\\
	\textbf{	Allowed Phase Space Volume}=&\int \mathrm{d} \tau=\int \mathrm{dxdydz} \int \mathrm{dp}_{\mathrm{x}} \mathrm{dp}_{\mathrm{y}} \mathrm{dp}_{\mathrm{z}}\\
	&=V \int p^{2} d p \int_{0}^{\pi} \sin \theta d q \int_{0}^{2 \pi} d \phi=V \frac{4}{3} \pi p^{3}\\
\text{	No. of microstate }&=\frac{\mathrm{V}}{\mathrm{h}^{3}} \times \frac{4}{3} \pi \mathrm{p}^{3}
\intertext{Note : for 2. dimension.}
\int \mathrm{dx}&=\int \mathrm{dx} \int \mathrm{dy} \int \mathrm{dp}_{\mathrm{x}} \int \mathrm{dp}_{\mathrm{x}}\\
&=\int \mathrm{d}^{2} \mathrm{q} \cdot \mathrm{d}^{2} \mathrm{p} \quad=\mathrm{A} \cdot \pi \mathrm{p}^{2} \quad\{\text{ where }\mathrm{A}= \text{area} \}
\intertext{Since wave polarize in 2 dimension}
\int \mathrm{dr}&=2 \mathrm{~A} \pi \mathrm{p}^{2}
\intertext{If momentum lies between $p$ and $p+d p$}
\therefore\text{ No. of microstate }&=\frac{2 \mathrm{~A}}{\mathrm{~h}^{2}} 2 \pi \mathrm{pdp}=\frac{4 \mathrm{~A} \pi}{\mathrm{h}^{2}} \mathrm{pdp}
\intertext{If momentum lies between $\mathrm{p}$ and $\mathrm{p}+\mathrm{dp}$ then no. of microstate $=\frac{\mathrm{V}}{\mathrm{h}^{3}} 4 \pi \mathrm{p}^{2} \mathrm{dp}$}
\text{Density of state }&=\frac{\text { no. of state }}{\text { Volume }}\text{ or} \frac{\text { no. of state }}{\text { energy int erval }}
		\end{align*}
	\end{answer}
	\item Calculate the number of microstates accessible to the photon having frequency between $\nu$ and $\nu+d\nu$ confined to a 3 dimentional cavity of volume $V$
	 \begin{answer}
	 	\begin{align*}
	 	\intertext{No. of microstate in frequency range 0 to $\nu$}
	 	\Omega&=\frac{V}{h^{3}} \frac{4}{3} \pi(2 m E)^{3 / 2} \hspace{3cm}
	 	E=\frac{P^{2}}{2 m} \Rightarrow P=\sqrt{2 m} E\\
	 	&=\frac{\mathrm{V}}{\mathrm{h}^{3}} \frac{4}{3} \pi\left(\mathrm{P}^{2}\right)^{3 / 2}\hspace{3cm} \mathrm{P}=\frac{\mathrm{E}}{\mathrm{C}}=\frac{\mathrm{h\nu}}{\mathrm{C}}\\
	 	\Omega&=\frac{\mathrm{V}}{\mathrm{h}^{3}} \frac{4}{3} \pi \mathrm{P}^{3}=\frac{\mathrm{V}}{\mathrm{h}^{3}} \frac{4}{3} \pi\left(\frac{\mathrm{h\nu}}{\mathrm{c}}\right)^{3}=\frac{4}{3} \frac{\pi \mathrm{V\nu}^{3}}{\mathrm{c}^{3}}
	 	\intertext{No. of microstates in the frequency range $\nu$ and $v+d \nu$ is}
	 	\mathrm{d} \Omega&=\frac{4}{3}\  \frac{\pi \mathrm{V}}{\mathrm{c}^{3}} 3 \mathrm{\nu}^{2} \mathrm{~d} \mathrm{v}=\frac{4 \pi \mathrm{V}}{\mathrm{c}^{3}} \mathrm{\nu}^{2} \mathrm{~d} \mathrm{\nu}
	 	\intertext{Density of state}
	 	g(E)&=\text{ no. of states per unit energy range}\\
	 \text{	we have }\quad
	 	\Omega&=\frac{V}{h^{3}} \ \frac{4}{3} \ \pi(2 m E)^{3 / 2}\\
	 	\mathrm{d} \Omega&=\frac{\mathrm{V}}{\mathrm{h}^{3}} \ \frac{4}{3}\  \cdot \frac{3}{2} \pi(2 \mathrm{mE})^{1 / 2} 2 \mathrm{~m} \mathrm{dE}\\
	 	&=\frac{V}{h^{3}} \cdot 2 \pi(2 m)^{3 / 2} E^{1 / 2} d E\\
	 	&\left[\mathrm{E}=\frac{\mathrm{P}^{2}}{2 \mathrm{~m}} \Rightarrow 2 \mathrm{~m}\frac{P^2}{E}-\frac{E^2}{c^2 E}=\frac{E}{c^2}\right]\\
	 	&=\frac{V}{h^{3}} 2 \pi \left(\frac{E}{c^2} \right) ^{\frac{3}{2}}E^{\frac{1}{2}}dE\\
	 	&=\frac{V}{h^3}\ \frac{2\pi}{c^3}E^2dE\\
	 	g(E)&=\frac{d\Omega}{dE}=\frac{2\pi VE^2}{h^3 c^3}
	 	\end{align*}
	 \end{answer}
 \section{Canonical Ensemble}
	\item Show that the partition fuction of two independent (non-interacting)
	system $i$ and $j$ is given by 
	$$Z_{ij}=Z_i\times Z_j$$
	\begin{answer}
		\begin{align*}
		\intertext{We know that}
		P_{i}&=\frac{e^{-\beta E_{i}}}{Z_{i}}, P_{j}=\frac{e^{-\beta E_{j}}}{Z_{j}}\\
		E&=E_{i}+E_{j}, P_{i j}=\frac{e^{-\beta\left(E_{i}+E_{j}\right)}}{Z_{i j}}\\
		P_{i j}&=\frac{e^{-\beta E}}{Z_{i i}}=\frac{e^{-\beta\left(E_{i}+E_{j}\right)}}{Z_{i i}}=\frac{e^{-\beta E_{i}} \times e^{-\beta E_{j}}}{\dot{Z}_{i i}}\\
		\mathbf{P}_{\mathrm{ij}}&=\mathrm{P}_{\mathrm{i}} \times \mathrm{P}_{\mathrm{j}}\\
		\frac{e^{-\beta E_{i}} \times e^{-\beta E_{j}}}{Z_{i j}}&=\frac{e^{-\beta E_{i}}}{Z_{i}} \times \frac{e^{-\beta E_{j}}}{Z_{i}}\\
		\text{So  }\quad Z_{\mathrm{ij}}&=Z_{\mathrm{i}} \times Z_{\mathrm{j}}\\
	\text{	In general }Z_{i j k} \cdots \cdots&=Z_{i} \times Z_{j} \times Z_{k} \times \cdots \\
	\text{Total partition function }Z&=Z_{\text {Rotational motion }} \times Z_{\text {Translational motion }} \times Z_{\text {Vibrational motion }} \times \ldots \ldots \ldots \ldots \ldots
		\end{align*}
	\end{answer}
	\item A system consist of three independent particles localised in space. Each particle have two states of energy $O$ and $E$. When the system is in thermal equilibrium with a heat bath at temperature $T$. Calculate its partition function? 
 	\begin{answer}
 		Method :-1
 		\begin{align*}
 		\mathrm{Q}_{\mathrm{N}}(\mathrm{V}, \mathrm{T})&=\sum_{\mathrm{r}} \mathrm{g}_{\mathrm{r}} \mathrm{e}^{-\beta \mathrm{E}_{\mathrm{r}}}\\
 		Q_1 (V,T)=\sum e^{\beta E_r}=1+e^{\beta E}\colorbox{red}{Not completed}
 		\end{align*}
 	\end{answer}
	\item The partition function for two Bose particle each of which can occupy any of the enetgy level $0$ and $E$ 
	 \begin{tasks}(2)
		\task[\textbf{a.}]$1+e^{-2 E / K T}+2 e^{-E / K T}$
		\task[\textbf{b.}] $1+e^{-2 E / K T}+e^{-E / K T}$
		\task[\textbf{c.}]$2 \mathrm{e}^{-2 \mathrm{E} / \mathrm{KT}}+\mathrm{e}^{-\mathrm{E} / \mathrm{KT}}$
		\task[\textbf{d.}] $e^{-2 E / K T}+e^{-E / K T}$
	\end{tasks}
	\begin{answer}$\left. \right. $\\
		\renewcommand*{\arraystretch}{2}
		\begin{tabular}{llllll} 
			& $\mathbf{0}$ & $\mathbf{E}$ & Total energy & \multicolumn{2}{c}{ degeneracy } \\
			1 & aa & 0 & 0 & 1 & $\left(g_{1}\right)$ \\
			2 & a & a & E & 1 & $\left(g_{2}\right)$ \\
			3 & 0 & a & 2 E & 1 & $\left(g_{3}\right)$
		\end{tabular}
	\begin{align*}
	\mathrm{Q}_{\mathrm{N}}(\mathrm{V}, \mathrm{T})&=\sum_{\mathrm{r}} \mathrm{g}_{\mathrm{r}} \mathrm{e}^{-\beta \mathrm{E}_{\mathrm{r}}}=\mathrm{g}_{1} \mathrm{e}^{-\beta \mathrm{E}_{1}}+\mathrm{g}_{2} \mathrm{e}^{-\beta \mathrm{E}_{2}}+\mathrm{g}_{3} \mathrm{e}^{-\beta \mathrm{E}_{3}}\\
	&=1+\mathrm{e}^{-\beta E}+\mathrm{e}^{-2 \beta E}
	\end{align*}
	Correct answer is option \textbf{(b)}
	\end{answer}
	\item The partition function of single gas molecule is $Z\alpha$. The partition fuction of $N$ such non-interacting gas molecule is given by
	 \begin{tasks}(2)
		\task[\textbf{a.}]$\frac{(Z \alpha)^{N}}{N !}$
		\task[\textbf{b.}]$(Z \alpha)^{N}$
		\task[\textbf{c.}] $\mathrm{NZ} \alpha$
		\task[\textbf{d.}] $(Z \alpha)^{N} / N$
	\end{tasks}
	\begin{answer}
		\begin{align*}
		\text{For indistinguishable particle }&\frac{(Z \alpha)^{N}}{N !}\\
		\text{for distinguishable particle }&(Z \alpha)^{N}
		\end{align*}
	\end{answer}
	\item A system has energy level $E_0,2E_0,3E_0....$ where the excited state are triply degenerate. Four non-interacting Bosons are placed in this system. If the total energy of these Bosons is $5E_0$. The number of microstate is 
	 \begin{tasks}(2)
		\task[\textbf{a.}]2
		\task[\textbf{b.}]3
		\task[\textbf{c.}]4
		\task[\textbf{d.}]5
	\end{tasks}
	\begin{answer}
		\begin{align*}
		\text{Total energy }&=5 \mathrm{E}_{0}
		\intertext{For this 3 bosons must be in $E_0$ state and $1$ in $2E_0$ state.}
		\end{align*}
		\begin{figure}[H]
			\centering
			\includegraphics[height=4cm,width=4cm]{SM-problem-01}
		\end{figure}
		Correct answer is option \textbf{(b)}
	\end{answer}
	\item An ensemble of quantum harmonic oscillator is kept at a finite temperature $T$ 
	$$\mathrm{T}=\frac{1}{\mathrm{k}_{\mathrm{B}} \beta} \quad
 k_B	\text{-Boltzmann constant}$$
 The partition function of a single oscillator with energy $\left( n+1/2\right) \hbar\omega$ is given by
  \begin{tasks}(2)
 	\task[\textbf{a.}] $Z=\frac{e^{-\beta \hbar \omega / 2}}{1-e^{-\beta \hbar \omega}}$
 	\task[\textbf{b.}]$Z=\frac{e^{-\beta \hbar \omega / 2}}{1+e^{-\beta \hbar \omega}}$
 	\task[\textbf{c.}]$Z=\frac{1}{1-e^{-\beta \hbar \omega}}$
 	\task[\textbf{d.}] $Z=\frac{1}{1+e^{-\beta \hbar \omega}}$
 \end{tasks}
	\begin{answer}
		\begin{align*}
		\text{(i) }Q_{N}\left(V_{1} T\right)&=\sum_{r} e^{-\beta E}=\sum_{n} e^{-\beta(n+1 / 2) \hbar w}\\
		&=e^{-\beta \hbar \omega / 2}+e^{-3 / 2 \beta \hbar \omega}+e^{-5 / 2^{\beta \hbar \omega}}+\ldots \ldots\\
		&=\mathrm{e}^{-\beta \hbar \omega / 2}\left(1+\mathrm{e}^{-\beta \hbar \omega}+\mathrm{e}^{-\beta \hbar \omega}+\ldots \ldots\right.\\
		&=\mathrm{e}^{-\beta \hbar \omega / 2}\left(\frac{1}{1-\mathrm{e}^{-\beta \hbar \omega}}\right)\text{ use G.P, formula}\hspace{0.5cm}\left\{\begin{array}{lll}R=1+r+\pi^{2} \ldots \ldots \ldots \ldots \ldots \infty & \text { G.P. } \\ \sum_{0}^{\infty} R=\frac{1}{1-r} & r<1 &  \end{array}\right\}
		\end{align*}
	\end{answer}
	\item The average number of energy quanta of the oscillator is given by 
	 \begin{tasks}(2)
		\task[\textbf{a.}] $\langle\mathrm{n}\rangle=\frac{1}{\mathrm{e}^{\beta \hbar \omega}-1}$
		\task[\textbf{b.}]$\langle\mathrm{n}\rangle=\frac{\mathrm{e}^{-\beta \hbar \omega}}{\mathrm{e}^{\beta \hbar \omega}-1}$
		\task[\textbf{c.}]$\langle\mathrm{n}\rangle=\frac{1}{\mathrm{e}^{\beta \hbar \omega}+1}$
		\task[\textbf{d.}] $\langle n\rangle=\frac{e^{-\beta \hbar \omega}}{e^{\beta \hbar \omega}+1}$
	\end{tasks}
	\begin{answer}
		\begin{align*}
		\intertext{(i) The average energy of a quantum oscillator is given by}
		\mathrm{U}&=\mathrm{N}\left[1 / 2 \hbar \omega+\frac{\hbar \omega}{\mathrm{e}^{\beta \hbar \omega-1}}\right]
		\intertext{If we neglect the zero point energy the average energy bocome}
	\mathrm{U}&=\frac{\mathrm{N} \hbar \omega}{\mathrm{e}^{\beta \hbar \omega}-1}\\
	\text{Avarege number of energy quanta}&=\frac{U}{N\hbar \omega}\\\colorbox{red}{Not completed}
		\end{align*}
	\end{answer}
	\item In a particular salt among $n$ atoms each atom has a spin $1/2$ and is associated with a magnetic dipolemoment $\mu_B$. The distance between the magnetic atoms is large enough to make the interaction between them negligible. In otherwords our system is an idealised spin system. The salt is placed in an external magnetic field $B$. Find the partition function of the individual atom and that for the salt 
	\begin{answer}
		\begin{align*}
		\mathrm{Q}_{1}(\mathrm{~V}, \mathrm{~T})&=\sum_{\mathrm{r}} \mathrm{e}^{-\beta \mathrm{E}_{\mathrm{r}}}\qquad E=\mu_B\ B =\text{for} S=-\frac{1}{2}\qquad \left\lbrace\text{ where }\mu_B=\text{ Bohr magneton }\right\rbrace \\
		&=e^{-\beta\mu_B\ B}+e^{\beta\mu_B\ B}
		\intertext{Since the salt consist of $N$ number of particles}
		Q_N(V,T)&=[Q_1(V,T)]^N=[e^{-\beta\mu_B\ B}+e^{\beta\mu_B\ B}]^N \rightarrow \text{distinguishable}\\
		\colorbox{red}{Not completed}
		\end{align*}
	\end{answer}
	\item Consider a system of two identical particles which may occupy any of the three levels. The lowest energy state $\varepsilon_0=0$ is double degenerates. The system is in thermal equilibrium at temperature T. Determine the partition function and everage energy of the system if particles obey
	(i) M-B statistics\\
	(ii) B-E statistics\\
	(iii) F-D statistics.
	\begin{answer}
		\begin{align*}
		\text{(i) For $M-B$ }&\text{statistics, $g_{i}=16$}\\
		\mathrm{Q}_{\mathrm{N}}(\mathrm{V}, \mathrm{T})&=\Sigma g_{r} e^{-\beta E_{r}}=4+4 e^{-\beta \epsilon}+5 e^{-2 \beta \epsilon}+2 e^{-3 \beta \epsilon}+e^{-4 \beta \epsilon}\\
		\langle\mathrm{E}\rangle&=-\frac{\partial}{\partial \beta} \ln \mathrm{Q}_{\mathrm{N}}(\mathrm{V}, \mathrm{T})=\frac{\in\left[4 e^{-\beta \epsilon}+10 e^{-2 \beta \epsilon}+6 e^{-3 \beta \epsilon}+4 e^{-4 \beta \epsilon}\right]}{Q_{N}(V, T)}\\
		\langle E\rangle&=\frac{\in\left[2 e^{-\beta \epsilon}+6 e^{-2 \beta \epsilon}+3 e^{-3 \beta \epsilon}+4 e^{-4 \beta \epsilon}\right]}{Q_{N}(V, T)}
		\intertext{(iii) For F-D statistics:}
		Q_{N}(V, T)&=1+2 e^{-\beta \epsilon}+2 e^{-\beta \epsilon}+e^{-3 \beta \epsilon}\\
		\langle E\rangle&=\frac{\in\left(2 e^{-\beta \epsilon}+4 e^{-2 \beta \epsilon}+3 e^{-3 \beta \epsilon}\right)}{Q_{N}(V, T)}
		\end{align*}
	\end{answer}
	
	
	
	
	
	
	
	
	
	
	
	
	
\end{enumerate}