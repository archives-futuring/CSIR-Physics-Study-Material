\chapter{Elementary Principles}
\section{Mechanics of a Particle}
\textbf{Linear Momentum}\\
Let $\textbf{r}$ be the radius vector of a particle from some given origin and $\textbf{v}$ its vector velocity.
$$\textbf{v}=\frac{d\textbf{r}}{dt}$$
The linear momentum $\textbf{p}$ of the particle is 
$$\textbf{p}=m\textbf{v}$$
Newton's second law of motion  states that there exist frames of reference in which the motion of the particle is described by the differential equation
$$\textbf{F}=\frac{d\textbf{p}}{dt}\equiv\dot{\textbf{p}}$$
$$\textbf{F}=\frac{d}{dt}(m\textbf{v})$$
If the mass of particle is constant
$$\textbf{F}=m\frac{d\textbf{v}}{dt}=m\textbf{a}$$
Where $\textbf{a}$ is the vector acceleration of the particle
The equation of motion is thus a differential equation of second order, assuming $\textbf{F}$ does not depend on higher-order derivatives.\\\\
\textit{Conservation theorem for the Linear Momentum of a Particle: If the total force $\textbf{F}$ is zero, then $\dot{\textbf{p}}$, is conserved.}
\\\\
\textbf{Linear Momentum}\\	
The angular momentum of the particle about point $O$, denoted by $\textbf{L}$, is defined as
