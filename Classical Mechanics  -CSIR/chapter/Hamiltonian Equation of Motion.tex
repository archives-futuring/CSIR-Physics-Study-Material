\chapter{Hamiltonian Equation of Motion}
 Hamiltonian method providing a framework for theoretical basis for further developments like Hamiltonian Jacobi theory, perturbation approaches and chaos. Outside classical mechanics Hamiltonian formulation provides much of the language with which present-day statistical mechanics and quantum mechanics is constructed. Throughout the chapter we are assuming mechanical systems are holonomic and the forces are monogenic.
 \begin{itemize}
 	\item In Hamiltonian formulation there can be no constraint equation among the coordinates.
 	\item If $n$ coordinates are not independent, a reduced set of $m$ coordinates, with $m<n$, must be used for the formulation of the problem
 \end{itemize}
\section{Hamiltonian Formulation}
\begin{itemize}
	\item Describe the motion in terms of first order equations of motion
	\item Number of initial conditions determining the motion are stil be $2n$
	\item $2n$ independent first order equations expressed in terms of $2n$ independent variables
	\item $2n$ equations of motion describe behavior of the system point in a phase space
	\item Out of $2n$ independent quantities half of them are $n$ generalized coordinates and other half set to be the generalized or conjugate momenta $p_i$
	\begin{equation}
	  p_i=\frac{\partial L(q_j,\dot{q}_j, t)}{\partial \dot{q}_j}
	\end{equation}
	where $j$ index shows the set of $q$'s and $\dot{q}$'s
	\item The quantities $(p,q)$ are known as the canonical variables
\end{itemize}
\section{Legendre Transformation}
Consider a function of only two variables $f(x,y)$ so that a differential of $f$ has the form
$$df=udx+vdy$$
where
$$u=\frac{\partial f}{dx},\quad v=\frac{\partial f}{dy}$$
To change the basis of description from $x,y$ to a new distinct set of variables $u,y$ so that differential quantities are expressed in terms of $du$ and $dy$
\begin{align*}
\intertext{Let $g$ be a function $u$ and $y$ defined by the equation}
g&=f-ux
\intertext{A differential of $g$ is then given as}
dg&=df-udx-xdu
\intertext{or}
dg&=vdy-xdu
\intertext{and we get}
x=\frac{-\partial g}{\partial u},&\quad v=\frac{\partial g}{\partial y}
\end{align*}
\section{Hamilton Equation of Motion}
Mathematically the transition from Lagrangian to Hamiltonian formulation corresponds to changing the variables in our mechanical functions from $(q, \dot{q}_j,t)$ to $(q,p,t)$ when $p$ is related to $q$ and $\dot{q}$ by
$$p_i=\frac{\partial L(q_j,\dot{q}_j,t)}{\partial \dot{q}_i}$$
The proceedure for switching variables in this manner is provided by the 'Legendre  transformation'.
\begin{align}
\text{Consider Lagrangian }&\ L(q,\dot{q},t)\ \text{ then}\notag\\
dL=\frac{\partial L}{\partial q_i}dq_i&+\frac{\partial L}{\partial \dot{q}_1}+\frac{\partial L}{\partial t}dt\\
\text{The canonical }&\text{momentum  was defined as }p_i=\frac{\partial L}{\partial \dot{q}_i}\notag\\
\text{Substituting it in }&\text{Lagrange equation we obtain}
\dot{p}_i=\frac{\partial L}{\partial {q}_i}\notag\\
\therefore dL&=\dot{p}_idq_i+p_id\dot{q}_i+\frac{\partial L}{\partial t}dt
\intertext{The Hamiltonian $H(q,p,t)$ is generated by the Legendre transformation}
H(q,p,t)&=\dot{q}_ip_i-L(q,\dot{q},t)
\intertext{and has the differential}
dH&=\dot{q}_1dp_i-\dot{p}_1dq_1-\frac{\partial L}{\partial t}dt
\intertext{Where the term $p_id\dot{q}_i$ removed by Legendre transformation since $dH$ can also written as }
dH&=\frac{\partial H}{\partial q_i}dq_1+\frac{\partial H}{\partial p_i}dp_i+\frac{\partial H}{dt}dt
\intertext{We obtain $2n+1$ relations}
&\left.\begin{array}{rl}\dot{q}_{i} & =\frac{\partial H}{\partial p_{i}} \label{HM-07}\\\\ -\dot{p}_{i} & =\frac{\partial H}{\partial q_{i}}\end{array}\right\}\\\notag\\
&-\frac{\partial L}{\partial t}=\frac{\partial H}{\partial t}\notag
\end{align}
Equation $\ref{HM-07}$ are known as the canonical equations of Hamiltonian
\section{Steps to construct Hamiltonian}
Hamiltonian for each problem must be constructed via the Lagrangian formulation.
\begin{enumerate}
	\item With chosen set of generalized coordinates, $q_i$ the Lagrangian $L(\dot{q},\dot{q}_i,t)$=T-V is constructed
	\item The conjugate momenta are defined as function of $q_i,\dot{q}_i$ and $t$ by equation
	$$p_i=\frac{\partial L}{\partial \dot{q}_i}$$
	\item Legendre transformation used to form the Hamiltonian. At this stage we have some mixed functions of $q_i,\dot{q}_i,p_i$ and $t$
	\item $p_i=\frac{\partial L}{\partial \dot{q}_i}$ then converted to obtain $\dot{q}_i$ as functions of $(q,p,t)\ (ie\quad \dot{q}_i=\frac{\partial H}{\partial p_i})$
	\item The result of the previous steps are then applied to eliminate $\dot{q}_i$ from $H$ so to express it solely as a function of $(q,p,t)$
\end{enumerate}
\begin{note}
	In many problems Lagrangian is the sum of functions each homogeneous in the generalized velocities of degree $0,1$ and $2$ respectively in that are
	$$ H=\dot{q}_1p_i-L=\dot{q}_ip-[L_0(q,t)+L_i(q,t)\dot{q}_k+L_1(q_i,t)\dot{q}_k\dot{q}_m]$$
	If equations defining generalized coordinates don't depend on time then $L_2 \dot{q}_k\dot{q}_m=T(K.E)$\\
	If forces are derivable from a consevative potential $V$ (ie work is independent of path), then $L_0=-V$
	When both there conditions are satisfied the Hamiltonian is automatically the total energy
	$$H=T+V=
	E$$
\end{note}





