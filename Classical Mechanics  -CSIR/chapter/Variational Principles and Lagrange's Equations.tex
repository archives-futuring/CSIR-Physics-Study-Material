\chapter{Variational Principles and Lagrange's Equations}
\section{Hamilton's Principle}
\subsection{Configuration Space}
\begin{itemize}
	\item The Instantaneous configuration of a system is described by the values of $n$ generalized coordinates $q_1,    q_n$ and corresponds to a particular point in a cartesian hyperspace where the q's from the $n$  diamensional space is known as configuration space.
	\item As time goes by, the state of the system changes and the system point moves in configuration space tracing out a curve, described as "the path of motion of the system".
	\item The motion of the system then refered to the motion of the system point along this path in configuration space
	\item Time can be considered formally as a parameter of the curve to each point on the path there is associated one or more values of the time. Each point on the path represent the entire system  configuration at some given instant of time.\\\\
	\textbf{Monogenic System}: The mechanical systems whose motion for which all forces  (except the forces of constraint) are derivable from a generalized scalar potential that may be a function of the coordinates, velocities and time.
	\subsection{Hamilton's Principle}
	The motion of the system from time $t_1 $ to time $t_2$ is such that the line integral called the avtion or the action integral.
	$$I=\int\limits_{t_1}^{t_2}Ldt$$
	where $L$ is $T-V$ has stationary value for the actual path of the motion.\\\\
	ie. out of all possible path by which the system point could travel from it's position at time $t_1$ to its position at time $t_2$. It will travel along that path for which the value of integral is stationary.\\\\
	$\therefore$ Hamilton's principle can summarize by saying that the motion is such that the variation of the line integral $I$ for fixed $t_1$ and $t_2$ is zero.
	$$\delta I=\delta\int\limits_{t_1}^{t_2}L(q_1,.....q_n,\dot{q}_1.....\dot{q}_n,t)dt=0$$
	Where the system constraints are holonomic, Hamilton's principle is both a necessary and suffitient condition for lagrange's equations.
	
	
	
	
	
	
	
	
	
	
	
	
	
	
	
	
	
	
	
	
	
	
	
	
\end{itemize}