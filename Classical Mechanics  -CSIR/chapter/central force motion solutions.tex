\chapter{central force motion solutions}
\begin{abox}
	Practice set 1 solutions
	\end{abox}
\begin{enumerate}
	\begin{minipage}{\textwidth}
		\item The acceleration due to gravity $(g)$ on the surface of Earth is approximately $2.6$ times that on the surface of Mars. Given that the radius of Mars is about one half the radius of Earth, the ratio of the escape velocity on Earth to that on Mars is approximately
		\exyear{NET JUNE 2011}
	\end{minipage}
	\begin{tasks}(2)
		\task[\textbf{A.}] $1.1$
		\task[\textbf{B.}]$1.3$
		\task[\textbf{C.}]$2.3$
		\task[\textbf{D.}]$5.2$
	\end{tasks}
\begin{answer}
	$\text { Escape velocity }=\sqrt{2 g R}$\\
	$\frac{\text { Escape velocity of Earth }}{\text { Escape velocity of Mass }}=\sqrt{\frac{g_{e} R_{e}}{g_{m} R_{m}}}=2.3 \quad \text { where } \frac{R_{e}}{R_{m}}=2 \text { and } \frac{g_{e}}{g_{m}}=2.6$\\
	THe correct option is \textbf{(c)}
\end{answer}
\begin{minipage}{\textwidth}
	\item Two particles of identical mass move in circular orbits under a central potential $V(r)=\frac{1}{2} k r^{2}$. Let $l_{1}$ and $l_{2}$ be the angular momenta and $r_{1}, r_{2}$ be the radii of the orbits respectively. If $\frac{l_{1}}{l_{2}}=2$, the value of $\frac{r_{1}}{r_{2}}$ is:
	\exyear{NET DEC 2011}
\end{minipage}
\begin{tasks}(2)
	\task[\textbf{A.}] $\sqrt{2}$
	\task[\textbf{B.}]$1 / \sqrt{2}$
	\task[\textbf{C.}] 2
	\task[\textbf{D.}] $1 / 2$
\end{tasks}
\begin{answer}
	 $V_{e f f}=\frac{l^{2}}{2 m r^{2}}+\frac{1}{2} k r^{2}$, where $l$ is angular momentum.\\\\
	Condition for circular orbit $\frac{\partial V_{e f f}}{\partial r}=0 \Rightarrow-\frac{l^{2}}{m r^{3}}+k r=0 \Rightarrow l^{2} \propto r^{4} \Rightarrow l \propto r^{2}$.\\
	Thus $\frac{l_{1}}{l_{2}}=\left(\frac{r_{1}}{r_{2}}\right)^{2} \Rightarrow \frac{r_{1}}{r_{2}}=\sqrt{\frac{l_{1}}{l_{2}}} \Rightarrow \frac{r_{1}}{r_{2}}=\sqrt{2}$\\ since $\frac{l_{1}}{l_{2}}=2$.\\
	The correct option is \textbf{(a)}
\end{answer}
\begin{minipage}{\textwidth}
	\item A planet of mass $m$ moves in the inverse square central force field of the Sun of mass $M$. If the semi-major and semi-minor axes of the orbit are $a$ and $b$, respectively, the total energy of the planet is:
	\exyear{NET DEC 2011}
\end{minipage}
\begin{tasks}(2)
	\task[\textbf{A.}] $-\frac{G M m}{a+b}$
	\task[\textbf{B.}]$-G M m\left(\frac{1}{a}+\frac{1}{b}\right)$
	\task[\textbf{C.}]$-\frac{G M m}{a}\left(\frac{1}{b}-\frac{1}{a}\right)$
	\task[\textbf{D.}]$-G M m\left(\frac{a-b}{(a+b)^{2}}\right)$
\end{tasks}
\begin{answer}
 Assume Sun is at the centre of elliptical orbit. Conservation of energy\\
  $\frac{1}{2} m v_{1}^{2}-\frac{G M m}{a}=\frac{1}{2} m v_{2}^{2}-\frac{G M m}{b}$\\ Conservation of momentum $L=m v_{1} a=m v_{2} b$\\
  \begin{figure}[H]
  	\centering
  	\includegraphics[height=3cm,width=5cm]{diagram-20210926(12)-crop}
  \end{figure}
 \begin{align*}
 	&v_{2}=v_{1}\left(\frac{a}{b}\right) \\
 	&\frac{1}{2} m v_{1}^{2}-\frac{1}{2} m v_{2}^{2}=\frac{G M m}{a}-\frac{G M m}{b} \Rightarrow \frac{1}{2} m\left(v_{1}^{2}-v_{1}^{2} \frac{a^{2}}{b^{2}}\right)=G M m\left(\frac{b-a}{a b}\right) \\
 	&\frac{1}{2} m v_{1}^{2}\left(\frac{b^{2}-a^{2}}{b^{2}}\right)=G M m\left(\frac{b-a}{a b}\right) \Rightarrow \frac{1}{2} m v_{1}^{2}=G M m\left(\frac{b}{a}\right) \cdot \frac{1}{(b+a)} \\
 	&E=\frac{1}{2} m v_{1}^{2}-\frac{G M m}{a}=G M m \frac{b}{a} \frac{1}{(b+a)}-\frac{G M m}{a} \\
 	&=\frac{G M m}{a}\left(\frac{b}{(b+a)}-1\right)=\frac{G M m}{a}\left(\frac{b-b-a}{(b+a)}\right)=-\frac{G M m}{(b+a)}
 \end{align*}
 The correct option is \textbf{(a)}	
\end{answer}
\begin{minipage}{\textwidth}
	\item A planet of mass $m$ moves in the gravitational field of the Sun (mass $M$ ). If the semimajor and semi-minor axes of the orbit are $a$ and $b$ respectively, the angular momentum of the planet is
	\exyear{NET DEC 2012}
\end{minipage}
\begin{tasks}(2)
	\task[\textbf{A.}]$\sqrt{2 G M m^{2}(a+b)}$
	\task[\textbf{B.}]$\sqrt{2 G M m^{2}(a-b)}$
	\task[\textbf{C.}]$\sqrt{\frac{2 G M m^{2} a b}{a-b}}$
	\task[\textbf{D.}]$\sqrt{\frac{2 G M m^{2} a b}{a+b}}$
\end{tasks}
\begin{answer}
	 Assume Sun is at the centre of elliptical orbit.\\\\
	Conservation of energy $\frac{1}{2} m v_{1}^{2}-\frac{G M m}{a}=\frac{1}{2} m v_{2}^{2}-\frac{G M m}{b}$\\\\
	Conservation of momentum $L=m v_{1} a=m v_{2} b$
	$$
	v_{2}=v_{1}\left(\frac{a}{b}\right)
	$$
	\begin{figure}[H]
		\centering
		\includegraphics[height=3cm,width=5cm]{diagram-20210926(12)-crop}
	\end{figure}
	\begin{align*}
		&\frac{1}{2} m v_{1}^{2}-\frac{1}{2} m v_{2}^{2}=\frac{G M m}{a}-\frac{G M m}{b} \Rightarrow \frac{1}{2} m\left(v_{1}^{2}-v_{1}^{2} \frac{a^{2}}{b^{2}}\right)=G M m\left(\frac{b-a}{a b}\right) \\
		&\frac{1}{2} m v_{1}^{2}\left(\frac{b^{2}-a^{2}}{b^{2}}\right)=G M m\left(\frac{b-a}{a b}\right) \Rightarrow \frac{1}{2} m v_{1}^{2}=G M m\left(\frac{b}{a}\right) \cdot \frac{1}{(b+a)} \\
		&v_{1}=\sqrt{2 G M\left(\frac{b}{a}\right) \cdot \frac{1}{(b+a)}} \\
		&L=m v_{1} a=m \sqrt{2 G M\left(\frac{b}{a}\right) \cdot\left(\frac{1}{b+a}\right)} \cdot a=m \sqrt{\frac{2 G M a b}{(b+a)}} \Rightarrow L=\sqrt{\frac{2 G M m^{2} a b}{a+b}}
	\end{align*}
	The correct option is \textbf{(d)}
\end{answer}
\begin{minipage}{\textwidth}
	\item A planet of mass $m$ and an angular momentum $L$ moves in a circular orbit in a potential, $V(r)=-k / r$, where $k$ is a constant. If it is slightly perturbed radially, the angular frequency of radial oscillations is
	\exyear{NET JUNE 2013}
\end{minipage}
\begin{tasks}(2)
	\task[\textbf{A.}] $m k^{2} / \sqrt{2} L^{3}$
	\task[\textbf{B.}]$m k^{2} / L^{3}$
	\task[\textbf{C.}]$\sqrt{2} m k^{2} / L^{3}$
	\task[\textbf{D.}]$\sqrt{3} m k^{2} / L^{3}$
\end{tasks}
\begin{answer}$\left. \right. $\\
	\begin{minipage}{0.5\textwidth}
	$V_{e f f}=\frac{L^{2}}{2 m r^{2}}-\frac{k}{r} .$\\
	For circular orbit $\frac{\partial V_{e f f}}{\partial r}=-\frac{L^{2}}{m r^{3}}+\frac{k}{r^{2}}=0$ \\
	$\Rightarrow \frac{L^{2}}{m r^{3}}=\frac{k}{r^{2}} .$ \\
	Thus $r=r_{0}=\frac{L^{2}}{m k} \Rightarrow \omega=\sqrt{\frac{k}{m}}$,
	\end{minipage}
\begin{minipage}{0.5\textwidth}
\begin{figure}[H]
	\centering
	\includegraphics[height=4cm,width=5cm]{diagram-20210926(17)-crop}
\end{figure}
\end{minipage}
 $$k=\left.\frac{d^{2} V_{e f f}}{d r^{2}}\right|_{r=r_{0}}=+\frac{3 L^{2}}{m r^{4}}-\left.\frac{2 k}{r^{3}}\right|_{r=r_{0}}=\frac{3 L^{2}}{m\left(\frac{L^{2}}{m k}\right)^{4}}-\frac{2 k}{\left(\frac{L^{2}}{m k}\right)^{3}}=\frac{3 m^{3} k^{4}}{L^{6}}-\frac{2 m^{3} k^{4}}{L^{6}}=\frac{m^{3} k^{4}}{L^{6}}$$
 $$\omega=\sqrt{\frac{\left.\frac{d^{2} V}{d r^{2}}\right|_{r=r_{0}}}{m}} \Rightarrow \omega=\frac{m k^{2}}{L^{3}}$$
 The correct option is \textbf{(b)}	
\end{answer}
\begin{minipage}{\textwidth}
\item The radius of Earth is approximately $6400 \mathrm{~km}$. The height $h$ at which the acceleration due to Earth's gravity differs from $g$ at the Earth's surface by approximately $1 \%$ is
	\exyear{NET DEC 2014}
\end{minipage}
\begin{tasks}(2)
	\task[\textbf{A.}] $64 \mathrm{~km}$
	\task[\textbf{B.}] $48 \mathrm{~km}$
	\task[\textbf{C.}]$32 \mathrm{~km}$
	\task[\textbf{D.}]$16 \mathrm{~km}$
\end{tasks}
\begin{answer}
$\frac{g}{g^{\prime}}=1+\frac{2 h}{R} \Rightarrow \frac{g}{g^{\prime}}-1=\frac{2 h}{R} \Rightarrow \frac{\Delta g}{g^{\prime}}=\frac{2 h}{R} \Rightarrow h=32 k \cdot m$	
\end{answer}
\begin{minipage}{\textwidth}
	\item The probe Mangalyaan was sent recently to explore the planet Mars. The inter-planetary part of the trajectory is approximately a half-ellipse with the Earth (at the time of launch),Sun and Mars (at the time the probe reaches the destination) forming the major axis. Assuming that the orbits of Earth and Mars are approximately circular with radii $R_{E}$ and $R_{M}$, respectively, the velocity (with respect to the Sun) of the probe during its voyage when it is at a distance
		\begin{figure}[H]
		\centering
		\includegraphics[height=3cm,width=5cm]{diagram-20210926(22)-crop(1)}
	\end{figure}
	 $r\left(R_{E}<<r<<R_{M}\right) \text { from the Sun, neglecting the effect of Earth and Mars, is }$
	\exyear{NET DEC 2014}
\end{minipage}
\begin{tasks}(2)
	\task[\textbf{A.}] $\sqrt{2 G M \frac{\left(R_{E}+R_{M}\right)}{r\left(R_{E}+R_{M}-r\right)}}$
	\task[\textbf{B.}]$\sqrt{2 G M \frac{\left(R_{E}+R_{M}-r\right)}{r\left(R_{E}+R_{M}\right)}}$
	\task[\textbf{C.}]$\sqrt{2 G M \frac{R_{E}}{r R_{M}}}$
	\task[\textbf{D.}]$\sqrt{\frac{2 G M}{r}}$
\end{tasks}
\begin{answer}
	$\text { Total energy } E=-K / 2 a \text { where } 2 a \text { major axis and } 2 a=R_{E}+R_{M} \text {. }$
	$$\frac{1}{2} m v^{2}-\frac{G M m}{r}=-\frac{G M m}{\left(R_{E}+R_{M}\right)} \Rightarrow v=\sqrt{2 G M \frac{\left(R_{E}+R_{M}-r\right)}{r\left(R_{E}+R_{M}\right)}}$$
	The correct option is \textbf{(b)}
\end{answer}
\begin{minipage}{\textwidth}
	\item After a perfectly elastic collision of two identical balls, one of which was initially at rest, the velocities of both the balls are non zero. The angle $\theta$ between the final, velocities (in the lab frame) is
	\exyear{NET DEC 2016}
\end{minipage}
\begin{tasks}(2)
	\task[\textbf{A.}] $\theta=\frac{\pi}{2}$
	\task[\textbf{B.}]$\theta=\pi$
	\task[\textbf{C.}]$0<\theta \leq \frac{\pi}{2}$
	\task[\textbf{D.}] $\frac{\pi}{2}<\theta \leq \pi$
\end{tasks}
\begin{answer}
tion: Angle between two particle $\theta_{1}+\theta_{2}=0$
Conservation of momentum
\begin{align*}
&m u=m v_{1} \cos \theta_{1}+m v_{2} \cos \theta_{2} \\
&0=m v_{1} \sin \theta_{1}-m v_{2} \sin \theta_{2}
\end{align*}
conservation of kinetic energy\\
$
\frac{1}{2} m u^{2}=\frac{1}{2} m v_{1}^{2}+\frac{1}{2} m v_{2}^{2}
$
\begin{align*}
	&u^{2}=v_{1}^{2}+v_{2}^{2}+2 v_{1} v_{2}\left(\cos \theta_{1} \cos \theta_{2}-\sin \theta_{1} \sin \theta_{2}\right) \\
	&u^{2}=v_{1}^{2}+v_{2}^{2}+2 v_{1} v_{2} \cos \left(\theta_{1}+\theta_{2}\right) \\
	&u^{2}=v_{1}^{2}+v_{2}^{2} \\
	&v_{1}^{2}+v_{2}^{2}=v_{1}^{2}+v_{2}^{2}+2 v_{1} v_{2} \cos \left(\theta_{1}+\theta_{2}\right) \\
	&\cos \left(\theta_{1}+\theta_{2}\right)=0 \\
	&\theta_{1}+\theta_{2}=\frac{\pi}{2} \Rightarrow \theta=\frac{\pi}{2}
\end{align*}
The correct option is \textbf{(a)}	
\end{answer}
\begin{minipage}{\textwidth}
	\item Consider circular orbits in a central force potential $V(r)=-\frac{k}{r^{n}}$, where $k>0$ and $0<n<2$. If the time period of a circular orbit of radius $R$ is $T_{1}$ and that of radius $2 R$ is $T_{2}$, then $\frac{T_{2}}{T_{1}}$
	\exyear{NET DEC 2016}
\end{minipage}
\begin{tasks}(2)
	\task[\textbf{A.}] $2^{\frac{n}{2}}$
	\task[\textbf{B.}]$2^{\frac{2}{3} n}$
	\task[\textbf{C.}]$2^{\frac{n}{2}+1}$
	\task[\textbf{D.}]$2^{n}$
\end{tasks}
\begin{answer}
	$V_{e f f}=\frac{J^{2}}{2 m r^{2}}-\frac{k}{r^{n}}, \frac{\partial V_{e f f}}{\partial r}=-\frac{J^{2}}{m r^{3}}+\frac{n k}{r^{n+1}}=0$\\
	\begin{align*}
		&\because J=m r^{2} \omega \Rightarrow \frac{m^{2} \omega^{2} r^{4}}{r^{3}}=\frac{n k}{r^{n+1}} \Rightarrow \omega^{2} \propto \frac{1}{r^{n+2}} \Rightarrow \omega \propto r^{-(n+2) / 2} \Rightarrow T \propto r^{\frac{n}{2}+1} \\
		&\frac{T_{2}}{T_{1}}=\left(\frac{2 R}{R}\right)^{\frac{n+2}{2}}=2^{\frac{n}{2}+1}
	\end{align*}
	The correct option is \textbf{(c)}
\end{answer}
\begin{minipage}{\textwidth}
	\item A ball weighing $100 \mathrm{gm}$, released from a height of $5 \mathrm{~m}$, bounces perfectly elastically off a plate. The collision time between the ball and the plate is $0.5 \mathrm{~s}$. The average force on the plate is approximately
	\exyear{NET JUNE 2017}
\end{minipage}
\begin{tasks}(2)
	\task[\textbf{A.}] $3 N$
	\task[\textbf{B.}]$2 N$
	\task[\textbf{C.}]$5 N$
	\task[\textbf{D.}]$4 N$
\end{tasks}
\begin{answer}
$m=\frac{100}{1000}=0.1 \mathrm{~kg}$\\
$$
\begin{array}{rl}
m g h=\frac{1}{2} m v^{2} & v=\sqrt{2 g h} \\
v & =10 \mathrm{~m} / \mathrm{sec}
\end{array}
$$
change in momentum during collision, $(m v)-(-m v)=2 k . g m / \mathrm{sec}$
$$
f=\frac{\Delta P}{\Delta t}=\frac{2}{0.5}=4 N
$$	
\end{answer}
\end{enumerate}