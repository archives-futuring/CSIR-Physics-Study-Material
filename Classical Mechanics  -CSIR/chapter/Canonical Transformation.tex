\chapter{Canonical Transformation}
For each problem there may be one particular choice for which all coordinates $q_i$ are cyclic. Then the conjugate momenta $p_i$ are all constant:
$$p_i=\alpha_i$$
Consider a situation in which Hamiltonian is a constant of motion,
then 
$$H=H(\alpha_1,.......\alpha_n)$$
so the Hamiltonian equations for $\dot{q}_i$ are 
$$\dot{q}_i=\frac{\partial H}{\partial \alpha_i}=\omega_i$$
Where $\omega_i$'s are functions of $\omega_i$'s only 
$\therefore $ solutions are
$$ q_i=\omega_i t+\beta_i$$
where $\beta_i$'s are constants of integration.\\
\par Since the obivious generalized coordinates suggested by the problem will
not normally be cyclic, we must have a specific proceedure for transforming from one set of variables to some other set that may be more suitable.\\
\par In the Hameltonian formulation the momenta are also independent variable on the same level as the generalized coordinates. The similtaneous transformation of the independent coordinates and momenta $q_i,p_i$ to a new se t  $Q_i,P_i$ with (invertible) equation of transformation:
$$Q_i=Q_i (q,p,t)$$
$$P_i=P_i (q,p,t)$$
Which define a point transformation in phase space.