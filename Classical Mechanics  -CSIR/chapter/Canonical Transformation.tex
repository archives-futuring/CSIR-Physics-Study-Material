\chapter{Canonical Transformation}
For each problem there may be one particular choice for which all coordinates $q_i$ are cyclic. Then the conjugate momenta $p_i$ are all constant:
$$p_i=\alpha_i$$
Consider a situation in which Hamiltonian is a constant of motion,
then 
$$H=H(\alpha_1,.......\alpha_n)$$
so the Hamiltonian equations for $\dot{q}_i$ are 
$$\dot{q}_i=\frac{\partial H}{\partial \alpha_i}=\omega_i$$
Where $\omega_i$'s are functions of $\omega_i$'s only 
$\therefore $ solutions are
$$ q_i=\omega_i t+\beta_i$$
where $\beta_i$'s are constants of integration.\\
\par Since the obivious generalized coordinates suggested by the problem will
not normally be cyclic, we must have a specific proceedure for transforming from one set of variables to some other set that may be more suitable.\\
\par In the Hameltonian formulation the momenta are also independent variable on the same level as the generalized coordinates. The similtaneous transformation of the independent coordinates and momenta $q_i,p_i$ to a new se t  $Q_i,P_i$ with (invertible) equation of transformation:
$$Q_i=Q_i (q,p,t)$$
$$P_i=P_i (q,p,t)$$
Which define a point transformation in phase space.
\section{Canonical Transformation}
	 In Hamiltonian mechanics the transformation should be such that $Q$ and $P$ are canonical coordinates
	We would like to change variables from the set $(a,p)$ to a new set $(Q,P )$ such that:
	\begin{enumerate}
		\item  Determinent of Jacobian matrix of transformation, $\left| \frac{\partial(Q,P)}{\partial(q,p)}\right|=+1 $
		(This ensures that it is volume and orientation are preserved during transformation)
		\item  Ensures structure of Hamilton's equation is not changed, so the exist a function $K=K(Q,P,t)$ 
		such that the equation of motion of new set are	
		$$\dot{Q_i}-\frac{\partial K}{\partial P_i}\quad,\quad\dot{P_i}=\frac{\partial K}{\partial Q_i}$$
		$K$ plays role of Hamiltonian in new coordinate set and known as 'kamiltonian'	
	\end{enumerate}
\begin{note}
	The transformation considered be problem independent that is to say $(Q,P)$ must be canonical coordinates not onlu for some specific mechanical systems, but for all systems of the same number of degrees of freedom.	
\end{note}
\textbf{Hamilton's Priciple of Transformed Coordinates}\vspace{0.1cm}\\
 As $Q_i$ and $P_i$ are canonical variable they must satisfy Hamiltonians principle
\begin{equation}
\delta\int\limits_{t_1}^{t_2}(P_i\dot{Q_i}-K(Q,P,t))dt=0
\end{equation}
as of old canonical variables
\begin{equation}
\delta\int\limits_{t_1}^{t_2}(p_i\dot{q_i}-H(Q,P,t))dt=0
\end{equation}
$\therefore$ we can say
$$\lambda (p_i\dot{q_i}-H)=P_i\dot{Q_i}-K+\frac{\partial F}{\partial t}$$
$F$ is a function of phase space coordinates with continuous second derivatives \\
$\lambda$ is a constant independent of canonical coordinates and the time and it is related to scale transformation. \\\\
For $\lambda=1$ we have
$$p_i\dot{q_i}-H=P_i\dot{Q_i}-K+\frac{dF}{dt}$$
Which is simply called canonical transformation 
\begin{note}
	A transformation of canonical coordinates for which is called extended canonical transformation
\end{note}
\section{Generating Function}
\begin{itemize}
	\item The term $\frac{df}{dt}$ in canonical transformation contributes to the variation of action integral only at the end points, and will vanish if $F$ is a function of $(q,p t)$ or $(Q,P,t)$or any mixture of the phase space coordinates since these have zero variation at the end points.
	\item Through equations of transformation and their inverses $F$ can be expressed partly in terms of the old set of variables and partly of the new.
	\item $F$ acts as a bridge between the two sets of canonical variables and is called the generating function of transformation.
\end{itemize}