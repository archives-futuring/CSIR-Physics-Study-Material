\chapter{Classical Problems}
\section{Review of Basic Concept}
\begin{enumerate}
	\item At the moment $t=0$ a stationary particle of mass $m$ experiences a time-dependent force $F=k t\left(t^{\prime}-t\right)$, where $k$ is a constant vector, $t^{\prime}$ is the time during which the given force acts. Find :\\
	(a) the momentum of the particle when the action of the force discontinued;\\
	(b) the distance covered by particle while the force acted.
	\begin{answer}
	(a) Since $F=k t\left(t^{\prime}-t\right)$, the direction of the force is fixed, as $k$ is a constant vector and the particle starts from rest, the motion will be in the direction of the force.
	\begin{align*}
	m \frac{d v}{d t}&=k t\left(t^{\prime}-t\right),\text{ from Newton's II law}\\
	\text{Integrating, }\int_{0}^{v} d v&=\frac{k}{m} \int_{0}^{t^{\prime}} t\left(t^{\prime}-t\right) d t=\frac{k}{m}\left[t^{\prime} \frac{t^{2}}{2}-\frac{t^{3}}{3}\right]_{0}^{t^{\prime}} \quad v=\frac{k}{m} \cdot \frac{t^{\prime 3}}{6}\\
\text{	Therefore, momentum of the particle }p&=m v=k t \frac{t^{\prime 3}}{6}.\\
\text{(b) }d s=v d t \Rightarrow s&=\int_{0}^{s} d s=\int_{0}^{t^{\prime}} v d t=\frac{k t^{\prime 4}}{12 m}
	\end{align*}
		\end{answer}
	\item A block of mass $m_{1}=4 \mathrm{~kg}$ on a smooth inclined plane of $30^{\circ}$ is connected by a cord over a small, frictionless pulley to a second block of mass $m_{2}=5 \mathrm{~kg}$ hanging vertically. Calculate the acceleration with which the block moves and also the tension in the cord. Take $g=10 \mathrm{~m} / \mathrm{sec}^{2}$.
	\begin{answer}
			The different forces acting on the masses are shown in figure.
			\begin{figure}[H]
				\centering
				\includegraphics[height=3.5cm,width=6cm]{diagram-20220216-crop}
			\end{figure}
		\begin{align}
\text{We have }T-m_{1} g \sin \theta&=m_{1} a \label{1} \\
\text{ and }m_{2} g-T&=m_{2} a \label{2}\\
\intertext{Solving equations (\ref{1}) and (\ref{2}), we get}
a&=\frac{\left(m_{2}-m_{1} \sin \theta\right) g}{\left(m_{1}+m_{2}\right)}\label{3}\\
\text { and } \quad T & =m_{2} g\left[1-\frac{\left(m_{2}-m_{1} \sin \theta\right)}{\left(m_{1}+m_{2}\right)}\right] \notag\\
 \text { or } \quad T & =\frac{m_{1} m_{2}(1+\sin \theta) g}{\left(m_{1}+m_{2}\right)}\\
 \text{Here }m_{1}&=4 \mathrm{~kg}, m_{2}=5 \mathrm{~kg}, \theta=30^{\circ} \text{and }g=10 \mathrm{~m} / \mathrm{s}\notag\\
  \intertext{Substituting these values in equation (\ref{3}), we get}
 T&=\frac{5 \times 4\left(1+\frac{1}{2}\right) 10}{9}=\frac{300}{9}=33.33 \mathrm{~N}\notag
\end{align}
	\end{answer}
	\item A particle of mass $m$ is connected to two springs of unstretched length ' $l$ ' and spring constant $k$ as show in figure. Calculate acceleration of particle if it is slightly displaced along X-direction.
	\begin{answer}
			Let the particle is displaced by a distance $x$ along $+x$ direction as shown in figure.
			\begin{figure}[H]
				\centering
				\includegraphics[height=4.5cm,width=2cm]{diagram-20220216(1)-crop}
			\end{figure}
		\begin{figure}[H]
			\centering
			\includegraphics[height=3cm,width=2cm]{diagram-20220216(2)-crop}
		\end{figure}
		\begin{align*}
	\text{	Therefore, stretched length of spring }&=\sqrt{l^{2}+x^{2}}\\
		\text{elongation in springs }&=\sqrt{l^{2}+x^{2}}-l\\
	\text{	Restoring force on the particle due to each spring} F&=k\left(\sqrt{l^{2}+x^{2}}-l\right)
	\intertext{ Because of restoring force the particle moves back towards its initial position.}
\text{	Equation of motion of particle }F_{x}&=m \frac{d^{2} x}{d t^{2}}\\
	\text{or }-2 F \cos \theta&=m \frac{d^{2} x}{d t^{2}}\text{ or }-2 k\left(\sqrt{l^{2}+x^{2}}-l\right) \cdot \frac{x}{\sqrt{l^{2}+x^{2}}}=m \frac{d^{2} x}{d t^{2}}\\
	\text{Therefore, }\frac{d^{2} x}{d t^{2}}&=-\frac{2 k x}{m}\left(1-\frac{l}{\sqrt{l^{2}+x^{2}}}\right)\text{ or }\frac{d^{2} x}{d t^{2}}\\&=-\frac{2 k x}{m}\left[1-\left(1+\frac{x^{2}}{l^{2}}\right)^{-1 / 2}\right]\\
\text{	Since, }&x<l,\left(1+\frac{x^{2}}{l^{2}}\right)^{-1 / 2} \approx 1-\frac{x^{2}}{2 l^{2}}\\
\text{	Therefore, }\frac{d^{2} x}{d t^{2}}&=-\frac{2 k x}{m}\left[1-\left(1-\frac{x^{2}}{2 l^{2}}\right)\right]\text{ or }\frac{d^{2} x}{d t^{2}}=-\frac{k x^{3}}{m l^{2}}
		\end{align*}
	\end{answer}
\item At the moment $t=0$, the force $F=k t$ is applied to a small body of mass $m$ resting on smooth horizontal plane $(k=$ constant). The permanent direction of this force forms an angle $\theta$ with the horizontal (figure below). Find\\
\begin{figure}[H]
	\centering
	\includegraphics[height=3cm,width=10cm]{diagram-20220217-crop}
\end{figure}
(a) the velocity of the body at the moment of its breaking off the plane;\\
(b) the distance traversed by the body up to this moment.

\begin{answer}
		The free body diagram is shown in figure.
	\begin{align*}
	\text{From figure: }R+F \sin \theta&=m g\\
	\text{At breaking off, }R&=0, \therefore F=\frac{m g}{\sin \theta}\\
\text{	Now }k t&=\frac{m g}{\sin \theta}\text{ or }t=\frac{m g}{k \sin \theta}
\intertext{	(a) If $a$ be the acceleration, then $F \cos \theta=m a$ }
\text{	Therefore, }k t \cos \theta&=m \times\left(\frac{d v}{d t}\right) \quad\left(\because a=\frac{d v}{d t}\right) .
\intertext{Integrating this expression within proper limits, we get $m \int_{0}^{v} d v=k \cos \theta \int_{0}^{m g / k \sin \theta} t d t$}
\text{ or }\quad m v&=\frac{k \cos \theta}{2}\left[\frac{m g}{k \sin \theta}\right]^{2}\text{ or }v=\frac{m g^{2}}{2 k}\left(\frac{\cos \theta}{\sin ^{2} \theta}\right)\\
\text{(b) Without putting the limits, we have }v&=\frac{k \cos \theta}{2 m} t^{2}+C,\text{ where $C=$ constant of integration.}\\
\text{When }t&=0, v=0\text{ and hence }C=0.\\
\text{Now, }\frac{d s}{d t}&=\frac{k \cos \theta}{2 m} t^{2} \int_{0}^{s} d s=\frac{k \cos \theta}{2 m} \int_{0}^{m g / k \sin \theta} t^{2} d t\\
s&=\frac{k \cos \theta}{2 m}\left[\frac{t^{3}}{3}\right]_{0}^{m g / k \sin \theta} \quad\text{ or }s=\frac{m^{2} g^{3} \cos \theta}{6 k^{2} \sin ^{3} \theta} .
	\end{align*}
\end{answer}
\item Spherical particles of a given material of density $\rho$ are released from rest inside a liquid medium of lower density. The viscous drag force may be approximated by the Stoke's law, i.e, $F_{d}=6 \pi \eta R \mathrm{v}$, where $\eta$ is the viscosity of the medium, $R$ the radius of a particle and $v$ its instantaneous velocity. If $\tau(m)$ is the time taken by a particle of mass $m$ to reach half its terminal velocity, then the ratio $\tau(8 m) / \tau(m)$ is
\begin{answer}
	 Each particle has same density but different radii and masses. We have to calculate time of fall in terms of mass of particle therefore we will write our equations explicitly in terms of mass and we will remove radius from our equations wherever it appears.\\Drag force on particles
	 \begin{align}
	 F_{d}&=6 \pi \eta R \mathrm{v}, \quad\text{ Mass of a particle }m=\rho \cdot \frac{4}{3} \pi R^{3}\notag\\
	 \therefore R&=\left(\frac{3 m}{4 \pi \rho}\right)^{1 / 3} \quad \therefore F_{d}\notag\\&=6 \pi \eta\left(\frac{3 m}{4 \pi \rho}\right)^{1 / 3} v=K m^{1 / 3} v\text{, where }K=6 \pi \eta\left(\frac{3}{4 \pi \rho}\right)^{1 / 3}\notag\\
	\intertext{ If $\mathrm{B}$ is buoyancy force, then equation of motion of a particle is}\notag\\
	 m \frac{d v}{d t}&=m g-F_{d}-B\text{ where }\mathrm{B}=\frac{4}{3} \pi R^{3} \sigma g\notag\\&=\frac{4}{3} \pi R^{3} \rho g \cdot \frac{\sigma}{\rho}=m g \frac{\sigma}{\rho}, \sigma=\text{ density of medium}\notag\\
	 \therefore m \frac{d v}{d t}&=m g-K m^{1 / 3} v-m g \frac{\sigma}{\rho}, m \frac{d v}{d t}=m g\left(1-\frac{\sigma}{\rho}\right)-K m^{1 / 3} v \notag\\
	 \text { or } \frac{d v}{d t}&=g\left(1-\frac{\sigma}{\rho}\right)-K m^{-2 / 3} v\label{5}\\
	 \text{when terminal }&\text{velocity is reached }\frac{d v}{d t}=0\notag\\
	 \therefore 0&=g\left(1-\frac{\sigma}{\rho}\right)-K m^{-2 / 3} v_{t} \quad \therefore v_{t}=\frac{g\left(1-\frac{\sigma}{\rho}\right)}{K m^{-2 / 3}}\label{6}\\
	\intertext{ Now, if $\tau$ be the time to reach half the terminal velocity then from (\ref{5})}\notag\\
	 \int_{0}^{1 / 2} \frac{d v}{g\left(1-\frac{\sigma}{\rho}\right)-K m^{-2 / 3} v}&=\int_{0}^{\tau} d t, \quad \therefore-\frac{1}{K m^{-2 / 3}} \ln \left[\frac{g\left(1-\frac{\sigma}{\rho}\right)-\frac{K m^{-2 / 3} v_{t}}{2}}{g\left(1-\frac{\sigma}{\rho}\right)}\right]=\tau\notag\\
	 \text{using value of $v_{t}$ from (\ref{6}) we get, }\tau&=\frac{m^{2 / 3}}{K} \ln 2 \quad\text{ or }\quad \tau(m)=\frac{m^{2 / 3} \ln 2}{K}\notag\\
	  \therefore \tau(8 m) / \tau(m)&=(8 m)^{2 / 3} / m^{2 / 3}=4\notag
	 \end{align}
\end{answer}
\item A particle of unit mass is thrown vertically upward with initial speed $v_{0}$. It is acted upon by a drag force $b v^{2}$ in addition to gravity where $b$ is constant and $v$ is instantaneous velocity of particle. Calculate speed of the particle when it returns to the point from where it was thrown.
\begin{answer}
	In presence of drag force only quantity that is common in upward and downward motion is the distance covered.
	\begin{figure}[H]
		\centering
		\includegraphics[height=3cm,width=7cm]{c prb01}
	\end{figure}
	\begin{align*}
	\text{For upward motion, initial speed }&=v_{0},\\
	 \text{final speed }&=0,\text{ let height reached $=h$}\\
	 \text{ equation of motion }\frac{d v}{d t}&=-g-b v^{2}\\
	\text{or }\frac{d v}{d y} \cdot \frac{d y}{d t}&=-g-b v^{3}, \frac{d y}{d t}=v\text{ or }\frac{v d v}{g+b v^{2}}=-d y\\
	\therefore \int_{v}^{0} \frac{v d v}{g+b v^{2}}&=-\int_{0}^{h} d y\\
\text{	on integration we get, }\frac{1}{2 b} \ln \left(\frac{g+b v_{0}^{2}}{g}\right)&=h\\
\text{for downward motion, initial speed }&=0,\text{ final speed $=v($ let $)$, height descended $=h$ }\\
\text{equation of motion }\frac{d v}{d t}&=g-b v^{2}\text{ or }\frac{v d v}{g-b v^{2}}=d y\text{ or }\int_{0}^{v} \frac{v d v}{g-b v^{2}}=\int_{0}^{h} d y\\
\text{on integration we get, }\frac{1}{2 b} \ln \left(\frac{g}{g-b v^{2}}\right)&=h\\
\text{Comparing (a) and (b) we get, }\frac{g+b v_{0}^{2}}{g}&=\frac{g}{g-b v^{2}}, \quad g-b v^{2}=\frac{g^{2}}{g+b v_{0}^{2}}\\
\text{or }v&=\frac{v_{0}}{\sqrt{1+\frac{b v_{0}^{2}}{g}}} \Rightarrow v<v_{0}
	\end{align*}
	due to drag force the particle returns with a speed less than its initial speed. If drag force were absent $b=0$, then $v=v_{0} .$ Therefore in absence of drag force particle returns with same speed as its initial value.
\end{answer}
\item A small ring of mass $m$ can slide on a smooth circular wire of radius $r$ and center $O$, which is fixed in a vertical plane. From a point on the wire at a vertical distance $r / 2$ above $O$, the ring is given a velocity $\sqrt{(g r) }\text { along the downward tangent to the wire. Show that it will just reach the highest point of the wire. }$ Find the reaction between the ring and the wire when the ring is at a vertical distance $r / 2$ below.
\begin{answer}
	\begin{align}
\text{At point $C$, the velocity of the ring }&=\sqrt{(r g)}.\notag\\
\text{K.E. at }C&=\frac{1}{2} m v^{2}=\frac{1}{2} m r g\text{ and P.E. at }C\notag\\&=m g h=m g(A F)=m g(r+r / 2)=\frac{3 m g r}{2}.\notag\\
\text{Total energy }&=m g r\left[\frac{1}{2}+\frac{3}{2}\right]=2 m g r.\label{7}
	\end{align}
	If the ring has to reach at $D$, with kinetic energy zero, its potential energy $=m g(2 r)$.
	Becuase the particle at $C$ has this amount of energy and it will just reach at $D$ with zero velocity.\\
	\begin{figure}[H]
		\centering
		\includegraphics[height=4.8cm,width=4.5cm]{diagram-20220217(1)-crop}
	\end{figure}
	Now consider the ring at $B$, a distance $r / 2$ below $O$. Resolving $m g$ in two parts and considering the equilibrium, we get
	\begin{align}
	N-m g \cos \theta&=\frac{m v_{1}^{2}}{r}\notag\\
	N&=\frac{m v_{1}^{2}}{r}+m g \cos \theta\label{8}
	\intertext{Falling from $C$, i.e., through a distance $r$, the ring has lost a potential energy $m g r .$ This is the gain in kinetic energy.}
\text{	Hence, }\frac{1}{2} m v_{1}^{2}&=\frac{1}{2} m g r+m g r=\frac{3}{2} m g r \text{or }v_{1}^{2}=3 g r\text{ or }\frac{v_{1}^{2}}{r}=3 g.\notag\\
\text{ Fromequation (\ref{8}),} N&=m\left(3 g+g \times \frac{1}{2}\right)
	\left(\because \cos \theta=\frac{1}{2}\right)\notag\\\text{ or}& N=\left(\frac{7}{2}\right) \mathrm{mg}=3.5 \mathrm{mg}.\notag
	\end{align}
\end{answer}
\item A $2.0 \mathrm{~kg}$ block of mass, initially at rest, is dropped from a height of $0.40$ meter onto a spring whose force constant is $1960 \mathrm{nt} /$ meter. Find the maximum distance that the spring will be compressed.
\begin{answer}
	The situation is shown in figure. Let $m$ be the mass of the block and $k$ be the force constant of the spring.Let $l$ be the distance through which the spring is compressed. The total vertical fall of the block is $(h+l)$. Loss of the gravitational potential energy of the block $=m g(h+l)$.\\
	\begin{figure}[H]
		\centering
		\includegraphics[height=5cm,width=2.5cm]{diagram-20220217(2)-crop}
	\end{figure}
	\begin{align*}
	\text{Elastic potential energy }&\text{in the spring }=\left(\frac{1}{2}\right) k t^{2}.
\intertext{	By the law of conservation of energy}
	m g(h+l)&=\frac{1}{2} k t^{2} \text { or } \frac{2 m g h}{k}+\frac{2 m g}{k} l=l^{2} \text { or } l^{2}-\frac{2 m g}{k} l-\frac{2 m g h}{k}=0\\
	\therefore \quad l&=\frac{\left(\frac{2 m g}{k}\right) \pm \sqrt{\left[\left(\frac{2 m g}{k}\right)^{2}+\left(\frac{8 m g h}{k}\right)\right]}}{2}\\
	\text{According to given problem, }m&=2 \mathrm{~kg}, h=0.40 \mathrm{~m} \text{and $k=1960$ newton/meter. Hence,}\\
	l&=\frac{(2 \times 2 \times 9.8 / 1960) \pm \sqrt{(2 \times 2 \times 9.8 / 1960)^{2}+(8 \times 2 \times 9.8 \times 0.40 / 1960)}}{2} \\
	l&=\mathbf{0 . 1} \text { meter. }
	\end{align*}
\end{answer}
\item A particle of mass $3 \mathrm{~kg}$ is moving under the action of a central force whose potential energy is given by $U(r)=10 r^{3}$ joule. For what energy and angular momentum will the orbit be a circle of radius $10 \mathrm{~m}$ ? Calculate the time period of this motion.
\begin{answer}
	\begin{align*}
	\text{Given that }U(r)&=10 r^{3}.
	\intertext{So the force $F$ acting on the particle is given by}
	F&=\frac{\partial U}{\partial r}=-\frac{\partial}{\partial r}\left(10 r^{3}\right)=-10 \times 3 r^{2}=-30 r^{2}\\
	\text{For circular motion of the particle }F&=\frac{m v^{2}}{r}=30 r^{2}.\\
	\text{Substituting the given values, we have }\frac{3 \times v^{2}}{10}&=30 \times(10)^{2}\text{ or }v=100 \mathrm{~m} / \mathrm{s}.\\
	\text{The total energy in circular motion}
	E&=\text{ K.E. }+\text{ P.E. }=\frac{1}{2} m v^{2}+U(r)\\&=\frac{1}{2} \times 3 \times(100)^{2}+10 \times(10)^{3}=2.5 \times 10^{4}\text{ joule}\\
	\text{Angular momentum }&=m v r=3 \times 100 \times 10=3000 \mathrm{~kg}-\mathrm{m}^{2} / \mathrm{sec}\\
	\text{Time period }T&=\frac{2 \pi r}{v}=\frac{2 \times \pi \times 10}{100}=\frac{\pi}{5} \mathrm{sec} .
	\end{align*}
\end{answer}
\item In figure, $A B C D E$ is a channel in the vertical plane, part $B C D E$ being circular with radius $r$. A ball is released from $A$ and slides without friction and without rolling. Show that it will complete the loop path if $h$ is greater than $5 r / 2$.
\begin{figure}[H]
	\centering
	\includegraphics[height=3cm,width=5cm]{diagram-20220217(3)-crop}
\end{figure}
\begin{answer}
	\begin{align}
\intertext{	Let $m$ be the mass of the ball. When the ball comes down to $B$, its loses its potential energy $m g h$ which is converted into kinetic energy. Let $v_{B}$, be the velocity of the ball at $B$. Then,}\notag\\
	m g h&=\frac{1}{2} m v_{B}^{2}
\intertext{	The ball now rises to a point $D$, where its potential energy is $m g(h-2 r) .$ If $v_{D}$ be the velocity of the ball at $D$, then}\notag\\
	m g(h-2 r)&=\frac{1}{2} m v_{D}^{2}\label{09}
\intertext{	Now to complete the circular path, it is necessary that the centripetal force acting upward at point $D$ should be equal or greater than the force $m g$ acting downward. Therefore,}\notag\\
	\frac{m v_{D}^{2}}{r} \geq m g&\text{ or }v_{D}^{2} \geq r g\\
\text{	Fromequation (\ref{09}) }v_{D}^{2}&=2 g(h-2 r)\notag\\
	\therefore \quad 2 g(h-2 r) \geq &r g, h \geq \frac{5}{2} r .\notag
	\end{align}
\end{answer}
\item A moving particle of mass $m$ collides head-on with a particle of mass $2 m$ which is initially at rest. Show that the particle $m$ will loose $8 / 9$ th part of its initial kinetic energy after the collision.
\begin{answer}
	Let $u_{1}$ be the initial velocity of mass $m$ before collision and $v_{1}$ and $v_{2}$ the velocities of masses $m$ and $2 m$ after collision respectively.\\
	According to the law of conservation of kinetic energy, we have
	\begin{align}
 \frac{1}{2} m u_{1}^{2}&=\frac{1}{2} m v_{1}^{2}+\frac{1}{2}(2 m) v_{2}^{2}\notag \\ \text { or } & u_{1}^{2}-v_{1}^{2}=2 v_{2}^{2} \notag\\ \text { or } & \left(u_{1}-v_{1}\right)\left(u_{1}+v_{1}\right)=2 v_{2}^{2}\label{10}
\intertext{ By the law of conservation of momentum, we have}\notag
 m u_{1}&=m v_{1}+(2 m) v_{2} \notag\\
 \left(u_{1}-v_{1}\right)&=2 v_{2}\label{11}\\
\text{ From eqs. (\ref{10}) and (\ref{11}), we get }\left(u_{1}+v_{1}\right)&=v_{2}\label{12}
\intertext{ Substituting the value of $v_{2}$ from eq. (\ref{12}) in eq. (\ref{11}), we get}\notag\\
u_{1}-v_{1}=2\left(u_{1}+v_{1}\right) \text { or }-3 v_{1}&=u_{1} \text { or } v_{1}=-\left(\frac{1}{3}\right) u_{1}\label{13}
\intertext{Now the initial and final kinetic energies of mass $m$ are}
K_{i}=\left(\frac{1}{2}\right) m u_{1}^{2} \text { and } K_{f}&=\left(\frac{1}{2}\right) m v_{1}^{2}=\left(\frac{1}{2}\right) m\left(\frac{u_{1}^{2}}{9}\right)\notag\\
\text{Therefore, fraction loss }&=\frac{K_{i}-K_{f}}{K_{i}}=\frac{\frac{1}{2} m u_{1}^{2}\left(1-\frac{1}{9}\right)}{\frac{1}{2} m u_{1}^{2}}=\frac{8}{9}\notag
	\end{align}
\end{answer}
\item A solid hemisphere of mass $M$ and radius $R$ is placed against a smooth wall as shown in the figure. what is normal reaction on the sphere due to the wall.
\begin{answer}
	Centre of mass of the hemisphere lies at a distance $\frac{3 R}{8}$ from
	its base centre. The sphere is in equilibrium therefore torque about point A must be zero.
	\begin{figure}[H]
		\centering
		\includegraphics[height=3.2cm,width=9cm]{diagram-20220217(4)-crop}
	\end{figure}
	\begin{align*}
	\therefore N R-M g \cdot \frac{3 R}{8}&=0\\
	\text{or, }N&=\frac{3 M g}{8}
	\end{align*}
\end{answer}
\item A thin rod of mass $M$ and length $L$ is suspended from one end while its other end lies on a smooth horizontal plane and makes an angle $\alpha$ with the plane. Find the normal reaction from the plane and tension in the string from which it is suspended.
\begin{figure}[H]
	\centering
	\includegraphics[height=3cm,width=5cm]{diagram-20220217(5)-crop}
\end{figure}
\begin{align}
\text{For translational equilibrium }T+N&=M g \label{14}\\
\text{For rotational }&\text{equilibrium }\notag\\
\text{Torque about lower}&\text{ end of the rod is zero}\notag\\
	\therefore T . L \cos \alpha-M g \frac{L}{2} \cos \alpha&=0 \quad \therefore \quad T=\frac{M g}{2}\notag\\
	\text{from (\ref{14}) }N&=M g-T=M g-\frac{M g}{2}=\frac{M g}{2}\notag
\end{align}
Example: A uniform ladder of length 2L and mass 'm' leans against a wall in a vertical plane at an angle $\theta$ to the horizontal. The floor is rough, having a coefficient of static friction $\mu$
A person of mass $M$ stands on the ladder at a distance $D$ from its base (see figure). If the wall is frictionless, the maximum distance $\left(D_{\max }\right)$ up the ladder that the person can reach before the ladder slips is
\begin{figure}[H]
	\centering
	\includegraphics[height=4cm,width=4cm]{diagram-20220217(6)-crop}
\end{figure}
\begin{answer}
	Let us take ladder plus person as system. Various forces acting on the system are shown in the figure. For translational equilibrium we must have
	\begin{align}
	N_{1}&=(M+m) g\label{15}\\
	N_{2}&=f_{r}\label{16}\\
	\intertext{	For rotational equilibrium torque about lower end of the ladder must be zero. Therefore,}\notag\\
	M g D \cos \theta+m g L \cos \theta-N_{2} .2 L \sin \theta&=0\notag\\
	\therefore N_{2}&=\frac{g \cot \theta}{2}\left[\frac{M D}{L}+m\right]\notag\\
	\therefore\text{ from (\ref{16}) we get, }f_{r}&=\frac{g \cot \theta}{2}\left[\frac{D}{L} M+m\right]\notag\\
	\text{since }f_{r} \leq \mu N_{1}, \quad &\therefore \frac{g \cot \theta}{2}\left[\frac{D}{L} M+m\right] \leq \mu(M+m) g\notag\\
\text{	or }D \leq \frac{L}{M}[2 \mu(M+m) \tan \theta-m] \quad &\therefore D_{\max }=L\left[2 \mu\left(1+\frac{m}{M}\right) \tan \theta-\frac{m}{M}\right]\notag
	\end{align}
 \begin{figure}[H]
 	\centering
 	\includegraphics[height=4.5cm,width=4.5cm]{diagram-20220217(7)-crop}
 \end{figure}
\end{answer}
\item A body of mass $m$ rests on a horizontal plane with the friction coefficient $\mu .$ At the moment $t=0$, a horizontal force is applied to it, which varies with time $F=k t$, where $k$ is a constant vector. Find the distance traversed by the body during first/seconds after the force action began.
\begin{answer}
	Here just after applying the force, the motion does not start due to friction force. As the applied force is proportional to time, let after a time $t_{0}$, the motion starts.
	\begin{align*}
	\text{Now, }F=k t_{0}=\mu m g\text{ or }t_{0}=\left(\frac{\mu m g}{k}\right)
\intertext{	If $t \leq t_{0}$, the distance traversed by the body $s=0 .$}
\text{	When }t \leq t_{0},\text{ then }F&=k\left(t-t_{0}\right)
	\end{align*}
		\begin{align}
	\therefore \quad m \frac{d v}{d t}&=k\left(t-t_{0}\right)\text{ or }m d v=k\left(t-t_{0}\right) d t \ldots \label{17}\\
\text{	Integrating equation (\ref{17}), we get } m v&=\frac{k}{2}\left(t-t_{0}\right)^{2}+C_{1}\notag\\
	\text{When }t&=t_{0}, v=0, \therefore C_{1}=0\notag\\
	\therefore \quad m v&=\frac{k}{2}\left(t-t_{0}\right)^{2}\\
	\text{Again }m \frac{d s}{d t}&=\frac{k}{2}\left(t-t_{0}\right)^{2}\text{ or }m d s=\frac{k}{2}\left(t-t_{0}\right)^{2} d t\label{19}\\
	\text{Integrating equation (\ref{19}), we get }s&=\frac{k}{6 m}\left(t-t_{0}\right)^{3}+C_{2}\notag\\
\text{	Here }C_{2}&=0\text{ because when }t=t_{0}, s=0\notag\\
	\therefore \quad s&=\frac{k}{6 m}\left(t-t_{0}\right)^{3} .\notag
	\end{align}
\end{answer}
\item A falling rain drop accumulates moisture due to which its radius increases at a constant rate $k$. Neglecting drag calculate the speed of rain drop after it has fallen for a time $t$.
\begin{answer}
	\begin{align}
	\text{Equation of motion }\frac{d p}{d t}&=m g\label{20}\\
	\text{given }\frac{d r}{d t}&=k \Rightarrow r=k t\text{ (if initial radius is zero)}\notag\\
	\text{since }m&=\frac{4}{3} \pi r^{3} \rho \quad \therefore \frac{d m}{d t}=4 \pi r^{2} \cdot \frac{d r}{d t} \cdot \rho\notag
\intertext{	$\rho$ is density, we assume it to be constant}\notag
	\frac{d m}{d t} &=\left(\frac{4}{3} \pi r^{3} \rho\right) \cdot \frac{3 k}{r} \notag\\
	\frac{d m}{d t} &=\frac{3 m k}{r}\label{21}\\
\intertext{	Dividing (\ref{20}) by (\ref{21}) we get}\notag
\frac{d p}{d m}&=\frac{g}{3 k} r \quad d p=\left(\frac{g}{3 k}\right) \cdot\left(\frac{3 m}{4 \pi \rho}\right)^{1 / 3} d m\notag\\
\intertext{on integration we get}\notag
\therefore \quad p&=\left(\frac{g}{3 k}\right)\left(\frac{3}{4 \pi \rho}\right)^{1 / 3} \cdot \frac{m^{4 / 3}}{4 / 3}+c\notag\\
\text{taking }p&=m=0\text{ at }t=0,\text{ we get }c=0\notag\\
\therefore \quad p&=\frac{g}{4 k}\left(\frac{3}{4 \pi \rho}\right)^{1 / 3} \cdot m^{4 / 3}\notag\\
\text{or }v&=\left(\frac{g}{4 k}\right) \cdot\left(\frac{3 m}{4 \pi \rho}\right)^{1 / 3}\notag\\
v&=\frac{g}{4 k} r \Rightarrow v=\frac{g}{4} t\notag
	\end{align}
\end{answer}
	\item A particle of mass $2 \mathrm{~kg}$ is moving such that at time $t$, its position, in metre, is given by $\vec{r}(t)=5 \hat{i}-2 t^{2} \hat{j}$. The angular momentum of the particle at $t=2 \mathrm{~s}$ about the origin, in $\mathrm{kg} \mathrm{m}^{2} \mathrm{~s}^{-1}$, is
	 \begin{tasks}(4)
		\task[\textbf{a.}]$-40 \hat{k}$
		\task[\textbf{b.}]$-80 \hat{k}$
		\task[\textbf{c.}]$80 \hat{k}$
		\task[\textbf{d.}]  $40 \hat{k}$
	\end{tasks}
	\begin{answer}
		\begin{align*}
		\vec{r} &=5 \hat{i}-2 t^{2} \hat{j} \\
		\vec{v} &=\frac{d \vec{r}}{d t}=-4 t \hat{j} \\
		\vec{p} &=m \vec{v}=2(-4 t \hat{j})=-8 t \hat{j} \\
		\vec{L} &=\vec{r} \times \vec{p}=\left(5 \hat{i}-2 t^{2} \hat{j}\right) \times(-8 t \hat{j}) \\
		&=-40 t \hat{k}=-40 \times 2 \hat{k}=-80 \hat{k}
		\end{align*}
			Hence correct answer is (b)
	\end{answer}
	\item The scalar potential corresponding to the force field $\vec{F}=\hat{i}(y+z)$
	 \begin{tasks}(4)
		\task[\textbf{a.}]Is $y^{2} / 2$
		\task[\textbf{b.}] Is 1
		\task[\textbf{c.}]Is zero
		\task[\textbf{d.}] Does not exist
	\end{tasks}
	\begin{answer}
		\begin{align*}
		\vec{F} &=\hat{i}(y+z) \\
	\vec{\nabla} \times \vec{F}&=\left|\begin{array}{ccc}\hat{i} & \hat{j} & \hat{k} \\ \frac{\partial}{\partial x} & \frac{\partial}{\partial y} & \frac{\partial}{\partial z} \\ y+z & 0 & 0\end{array}\right|=\hat{i}(0-0)-\hat{j}(0-1)+\hat{k}(0-1)=-\hat{j}-\hat{k} \neq 0
		\end{align*}
			Force is not conservative so we cannot define potential. \\Hence, correct answer is (d)
	\end{answer}
\end{enumerate}
\section{Stability Analysis}
\begin{enumerate}
	\item  A particle of mass $m$ is moving under a one dimensional potential $V(x)=-a x+b x^{2}$ where $a>$ 0 and $b>0$. Find the equilibrium points and find frequency of oscillation about the stable equilibrium.
	\begin{answer}
		\begin{align*}
		\text{At equilibrium point, }\left.\frac{d V}{d x}\right|_{x=x_{0}}=0\\
		\therefore-a+2 b x_{0}=0 \quad\text{ or }\quad x_{0}&=\frac{a}{2 b}\\
	\text{	Thus, }x_{0}&=\frac{a}{2 b}\text{ is an equilibrium point.}
\intertext{	To know whether it is stable or To know whether it is stable or unstable equilibrium point let us calculate second derivative of potential at equilibrium point.}
	\left.\frac{d^{2} V}{d x^{2}}\right|_{x=x_{0}}&=2 b>0\\
	\text{Therefore }x_{0}=\frac{a}{2 b}&\text{ is stable equilibrium point, force constant }k=\left.\frac{d^{2} V}{d x^{2}}\right|_{x=x_{0}}=2 b
	\intertext{Therefore, frequency of oscillation is}
	\omega&=\sqrt{\frac{k}{m}}=\sqrt{\frac{2 b}{m}}
		\end{align*}
	\end{answer}
	\item  Potential corresponding to force between the atoms of a diatomic molecule is $V(r)=\frac{a}{r^{12}}-\frac{b}{r^{6}}$ where $a$ and $b$ are positive constants and $r$ is separation between the atoms. Calculate bond length for stable configuration and also calculate frequency of oscillation of atoms if mass of each atom be $m$.
	\begin{answer}
		\begin{align*}
	\text{	For stable configuration }\left.\frac{d V}{d r}\right|_{r=r_{0}}&=0\\
		-\frac{12 a}{r_{0}^{13}}+\frac{6 b}{r_{0}^{7}}=0 \Rightarrow r_{0}&=\left(\frac{2 a}{b}\right)^{1 / 6}\\
		\text{Therefore bond length for stable configuration is }&\left(\frac{2 a}{b}\right)^{1 / 6}\\
		\text{Force constant }k=\left.\frac{d^{2} V}{d r^{2}}\right|_{r=r_{0}}&=\frac{12 \times 13 a}{r_{0}^{14}}-\frac{6 \times 7 b}{r_{0}^{8}}=\frac{1}{r_{0}^{8}}\left(\frac{12 \times 13 a}{r_{0}^{6}}-6 \times 7 b\right)\\
		&=\left(\frac{b}{2 a}\right)^{8 / 6}\left(\frac{12 \times 13 a}{2 a / b}-42 b\right)\\&=\left(\frac{b}{2 a}\right)^{4 / 3} \cdot 36 b=\frac{18}{2^{1 / 3}} \cdot \frac{b^{7 / 3}}{a^{4 / 3}}\\
		\text{Reduced mass of system }\mu&=\frac{m \cdot m}{m+m}=m / 2\\
		\text{Frequency of oscillation} \omega=\sqrt{\frac{k}{\mu}}&=\sqrt{\frac{18}{2^{1 / 3}} \cdot \frac{b^{7 / 3}}{m / 2 a^{4 / 3}}}=6\left(\frac{b^{7}}{2 m^{3} a^{4}}\right)^{1 / 6}
		\end{align*}
	\end{answer}
	
	\item  A particle of mass ' $m$ ' is moving under potential $V(x)=a x^{3}-b x^{2}$. Initially the particle is at at stable point. What minimum speed be given to the particle so that it reaches unstable point. Plot potential versus $x$.
	\begin{answer}
		\begin{align*}
		\text{For equilibrium point }\left.\frac{d V}{d x}\right|_{x=x_{0}}&=0\\
		3 a x_{0}^{2}-2 b x_{0}&=0 \Rightarrow x_{0}=0, \frac{2 b}{3 a}\\
		\left.\frac{d^{2} V}{d x^{2}}\right|_{x=x_{0}}&=6 a x_{0}-2 b,\left.\frac{d^{2} V}{d x^{2}}\right|_{x=0}=-2 b<0, \therefore x_{0}=0\text{ is unstable point }\\
		\left.\frac{d^{2} V}{d x^{2}}\right|_{x=\frac{2 b}{3 a}}&=2 b>0, \therefore x_{0}=\frac{2 b}{3 a}\text{ is stable point}
		\intertext{To calculate speed let us apply conservation of energy.}
		\text{Total energy initial }&=\text{ Total energy final}
		\intertext{(Kinetic Energy + Potential Energy) at $x=\frac{2 b}{a}=($ Kinetic Energy $+$ Potential Energy) at $x=0$ for minimum speed $(u)$ at stable point the particle will just reach unstable point and stops there.}
			\therefore \frac{1}{2} m u^{2}+V\left(x=\frac{2 b}{3 a}\right)&=\frac{1}{2} m .0^{2}+V(x=0)\\
		\frac{1}{2} m u^{2}+a\left(\frac{2 b}{3 a}\right)^{3}-b\left(\frac{2 b}{3 a}\right)^{2}&=0+0\\
		u^{2}=-\frac{2}{m}\left(\frac{2 b}{3 a}\right)^{2}\left(\frac{2 b}{3}-b\right)&=\frac{2}{m} \cdot \frac{4 b^{2}}{9 a^{2}} \cdot \frac{b}{3}=\frac{8 b^{3}}{27 m a^{2}}\\
		\therefore u&=\sqrt{\frac{8 b^{3}}{27 m a^{2}}}
		\end{align*}
		\begin{figure}[H]
			\centering
			\includegraphics[height=3cm,width=5cm]{stability001}
		\end{figure}
	\end{answer}
	\item  A particle is moving under potential $V(r)=\frac{a}{r^{2}}-\frac{b}{r} .$ Calculate the minimum value of potential energy.
	\begin{answer}
		\begin{align*}
	\text{	For potential to be minimum }\left.\frac{d V}{d r}\right|_{r=r_{0}}=0\\
	 -\frac{2 a}{r_{0}^{3}}+\frac{b}{r_{0}^{2}}=0 \quad \therefore r_{0}&=\frac{2 a}{b}\\
	 \left.\frac{d^{2} V}{d r^{2}}\right|_{r=r_{0}}=\frac{6 a}{r_{0}^{4}}-\frac{2 b}{r_{0}^{3}}&=\frac{1}{r_{0}^{3}}\left(\frac{6 a}{r_{0}}-2 b\right)=\frac{1}{r_{0}^{3}}\left(\frac{6 a}{2 a / b}-2 b\right)=\frac{b}{r_{0}^{3}}>0\\
	 \text{Therefore at }r&=r_{0}\text{ potential is minimum}\\
	 \therefore \quad V_{\min }&=V\left(r_{0}\right)=\frac{1}{r_{0}}\left(\frac{a}{r_{0}}-b\right)=-\frac{b^{2}}{4 a}
		\end{align*}
	\end{answer}
	\item  A cube is placed on the top of a fixed hemisphere as shown in figure. What should be relation between length of side of cube and radius of hemisphere so that cube has stable equilibrium.
		\begin{figure}[H]
		\centering
		\includegraphics[height=3cm,width=6cm]{diagram-20220219(2)-crop}
	\end{figure}
	\begin{answer}
		$\left. \right. $
		\begin{figure}[H]
			\centering
			\includegraphics[height=5.5cm,width=9cm]{diagram-20220219(3)-crop}
		\end{figure}
		To discuss equilibrium of cube we first write its potential energy as function of angle from its equilibrium position. As shown in the figure below, initially point A was in contact with spherical surface but now in displaced position point $\mathrm{B}$ is in contact. Therefore,
		\begin{align*}
		A B&=R \theta
		\intertext{Height of centre of cube from centre level of hemisphere is}
		h&=O C+O^{\prime} D \\
		&=O O^{\prime} \sin \theta+O^{\prime} S \cos 0 \\
		&=A B \sin \theta+\left(O^{\prime} B+B S\right) \cos \theta \\
		&=R \theta \sin \theta+\left(\frac{L}{2}+R\right) \cos \theta
		\intertext{Potential energy of the cube}
		V(\theta)&=M g h=M g\left[R \theta \sin \theta+\left(\frac{L}{2}+R\right) \cos \theta\right]
		\intertext{$\theta=0$ is equilibrium position of the cube. For this position to be stable equilibrium position, $\left.\frac{d^{2} V}{d \theta^{2}}\right|_{\theta=0}>0$}
	&\therefore \frac{d^{2}}{d \theta^{2}} M g\left[R \theta \sin \theta+\left(\frac{L}{2}+R\right) \cos \theta\right]_{\theta=0}>0\\
&\text{	or }\frac{d}{d \theta}\left[R \theta \cos \theta+R \sin \theta-\left(\frac{L}{2}+R\right) \sin \theta\right]_{\theta=0}>0\\
	&\text{or }\left[R \cos \theta-R \theta \sin \theta+R \cos \theta-\left(\frac{L}{2}+R\right) \cos \theta\right]_{\theta=0}>0\\
	&\text{or }\left[2 R-\left(\frac{L}{2}+R\right)\right]>0 \quad \therefore R-\frac{L}{2}>0\text{ or }\quad 2 R>L
	\intertext{Thus, cube can be in stable equilibrium position if its side length is less than diameter of hemisphere}
		\end{align*}
	\end{answer}
	\item  For the mass pulley system shown in figure, what should be relation between $m$ and $M$ so that system remains in stable equilibrium position. Pulley are smooth and strings are tight and inextensible
	\begin{figure}[H]
		\centering
		\includegraphics[height=4cm,width=5cm]{diagram-20220218(4)-crop}
	\end{figure}
	\begin{answer}
		The pulleys are fixed. Therefore we can write potential energy of the system by specifying position of blocks with respect to pulleys.\\
		Let ' $l$ ' be length of string and ' $d$ ' be the half distance between two pulleys. Therefore, $l$ and $d$ are constants If $x$ be distance of $m$ below the pulley as shown in figure then potential energy of system is
		\begin{figure}[H]
			\centering
			\includegraphics[height=3cm,width=6cm]{diagram-20220219(4)-crop}
		\end{figure}
		\begin{align}
		V(x)=-2 m g x-M g \sqrt{(l-x)^{2}-d^{2}} \quad &\therefore \frac{d V}{d x}=-2 m g+\frac{M g(l-x)}{\sqrt{(l-x)^{2}-d^{2}}}\notag\\
		\text{for equilibrium }\left.\frac{d V}{d x}\right|_{x=x_{0}}=0 \quad &\therefore-2 m g+\frac{M g\left(l-x_{0}\right)}{\sqrt{\left(l-x_{0}\right)^{2}-d^{2}}}=0\notag\\
	\text{	or }\frac{2 m}{M}&=\frac{l-x_{0}}{\sqrt{\left(l-x_{0}\right)^{2}-d^{2}}}\label{23}\\
	\left.\frac{d^{2} V}{d x^{2}}\right|_{x=x_{0}}&=\frac{M g d^{2}}{\left[\left(l-x_{0}\right)^{2}-d^{2}\right]^{3 / 2}}>0\text{ for all values of $d>0$}\notag\\
	\text{since, }\frac{l-x_{0}}{\sqrt{\left(l-x_{0}\right)^{2}-d^{2}}}>1 \quad &\therefore\text{ from (\ref{23}) }\frac{2 m}{M}>1\text{ or }2 m>M\notag
		\end{align}
	\end{answer}
\item The potential energy between two atoms are given $v(r)=\frac{a}{r^{12}}-\frac{b}{r^{6}}$ where $a, b$ positive constants.
(i) Find the equilibrium distance of two atoms.\\
(ii) Plot the potential\\
(iii) Calculate the frequency of small oscillation.\\
	\begin{answer}
		\begin{align*}
		\intertext{(i) For equilibrium potential energy should be minimum.}
		\frac{\mathrm{dU}}{\mathrm{dr}}=-\frac{12 \mathrm{a}}{\mathrm{r}^{13}}+\frac{6 \mathrm{~b}}{\mathrm{r}^{7}}&=0 \Rightarrow \mathrm{r}^{6}=\frac{2 \mathrm{a}}{\mathrm{b}} \\
		\mathrm{r}=\left(\frac{20}{\mathrm{~b}}\right)^{1 / 6}&=\mathrm{r}_{0} \\
		\mathrm{U}\left(\mathrm{r}_{0}\right)=\frac{\mathrm{ab}^{2}}{4 \mathrm{a}^{2}}-\frac{\mathrm{b} \cdot \mathrm{b}}{2 \mathrm{a}}&=\frac{\mathrm{b}^{2}}{4 \mathrm{a}}-\frac{\mathrm{b}^{2}}{2 \mathrm{a}}=-\frac{\mathrm{b}^{2}}{4 \mathrm{a}}\\
		\frac{\mathrm{dU}}{\mathrm{dr}}=-\frac{12 \mathrm{a}}{\mathrm{r}^{13}}+\frac{6 \mathrm{~b}}{\mathrm{r}^{7}}&=0 \Rightarrow \mathrm{r}^{6}=\frac{2 \mathrm{a}}{\mathrm{b}}\\ \mathrm{r}=\left(\frac{20}{\mathrm{~b}}\right)^{1 / 6}&=\mathrm{r}_{0} \quad[\text{ Position of stability equilibrium }]\\ \mathrm{U}\left(\mathrm{r}_{0}\right)=\frac{\mathrm{ab}^{2}}{4 \mathrm{a}^{2}}-\frac{\mathrm{b} \cdot \mathrm{b}}{2 \mathrm{a}}&=\frac{\mathrm{b}^{2}}{4 \mathrm{a}}-\frac{\mathrm{b}^{2}}{2 \mathrm{a}}=-\frac{\mathrm{b}^{2}}{4 \mathrm{a}}
			\end{align*}
			\begin{figure}[H]
				\centering
				\includegraphics[height=5cm,width=9cm]{diagram-20220219(5)-crop}
			\end{figure}
				\begin{align*}
		\mathrm{U}\left(\mathrm{r}_{0}\right)=\frac{-\mathrm{b}^{2}}{4 \mathrm{a}} ; \mathrm{U}(\mathrm{r})&=\mathrm{U}\left(\mathrm{r}_{0}\right)+\left.\left(\mathrm{r}-\mathrm{r}_{0}\right) \frac{\mathrm{dU}}{\mathrm{dr}}\right|_{\mathrm{r}_{0}}+\left.\frac{\left(\mathrm{r}-\mathrm{r}_{0}\right)^{2}}{2} \frac{\mathrm{d}^{2} \mathrm{U}}{\mathrm{dr}^{2}}\right|_{\mathrm{c}_{0}}\\
		\frac{\mathrm{d}^{2} \mathrm{U}}{\mathrm{dr}^{2}}=\frac{12 \times 13 \mathrm{a}}{\mathrm{r}^{14}}-\frac{42 \mathrm{~b}}{\mathrm{r}^{8}} ;\left.\frac{\mathrm{d}^{2} \mathrm{U}}{\mathrm{dr}^{2}}\right|_{\mathrm{k}_{0}}&=\frac{156 \mathrm{a}}{\mathrm{r}^{14}}-\frac{42 \mathrm{~b}}{\mathrm{r}^{8}}=\frac{156 \mathrm{a}}{(2 \mathrm{a})^{1 / 6}}-\frac{42 \mathrm{~b}}{\left(\frac{2 \mathrm{a}}{\mathrm{h}}\right)^{3 / 6}}\\
		&=\frac{156 \mathrm{a}}{\left(\frac{2 \mathrm{a}}{\mathrm{b}}\right)^{1 / 3}}-\frac{42 \mathrm{~b}}{\left(\frac{2 \mathrm{a}}{\mathrm{b}}\right)^{1 / 3}}=\text { constant }=c\\
	\intertext{	(Force constant equivalent to spring constant)}
		U(r)&=U\left(r_{0}\right)+\frac{c}{2}\left(r-r_{0}\right)^{2}\\
		\text{So, the force }&=-\frac{\mathrm{dU}}{\mathrm{dr}}=-\frac{\mathrm{c}}{2} \cdot 2\left(\mathrm{r}-\mathrm{r}_{0}\right)^{2}=-\mathrm{c}\left(\mathrm{r}-\mathrm{r}_{0}\right)\\
		\text{Force }&=-\nabla U ; U=U(r)\text{ only}\\
		\overrightarrow{\mathrm{F}}&=\mathrm{m} \overline{\mathrm{a}} \text{(Here introduced to harmonic oscillator maynecessary).}\\
		 \text{Equation of motion, }&\mathrm{m} \frac{\mathrm{d}^{2} \mathrm{r}}{\mathrm{dt}^{2}}=-\mathrm{c}\left(\mathrm{r}-\mathrm{r}_{0}\right) ; \frac{\mathrm{d}^{2} \mathrm{r}}{\mathrm{dt}^{2}}+\frac{\mathrm{c}}{\mathrm{m}}\left(\mathrm{r}-\mathrm{r}_{0}\right)=0\\
		 \text{ Take }\mathrm{c}&=m \omega^{2} ; \omega=\sqrt{\frac{\mathrm{c}}{\mathrm{m}}}
		 \intertext{A particle of mass ' $\mathrm{m}$ ' moving in a potential} \mathrm{V}(\mathrm{x})&=\frac{1}{2} \mathrm{~m} \omega_{0}^{2} \mathrm{x}^{2}+\frac{\mathrm{a}}{2 \mathrm{~m} \mathrm{x}^{2}} \quad\left(\omega_{0} \&\right. \text{are constant })
		 \intertext{ Find the angular trequency of small oscillation.}
		\frac{\mathrm{dv}}{\mathrm{dx}}&=m \omega_{0}^{2} \mathrm{x}-\frac{\mathrm{a}}{\mathrm{m} \mathrm{x}^{3}}=0\\
		m \omega_{0}^{2} x_{0}-\frac{a}{m x^{3}}&=0 ; x_{0}^{4}=\frac{a}{m^{2} \omega_{0}^{2}} ; \chi_{0}=\left(\frac{a}{m^{2} \omega_{0}^{2}}\right)^{1 / 4}\\
		\text{Equilibrium distance, }\frac{d^{2} z}{d x^{2}}&=m \omega_{0}^{2}+\frac{3 a}{m x^{4}}\\
		\left.\frac{\mathrm{d}^{2} \mathrm{r}}{\mathrm{dx}^{2}}\right|_{\mathrm{x}=\mathrm{x}_{0}}&=m \omega_{0}^{2}+\frac{3 \mathrm{a}}{\mathrm{ma}} \mathrm{m}^{2} \omega^{2}=4 \mathrm{~m} \omega_{0}^{2} \quad\left(\mathrm{x}_{0}^{4}=\frac{\mathrm{a}}{\mathrm{m}^{2} \omega_{0}^{2}}\right)\\
		\mathrm{U}(\mathrm{x})&=\mathrm{U}\left(\mathrm{x}_{0}\right)+\left.\left(\mathrm{x}-\mathrm{x}_{0}\right) \frac{\mathrm{dU}}{\mathrm{dx}}\right|_{\mathrm{x}=\mathrm{x}_{0}}+\left.\frac{\left(\mathrm{x}-\mathrm{x}_{0}\right)^{2}}{2} \frac{\mathrm{d}^{2} \mathrm{U}}{\mathrm{dx}^{2}}\right|_{\mathrm{x}=\mathrm{x}_{0}}\\
		&=\mathrm{U}\left(\mathrm{x}_{0}\right)+\left(\mathrm{x}-\mathrm{x}_{0}\right)^{2} 2 \mathrm{~m} \omega_{0}^{2}\\
		\text{Force }&=-\frac{\mathrm{dU}}{\mathrm{dx}}=-4 \mathrm{~m} \omega_{0}^{2}\left(\mathrm{x}-\mathrm{x}_{0}\right)\\
	\text{	Equation of motion, }\mathrm{m} \frac{\mathrm{d}^{2} \mathrm{x}}{\mathrm{dt}^{2}}&=\mathrm{F}=-4 \omega_{0}^{2}\left(\mathrm{x}-\mathrm{x}_{0}\right)\\
		\frac{d^{2} x}{d t^{2}}+4 \omega_{0}^{2}\left(x-x_{0}\right)&=0\\
		\text{Hence, the frequency of small oscillation, }\omega&=\sqrt{4 \omega_{0}^{2}}=2 \omega_{0}.
		\end{align*}
	\end{answer}
	\item A particle of mass ' $m$ ' is constrained to move in one dimension under a potential $u(x)=\frac{\alpha}{x^{2}}-\frac{\beta}{x}$, where $\alpha$ and $\beta$ are positive constants. Show that period of small oscillations about the equilibrium point is
	$T=4 \pi \sqrt{\frac{2 \alpha^{3} m}{\beta^{4}}}$
	\begin{answer}
		\begin{align*}
	\text{	At equilibrium point }&\left(x=x_{0}\right)\\
		\left.\frac{d u}{d x}\right|_{x=x_{0}}=0 \\
		-\frac{2 \alpha}{x_{0}^{3}}+\frac{\beta}{x_{0}^{2}}&=0
		\hspace{1cm}\therefore x_{0}=\frac{2 \alpha}{\beta}=
		 \intertext{time period of oscillation is given by}
		T&=2 \pi \sqrt{\frac{m}{k}}
	\intertext{	where $k$ is force constant and is given by}
		k&=\left.\frac{d^{2} u}{d x^{2}}\right|_{x=x_{0}} =\frac{6 \alpha}{x_{0}^{4}}-\frac{2 \beta}{x_{0}^{3}}=6 \alpha \cdot \frac{\beta^{4}}{16 \alpha^{4}}-\frac{2 \beta^{4}}{8 \alpha^{3}} \\
		&=\frac{3}{8} \cdot \frac{\beta^{4}}{\alpha^{3}}-\frac{2}{8} \cdot \frac{\beta^{4}}{\alpha^{3}} \Rightarrow k=\frac{\beta^{4}}{8 \alpha^{3}}\\
		\therefore T&=2 \pi \sqrt{\frac{8 \alpha^{3}}{\beta^{4}} \cdot m}=4 \pi \sqrt{\frac{2 \alpha^{3} m}{\beta^{4}}}
		\end{align*}
	\end{answer}
\end{enumerate}
\section{Central Force Motion}
\begin{enumerate}
	\item  Equation of the orbit of a particle moving under central force is $r \theta=\beta$, where $\beta$ is a constant. Find the force acting on the particle.
	\begin{answer}
		\begin{align*}
		r \theta=\beta,\text{ therefore, }u&=\frac{1}{r}=\frac{\theta}{\beta} \quad \therefore \frac{\partial^{2} u}{\partial \theta^{2}}=0\\
		\text{	Differential equation of orbit, }\frac{\partial^{2} u}{\partial \theta^{2}}+u&=\frac{-m f(r)}{L^{2} u^{2}}\\
		\therefore 0+u=\frac{-m f(r)}{L^{2} u^{2}} \quad \therefore \quad f(r)&=\frac{-L^{2} u^{3}}{m}=\frac{-L^{2}}{m r^{3}}
		\end{align*}
	\end{answer}
	\item  Equation of orbit of a particle moving under central force is $r^{n}=a \cos n \theta .$ Find the force on the particle.
	\begin{answer}
		\begin{align}
		\text{	Given, }r^{n}=a \cos n \theta,\text{ therefore, }u^{n}=\frac{1}{a \cos n \theta}&=\frac{1}{a} \sec n \theta \label{25}\\
		\text{ Taking $\ln$ both sides we get, }n \ln u&=\ln \left(\frac{1}{a}\right)+\ln \sec n \theta\notag\\
		\text{Differentiating w.r.t. $\theta$ we get, }\frac{n}{u} \frac{\partial u}{\partial \theta}&=n \tan n \theta \quad \therefore \frac{\partial u}{\partial \theta}=u \tan n \theta\notag\\
		\text{Differentiating again w.r.t. $\theta$ we get }\frac{\partial^{2} u}{\partial \theta^{2}}&=\frac{\partial u}{\partial \theta} \tan n \theta+u n \sec ^{2} n \theta\notag\\&=u \tan ^{2} n \theta+u n \sec ^{2} n \theta\notag\\
		\text{ Differential equation of orbit is }\frac{\partial^{2} u}{\partial \theta^{2}}+u&=\frac{-m f(r)}{L^{2} u^{2}}\notag\\
		u \tan ^{2} n \theta+u n \sec ^{2} n \theta+u&=\frac{-m f(r)}{L^{2} u^{2}}, u \sec ^{2} n \theta+u n \sec ^{2} n \theta=\frac{-m f(r)}{L^{2} u^{2}}\notag\\
		\therefore f(r)&=-\frac{L^{2} u^{3}(1+n) \sec ^{2} n \theta}{m}\notag\\
		\text{from (\ref{25}) }\sec ^{2} n \theta&=a^{2} u^{2 n}\notag\\
		\therefore f(r)&=\frac{-L^{2}(n+1) a^{2} u^{2 n+3}}{m}\notag\\
		\therefore \quad f(r)&=\frac{-L^{2}(n+1) a^{2}}{m} \cdot \frac{1}{r^{2 n+3}}\notag\\
		\text{Or }
		f(r) \propto \frac{1}{r^{2 n+3}}&
		\text{and the force is attractive in nature.}\notag
		\end{align}
	\end{answer}
	\item  A particle of mass $m$ is moving under a central force. The angular momentum of the partix $\mathrm{L}$ and its equation of orbit is $r=A e^{k 0}$. Calculate potential energy of the particle.
	\begin{answer}
		\begin{align}
		\text{Potential energy is given as }V(r)&=-\int f(r) d r\label{26}
		\intertext{	Therefore, let us first find $f(r)$.}\notag
		\text{Given }r=A e^{k \theta},\text{ thereforc, }u&=\frac{1}{A} e^{-k \theta} \quad \therefore \frac{\partial^{2} u}{\partial \theta^{2}}=k^{2} u\notag\\
		\text{Differential equation of orbit is }\frac{\partial^{2} u}{\partial \theta^{2}}+u&=\frac{-m f(r)}{L^{2} u^{2}} \quad \therefore\left(k^{2}+1\right) u=\frac{-m f(r)}{L^{2} u^{2}}\notag\\
		\therefore f(r)&=\frac{-L^{2}\left(k^{2}+1\right) u^{3}}{m} \text { or } f(r)=\frac{-L^{2}\left(k^{2}+1\right)}{m r^{3}}\notag\\
		\text{Therefore, from (\ref*{26}) }V(r)&=-\int f(r) d r=\frac{-L^{2}\left(k^{2}+1\right)}{2 m r^{2}}\notag
		\end{align}
	\end{answer}
	\item  For a particle moving under gravitational force pericentre distance in parabolic orbit is $r_{p}$ while the radius of the circular orbit with same angular momentum is $r_{c}$. What is relation between $r_{p}$ and $r_c$ ?
	\begin{answer}
		Pericentre distance is the minimum distance $(\cos \theta=\max =1)$ and for parabolic orbit $e=1 .$
		\begin{align*}
		\text{Therefore, }r_{\min }&=\frac{l}{1+e \cos \theta}=\frac{l}{2}=r_{p}\\
		\text{for circular orbit $e=0$, therefore }r_{c}&=\frac{l}{1+e \cos \theta}=l \quad \therefore r_{p}=\frac{r_{c}}{2}
		\end{align*}
	\end{answer}
	\item  Ratio of maximum to minimum speed of a planet revolving around the sun in an elliptical orbit is $2: 1$, What is eccentricity of the orbit?
	\begin{answer}
		\begin{align*}
		\text{	Given }\frac{v_{\max }}{v_{\min }}=\frac{2}{1} \therefore \frac{\sqrt{\frac{G M}{a}\left(\frac{1+e}{1-e}\right)}}{\sqrt{\frac{G M}{a}\left(\frac{1-e}{1+e}\right)}}=\frac{2}{1} \quad\text{ or }\quad \frac{1+e}{1-e}=\frac{2}{1} \quad \therefore e=\frac{1}{3}
		\end{align*}
	\end{answer}
	\item  A planet is revolving around the sun in a circular orbit. Due to some reason the speed of the planet suddenly becomes double. What is new orbit of the planet.
	\begin{answer}
		\begin{align*}
		\text{	Orbital speed of the planet is }&\sqrt{\frac{G M}{r}}\\
		\text{New speed of the planet }&=2 \sqrt{\frac{G M}{r}}\\
		\text{Therefore, new energy of the planet }&=\frac{1}{2} m v^{2}-\frac{G M m}{r}=\frac{1}{2} m \cdot \frac{4 G M}{r}-\frac{G M m}{r}=\frac{2 G M m}{r}>0
		\intertext{	Total energy of the planet becomes positive on doubling its speed therefore new orbit of the planet will be hyperbolic.}
		\end{align*}
	\end{answer}
\end{enumerate}
\begin{enumerate}
	\item  A planet of mass $m$ and angular momentum $\mathrm{L}$ moves in a circular orbit in a potential, $V(r)=-k / r$ where $k$ is a positive constant find the radius of stable circular orbit. If the planet is slightly perturbed, find angular frequency of radial oscillation.
	\begin{answer}$\left. \right. $
		\begin{figure}[H]
			\centering
			\includegraphics[height=1.5cm,width=6cm]{diagram-20220221(2)-crop}
		\end{figure}
		\begin{align*}
		V_{e f f}=\frac{L^{2}}{2 m r^{2}}+V(r)&=\frac{L^{2}}{2 m r^{2}}-\frac{k}{r}\\
		\text{For stable point (orbit) }\left.\frac{\partial V_{e f f}}{\partial r}\right|_{r=r_{0} .}&=0\\
		\therefore-\frac{L^{2}}{m r_{0}^{3}}+\frac{k}{r_{0}^{2}}&=0 \text { or } r_{0}=\frac{L^{2}}{m k}\\
		\text{	Therefore radius of stable circular orbit is }L^{2} / \mathrm{mk}
		\end{align*}
		Angular frequency of oscillation about stable point:
		\begin{figure}[H]
			\centering
			\includegraphics[height=0.7cm,width=6.4cm]{diagram-20220221-crop}
		\end{figure}
		\begin{align*}
		\omega&=\sqrt{\frac{\left.\frac{\partial^{2} V_{e f f}}{\partial r^{2}}\right|_{r=r_{0}}}{m}}=\sqrt{\frac{\left(\frac{3 L^{2}}{m r_{0}^{4}}-\frac{2 k}{r_{0}^{3}}\right)}{m}}\\
		&=\sqrt{\frac{\frac{1}{r_{0}^{3}}\left(\frac{3 L^{2}}{m \cdot \frac{L^{2}}{m k}}-2 k\right)}{m}}=\sqrt{\frac{k}{m r_{0}^{3}}}=\sqrt{\frac{k^{4} m^{2}}{L^{6}}}=\frac{m k^{2}}{L^{3}}
		\end{align*}
		actual radial oscillation is as shown in figure.
		\begin{figure}[H]
			\centering
			\includegraphics[height=4cm,width=4cm]{diagram-20220221(1)-crop}
		\end{figure}
	\end{answer}
	\item  A small asteroid is approaching a massive star with a speed $v$ from very large distance, at an impact parameter ' $b$ ' as shown in figure. If the mass of the star is $M$ and its radius is $R$, then what is the minimum value of $b$ such that the asteroid will miss the star?
	\begin{answer}
		Angular momentum of asteroid about centre of star is $L=m v b$
		\begin{figure}[H]
			\centering
			\includegraphics[height=3.5cm,width=5cm]{diagram-20220221(3)-crop}
		\end{figure}
		\begin{align*}
		\text{Effective potential energy }V_{e f f}&=\frac{-G M m}{r}+\frac{L^{2}}{2 m r^{2}}=\frac{-G M m}{r}+\frac{m v^{2} b^{2}}{2 r^{2}}
		\end{align*}
		For minimum value of $b$ we will have to assume that asteroid just misses the star as shown in figure. Equivalent 1-d problem is also shown in figure
		\begin{figure}[H]
			\centering
			\includegraphics[height=2.5cm,width=12cm]{diagram-20220221(7)-crop}
		\end{figure}
		\text{Total energy is conserved in central force motion }
		\begin{align*}
		\text{Therefore, }E_{1}&=E_{2}\\
		\text{or }&\left(\frac{1}{2} m \dot{r}^{2}+V_{e f f}\right)_{\mathrm{at} r=\infty}=\left(\frac{1}{2} m \dot{r}^{2}+V_{e f f}\right)_{\mathrm{at} r=R}\\
		\frac{1}{2} m v^{2}-\frac{G M m}{\infty}+\frac{m v^{2} b^{2}}{\infty}&=\left(\frac{1}{2} m o^{2}-\frac{G M m}{R}+\frac{m v^{2} b^{2}}{2 R^{2}}\right)\\
		\therefore v^{2}+\frac{2 G M}{R}&=\frac{v^{2} b^{2}}{R^{2}} \quad \text { or } b=R \sqrt{1+\frac{2 G M}{R v^{2}}}
		\end{align*}
	\end{answer}
	\item  A particle is thrown in radially outward direction from earth's surface with initial speed $\sqrt{\frac{3 G M}{4 R}}$ where $M$ is mass of earth $R$ is radius of earth. Find the height upto which particle goes.
	\begin{answer}
		Particle has been thrown radially outward 
		\begin{figure}[H]
			\centering
			\includegraphics[height=4cm,width=6.5cm]{diagram-20220221(6)-crop}
		\end{figure}
		\begin{align*}
		\text{therefore }L&=0, V_{e f f}=V(r)+\frac{L^{2}}{2 m r^{2}}\\
		\therefore V_{e f f}&=\frac{-G M m}{r}+0\\
		\intertext{from conservation of energy we get }E_{1}&=E_{2}\\
		\text { or }\left(\frac{1}{2} m \dot{r}^{2}+V_{e f f}\right)_{r=R}&=\left(\frac{1}{2} m \dot{r}^{2}+V_{d f}\right)_{r=r} \\
		\frac{1}{2} m \cdot \frac{3 G M}{4 R}-\frac{G M m}{R}&=0-\frac{G M m}{r} \\
		\frac{3}{8 R}-\frac{1}{R}&=-\frac{1}{r} \quad \therefore r=\frac{8 R}{5}
		\end{align*}
		Therefore, maximum height from earth's surface $=r-R=\frac{3 R}{5}$
	\end{answer}
	\item  A particle is thrown from the earth's surface with speed $\sqrt{\frac{G M}{R}}$ where $M=$ mass of earth
	$R=$ radius of earth. If direction of initial velocity makes an angle $\alpha$ with the outward radial direction, find the maximum distance of the particle from centre of earth.
	\begin{answer}
		Angular momentum of the particle is
		\begin{figure}[H]
			\centering
			\includegraphics[height=5cm,width=8cm]{diagram-20220221(5)-crop}
		\end{figure}
		\begin{align*}
		L &=m v \sin \alpha R=m \sqrt{\frac{G M}{R}} \sin \alpha R \\
		&=m \sqrt{G M R} \sin \alpha \\
		\therefore V_{e f f} &=V(r)+\frac{L^{2}}{2 m r^{2}}=\frac{-G M m}{r}+\frac{G M m R}{2 r^{2}} \sin ^{2} \alpha
		\end{align*}
		From conservation of energy we get $\mathrm{E}_{1}=\mathrm{E}_{2}$
		\begin{align*}
		\text{	or }\left(\frac{1}{2} m \dot{r}^{2}+V_{e f f}\right)_{r=R}&=\left(\frac{1}{2} m \dot{r}^{2}+V_{e f f}\right)_{r=r}\text{ (see the figure)}\\
		\frac{1}{2} m \frac{G M}{R} \cos ^{2} \alpha-\frac{G M m}{R}+\frac{G M m}{2 R} \sin ^{2} \alpha&=\frac{-G M m}{r}+\frac{G M m R}{2 r^{2}} \sin ^{2} \alpha\\
		\text{Divide both sides by }\frac{G M m}{2 R}\text{ to get }&\cos ^{2} \alpha-2+\sin ^{2} \alpha=-2 \frac{R}{r}+\frac{R^{2}}{r^{2}} \sin ^{2} \alpha\\
		\text{or }-1=-2\left(\frac{R}{r}\right)+\left(\frac{R}{r}\right)^{2} \sin ^{2} \alpha \quad &\therefore\left(\frac{R}{r}\right)^{2} \sin ^{2} \alpha-2\left(\frac{R}{r}\right)+1=0\\
		\therefore \frac{R}{r}=\frac{+2 \pm \sqrt{4-4 \sin ^{2} \alpha}}{2 \sin ^{2} \alpha} \quad \frac{R}{r}&=\frac{+1 \pm \cos \alpha}{\sin ^{2} \alpha} \\
		\therefore r=\frac{R \sin ^{2} \alpha}{1 \pm \cos \alpha}=\frac{R\left(1-\cos ^{2} \alpha\right)}{(1 \pm \cos \alpha)} \quad &\therefore r=R(1-\cos \alpha) \quad \text { or } \quad r=R(1+\cos \alpha)
		\end{align*}
		Since $r$ must be greater than $r$. Therefore, $r=R(1+\cos \alpha)$
	\end{answer}
	\item  A particle moves under a central potential $V(r)=\frac{-k}{r^{m}} .$ What should be value of $m$ for its orbit to be stable.
	\begin{answer}
		\begin{align*}
		\text{Corresponding is }f(r)&=-\frac{\partial V}{\partial r}=\frac{-m k}{r^{m+1}}=-m k r^{-(m+1)}\\
		\text{	We know that for }f&=-k r^{n}\text{ condition for stability is $n>-3$.} 
		\intertext{Therefore for orbit to be stable under given potential we must have.}
		-(m+1)>-3&\text{ or }m+1<3 \quad \therefore m<2
		\end{align*}
	\end{answer}
\end{enumerate}
\section{Lagrangian}
\begin{enumerate}
	\item The Lagrangian of a particle of charge e and mass $m$ in applied electric and magnetic fields is given by $L=\frac{1}{2} \mathrm{~m} \vec{v}^{2}+\mathrm{e} \overrightarrow{\mathrm{A}} \cdot \vec{v}-\mathrm{e} \phi$, where $\overrightarrow{\mathrm{A}}$ and $\phi$ are the vector and scalar potentials corresponding to the magnetic and electric fields, respectively. Which of the following statements is correct?
	 \begin{tasks}(1)
		\task[\textbf{a.}] The carionically conjugate momentum of the particle is given by $\overrightarrow{\mathrm{p}}=\mathrm{m} \overrightarrow{\mathrm{v}}$
		\task[\textbf{b.}]The Hamiltonian of the particle is given by $\mathrm{H}=\frac{\overrightarrow{\mathrm{p}}^{2}}{2 \mathrm{~m}}+\frac{\mathrm{e}}{\mathrm{m}} \cdot \overrightarrow{\mathrm{A}} \cdot \overrightarrow{\mathrm{p}}+\mathrm{e} \phi$
		\task[\textbf{c.}]L remains unchanged under a gauge transformation of the potentials.
		\task[\textbf{d.}]  Under a gauge transformation of the potentials, $\mathrm{L}$ changes by the total time derivative of a function of $\overrightarrow{\mathrm{r}}$ and $\mathrm{t}$.
	\end{tasks}
	\begin{answer}
		\begin{align*}
		\mathrm{L}&=\frac{1}{2} \mathrm{~m} \overrightarrow{\mathrm{v}}-\mathrm{e} \phi+\mathrm{e} \overrightarrow{\mathrm{A}} \cdot \overrightarrow{\mathrm{v}}
	\intertext{	We know that $\vec{E}$ and $\vec{B}$ fields are in variant under gauge transformation.}
		\overrightarrow{\mathrm{A}}(\overrightarrow{\mathrm{x}}, \mathrm{t}) \rightarrow \overrightarrow{\mathrm{A}}^{\prime}&=\overrightarrow{\mathrm{A}}+\vec{\nabla} \lambda(\overrightarrow{\mathrm{x}}, \mathrm{t}) ; \quad \varphi(\overrightarrow{\mathrm{x}}, \mathrm{t}) \rightarrow \varphi^{\prime}=\varphi-\frac{\partial \lambda}{\partial \mathrm{t}}(\overrightarrow{\mathrm{x}}, \mathrm{t})
	\intertext{	where $\lambda(\vec{x}, t)$ is an arbitrary scalar function.}
		\therefore \quad \mathrm{L} \rightarrow \mathrm{L}^{\prime}&=\mathrm{L}+\mathrm{e}\left(\frac{\partial \lambda}{\partial \mathrm{t}}(\overrightarrow{\mathrm{x}}, \mathrm{t})+\overrightarrow{\mathrm{v}} \cdot \vec{\nabla} \lambda(\overrightarrow{\mathrm{x}}, \mathrm{t})\right)
		\end{align*}
		The expression in bracket is just the total time derivative of $\lambda(\vec{x}, t)$.\\
		If we add a total time derivative of a function of $\overrightarrow{\mathrm{x}}$ and $\mathrm{t}$ to the Lagrangian, the equations of motion do not change.\\
		Correct answer is option \textbf{(d)}
	\end{answer}
	\item The Hamiltonian o fa system with $n$ degrees of freedom is given by \\$H\left(q_{i}, \ldots \ldots \ldots ., q_{n} ; p_{i}, \ldots \ldots \ldots \ldots, p_{n} ; t\right)$\\
	with an explicit dependence on the time $t$. Which of the following is correct?
	 \begin{tasks}(1)
		\task[\textbf{a.}]Different phase trajectories cannot intersect each other
		\task[\textbf{b.}]H always represents the total energy of the system and is a constant of the motion.
		\task[\textbf{c.}]The equations $\dot{\mathrm{q}}_{\mathrm{i}}=\partial \mathrm{H} / \partial \mathrm{p}_{\mathrm{i}}, \dot{\mathrm{p}}_{\mathrm{i}}=-\partial \mathrm{H} / \partial \mathrm{q}_{\mathrm{i}}$ are not valid since H has explicit time dependence.
		\task[\textbf{d.}] Any initial volume element in phase space remains unchanged in magnitude under time evolution.
	\end{tasks}
\begin{answer}
	According to Liouville's theorem, the phase volume occupied by a collection of system evolve according to - Hamilton's equation of motion, and will be preserved in time.\\
	Correct answer is option \textbf{(d)}
\end{answer}
	\item A double pendulum consists of two point masses $m$ attached by massless strings of length $l$ as shown in the figure:
	\begin{figure}[H]
		\centering
		\includegraphics[height=4cm,width=4.5cm]{CM-02}
	\end{figure}
	The kinetic energy of the pendulum is :
	 \begin{tasks}(2)
		\task[\textbf{a.}]$\frac{1}{2} \mathrm{~m} \ell^{2}\left[\dot{\theta}_{1}^{2}+\dot{\theta}_{2}^{2}\right]$
		\task[\textbf{b.}]$\frac{1}{2} m \ell^{2}\left[2 \dot{\theta}_{1}^{2}+\dot{\theta}_{2}^{2}+2 \dot{\theta}_{1} \dot{\theta}_{2} \cos \left(\dot{\theta}_{1}-\dot{\theta}_{2}\right)\right]$
		\task[\textbf{c.}]$\frac{1}{2} m \ell^{2}\left[\dot{\theta}_{1}^{2}+2 \dot{\theta}_{2}^{2}+2 \dot{\theta}_{1} \dot{\theta}_{2} \cos \left(\dot{\theta}_{1}-\dot{\theta}_{2}\right)\right]$
		\task[\textbf{d.}] $\frac{1}{2} \mathrm{~m} \ell^{2}\left[2 \dot{\theta}_{1}^{2}+\dot{\theta}_{2}^{2}+2 \dot{\theta}_{1} \dot{\theta}_{2} \cos \left(\dot{\theta}_{1}+\dot{\theta}_{2}\right)\right]$
	\end{tasks}
	\begin{answer}
		\begin{align*}
	\mathrm{x}_{1}&=\ell \sin \theta_{1},  \mathrm{x}_{2}=\ell \sin \theta_{1}+\ell \sin \theta_{2} \\ \mathrm{y}_{1}&=-\ell \cos \theta_{1},  \mathrm{y}_{2}=-\ell \cos \theta_{1}-\ell \cos \theta_{2} \\ \dot{\mathrm{x}}_{1}&=\ell \cos \theta_{1} \dot{\theta}_{1},  \dot{\mathrm{y}}_{1}=\ell \sin \theta_{1} \dot{\theta}_{1}
		\end{align*}
		\begin{figure}[H]
			\centering
			\includegraphics[height=4.5cm,width=6.5cm]{CM-03}
		\end{figure}
		\begin{align*}
		\dot{x}_{2}&=\ell \cos \theta_{1} \dot{\theta}_{1}+\ell \cos \theta_{2} \dot{\theta}_{2} ; \quad \dot{y}_{2}=\ell \sin \theta_{1} \dot{\theta}_{1}+\ell \sin \theta_{2} \dot{\theta}_{2},\\
		T&=\frac{m}{2}\left(\dot{x}_{1}^{2}+\dot{y}_{1}^{2}\right)+\frac{m}{2}\left(\dot{x}_{2}^{2}+\dot{y}_{2}^{2}\right)\\
		&=\frac{\mathrm{m}}{2} \dot{\theta}_{1}^{2} \ell^{2}+\frac{\mathrm{m}}{2}\left[\dot{\theta}_{1}^{2} \ell^{2}+\dot{\theta}_{2}^{2} \ell^{2}+2 \ell^{2} \dot{\theta}_{1} \dot{\theta}_{2} \cos \left(\theta_{1}-\theta_{2}\right)\right]\\
		&=\frac{\mathrm{m} \ell^{2}}{2}\left[2 \dot{\theta}_{1}^{2}+\dot{\theta}_{2}^{2}+2 \dot{\theta}_{1} \dot{\theta}_{2} \cos \left(\theta_{1}-\theta_{2}\right)\right]
		\end{align*}
		Correct answer is \textbf{(b)}
	\end{answer}
	\item A particle of mass ' $\mathrm{m}$ ' moves inside a bowl. If the surface of the bowl is given by the equation $\mathrm{z}=\frac{1}{2} \mathrm{a}\left(\mathrm{x}^{2}+\mathrm{y}^{2}\right)$, where $a$ is a constant, the Lagrangian of the particle is:
	 \begin{tasks}(2)
		\task[\textbf{a.}] $\frac{1}{2} m\left(\dot{r}^{2}+r^{2} \dot{\phi}^{2}-g a r^{2}\right)$
		\task[\textbf{b.}] $\frac{1}{2} m\left[\left(1+a^{2} r^{2}\right) \dot{r}^{2}+r^{2} \dot{\phi}^{2}\right]$
		\task[\textbf{c.}]$\frac{1}{2} m\left(\dot{r}^{2}+r^{2} \dot{\theta}^{2}+r^{2} \sin ^{2} \dot{\phi}^{2}-g a r^{2}\right)$
		\task[\textbf{d.}]  $\frac{1}{2} m\left[\left(1+a^{2} r^{2}\right) \dot{r}^{2}+r^{2} \dot{\phi}^{2}-g a r^{2}\right]$
	\end{tasks}
	\begin{answer}$\left. \right. $
		\begin{figure}[H]
			\centering
			\includegraphics[height=1.5cm,width=2cm]{CM-04}
		\end{figure}
		\begin{align*}
		z&=\frac{1}{2} a\left(x^{2}+y^{2}\right)=\frac{1}{2} a r^{2} \quad \Rightarrow \dot{z}=a r \dot{r}\\
	\text{	K.E. }&=T=\frac{1}{2} m\left(\dot{x}^{2}+\dot{y}^{2}+\dot{z}^{2}\right)\\
	&=\frac{1}{2} m\left(\dot{r}^{2}+r^{2} \dot{\varphi}^{2}+\dot{z}^{2}\right)\\
	&=\frac{1}{2} m\left(\dot{r}^{2}+r^{2} \dot{\varphi}^{2}+a^{2} r^{2} \dot{r}^{2}\right) \qquad\text{ P.E.} =V=m g z=\frac{1}{2} m g a r^{2}\\
	L&=T-V=\frac{1}{2} m\left(\dot{r}^{2}+r^{2} \dot{\varphi}^{2}+a^{2} r^{2} \dot{r}^{2}-g a r^{2}\right)=\frac{1}{2} m\left[\left(1+a^{2} r^{2}\right) r^{2}+r^{2} \dot{\varphi}^{2}-g a r^{2}\right]
		\end{align*}
		Correct answer is option\textbf{(d)}
	\end{answer}
	\item A mass point glides without friction on a cycloid, which is given by $x=a(\vartheta-\sin \vartheta)$ and $y=a(1+\cos \vartheta)$ (with $0 \leq \vartheta \leq 2 \pi$ ). Determine\\
	(a) the Lagrangian, and\\
	(b) the equation of motion\\
	(c) Solve the equation of motion
	\begin{answer}
		The cycloid is represented by
		\begin{align*}
		x=a(\vartheta-\sin \vartheta), \quad y=a(1+\cos \vartheta),
		\end{align*}
		\begin{figure}[H]
			\centering
			\includegraphics[height=3.2cm,width=9cm]{CM-05}
		\end{figure}
		\begin{align*}
	\text{	where }&0 \leq \vartheta \leq 2 \pi.\text{ The kinetic energy is}\\
		T=\frac{1}{2} m\left(\dot{x}^{2}+\dot{y}^{2}\right)&=\frac{1}{2} m a^{2}\left\{[(1-\cos \vartheta) \dot{\vartheta}]^{2}+[-(\sin \vartheta) \dot{\vartheta}]^{2}\right\},\\
	\text{	and the potential energy is }V&=m g y=m g a(1+\cos \vartheta).\\
		\text{The Lagrangian is given by }L&=T-V=m a^{2}(1-\cos \vartheta) \dot{\vartheta}^{2}-m g a(1+\cos \vartheta).\\
		\text{The equation of motion then reads }&\frac{d}{d t}\left(\frac{\partial L}{\partial \dot{\vartheta}}\right)-\frac{\partial L}{\partial \vartheta}=0,\\
	\text{	i.e., }\quad &\frac{d}{d t}\left[2 m a^{2}(1-\cos \vartheta) \dot{\vartheta}\right]-\left[m a^{2}(\sin \vartheta) \dot{\vartheta}^{2}+m g a \sin \vartheta\right]=0\\
	\text{ or }\quad &\frac{d}{d t}[(1-\cos \vartheta) \dot{\vartheta}]-\frac{1}{2}(\sin \vartheta) \dot{\vartheta}^{2}-\frac{8}{2 a} \sin \vartheta=0,\\
	\text{ i.e.,} \quad&(1-\cos \vartheta) \ddot{\vartheta}+\frac{1}{2}(\sin \vartheta) \dot{\vartheta}^{2}-\frac{g}{2 a} \sin \vartheta=0.\\
\text{	By setting }u&=\cos \left(\frac{\vartheta}{2}\right), \text{one has }\frac{d u}{d t}=-\frac{1}{2} \sin \left(\frac{\vartheta}{2}\right) \dot{\vartheta}\text{ and } \frac{d^{2} u}{d t^{2}}\\&=-\frac{1}{2} \sin \left(\frac{\vartheta}{2}\right) \ddot{\vartheta}-\frac{1}{4} \cos \left(\frac{\vartheta}{2}\right) \dot{\vartheta}^{2} .\\ \text{Since }&\cot \left(\frac{\vartheta}{2}\right)=\sin \frac{\vartheta}{(1-\cos \vartheta)},\text{ we can write as}\\
\ddot{\vartheta}+\frac{1}{2} \cot \left(\frac{\vartheta}{2}\right) \dot{\vartheta}^{2}-\frac{g}{2 a} \cot \left(\frac{\vartheta}{2}\right)&=0,\text{ and therefore,} \frac{d^{2} u}{d t^{2}}+\frac{g}{4 a} u=0. 
\intertext{The solution of this differential equation is}
u&=\cos \left(\frac{\vartheta}{2}\right)=C_{1} \cos \sqrt{\frac{g}{4 a}} t+C_{2} \sin \sqrt{\frac{g}{4 a}} t 
\intertext{ The motion is just like the vibration of an ordinary pendulum of length $l=4 a$. The arrangement is therefore called a "cycloid pendulum".}
		\end{align*}
	\end{answer}
	\item 	Three particles of equal mass ' $m$ ' are connected by two identical massless springs of stiffiness constant ' $k$ ' as shown in the figure.\\
	If $x_{1}, x_{2}$ and $x_{3}$ denote the displacements of the masses from their respective equilibrium positions, the potential energy of the system is:
	 \begin{tasks}(2)
		\task[\textbf{a.}]$\frac{1}{2} k\left(x_{1}^{2}+x_{2}^{2}+x_{3}^{2}\right)$
		\task[\textbf{b.}]$\frac{1}{2} k\left[x_{1}^{2}+x_{2}^{2}+x_{3}^{2}-x_{2}\left(x_{1}+x_{3}\right)\right]$
		\task[\textbf{c.}]$\frac{1}{2} k\left[x_{1}^{2}+2 x_{2}^{2}+x_{3}^{2}+2 x_{2}\left(x_{1}+x_{3}\right)\right]$
		\task[\textbf{d.}] $\frac{1}{2} k\left[x_{1}^{2}+2 x_{2}^{2}+x_{3}^{2}-2 x_{2}\left(x_{1}+x_{3}\right)\right]$
	\end{tasks}
	\begin{answer}
		\begin{align*}
		\text{Potential energy }&=\frac{1}{2} k\left(x_{2}-x_{1}\right)^{2}\hspace{2cm}
		\text{first spring}\\
	\text{	Potential energy }&=\frac{1}{2} k\left(x_{3}-x_{2}\right)^{2}\hspace{2cm}
	\text{	second spring}
	\intertext{Potential energy of the system,}
	&=\frac{1}{2} k\left(x_{2}-x_{1}\right)^{2}+\frac{1}{2} k\left(x_{3}-x_{2}\right)^{2}=\frac{1}{2} k\left[x_{1}^{2}+x_{2}^{2}-2 x_{1} x_{2}+x_{2}^{2}+x_{3}^{2}-2 x_{2} x_{3}\right]\\
	&=\frac{1}{2} k\left[x_{1}^{2}+2 x_{2}^{2}+x_{3}^{2}-2 x_{2}\left(x_{1}+x_{3}\right)\right]
		\end{align*}
		Correct answer is option \textbf{(d)}
	\end{answer}
	\item The Lagrangian of a particle of mass $m$ moving in one dimensions is given by
	$$
	\mathrm{L}=\frac{1}{2} \mathrm{~m} \dot{\mathrm{x}}^{2}-\mathrm{bx}
	$$
	where $b$ is a positive constant. The coordinate of the particle $x(t)$ at time $t$ is given by: (in the following $c_{1}$ and $c_{2}$ are constants)
	 \begin{tasks}(2)
		\task[\textbf{a.}]$-\frac{b}{2 m} t^{2}+c_{1} t+c_{2}$
		\task[\textbf{b.}]$c_{1} t+c_{2}$
		\task[\textbf{c.}]$c_{1} \cos \left(\frac{b t}{m}\right)+c_{2} \sin \left(\frac{b t}{m}\right)$
		\task[\textbf{d.}] $c_{1} \cosh \left(\frac{b t}{m}\right)+c_{2} \sin h\left(\frac{b t}{m}\right)$
	\end{tasks}
	\begin{answer}
		\begin{align*}
		L&=\frac{1}{2} m \dot{x}^{2}-b x=\mathrm{T}-\mathrm{V}\\
		\therefore \mathrm{V}&=b x
		\intertext{ This is same as uniform gravitational potential. Therefore, solution must be of the form. This is same as uniform} \mathrm{S}&=\mathrm{S}_{0}+\mathrm{ut}+\frac{1}{2} \mathrm{at}^{2}
		\end{align*}
		Correct answer is option \textbf{(a)}
	\end{answer}
	\item A particle moves in a potential $V=x^{2}+y^{2}+\frac{z^{2}}{2}$. Which component (s) of the angular momentum is $/$ are constant (s) of motion?
	 \begin{tasks}(2)
		\task[\textbf{a.}]None
		\task[\textbf{b.}]$L_{x}, L_{y}$ and $L_{z}$
		\task[\textbf{c.}]Only $L_{x}$ and $L_{y}$
		\task[\textbf{d.}] Only $\mathrm{L}_{z}$
	\end{tasks}
	\begin{answer}
		\begin{align*}
		\text{Given }V(x, y, z)&=x^{2}+y^{2}+\frac{z^{2}}{2}
	\intertext{	In polar co-ordinate (spherical polar)}
	V&=r^{2} \sin ^{2} \theta+\frac{r^{2} \cos ^{2} \theta}{2} \quad\left[\begin{array}{l}x=r \cos \phi \sin \theta \\ y=r \sin \phi \sin \theta \\ z=r \cos \theta\end{array}\right.
\intertext{	Therefore, Lagrangian of system is}
	\mathrm{L}&=\mathrm{T}-\mathrm{V} \\
	L&=\frac{1}{2} m\left(\dot{r}^{2}+r^{2} \dot{\theta}^{2}+r^{2} \sin ^{2} \theta \dot{\phi}^{2}\right)-r^{2} \sin ^{2} \theta-\frac{r^{2} \cos ^{2} \theta}{2}
		\end{align*}
		$\phi$ is cyclic, therefore $p_{\phi}$ is constant of motion.\\
		$p_{\phi}$ is equal to $L_{z}$. Therefore, $L_{z}$ is constant of motion.\\
		Correct answer is option \textbf{(d)}
	\end{answer}
	\item	A particle of mass $m$ is projected with an initial velocity $u$ at an angle $\alpha$ with the horizontal. Use Lagran equations to describe the motion of projectile. The resistance of air is negligible.
	\begin{answer}
			Let $x, y$ be the coordinate of the particle at any instant. The system is holonomic because its positic confined to a plane and conservative because the only external force acting on the projectile is conse tive.
		\begin{align*}
		T&=\frac{1}{2} m\left(\dot{x}^{2}+\dot{y}^{2}\right)\text{ and }V=m g y\\
		\therefore L&=T-V=\frac{1}{2} m\left(\dot{x}^{2}+\dot{y}^{2}\right)-m g v\\
		\frac{\partial L}{\partial \dot{x}}&=m \dot{x}\text{ and } \frac{\partial L}{\partial x}=0\\
		\text{By Lagrange's equation, }\frac{d}{d t}\left(\frac{\partial L}{\partial \dot{x}}\right)&=\frac{\partial L}{\partial x}\\
		\therefore \quad \frac{d}{d t}(m \dot{x})&=0 \text { or } \ddot{x} =0 \\
		\frac{\partial L}{\partial \dot{y}}&=m y \text { and } \frac{\partial L}{\partial y} =-m g\\
		\text{By Lagrange's equation }\frac{d}{d t}\left(\frac{\partial L}{\partial \dot{y}}\right)&=\frac{\partial L}{\partial y}\\
		 \therefore \quad \frac{d}{d t}(m y)&=-m g\text{ or }\ddot{y}=-g
		 \intertext{Thus, $\ddot{x}=0, \ddot{y}=-g$ are the two equations describing the motion of a projectile. Proceeding with the first equation,}
		 \frac{d}{d t}\left(\frac{d x}{d t}\right)&=0 \text { or } d\left(\frac{d x}{d t}\right)=0\\
		 \text{Integrating, }\quad \frac{d x}{d t}&=c_{1}(\text{ a constant} )\\
		\text{ At }t&=0, \quad \frac{d x}{d t}=u \cos \alpha ; \quad \therefore u \cos \alpha=c_{1}\\
		\therefore \quad \frac{d x}{d t}&=u \cos \alpha, \quad \text { or } d x=u \cos \alpha d t\\
	\text{	Integrating, }x&=(u \cos \alpha) t+x_{0}(\text{ a constant })\\
\text{	At $t=0, x=0$, }&\text{therefore, }x_{0}=0 ; x=(u \cos \alpha) t
 \intertext{Proceeding with the second equation,}
	\frac{d}{d t}\left(\frac{d y}{d t}\right)&=-g, \text { or } d\left(\frac{d y}{d t}\right)=-g d t\\
\text{	Integrating, }\frac{d y}{d t}&=-g t+c_{2}(\mathrm{a}\text{ constant })\\
	\text{At }t&=0, \frac{d y}{d t}=u \sin \alpha ; \therefore u \sin \alpha=c_{2}\\
	\therefore \quad \frac{d y}{d t}&=-g t+u \sin \alpha, o r d y=-g t d t+u \sin \alpha d t\\
\text{	Integrating, }y&=-\frac{1}{2} g t^{2}+(u \sin \alpha) t+y_{0}\\
\text{	At }t&=0, y=0\text{ and therefore, $y_{0}=0$}\\
	\therefore \quad y&=(u \sin \alpha) t-\frac{1}{2} g t^{2}
		\end{align*}
		Thus, $x=(u \cos \alpha) t, y=(u \sin \alpha) t-\frac{1}{2} g t^{2}$ are the two solved equation describing the motion of a projectile.
	\end{answer}
	\item A particle of mass $m$ is moving in a plane under the influence of a force directed towards a fixed point and varying inversely as the square of the distance from that point. Set up the Lagrangian and equations of motion of the particle.
	\begin{answer}
		The problem is best solved in polar coordinates with respect to the fixed point as origin and any line through it as the reference line. Let $(r, \theta)$ be the polar coordinates of the particle at any time. Since there are two coordinates there will be two Lagrange's equations describing the motion of the particle. Motion is holonomic because the position of the particle is confined to a plane and conservative because the force is derivable from energy or energy is derivable from the force.
		\begin{figure}[H]
			\centering
			\includegraphics[height=2.5cm,width=5cm]{CM-06}
		\end{figure}
		\begin{align*}
		T&=\frac{1}{2} m \dot{r}^{2} \text{(linear kinetic energy)} +\frac{1}{2} I \dot{\theta}^{2} \text{(rotational $k$. energy)}\\
		\text{Or, }\quad T&=\frac{1}{2} m \dot{r}^{2}+\frac{1}{2}\left(m r^{2}\right) \dot{\theta}^{2} \quad\left(\because I=m r^{2}\right)
		\intertext{Since force varies inversely as the square of the distance and it is directed towards the origin, $F=-\left(k / r^{2}\right)$ where $k$ is a constant. But}
		F&=-\left(\frac{d V}{d r}\right)\\
		\therefore\quad&=\frac{k}{r^{2}}=-\frac{d V}{d r}\text{, or } d V=\frac{k d r}{r^{2}}\\
		\text{Integrating, }V&=-\frac{k}{r}+V_{0}\text{ (a constant)}\\
		\text{At }r&=\infty, V=0\text{ and therefore, }V_{0}=0 ; \therefore V=-\left(\frac{k}{r}\right)\\
		\therefore \quad L&=T-V=\frac{1}{2} m\left(\dot{r}^{2}+r^{2} \dot{\theta}^{2}\right)+\frac{k}{r}\\
		\frac{\partial L}{\partial \dot{r}}&=m \dot{r}\text{ and }\frac{\partial L}{\partial r}=m r \dot{\theta}^{2}-\frac{k}{r^{2}}\\
		\frac{\partial L}{\partial \dot{\theta}}&=m r^{2} \dot{\theta}\text{ and} \frac{\partial L}{\partial \theta}=0\\
		\text{By Lagrange's equations of motion, }&\left[\frac{d}{d t}\left(\frac{\partial L}{\partial q_{k}}\right)=\frac{\partial L}{\partial q_{k}}\right]\text{ we have }\\
		\frac{d}{d t}(m \dot{r})&=m r \dot{\theta}^{2}-\frac{k}{r^{2}}\text{ and } \frac{d}{d t}\left(m r^{2} \dot{\theta}\right)=0\\
		\text{Or, }\quad \ddot{r}&=r \dot{\theta}^{2}-\frac{k}{m r^{2}}\text{ and} r^{2} \dot{\theta}=\text{ constant}\\
	\text{	Or, }\quad \ddot{r}-r \dot{\theta}^{2}&=-\frac{k}{m r^{2}}\text{ or }\ddot{r}-r \dot{\theta}^{2}=-\frac{\omega^{2}}{r^{2}}\\
\text{	where, }\omega^{2}&=\frac{k}{m}\text{ and }r^{2} \dot{\theta}=h \text{(another constant).}
\intertext{	Thus equations describing the motion of the particles are} \ddot{r}-r \dot{\theta}^{2}&=-\frac{\omega^{2}}{r^{2}}\text{ and }r^{2} \dot{\theta}=h\text{ where }\omega^{2}\text{ and $h$ are constants.}
		\end{align*}
	\end{answer}
	\item Obtain the Lagrangian of a linear simple harmonic oscillator and the equation of motion in one dimension.
	\begin{answer}
		In a simple harmonic oscillator the particle is acted on by a force which is directed towards a fixed point (position of equilibrium) and whose magnitude varies linearly with the distance from the position ofequilibrium. That is, $F=-k x$ where $\mathrm{x}$ is the distance of the particle from the equilibrium position.
		\begin{align*}
		\text{But}
		F&=-\left(\frac{d V}{d x}\right) ; \quad \because d V=-F d x=h x d x\\
		\text{Integrating, }\quad V&=\frac{1}{2} k x^{2}+V_{0}(a\text{ constant })\\
	\text{	At }x=0, V&=0\text{ and therefore, }V_{0}=0\\
		\therefore \quad V&=\frac{1}{2} k x^{2} ; \quad T=\frac{1}{2} m \dot{x}^{2}\\
		\therefore \quad L&=T-V=\frac{1}{2} m x^{2}-\frac{1}{2} k x^{2}\\
	\text{	Now, }\quad \frac{\partial L}{\partial \dot{x}}&=m \dot{x}\text{ and } \frac{\partial L}{\partial \dot{x}}=-k x\\
		\text{By Lagrange's equation, }&\frac{d}{d t}\left(\frac{\partial L}{\partial \dot{x}}\right)=\frac{\partial L}{\partial x}\\
		\therefore\quad\frac{d}{d t}(m x)&=-k x, \text { or } m \ddot{x}=-k x
		\intertext{This is the equation of a simple harmonic oscillator in one dimension.}
		\end{align*}
	\end{answer}
	\item Figure below shows a solid cylinder with centre $G$ and radius $a$ rolling on the rough inside surface of a fixed cylinder with centre $O$ and radius $b>a$. Find the Lagrange equation of motion and deduce the period of small oscillations about the equilibrium position.
	\begin{figure}[H]
		\centering
		\includegraphics[height=4cm,width=7cm]{Lagrangian 01}
	\end{figure}
	\begin{answer}
		If the cylinder were not obliged to roll, the system would have two degrees of freedom with generalised coordinates $\theta$ (the angle between $O G$ and the downward vertical) and $\phi$ (the rotation angle of the cylinder measured from some reference position).\\
		The roling condition imposes the kinematical constraint.
		\begin{align*}
		(b-a) \dot{\theta}-a \dot{\phi}&=o
	\intertext{	This constraint is integrable and is equivalent to the geometrical constraint.}
	(b-a) \theta-a \phi&=o
\intertext{	on taking $\phi=0$ when $\theta=0$. Thus the rolling cylinder is a standard conservative system with one degree of freedom.}
\intertext{Take $\theta$ as the generalised coordinate. Then the kinetic energy is given by}
T &=\frac{1}{2} m((b-a) \dot{\theta})^{2}+\frac{1}{2}\left(\frac{1}{2} m a^{2}\right) \dot{\phi}^{2} \\
&=\frac{1}{2} m((b-a) \dot{\theta})^{2}+\frac{1}{2}\left(\frac{1}{2} m a^{2}\right)\left(\frac{b-a}{a}\right)^{2} \dot{\theta}=\frac{3}{4} m(b-a)^{2} \dot{\theta}^{2}\\
\text{and the potential energy by }V&=-m g(b-a) \cos \theta
\intertext{There is only one Lagrange equation, namely}
\frac{d}{d t}\left[\frac{3}{2} m(b-a)^{2} \dot{\theta}\right]-0*=-m g(b-a) \sin \theta
\intertext{which simplifies to give}
\ddot{\theta}+\frac{2 g}{3(b-a)} \sin \theta&=0
\intertext{Interestingly, this equation is identical to the exact equation for the oscillations of a simple pendulum of length $3(b-a) / 2$ as obtaind.}
\intertext{The linearised equation governing small oscillations of the cylinder about $\theta=0$ is is}
\ddot{\theta}+\frac{2 g}{3(b-a)} \theta&=0
\intertext{so that the period $\tau$ of small oscillation is given by}
\tau&=2 \pi\left(\frac{3(b-a)}{2 g}\right)^{1 / 2}
		\end{align*}
\end{answer}
\item A pendulum of mass $m$ is attached to a block of mass $M$. The block slides on a horizontal frictionless surface. Find the Lagrangian and equation of motion of the pendulum. For small amplitude oscillations, derive an expression for periodic time.
\begin{figure}[H]
	\centering
	\includegraphics[height=4cm,width=5cm]{Lagrangian 02}
\end{figure}
\begin{answer}
		Let at any time $t$ the coordinates of $M$ and $m$ be $\left(x_{1}, 0\right)$ and $\left(x_{2}, y_{2}\right)$ respectively.
	\begin{align}
	\text{Here, }\quad x_{2}&=x_{1}+\ell \sin \theta\text{ and }ay_{2}=-\ell \cos \theta.\text{ The suitable generalized coordinates are $x_{1}$ and $\theta$.}\notag\\
	 T &=\frac{1}{2} M \dot{x}_{1}^{2}+\frac{1}{2} m\left(\dot{x}_{2}^{2}+\dot{y}_{2}^{2}\right) \notag\\ &=\frac{1}{2} M \dot{x}_{1}^{2}+\frac{1}{2} m\left[\left(\dot{x}_{1}+\ell \dot{\theta} \cos \theta\right)^{2}+(\ell \dot{\theta} \sin \theta)^{2}\right] \notag\\ &=\frac{1}{2} M \dot{x}_{1}^{2}+\frac{1}{2} m\left(\dot{x}_{1}^{2}+\ell^{2} \dot{\theta}^{2}+2 \ell \dot{x}_{1} \dot{\theta} \cos \theta\right)\notag\\
	 (\text{because }\dot{x}_{2}&=\dot{x}_{1}+\ell \cos \theta \dot{\theta} \text{and }\dot{y}_{2}=\ell \sin \theta \dot{\theta} )\notag\\
	 V&=-m g \cos \theta\notag\\
	\text{ Hence, }\quad L&=T-V=\frac{1}{2}(M+m) \dot{x}_{1}^{2}+\frac{1}{2} m \ell^{2} \dot{\theta}^{2}+m \ell \dot{x}_{1} \dot{\theta} \cos \theta+m g \ell \cos \theta\notag
	\intertext{We see that $x_{1}$ is cyclic coordinate, so}\notag
	\text{Here, }\frac{\partial L}{\partial x_{1}}&=0\text{ and }\frac{\partial L}{\partial \dot{x}_{1}}=(M+m) \dot{x}_{1}+m \ell \dot{\theta} \cos \theta\text{ is conserved.}\notag\\
	\frac{\partial L}{\partial \theta}&=m \ell\left(\dot{x}_{1} \dot{\theta}+g\right)(-\sin \theta) \text { and } \frac{\partial L}{\partial \dot{\theta}}=m \ell^{2} \dot{\theta}+m \ell \dot{x}_{1} \cos \theta\notag
\intertext{	Equation of motion in $\theta$ is}\notag
	m \ell^{2} \ddot{\theta}&+m \ell \ddot{x}_{1} \cos \theta+m \ell(-\sin \theta) \dot{\theta} \dot{x}_{1}-m \ell(-\sin \theta) \dot{x}_{1} \dot{\theta}+m g \ell \sin \theta=0\notag\\
\text{	Or, }\quad m \ell^{2} \ddot{\theta}&+m \ell \cos \theta \ddot{x}_{1}+m g \ell \sin \theta=0\notag\\
	\text{If }\theta\text{ is small, }&\sin \theta \approx \theta\text{ and also }\cos \theta \approx 1\text{, then}\notag\\
	m \ell^{2} \ddot{\theta}&+m \ell \ddot{x}_{1}+m g \ell \theta=0\notag\\
\text{	Or,}\quad 
	\ddot{\theta}+\frac{\ddot{x}_{1}}{\ell}&+\frac{g}{\ell} \theta=0 \label{CMP-25}
\intertext{	Equation of motion in $x_{1}$ is}\notag
(M+m) \ddot{x}_{1}&+m \ell\left(\ddot{\theta} \cos \theta-\dot{\theta}^{2} \sin \theta\right)=0\notag\\
\text{For small }\theta,&\left(\cos \theta \cong 1, \sin \theta \cong \theta\right.\text{ and }\dot{\theta}^{2} \theta\text{ is negligible })\notag\\
(M+m) \ddot{x}_{1}&+m \ell \ddot{\theta}=0\label{CMP-26}
\intertext{Fromequations (\ref{CMP-25}) and (\ref{CMP-26}), we have}\notag
\ddot{\theta}-\frac{m \ddot{\theta}}{M+m}&+\frac{g}{\ell} \theta=0\notag\\
\text{Hence,}\quad
\ddot{\theta}&=-\left[\frac{M+m}{M}\right] \frac{g}{\ell} \theta\notag
\intertext{This is the equation of simple harmonic motion whose period is given by}\notag
T&=2 \pi \sqrt{\frac{\ell}{g}} \sqrt{\frac{M}{M+m}}\notag
	\end{align}
\end{answer}
\item Auniform disc of radius ' $\dot{a}$ ' and mass $m$, rotates about a fixed axis. A mass less rope is fixed to a points on the out side 'circumference' and leads to mass less spring which is intwin fastened to a fixed point. At a radius $a / 2$ another cord is fastenped to a spring which connects to a $m a, m$. Set up the Lagrange's eq. of the DISC and the mass.
\begin{answer}$\left. \right. $
	\begin{figure}[H]
		\centering
		\includegraphics[height=6cm,width=5cm]{Lagrangian 03}
	\end{figure}
	Let $b_{1}$ and $b_{2}$ be the original length of the spring 1 and 2 . Which the whole system is in eq. and the DISC is stationary.\\
	Let $\theta$ be the angle the DISC has twined away from the eq. position. Then extension of the spring $-1$ is
	\begin{align*}
	x_{1}&=a \theta
	\intertext{If the spring $-2$ is stretched by a distance $x_{2}$ the mass $m$ is low ered by a distance}
	x_{3}&=\frac{a}{2} \theta+x_{2}\\
	K.E.\hspace{2cm}
	\mathrm{T}&=\mathrm{K} \cdot \mathrm{E} \cdot\text{ of }\mathrm{DISC}+\mathrm{K} \cdot \mathrm{E} \cdot\text{ of mass}\\
	&=\frac{1}{2}(\dot{\theta})^{2}+\frac{1}{2} m \dot{x}_{3}^{2}=\frac{1}{2} m a^{2} \cdot \dot{\theta}^{2}+\frac{1}{2} m\left(\frac{a}{2} \dot{\theta}+\dot{x}_{2}\right)^{2}\\
	P.E.
	\hspace{2cm}V&=\frac{1}{2} k x_{1}{ }^{2}+\frac{1}{2} k x_{2}{ }^{2}-m g x_{3} .
\intertext{	Hence the Lagrangian for the system is}
\mathrm{L}&=\mathrm{T}-\mathrm{V} \\
\mathrm{L}&=\frac{1}{2} m a^{2} \dot{\theta}^{2}+\frac{1}{2} m\left(\frac{a^{2}}{4} \dot{\theta}^{2}+a \dot{\theta} \dot{x}_{2}+\dot{x}_{2}^{2}\right)-\frac{1}{2} k x_{1}^{2}-\frac{1}{2} k x_{2}^{2}+m g\left(\frac{a}{2} \theta+x_{2}\right)\\
\text{Lagrangian eq. of motion, }&\frac{d}{d t}\left(\frac{\partial L}{\partial \dot{x}_{2}}\right)-\left(\frac{\partial L}{\partial x_{2}}\right)=0 \Rightarrow \frac{d}{d t}\left(\frac{\partial L}{\partial \dot{\theta}}\right)-\left(\frac{\partial L}{\partial \theta}\right)\\
\text { Or } \quad \frac{d}{d t}\left[\frac{1}{2} m a \theta+m \dot{x}_{2}\right]&-\left[-k x_{2}+m g\right]=0  \Rightarrow m \ddot{x}_{2}+k x_{2}+\frac{1}{2} m(a \ddot{\theta}-2 g)=0 \\ 
\text { Or } \quad \ddot{x}+\left(\frac{k}{m}\right) x_{2}&+\frac{1}{2}(a \ddot{\theta}-2 g)=0 \quad  \Rightarrow \frac{d}{d t}\left[\frac{1}{2} m a^{2} \dot{\theta}+\frac{1}{2} m a^{2}+\dot{\theta} \max _{2}\right]-\left[-k a^{2} \theta+\frac{m g a}{2}\right]\\
\text{Or }\quad \frac{3}{4} m a^{2} \ddot{\theta}&+R a^{2} \theta+\frac{1}{2} m a\left(\ddot{x}_{2}-g\right)=0\\
\text{Or }\quad \ddot{\theta}&+\left(\frac{4}{3} \frac{k}{m}\right) \theta+\frac{2}{3 a}\left(\ddot{x}_{2}-g\right)=0
\intertext{Which are the required equation of motion.}
	\end{align*}
\end{answer}
\item Set up the lagrangian for the following system, where the pully is massless.
\begin{figure}[H]
	\centering
	\includegraphics[height=6cm,width=4cm]{Lagrangian 04}
\end{figure}
\begin{answer}
	\begin{align*}
	\mathrm{T}&=\frac{1}{2} m_{1} \dot{x}_{1}^{2}+\frac{1}{2} m_{2} \dot{x}_{2}^{2}\\
	\mathrm{~V}&=-m_{1} g x_{1}-m_{2} g x_{2}+\frac{1}{2} k x_{1}^{2}\\
	\mathrm{~L}&=\frac{1}{2} m_{1} \dot{x}_{1}^{2}+\frac{1}{2} m_{2} \dot{x}_{2}^{2}+m_{1} g x_{1}+m_{2} g x_{2}-\frac{1}{2} k x_{1}^{2}\\
	\text{Thus, equation of motion is }&=\frac{d}{d t}\left(\frac{\partial L}{\partial \dot{x}_{1}}\right)-\left(\frac{\partial L}{\partial x_{1}}\right)=0\\
	\text{And }\frac{d}{d t}\left(\frac{\partial L}{\partial \dot{x}_{2}}\right)-\left(\frac{\partial L}{\partial x_{2}}\right)=0\\
	\Rightarrow \quad m_{1} \ddot{x}_{1}-m_{1} g+k x_{1}=0\\
	\text{Say, }m_{2} \ddot{x}_{2}-m_{2} g=0
	\end{align*}
\end{answer}
\item Set up the lagrangian for the following system, the disk also have mass ' $m$ '.
\begin{figure}[H]
	\centering
	\includegraphics[height=6cm,width=2.7cm]{Lagrangian 11}
\end{figure}
\begin{answer}$\left. \right. $
	\begin{figure}[H]
		\centering
		\includegraphics[height=6cm,width=4cm]{Lagrangian 04}
	\end{figure}
		Let $\theta$ be the angle one disc has turned away from one equilibrium position. Then extension of the spring $x_{1}=a \theta$\\
	Mass $m$ is lowered by a distance $x_{2}=\mathrm{a} \theta$
	\begin{align*}
\therefore \quad \mathrm{T} &=\frac{1}{2} m(a \dot{\theta})^{2}+\frac{1}{2} \mathrm{I} \dot{\theta}^{2} \\ \mathrm{~V} &=-\mathrm{mga} \theta+\frac{1}{2} k \theta^{2} a^{2} \\ \mathrm{~L} &=\frac{1}{2} m a^{2} \dot{\theta}^{2}+\frac{1}{4} m a^{2} \dot{\theta}^{2}+m g a \theta-\frac{1}{2} k a^{2} \theta^{2} \\
\text{Eq. of motion is}&
\frac{d}{d t}\left(\frac{\partial L}{\partial \theta}\right)-\frac{\partial L}{\partial \theta}=0\\
\Rightarrow \quad &m a^{2} \ddot{\theta}+\frac{1}{2} m a^{2} \ddot{\theta}-m g a+k a^{2} \theta=0\\
\Rightarrow \quad &\frac{3}{2} m a \ddot{\theta}-m g+k a \theta=0\\
\Rightarrow \quad &\ddot{\theta}-\frac{2}{3} g / a+\frac{2}{3}\left(\frac{k}{m}\right) \theta=0
	\end{align*}
\end{answer}
\item Two blocks connected by a spring of spring constant $k$ are free to slide frictionlessly along a horizontal surface, as shown in figure. The unstretched length of the spring is $a$.
\begin{figure}[H]
	\centering
	\includegraphics[height=1cm,width=6cm]{Lagrangian 06}
\end{figure}
Two masses connected by a spring sliding horizontally along a frictionless surface.\\
(a) Identify a set of generalized coordinates and write the Lagrangian.
\begin{answer}
	As generalized coordinates I choose $X$ and $u$, where $X$ is the position of the right edge of the block of mass $M$, and $X+u+a$ is the position of the left edge of the block of mass $m$, where $a$ is the unstretched length of the spring. Thus, the extension of the spring is $u$. The Lagrangian is then
	\begin{align*}
	L&=\frac{1}{2} M \dot{X}^{2}+\frac{1}{2} m(\dot{X}+\dot{u})^{2}-\frac{1}{2} k u^{2}=\frac{1}{2}(M+m) \dot{X}^{2}+\frac{1}{2} m \dot{u}^{2}+m \dot{X} \dot{u}-\frac{1}{2} k u^{2}\\
	&\text{(b) Find the equation of motion.}\\
	&\text{The canonical momenta are}\\
	p_{X} &\equiv \frac{\partial L}{\partial \dot{X}}=(M+m) \dot{X}+m \dot{u}, p_{u} \equiv \frac{\partial L}{\partial \dot{u}}=m(\dot{X}+\dot{u})\\
	&\text{The corresponding equations of motion are then}\\
	&\dot{p}_{X}=F_{X}=\frac{\partial L}{\partial X} \quad \Rightarrow \quad(M+m) \ddot{X}+m \ddot{u}=0 \\
	&\dot{p}_{u}=F_{u}=\frac{\partial L}{\partial u} \quad \Rightarrow \quad m(\ddot{X}+\ddot{u})=-k u .
	\end{align*}
	(c) Find all conserved quantities.\\
	There are two conserved quantities. One is $p_{X}$ itself, as is evident from the fact that $L$ is cyclic in $X$. This is the conserved 'charge' $\Lambda$ associated with the continuous symmetry $X \rightarrow X+\zeta$. i.e. $\Lambda=p_{X}$. The other conserved quantity is the Hamiltonian $H$, since $L$ is cyclic in $t$. Furthermore, because the kinetic energy is homogeneous of degree two in the generalized velocities, we have that $H=E$, with
	\begin{align*}
	E&=T+U=\frac{1}{2}(M+m) \dot{X}^{2}+\frac{1}{2} m \dot{u}^{2}+m \dot{X} \dot{u}+\frac{1}{2} k u^{2} .\\
	\intertext{It is possible to eliminate $\dot{X}$, using the conservation of $\Lambda$ :}
	\dot{X}&=\frac{\Lambda-m \dot{u}}{M+m} .
	\intertext{This allows us to write}
	E&=\frac{\Lambda^{2}}{2(M+m)}+\frac{M m \dot{u}^{2}}{2(M+m)}+\frac{1}{2} k u^{2}
	\end{align*}
	(d) Find a complete solution to the equations of motion. As there are two degrees of freedom, your solution should involve 4 constants integration. You need not match initial conditions and you need not choose the quantities in part (c) to be among the constants.\\
	Using conservation of $\Lambda$, we may write $\ddot{X}$ in terms of $\ddot{x}$, in which case
	\begin{align*}
	\frac{M m}{M+m} \ddot{u}&=-k u \Rightarrow u(t)=A \cos (\Omega t)+B \sin (\Omega t)\\
\text{}\Omega&=\sqrt{\frac{(M+m) k}{M m}} .
\intertext{	For the $X$ motion, we integrate equation above, obtaining}
X(t)&=X_{0}+\frac{\Lambda t}{M+m}-\frac{m}{M+m}(A \cos (\Omega t)-A+B \sin (\Omega t))
\intertext{There are thus four constant: $X_{0}, \Lambda, A$ and $B$. Note that conservation of energy says}
E&=\frac{\Lambda^{2}}{2(M+m)}+\frac{1}{2} k\left(A^{2}+B^{2}\right)
	\end{align*}
	\textbf{Alternate solution :} We could choose $X$ as the position of the left block and $x$ as the position of the right block. In this case,
	\begin{align*}
	L&=\frac{1}{2} M \dot{X}^{2}+\frac{1}{2} m \dot{x}^{2}-\frac{1}{2} k(x-X-b)^{2} .
\intertext{	Here, $b$ includes the unstretched length $a$ of the spring, but may also include the size of the blocks if say, $X$ and $x$ are measured relative to the blocks midpoint. The canonical momenta are}
p_{X}&=\frac{\partial L}{\partial \dot{X}}=M \dot{X}, \quad p_{x}=\frac{\partial L}{\partial \dot{x}}=m \dot{x}
\intertext{The equation of motion are then}
\dot{p}_{X}&=F_{X}=\frac{\partial L}{\partial X} \quad \Rightarrow \quad M \ddot{X}=k(x-X-b) \\
\dot{p}_{x}&=F_{x}=\frac{\partial L}{\partial x} \quad \Rightarrow \quad m \ddot{x}=-k(x-X-b) .
\intertext{The one parameter family which leaves $L$ invariant is $X \rightarrow X+\zeta$ and $x \rightarrow x+\zeta$, i.e. simultaneous and identical displacement of both of the generalized coordinates. Then}
\Lambda&=M \dot{X}+m \dot{x}
\intertext{which is simply the $x$-components of the total momentum. Again, the energy is conserved.}
E&=\frac{1}{2} M \dot{X}^{2}+\frac{1}{2} m \dot{x}^{2}+\frac{1}{2} k(x-X-b)^{2} .
\intertext{We can combine the equations of motion to yield}
M m \frac{d^{2}}{d t^{2}}(x-X-b)&=-k(M+m)(x-X-b),
\intertext{which yields}
x(t)-X(t)&=b+A \cos (\Omega t)+B \sin (\Omega t)
\intertext{From the conservation of $\Lambda$, we have}
M X(t)+m x(t)&=\Lambda t+C
\intertext{where $C$ is another constant. Thus, we have the motion of the system in terms of four constant: $A, B, \Lambda$ and $C$ }
X(t)&=-\frac{m}{M+m}(b+A \cos (\Omega t)+B \sin (\Omega t))+\frac{\Lambda t+C}{M+m}\\
x(t)&=-\frac{M}{M+m}(b+A \cos (\Omega t)+B \sin (\Omega t))+\frac{\Lambda t+C}{M+m}
	\end{align*}
\end{answer}
\item A particle of charge $e$ moves in three dimensions in the presence of a uniform magnetic field $B=B_{0} \hat{z}$ and a uniform electric field $E=E_{0} \hat{x}$. The potential energy is
$$
U(r, \dot{r})=-e E_{0} x-\frac{e}{c} B_{0} x \dot{y},
$$
where we have chosen the gauge $A=B_{0} x \hat{y}$.\\
(a) Find the canonical momenta $p_{x}, p_{y}$ and $p_{z}$\\
(b) Identify all conserved quantities\\
(c) Find a complete, general solution for the motion of the system $\{x(t), y(t), x(t)\}$.
\begin{answer}
	\begin{align*}
	\text{( The Lagrangian is )}L&=\frac{1}{2} m\left(\dot{x}^{2}+\dot{y}^{2}+\dot{z}^{2}\right)+\frac{e}{c} B_{0} x \dot{y}+e E_{0} x. 
	\intertext{The canonical momenta are}
	p_{x}=\frac{\partial L}{\partial \dot{x}}&=m \dot{x} ; \quad p_{y}=\frac{\partial L}{\partial \dot{y}}=m \dot{y}+\frac{e}{c} B_{0} x ; \quad p_{x}=\frac{\partial L}{\partial \dot{z}}=m \dot{z}
\intertext{	(b) There are three conserved quantities. First is the momentum $p_{y}$, since $F_{y}=\frac{\partial L}{\partial y}=0 .$ Second is the momentum $p_{z}$, since $F_{y}=\frac{\partial L}{\partial z}=0$. The third conserved quantity is the Hamiltonian, since $\frac{\partial L}{\partial t}=0$. We have }
H&=p_{x} \dot{x}+p_{y} \dot{y}+p_{z} \dot{z}-L\\
\Rightarrow \quad H&=\frac{1}{2} m\left(\dot{x}^{2}+\dot{y}^{2}+\dot{z}^{2}\right)-e E_{0} x\\
\text{(c) The equations of motion are }\ddot{x}-\omega_{c} \dot{y}&=\frac{e}{m} E_{0} ; \ddot{y}+\omega_{c} \dot{x}=0 ; \ddot{z}=0.
\intertext{The second equation can be integrated once to yield $\dot{y}=\omega_{c}\left(x_{0}-x\right)$, where $x_{0}$ is a constant. Substituting this into the first equation gives}
\ddot{x}+\omega_{c}^{2} x&=\omega_{c}^{2} x_{0}+\frac{e}{m} E_{0} .
\intertext{This is the equation of a constantly forced harmonic oscillator. We can therefore write the generalsolution as}
x(t)&=x_{0}+\frac{e E_{0}}{m \omega_{c}^{2}}+A \cos \left(\omega_{c} t+\delta\right) \\
y(t)&=y_{0}-\frac{e E_{0}}{m \omega_{c}} t-A \sin \left(\omega_{c} t+\delta\right) \\
z(t)&=z_{0}+\dot{z}_{0} t
\intertext{Note that there are six constants, $\left\{A, \delta, x_{0}, y_{0}, z_{0}, \dot{z}_{0}\right\}$, are required for the general solution of three coupled second order ODEs.}
	\end{align*}
\end{answer}
\item A point mass $m$ slides frictionlessly, under the influence of gravity, along a massive ring of radius $a$ and mass $M$. The ring is affixed by horizontal springs to two fixed vertical surfaces, as depicted in figure. Allmotion is within the plane of the figure.
\begin{figure}[H]
	\centering
	\includegraphics[height=2cm,width=6cm]{Lagrangian 07}
\end{figure}
A point mass $m$ slides frictionlessly along a massive ring of radius $a$ and mass $M$, which is affixed by horizontal springs to two fixed vertical surfaces.
\begin{answer}
	(a) Choose as generalized coordinates the horizontal displacement $X$ of the center of the ring with respect to equilibrium, and the angle $\theta$ a radius to the mass $m$ makes with respect to the vertical (see figure). You may assume that at $X=0$ the spring are both unstretched. Find the Lagrangian $L(X, \theta, \dot{X}, \dot{\theta}, t)$.\\
	The coordinates of the mass point are $x=X+a \sin \theta, y=-a \cos \theta$.\\
	The kinetic energy is
	\begin{align*}
T &=\frac{1}{2} M \dot{X}^{2}+\frac{1}{2} m(\dot{X}+a \cos \theta \dot{\theta})^{2}+\frac{1}{2} m a^{2} \sin ^{2} \theta \dot{\theta}^{2} \\ &=\frac{1}{2}(M+m) \dot{X}^{2}+\frac{1}{2} m a^{2} \dot{\theta}^{2}+m a \cos \theta \dot{X} \dot{\theta} .\\
\text{The potential energy is }U&=k X^{2}-m g a \cos \theta.\\
\text{Thus, the Lagrangian is}&=\frac{1}{2}(M+m) \dot{X}^{2}+\frac{1}{2} m a^{2} \dot{\theta}^{2}+m a \cos \theta \dot{X}-k X^{2}+m g a \cos \theta.
\intertext{(b) Find the generalized momenta $p_{X}$ and $p_{\theta}$, and the generalized forces $F_{X}$ and $F_{\theta}$}
\text{We have }p_{X}&=\frac{\partial L}{\partial \dot{X}}=(M+m) \dot{X}+m a \cos \theta \dot{\theta}, p_{\theta}=\frac{\partial L}{\partial \dot{\theta}}=m a^{2} \dot{\theta}+m a \cos \theta \dot{X}.\\
\text{For the forces, }F_{X}&=\frac{\partial L}{\partial X}=-2 k X, \quad F_{\theta}=\frac{\partial L}{\partial \theta}=-m a \sin \theta \dot{X} \dot{\theta}-m g a \sin \theta.
\intertext{(c) Derive the equations o motion.}
\text{The equations of motion arc }&\frac{d}{d t}\left(\frac{\partial L}{\partial \dot{q}_{\sigma}}\right)=\frac{\partial L}{\partial q_{\sigma}}, \text{for each generalized coordinate }q_{\sigma} .\text{ For $X$ we have }\\
(M+m) \ddot{X}+m a \cos \theta \ddot{\theta}-m a \sin \theta \dot{\theta}^{2}&=-2 k X .\\
\text{For }\theta, m a^{2} \ddot{\theta}+m a \cos \theta \ddot{X}&=-m g a \sin \theta.
\intertext{(d) Find expression for all conserved $_{1}$ uantities.}
\intertext{Horizontal and vertical translationa $^{1}$ symmetries are broken by the springs and by gravity, respectively. The remaining symmetry is that of time translation. From $\frac{d H}{d t}=-\frac{\partial L}{\partial t}$, we have that $H=\sum_{\sigma} p_{\sigma} \dot{q}_{\sigma}-L$ is conserved. For this problem, the kinetic energy is a homogeneous function of degree 2 in the generalized velocities, and the potential is velocity-independent. Thus,}
H=T+U&=\frac{1}{2}(M+m) \dot{X}^{2}+\frac{1}{2} m a^{2} \dot{\theta}^{2}+m a \cos \theta \dot{X} \dot{\theta}+k X^{2}-m g a \cos \theta
	\end{align*}
\end{answer}
\item A point particle of mass $m$ moves in three dimensions in a helical potential
$$
U(\rho, \phi, z)=U_{0} \rho \cos \left(\phi-\frac{2 \pi z}{b}\right)
$$
We call $b$ the pitch of the helix.
\begin{answer}
	\begin{align*}
	\intertext{(a) Write down the Lagrangian, choosing $(\rho, \phi, z)$ as generalized coordinates.}
\text{	The Lagrangian is }L&=\frac{1}{2} m\left(\dot{\rho}^{2}+\rho^{2} \dot{\phi}^{2}+\dot{z}^{2}\right)-U_{0} \rho \cos \left(\phi-\frac{2 \pi z}{b}\right)
\intertext{	(b) Find the equation of motion.}
	\text{Clearly }p_{\rho}&=m \dot{\rho} ; p_{\phi}=m \rho^{2} \dot{\phi} ; p_{z}=m \dot{z},\text{ and}\\
	F_{\rho}=m \rho \dot{\phi}^{2}-U_{0} \cos \left(\phi-\frac{2 \pi z}{b}\right), F_{\phi}&=U_{0} \rho \sin \left(\phi-\frac{2 \pi z}{b}\right), F_{z}=-\frac{2 \pi U_{0}}{b} \rho \sin \left(\phi-\frac{2 \pi z}{b}\right) \text {. }
\intertext{	Thus, the equation of motion are}
	m \ddot{\rho}&=m \rho \dot{\phi}^{2}-U_{0} \cos \left(\phi-\frac{2 \pi z}{b}\right) \\
	m \rho^{2} \ddot{\phi}+2 m \rho \dot{\rho} \dot{\phi}&=U_{0} \rho \sin \left(\phi-\frac{2 \pi z}{b}\right)\\
	m \ddot{z}&=-\frac{2 \pi U_{0}}{b} \rho \sin \left(\phi-\frac{2 \pi z}{b}\right)
	\intertext{(c) Show that there exists a continuous one-parameter family of coordinate transformations which leaves $L$ invariant. Find the associated conserved quantity, $\Lambda$. Is anything else conserved?}
	\intertext{Due to the helical symmetry, we have that $\phi \rightarrow \phi+\zeta, z \rightarrow z+\frac{b}{2 \pi} \zeta$ is such a continuous one-parameter}
	\intertext{family of coordinate transformations. Since it leaves the combination $\phi-\frac{2 \pi z}{\dot{b}}$ unchanged, we have that $\frac{d L}{d \zeta}=0$, and}
\Lambda &=\left.p_{\rho} \frac{\partial \rho}{\partial \zeta}\right|_{\zeta=0}+\left.p_{\phi} \frac{\partial \phi}{\partial \zeta}\right|_{\zeta=0}+\left.p_{z} \frac{\partial z}{\partial \zeta}\right|_{\zeta=0} \\ &=p_{\phi}+\frac{b}{2 \pi} p_{z} \\ &=m \rho^{2} \dot{\phi}+\frac{m b}{2 \pi} \dot{z}
\intertext{is the conserved Noether 'charge'. The other conserved quantity is the Hamiltonian,}
H&=\frac{1}{2} m\left(\dot{\rho}^{2}+\rho^{2} \dot{\phi}^{2}+\dot{z}^{2}\right)+U_{0} \rho \cos \left(\phi-\frac{2 \pi z}{b}\right)
\intertext{Note that $H=T+U$ because $T$ is homogeneous of degree 2 and $U$ is homogeneous of degree 0 in the generalized velocities.} 
	\end{align*}
\end{answer}
\textbf{Statement for Linked Answer Q.12 and Q.13 :}\\
A particle of mass $m$ is constrained to move in a vertical plane along a trajectory given by $x=\mathrm{A} \cos \theta$, $y=\mathrm{A} \sin \theta$, where $\mathrm{A}$ is a constant.
\item The Lagrangian of the particle is
 \begin{tasks}(2)
	\task[\textbf{a.}]$\frac{1}{2} m \mathrm{~A}^{2} \dot{\theta}^{2}-m g \mathrm{~A} \cos \theta$
	\task[\textbf{b.}]$\frac{1}{2} m \mathrm{~A}^{2} \dot{\theta}^{2}-m g \mathrm{~A} \sin \theta$
	\task[\textbf{c.}] $\frac{1}{2} m \mathrm{~A}^{2} \dot{\theta}^{2}$
	\task[\textbf{d.}] $\frac{1}{2} m \mathrm{~A}^{2} \dot{\theta}^{2}+m g \mathrm{~A} \cos \theta$
\end{tasks}
\begin{answer}$\left. \right. $
	\begin{figure}[H]
		\centering
		\includegraphics[height=4.2cm,width=4.5cm]{Lagrangian 08}
	\end{figure}
	\begin{align*}
	L&=T-V\\
	L&=\frac{1}{2} m\left(\dot{x}^{2}+\dot{y}^{2}\right)-m g y^{x}\\
	x&=A \cos \theta, \quad y=A \sin \theta\\
	\dot{x}&=-A \sin \theta \dot{\theta}, \quad \dot{y}=A \cos \theta \dot{\theta}\\
	 \therefore \quad L &=\frac{1}{2} m\left(A^{2} \sin ^{2} \theta \dot{\theta}^{2}+A^{2} \cos ^{2} \theta \dot{\theta}^{2}\right)-m g A \cos \theta \\ L &=\frac{1}{2} m A^{2} \dot{\theta}^{2}-m g A \cos \theta 
	\end{align*}
	 Correct answer is option \textbf{(a)}
\end{answer}
\item The equation of motion of the particle is
 \begin{tasks}(2)
	\task[\textbf{a.}]$\ddot{\theta}-\frac{g}{\mathrm{~A}} \cos \theta=0$
	\task[\textbf{b.}]$\ddot{\theta}+\frac{g}{\mathrm{~A}} \sin \theta=0$
	\task[\textbf{c.}] $\ddot{\theta}=0$
	\task[\textbf{d.}]  $\ddot{\theta}-\frac{g}{\mathrm{~A}} \sin \theta=0$
\end{tasks}
\begin{answer}
	Equation of motion
	\begin{align*}
	\frac{d}{d t}\left(\frac{\partial L}{\partial \dot{\theta}}\right)-\frac{\partial L}{\partial \theta}&=0\\
	\frac{d}{d t}\left(m A^{2} \dot{\theta}\right)-m g A \sin \theta&=0\\
	m A^{2} \ddot{\theta}-m g A \sin \theta&=0\\
	\ddot{\theta}-\frac{g}{A} \sin \theta&=0
	\end{align*}
	Correct answer is option \textbf{(d)}
\end{answer}
\item A particle of mass $M$ is attached to two identical springs of unstretched length $L_{0}$ and spring constant $k$. The entire system is placed on a horizontal frictionless table as shown in the figure. The mass is slightly puled along the surface of the table and perpendicular to the lengths of the springs and then let go. Using the Lagrangian equation (s) of motion, show whether the mass will execute simple harmonic motion. If so, find the time period.
\begin{figure}[H]
	\centering
	\includegraphics[height=2.5cm,width=7cm]{Lagrangian 09}
\end{figure}
\begin{answer}$\left. \right. $
	\begin{figure}[H]
		\centering
		\includegraphics[height=2.5cm,width=6cm]{Lagrangian 10}
	\end{figure}
	\begin{align*}
\text{	Kinetic energy }T&=\frac{1}{2} m \dot{x}^{2}\\
\text{	potential energy }V&=\frac{1}{2} k\left(\sqrt{\ell_{0}^{2}+x^{2}}-\ell_{0}\right)^{2}
\intertext{Number gravitational potential energy term because motion is taking place in horizontal plane.}
\text{Lagrangian }L&=T-V=\frac{1}{2} m \dot{x}^{2}-\frac{1}{2} k\left(\sqrt{\ell_{0}^{2}+x^{2}}-\ell_{0}\right)^{2}
\intertext{Lagrange's equation}
&\frac{d}{d t}\left(\frac{\partial L}{\partial \dot{x}}\right)-\frac{\partial L}{\partial x}=0 \\
&m \ddot{x}+2 k\left(\sqrt{\ell_{0}^{2}+x^{2}}-\ell_{0}\right) \frac{x}{\sqrt{\ell_{0}^{2}+x^{2}}}=0 \\
&m \ddot{x}+2 k x\left(1-\frac{\ell_{0}}{\sqrt{\ell_{0}^{2}+x^{2}}}\right)=0\\
\text{for }&x<<\ell_{0} \quad m \ddot{x}+2 k x\left(1-\left(1+\frac{x^{2}}{\ell_{0}^{2}}\right)^{-1 / 2}\right)=0\\
m \ddot{x}+2 k x\left(1-1+\frac{x^{2}}{2 \ell_{0}^{2}}\right)&=0 \Rightarrow m \ddot{x}+\frac{k x^{3}}{\ell_{0}^{2}}=0 \ddot{x} \alpha-x^{3}
\intertext{Motion is not simple harmonic.}
	\end{align*}
\end{answer}
\end{enumerate}
\section{Hamiltonian Mechanics}
\begin{enumerate}
	\item Show that $\frac{d H}{d t}=\frac{\partial H}{\partial t}$
	\begin{answer}
		\begin{align*}
		\text{Since }H&=H\left(q_{j}, p_{j}, t\right)\\
		\therefore \quad \frac{d H}{d t} & =\frac{\partial H}{\partial t}+\sum_{j} \frac{\partial H}{\partial q_{j}}+\sum_{j} \frac{\partial H}{\partial p_{j}} \dot{p}_{j} \Rightarrow \frac{d H}{d t}=\frac{\partial H}{\partial t}+\sum_{j}\left(-\dot{p}_{j} \dot{q}_{j}\right)+\sum_{k}\left(\dot{q}_{j} \dot{p}_{j}\right) \\
		\therefore \quad \frac{d H}{d t} & =\frac{\partial H}{\partial t}
		\end{align*}
	\end{answer}
	\item Obtain the Hamiltonian function of a compound pendulum and hence obtain the equation describing its motion.
	\begin{answer}
		The pendulum is a harmonic conservative system. It is holonomic because the constraint is on the position of the pendulum in the sense that it must vibrate in its own plane and conservative because the gravitational force acting on it is conservative. With reference to a frame fixed at the point of suspension with downward vertical as the reference line, we have
		\begin{align*}
		T&=\frac{1}{2} I \dot{\theta}^{2}
	\intertext{	where $I$ is the moment of inertia of the pendulum about the axis passing through the point suspension and perpendicular to its plane.}
		V&=-m g \ell \cos \theta
	\intertext{	where $\ell$ is the distance of the centre of gravity from the centre of suspension.}
\therefore & H=T+V=\frac{1}{2} I \dot{\theta}^{2}-m g \ell \cos \theta \\ \text { Now, } & p_{\theta}=\frac{\partial T}{\partial \dot{\theta}}=I \dot{\theta} \\ \therefore & H=\frac{1}{2} I \frac{p_{\theta}^{2}}{I^{2}}-m g \ell \cos \theta \\ \text { Or, } & H=\frac{p_{\theta}^{2}}{I^{2}}-m g \ell \cos \theta \\ & \frac{\partial H}{\partial p_{\theta}}=\frac{p_{\theta}}{I} \text { and } \frac{\partial H}{\partial p_{\theta}}=-m g \ell(-\sin \theta)=m g \ell \sin \theta\\
\text{By Hamilton's general equations }&\left(\dot{p}_{k}=-\frac{\partial H}{\partial q_{k}}\right.\text{ and }
\left.\dot{q}_{k}=\frac{\partial H}{\partial p_{k}}\right)\\
\text{We have here }\dot{p}_{\theta}&=-\frac{\partial H}{\partial \theta}=-m g \ell \sin \theta\text{ and }\dot{\theta}=\frac{\partial H}{\partial p_{\theta}}=\frac{p_{\theta}}{I}
\intertext{These are the two equations describing the motion of a compound pendulum. However, these two equaions can be combined into a signle equation. Differentiating the second equation and eliminating $\dot{p}_{\theta}$,}
\intertext{we have}
\ddot{\theta}&=\frac{\dot{p}_{\theta}}{I} \sin \theta ; \text { or } \ddot{\theta}+\frac{m g I}{I} \sin \theta=0
\intertext{This is the required single equation that describes the motion of a compound pendulum.}
		\end{align*}
	\end{answer}
	\item A particle moves in the $x y$ plane under the influence of a central force depending on its distance from the origin.\\
	(a) Set up the Hamiltonian for the system\\
	(b) Obtain Hamilton's equations of motion
	\begin{answer}
	\begin{align}
	\text{(a) }T \text{(kinetic energy of of the particle) }&=\frac{1}{2 m \dot{r}^{2}}+\frac{1}{2} I \dot{\theta}^{2}=\frac{1}{2} m \dot{r}^{2}+\frac{1}{2} m r^{2} \dot{\theta}^{2}\quad
	\left(\because I=m r^{2}\right)\notag\\
\therefore & H=T+V=\frac{1}{2} m \dot{r}^{2}+\frac{1}{2} m r^{2} \dot{\theta}^{2}+V(r)\notag \\ \text { Now, } & p_{r}=\frac{\partial T}{\partial \dot{r}}=m \dot{r} \text { and } p_{\theta}=\frac{\partial T}{\partial \dot{\theta}}=I \dot{\theta}=m r^{2} \dot{\theta} \notag\\ \therefore & H=\frac{1}{2} m \cdot \frac{p_{r}^{2}}{m^{2}}+\frac{1}{2} m r^{2} \frac{p_{\theta}^{2}}{m^{2} r^{4}}+V(r) \notag\\ \text { Or, } & H=\frac{p_{r}^{2}}{2 m}+\frac{p_{\theta}^{2}}{2 m r^{2}}+V(r)\notag
\intertext{(b) We have from Hamilton's equation,}
&\left(\dot{p}_{k}=-\frac{\partial H}{\partial q_{k}} \text { and } \dot{q}_{k}=\frac{\partial H}{\partial p_{k}}\right) \notag\\
&\dot{p}_{r}=-\frac{\partial H}{\partial r}=-\left(-\frac{p_{\theta}^{2}}{m r^{3}}+\frac{\partial V}{\partial r}\right)=\frac{p_{\theta}^{2}}{m r^{3}}-\frac{\partial V}{\partial r} \label{HM-29}\\
&\dot{r}=\frac{\partial H}{\partial p_{r}}=\frac{p_{r}}{m} \label{HM-30}\\
&\dot{p}_{\theta}=-\frac{\partial H}{\partial \theta}=0 \label{HM-31}\\
&\dot{\theta}=\frac{\partial H}{\partial p_{\theta}}=\frac{p_{\theta}}{m r^{2}}\label{HM-32}
\intertext{Equations (\ref{HM-29}), (\ref{HM-30}), (\ref{HM-31}) and (\ref{HM-32}) are the required Hamilton's equations. However they can be combined to reduce the number of equations describing the motion. Equations (\ref{HM-29}) and (\ref{HM-30}) combine into (after elimination of $\dot{p}_{r}$ )}\notag
\ddot{r}&=\frac{p_{\theta}^{2}}{m^{2} r^{3}}-\frac{1}{m} \frac{\partial V}{\partial r}=\frac{m^{2} r^{4} \dot{\theta}^{2}}{m^{2} r^{2}}-\frac{1}{m} \frac{\partial V}{\partial r}\notag
\intertext{and (\ref{HM-31}) and (\ref{HM-32}) combine into (after elimination of $\dot{p}_{\theta}$ ) $\ddot{\theta}=0$ Thus the equations describing motion of the particle are}
\ddot{r}-r \dot{\theta}^{2}&=-\frac{1}{m} \frac{\partial V}{\partial r} \text { and } \ddot{\theta}=0\notag\\
\text{Or, }\quad \ddot{r}-r \dot{\theta}^{2}&=-\frac{1}{m} \frac{\partial V}{\partial r}\text{ and }\dot{\theta}=\frac{p_{\theta}}{m r^{2}}=\frac{a \text { constant }}{m r^{2}}\notag\\
\quad\left(\because p_{\theta}=a\right.&\text{ constant from }\left.(i i i)\right)\text{
	Or, }\quad m r^{2} \dot{\theta}=a\text{ constant}\notag
	\end{align}
	\end{answer}
	\item The Hamiltonian of a simple pendulum consisting of a mass ' $\mathrm{m}$ ' attached to a massless string of length $l$ is $\mathrm{H}=\frac{\mathrm{p}_{\theta}^{2}}{2 \mathrm{~m} \ell^{2}}+\mathrm{mg} \ell(1-\cos \theta)$. If L denotes the Lagrangian, the value of $\frac{\mathrm{dL}}{\mathrm{dt}}$ is :
	 \begin{tasks}(2)
		\task[\textbf{a.}]$-\frac{2 g}{\ell} p_{\theta} \sin \theta$
		\task[\textbf{b.}]$-\frac{\mathrm{g}}{\ell} \mathrm{p}_{\theta} \sin 2 \theta$
		\task[\textbf{c.}] $\frac{\mathrm{g}}{\ell} \mathrm{p}_{\theta} \cos \theta$
		\task[\textbf{d.}] $\ell \mathrm{p}_{\theta}^{2} \cos \theta$
	\end{tasks}
	\begin{answer}
		\begin{align*}
		\mathrm{H}&=\frac{\mathrm{p}_{\theta}^{2}}{2 \mathrm{~m} \ell^{2}}+\mathrm{mg} \ell(1-\cos \theta) \Rightarrow \mathrm{L}=\sum_{\mathrm{i}} \mathrm{p}_{\mathrm{i}} \dot{\mathrm{q}}_{\mathrm{i}}-\mathrm{H}=\mathrm{p}_{\theta} \dot{\theta}-\mathrm{H}\\
		\dot{\theta}&=\frac{\partial \mathrm{H}}{\partial \mathrm{p}_{\theta}}=\frac{\mathrm{p}_{\theta}}{\mathrm{m} \ell^{2}}\text{ and }\dot{\mathrm{p}}_{\theta}=-\frac{\partial \mathrm{H}}{\partial \theta}=-\mathrm{mg} \ell \sin \theta\\
		\text{Therefore, }\mathrm{L}&=\mathrm{p}_{\theta} \frac{\mathrm{p}_{\theta}}{\mathrm{m} \ell^{2}}-\left[\frac{\mathrm{p}_{\theta}^{2}}{2 \mathrm{~m} \ell^{2}}+\mathrm{mg} \ell(1-\cos \theta)\right] \Rightarrow \mathrm{L}=\frac{\mathrm{p}_{\theta}^{2}}{2 \mathrm{~m} \ell^{2}}-\mathrm{mg} \ell(1-\cos \theta)\\
		\text{Since, }L&=L\left(\theta, p_{\theta}\right)\text{Since, }
		 -\frac{L}{1}=\frac{\partial L}{\partial p_{\theta}} \dot{p}_{\theta}+\frac{\partial L}{\partial \theta} \dot{\theta}\\
		 \text{Now, }\frac{\partial \mathrm{L}}{\partial \mathrm{p}_{\theta}}&=\frac{\mathrm{p}_{\theta}}{\mathrm{m} \ell^{2}}, \quad \dot{\mathrm{p}}_{\theta}=-1.1 \mathrm{~g} \ell \sin \theta, \quad \frac{\partial \mathrm{L}}{\partial \theta}=-\mathrm{mg} \ell \sin \theta, \quad \dot{\theta}=\frac{\mathrm{p}_{\theta}}{\mathrm{m} \ell^{2}} \\
		 \text{Then} \frac{\mathrm{dL}}{\mathrm{dt}}&=\frac{\mathrm{p}_{\theta}}{\mathrm{m} \ell^{2}}(-\mathrm{mg} \ell \sin \theta)-\mathrm{mg} \ell \quad \mathrm{n} \theta \frac{\mathrm{p}_{\theta}}{\mathrm{m} \ell^{2}}=-\frac{2 \mathrm{~g}}{\ell} \mathrm{p}_{\theta} \sin \theta
		\end{align*}
		Correct option is \textbf{(a)}
	\end{answer}
	\item The particel of mass $m$ is constrained to move on the surface of a cylinder of radius $a$ under an attractive central force $F$, given by
	$$
	F=-k r
	$$
	where $k$ is the force constant. This force is proportional to the distance of the particle from the origin
\begin{answer}
		The motion of the particle can be described in terms of the cartesian coordinates $\mathrm{x}, \mathrm{y}, \mathrm{z}$ or cylindrical $\rho, \theta, z$ (See figure). The equation of constraint is
		\begin{figure}[H]
			\centering
			\includegraphics[height=7cm,width=4.5cm]{Lagrangian 12}
		\end{figure}
	\begin{align}
	\rho^{2}&=x^{2}+y^{2}=a^{2}\label{HM-33-}
\intertext{	To find velocity in cylinderical coordinates, we can proceed as follows}\notag
\text{The change in position along }\rho&=d \rho\notag\\
\text{The change in position along }\theta&=\rho d \theta\notag\\
\text{The change in position along }z&=d z
\text{So, }v_{\rho}=\dot{\rho}, v_{\theta}=\rho \dot{\theta}\text{and} v_{z}=\dot{z}\notag\\
\text{So, we have }v^{2}&=v_{\rho}^{2}+v_{\theta}^{2}+v_{z}^{2}.\notag\\
\text{The kinetic energy, }T&=\frac{1}{2} m v^{2}=\frac{1}{2}\left(\dot{\rho}^{2}+\rho^{2} \dot{\theta}^{2}+\dot{z}^{2}\right)\notag\\
\text{Here, }\rho&=a\notag\\
\text{Therefore, }\dot{\rho}&=0,\text{ and hence,}\notag\\
T &=\frac{1}{2} m\left(a^{2} \dot{\theta}^{2}+\dot{z}^{2}\right) \notag\\ \text{and}\quad V &=\frac{1}{2} k r^{2}=\frac{1}{2} k\left(x^{2}+y^{2}+z^{2}\right)=\frac{1}{2} k\left(a^{2}+z^{2}\right)\notag\\
\therefore \quad L=T-V&=\frac{1}{2} m\left(a^{2} \dot{\theta}^{2}+\dot{z}^{2}\right)-\frac{1}{2} k\left(a^{2}+z^{2}\right)\label{HM-34-}\\
\text{Hence,}
p_{\theta}&=\frac{\partial L}{\partial \dot{\theta}}=m a^{2} \dot{\theta}\text{ or} \dot{\theta}=\frac{p_{\theta}}{m a^{2}}\notag\\
\text{and }\quad p_{z}&=\frac{\partial L}{\partial \dot{z}}=m \dot{z}\text{ or} \dot{z}=\frac{p_{z}}{m}\notag\\
\text{Therefore, }\quad H&=T+V=\frac{1}{2} m\left(a^{2} \dot{\theta}+\dot{z}^{2}\right)+\frac{1}{2} k\left(a^{2}+z^{2}\right)\notag\\
\text { Or, } \quad H&=\frac{p_{\theta}^{2}}{2 m a^{2}}+\frac{p_{z}^{2}}{m}+\frac{1}{2} k\left(a^{2}+z^{2}\right)\label{HM-35-}
\intertext{Hence, the Hamilton's equation are}\notag\\
\dot{z}&=\frac{\partial H}{\partial p_{z}}=\frac{p_{z}}{m}\text{ or }p_{z}=m \dot{z}\label{HM-36-}\\
\dot{\theta}&=\frac{\partial H}{\partial p_{\theta}}=\frac{p_{\theta}}{m a^{2}}\text{ or }p_{\theta}=m a^{2} \dot{\theta}\label{HM-37-}\\
-\dot{p}_{z}&=\frac{\partial H}{\partial z}=k z \quad\text{ or }\quad \dot{\mathrm{p}}_{z}=-k z\label{HM-38-}\\
-\dot{p}_{\theta}&=\frac{\partial H}{\partial \theta}=0 \quad\text{ or }
p_{\theta}= \text{constant}\label{HM-39-}
\intertext{From equation (\ref{HM-36-}) and (\ref{HM-38-}), we get}
m \ddot{z}+k z&=0\notag
\intertext{which shows that the motion of the particle in z direction is simple harmonic with period $T$, given by}\notag
T&=2 \pi \sqrt{\frac{m}{k}}\notag
\intertext{From equation (\ref{HM-37-}) and (\ref{HM-39-}), we get}\notag
p_{\theta}&=m a^{2} \dot{\theta}=\text { constant }\notag
\intertext{Thus the angular momentum about Z-axis is a constant of motion.}\notag
	\end{align}
\end{answer}
	\item Find equations of motion of particle moving near the surface of earth.
	\begin{answer}
		\begin{align*}
		\intertext{Let us consider z-axis along upward vertical direction, the kinetic energy is}
		T&=\frac{1}{2} m\left(\dot{x}^{2}+\dot{y}^{2}+\dot{z}^{2}\right)
	\intertext{	Further the applied force on the body is its weight acting in negative z-direction, i.e.,}
		F&=F_{z}-m g=-\frac{\partial V}{\partial Z}
		\intertext{This gives $V=m g Z$, on setting additive constant to zero Now} 
	\intertext{	Lagrangian is}
		L&=T-V=\frac{1}{2} m\left(\dot{x}^{2}+\dot{y}^{2}+\dot{z}^{2}\right)-m g Z\\
		\text{So that }\frac{\partial L}{\partial \dot{x}}&=\frac{\partial T}{\partial \dot{x}}=p_{x}=m \dot{x}\text{ giving }\dot{x}=\frac{p_{x}}{m}\\
	\text{	Also, }p_{y}&=m \dot{y}, p_{z}=m \dot{z}\\
		\therefore \quad \dot{y}&=\frac{p_{y}}{m}, \dot{z}=\frac{p_{z}}{m}\\
	\intertext{	Hamiltonian for such a system is conserved, i.e.,}
		H&=T+V=\frac{1}{2}\left(\dot{x}^{2}+\dot{y}^{2}+\dot{z}^{2}\right)+m g Z=\frac{1}{2 m}\left(p_{x}^{2}+p_{y}^{2}+p_{z}^{2}\right)+m g Z \\
		\intertext { giving equations of motion }
		\dot{p}_{x}&=-\frac{\partial H}{\partial x}=0, \quad \dot{p}_{y}=-\frac{\partial H}{\partial y}=0\\
		\dot{p}_{z}&=-\frac{\partial H}{\partial Z}=-m g\\
		\therefore \quad \dot{x}&=\frac{\partial H}{\partial p_{x}}=\frac{p_{x}}{m}, \quad \dot{y}=\frac{\partial H}{\partial p_{y}}=\frac{p_{y}}{m}, \dot{z}=\frac{\partial H}{\partial Z}=\frac{p_{z}}{m}
	\intertext{ so we finally get,}
		\ddot{x}&=\frac{\dot{p}_{y}}{m}=0, \quad \ddot{y}=\frac{\dot{p}_{y}}{m}=0, \quad \ddot{z}=\frac{\dot{p}_{z}}{m}=-g
		\end{align*}
	\end{answer}
	\item A mechanical system is described by the Hamiltonian $H(q, p)=\frac{p^{2}}{2 m}+\frac{1}{2} m \omega^{2} q^{2}$. As a result of the canonical transformation generated by $F(q, Q)=-\frac{Q}{q}$, the Hamiltonian in the new coordinate $Q$ and momentum $P$ becomes
	 \begin{tasks}(2)
		\task[\textbf{a.}]$\frac{1}{2 m} Q^{2} P^{2}+\frac{m \omega^{2}}{2} Q^{2}$
		\task[\textbf{b.}]$\frac{1}{2 m} Q^{2} P^{2}+\frac{m \omega^{2}}{2} P^{2}$
		\task[\textbf{c.}]$\frac{1}{2 m} P^{2}+\frac{m \omega^{2}}{2} Q^{2}$
		\task[\textbf{d.}] $\frac{1}{2 m} Q^{2} P^{4}+\frac{m \omega^{2}}{2} P^{-2}$
	\end{tasks}
	\begin{answer}
		\begin{align*}
		F(q, Q)&=-\frac{Q}{q}
	\intertext{	Differential relation for $F(q, Q)$}
		p&=\frac{\partial F}{\partial q}, \quad P=-\frac{\partial F}{\partial Q}\\
		\text{So, }\quad p&=\frac{Q}{q^{2}}, P=\frac{1}{q} \quad \Rightarrow p=Q P^{2}
	\intertext{	New Hamiltonian,}
		H^{\prime}&=H+\frac{\partial F}{\partial t}=H+0\\
		H^{\prime}(Q, P)&=H(q, p)=\frac{p^{2}}{2 m}+\frac{1}{2} m \omega^{2} q^{2}=\frac{1}{2 m} Q^{2} P^{4}+\frac{1}{2} m \omega^{2} P^{-2}
		\end{align*}
		Correct answer is option \textbf{(d)}
	\end{answer}
\end{enumerate}
\section{Small Oscillations}
\begin{enumerate}
	\item  A particle of mass $m$ moves in one dimension under the influence of a potential energy
	$$
	V(x)=-a\left(\frac{x}{\ell}\right)^{2}+b\left(\frac{x}{\ell}\right)^{4}
	$$
	where $a$ and $b$ are positive constants and $\ell$ is a characteristic length. The frequency of small oscillations about a point of stable equilibrium is:
	 \begin{tasks}(2)
		\task[\textbf{a.}] $\frac{1}{2 \pi \ell} \sqrt{\frac{b}{m}}$
		\task[\textbf{b.}]$\frac{2 b}{\pi \ell} \sqrt{\frac{1}{m a}}$
		\task[\textbf{c.}]$\frac{1}{\pi \ell} \sqrt{\frac{a^{2}}{m b}}$
		\task[\textbf{d.}] $\frac{1}{\pi \ell} \sqrt{\frac{a}{m}}$
	\end{tasks}
	\begin{answer}
		\begin{align*}
	\text{	We have,}
		V(x)&=-a\left(\frac{x}{\ell}\right)^{2}+b\left(\frac{x}{\ell}\right)^{4}
		\intertext{Here first we need to find the stable equilibrium. At equilibrium}
		\frac{\partial V(x)}{\partial x}&=0 \quad \Rightarrow \frac{-2 a x}{\ell^{2}}+\frac{4 b x^{3}}{\ell^{4}}=0 \quad \Rightarrow x\left(\frac{2 b x^{2}}{\ell^{2}}-a\right)=0\\
		\Rightarrow \quad x&=0, x=\pm \ell \sqrt{\frac{a}{2 b}}\\
		\text{Now, }\quad \frac{\partial^{2} V}{\partial x^{2}}&=\frac{-2 a}{\ell^{2}}+\frac{12 b x^{2}}{\ell^{4}}\\
		\text{We see that for }x&=0, \frac{\partial^{2} V}{\partial x^{2}}<0
		\text{hence, here there is unstable equilibrium}\\
		\text{For}\qquad x&=\pm \ell \sqrt{\frac{a}{2 b}},\left(\frac{\partial^{2} V}{\partial x^{2}}\right)_{x=\pm \ell \sqrt{\frac{a}{2 b}}}=-\frac{2 a}{\ell^{2}}+\frac{12 b}{\ell^{4}} \ell^{2} \times \frac{a}{2 b}=\frac{4 a}{\ell^{2}}>0
		\intertext{hence, $x=\pm \ell \sqrt{\frac{a}{2 b}}$ corresponds to stable equilibrium.. Now to find the frequeny, we see that the problemis one dimensional, so matrices $\mathbf{V}$ and $\mathbf{T}$ both have one element each, i.e. $V_{11}$ and $T_{11}$ respectively.}
		V_{11}&=\left(\frac{\partial^{2} V}{\partial x^{2}}\right)_{x=\pm f \sqrt{\frac{a}{2 b}}}=\frac{4 a}{\ell^{2}}\\
		\text{Now, the kinetic energy }&=\frac{1}{2} m \dot{x}^{2},\text{ so }T_{11}=m\\
	\text{	Now, we have, }V_{11}-\omega^{2} T_{11}&=0 \qquad\Rightarrow \omega^{2}=\frac{V_{11}}{T_{11}}=\frac{4 a}{m \ell^{2}} \qquad\Rightarrow \omega=\frac{2}{\ell} \sqrt{\frac{a}{m}}\\
	\text{So, the frequency }v&=\frac{\omega}{2 \pi}=\frac{1}{\pi \ell} \sqrt{\frac{a}{m}}
	\intertext{Note: As we have seen in this problem, first we should find the equilibrium configuration of the system then expand potential energy about this configuration in Taylor series.}
		\end{align*}
	\end{answer}
	\item  A particle of mass $m$ is moving in a potential of the form $V(x, y, z)=\frac{1}{2} m \omega^{2}\left(3 x^{2}+3 y^{2}+2 z^{2}+2 x\right)$. The oscillation frequencies of the three normal modes of the particles are given by
	\begin{answer}
		\begin{align*}
		\text{The potential is, }V(x, y, z)&=\frac{1}{2} m \omega^{2}\left(3 x^{2}+3 y^{2}+2 z^{2}+2 x y\right)\\
	\text{	Now, }\quad \frac{\partial V}{\partial x}&=0 \quad \Rightarrow 6 x+2 y=0 \quad \Rightarrow y=-3 x\\
	\frac{\partial V}{\partial y}&=0 \quad \Rightarrow 6 y+2 x=0 \quad \Rightarrow y=\frac{-x}{3}\\
	\frac{\partial V}{\partial z}&=0 \quad \Rightarrow z=0\\
	\text{Now, }\quad \frac{\partial^{2} V}{\partial x^{2}}&=3 m \omega^{2}>0 \text{for all $x$}\\
	\frac{\partial^{2} V}{\partial y^{2}}&=3 m \omega^{2}>0\text{ for all $y$}\\
		\text{and }\frac{\partial^{2} V}{\partial z^{2}}&=4>0\text{ for all $z$}
		\intertext{All these conditions tell us that equilibrium point is $(0,0,0)$, because $y=-3 x=-\frac{x}{3}$ satisfies only $x=y=0$ So, the given potential is in the form of expansion about $(0,0,0)$ and the matrix $V_{i j}$ can be written just by inspection.}
		V &=\frac{1}{2} m \omega^{2}\left(3 x^{2}+3 y^{2}+2 z^{2}+2 x y\right) \\ &=\frac{1}{2} m \omega^{2}\left[\begin{array}{lll}x & y & z\end{array}\right]\left[\begin{array}{lll}3 & 1 & 0 \\ 1 & 3 & 0 \\ 0 & 0 & 2\end{array}\right]\left[\begin{array}{l}x \\ y \\ z\end{array}\right]\\
\text{		because}\quad
		&V=\frac{1}{2}\left(3 m \omega^{2} x^{2}+3 m \omega^{2} y^{2}+2 m \omega^{2} z^{2}+m \omega^{2} x y+m \omega^{2} y x\right) \\
		\text{or,}\quad&V=\frac{1}{2}\left(V_{11} x^{2}+V_{22} y^{2}+V_{33} z^{2}+V_{12} x y+V_{21} y x\right)\\
	\text{	Kinetic energy }T&=\frac{1}{2} m\left(\dot{x}^{2}+\dot{y}^{2}+\dot{z}^{2}\right)=\frac{1}{2}\left[\begin{array}{lll}\dot{x} & \dot{y} & \dot{z}\end{array}\right]\left[\begin{array}{lll}m & 0 & 0 \\ 0 & m & 0 \\ 0 & 0 & m\end{array}\right]\left[\begin{array}{l}\dot{x} \\ \dot{y} \\ \dot{z}\end{array}\right]\\
	\text{Now,}
	\left|\mathbf{V}-\Omega^{2} \mathbf{T}\right|&=0\quad \text{where $\Omega$ is normal mode freuency.}\\
	\Rightarrow \quad&\left|\begin{array}{ccc}3 m \omega^{2}-\Omega^{2} m & m \omega^{2} & 0 \\ m \omega^{2} & 3 m \omega^{2}-\Omega^{2} m & 0 \\ 0 & 0 & 2 m \omega^{2}-\Omega^{2} m\end{array}\right|=0\\
	\Rightarrow \quad\left(2 \omega^{2}-\Omega^{2}\right)&\left[\left(3 \omega^{2}-\Omega^{2}\right)^{2}-\omega^{4}\right]=0 \quad \Rightarrow \Omega_{1}^{2}=2 \omega^{2}\\
\text{	and }\quad \Omega^{2}&=3 \omega^{2} \pm \omega^{2} \quad \Rightarrow \Omega_{2}^{2}=2 \omega^{2}\text{ and }\Omega_{3}^{2}=4 \omega^{2}
\intertext{So, the frequencies are $\omega \sqrt{2}, \omega \sqrt{2}$ and $2 \omega$}
\intertext{	Note that frequencies are always positive, hence we shouldn't write $\Omega_{1}=\pm \omega \sqrt{2}$ etc.}
		\end{align*}
	\end{answer}
	\item \textbf{Coupled mass point on a circle:}\\
	Four mass points of mass $m$ move on a circle of radius $R$. Each mass point is coupled to its two neighboring points by a spring with spring constant $k$ (see figure below). Find the Lagrangian of the system and derive the equations of motion of the system. Calculate the eigenfrequencies of the system, and discuss the related eigenvibrations.
	\begin{answer}
		\begin{align*}
	\intertext{	The kinetic energy of the system is given by}
		T&=\frac{1}{2} m \sum_{v=1}^{4} \dot{s}_{v}^{2}
	\intertext{	For small displacement from the equilibrium position, the potential reads}
		V&=\frac{1}{2} k \sum_{v=1}^{4}\left(s_{v+1}-s_{v}\right)^{2}, \quad s_{4+1}=s_{1}
	\intertext{	We set $s_{v}=R \varphi_{v}$, and take the angles $\varphi_{v}$ as generalized coordinates. Then the Lagrangian is}
		L&=T-V=\frac{1}{2} m R^{2} \sum_{v=1}^{4} \dot{\varphi}_{v}^{2}-\frac{1}{2} k R^{2} \sum_{v=1}^{4}\left(\varphi_{v+1}-\varphi_{v}\right)^{2}
	\intertext{	From the Lagrange equations}
		\frac{d}{d t} \frac{\partial L}{\partial \dot{\varphi}_{v}}&=\frac{\partial L}{\partial \varphi_{v}}
		\end{align*}
		\begin{figure}[H]
			\centering
			\includegraphics[height=4.8cm,width=5cm]{small oscillations-01}
		\end{figure}
		\begin{align*}
		\intertext{we find the equations of motion:}
		\frac{d}{d t} \frac{\partial L}{\partial \dot{\varphi}_{v}}=m R \ddot{\varphi}_{v}=-\frac{1}{2} k R^{2}\left[2\left(\varphi_{v}-\varphi_{v+1}\right)+2\left(\varphi_{v}-\varphi_{v+1}\right)\right]&=\frac{\partial L}{\partial \varphi_{v}} .
	\intertext{	For the case of four mass points, we then obtain}
	\ddot{\varphi}_{1}=\frac{k}{m}\left(\varphi_{2}-2 \varphi_{1}+\varphi_{4}\right), \quad \ddot{\varphi}_{2}&=\frac{k}{m}\left(\varphi_{3}-2 \varphi_{2}+\varphi_{1}\right)\\
	\ddot{\varphi}_{3}=\frac{k}{m}\left(\varphi_{4}-2 \varphi_{3}+\varphi_{2}\right), \ddot{\varphi}_{4}&=\frac{k}{m}\left(\varphi_{1}-2 \varphi_{4}+\varphi_{3}\right)
	\intertext{With the ansatz $\varphi_{v}=A_{v} \cos \omega t, \ddot{\varphi}_{v}=-A_{v} \omega^{2} \cos \omega t$, we are led to the following linear system of equations:}
	\left(\begin{array}{cccc}2 \frac{k}{m}-\omega^{2} & -\frac{k}{m} & 0 & -\frac{k}{m} \\ -\frac{k}{m} & 2 \frac{k}{m}-\omega^{2} & -\frac{k}{m} & 0 \\ 0 & -\frac{k}{m} & 2 \frac{k}{m}-\omega^{2} & -\frac{k}{m} \\ -\frac{k}{m} & 0 & -\frac{k}{m} & 2 \frac{k}{m}-\omega^{2}\end{array}\right)\left(\begin{array}{l}A_{1} \\ A_{2} \\ A_{3}\end{array}\right)&=0
	\intertext{For the nontrivial solutions, the determinant of the coefficient matrix must vanish. This condition leads to the determining equation for the eigenfrequencies:}
	\left(2 \frac{k}{m}-\omega^{2}\right)^{2}\left(4 \frac{k}{m}-\omega^{2}\right)\left(-\omega^{2}\right)=0
	\intertext{The frequencies are}
	\omega_{1}^{2}=0, \quad \omega_{2}^{2}&=4 \frac{k}{m}, \quad \omega_{3}^{2}=\omega_{4}^{2}=2 \frac{k}{m}
\intertext{	To calculate the related eigenvibrations, we insert these frequencies into the system of equations:}
	\end{align*}
\begin{align*}
	&\text{(1) }\omega_{1}^{2}=0: A_{1}=A_{2}=A_{3}=A_{4} : \text{The system does not vibrate but performs a uniform rotation}\\
	&\text{(2) }\omega_{2}^{2}=4 \frac{k}{m}: A_{1}=A_{3}=-A_{2}=-A_{4}:\text{ Two neighboring mass points perform an out-of-phase vibration}\\
	&\text{(3) }\omega_{3}^{2}=\omega_{4}^{2}=2 \frac{k}{m}: A_{1}=A_{2}=-A_{3}=-A_{4}\text{ or }A_{1}=A_{4}=-A_{2}=-A_{3} :\text{ Two neighboring mass points }\\&\text{vibrate in phase}
	\end{align*}
	\begin{figure}[H]
		\centering
		\includegraphics[height=8cm,width=11cm]{small oscillations-02}
	\end{figure}
	\end{answer}
	\item Two equal masses coupled by two equal springs
	Two equal masses move without friction on a plate. They are connected to each other and to the wall by two springs, as is indicated by Figure $7.3$. The two spring constants are equal, and the motion shall berestricted to a straight line (one-dimensional motion).
	Find\\
	(a) the equations of motion,\\
	(b) the normal frequencies, and\\
	(c) the anplitude ratios of the normal vibrations and the general solution.
	\begin{answer}
		\begin{align*}
		\intertext{(a) Let $x_{1}$ and $x_{2}$ be the displacements from the rest positions. The cquations of motion then read}
		m \ddot{x}_{1}&=-k x_{2}+k\left(x_{2}-x_{1}\right) \\
		m \dot{x}_{2}&=-k\left(x_{2}-x_{1}\right)
		\intertext{(b) For determining the normal frequencies, we use the ansatz}
		x_{1}&=A_{1} \cos \omega t, \quad x_{2}=A_{2} \cos \omega t
		\end{align*}
		\begin{figure}[H]
			\centering
			\includegraphics[height=2cm,width=6.5cm]{small oscillations-03}
		\end{figure}
		\begin{align}
	\intertext{	and thereby get from $\color{red}{(7.9)}$ and $(\color{red}{7.10})$ the equations}\notag\\
		&\left(2 k-m \omega^{2}\right) A_{1}-k A_{2}=0 \label{so-05}\\
		&-k A_{1}+\left(k-m \omega^{2}\right) A_{2}=0\notag
		\intertext{From therequirement for nontrivial solutions of the system of equations, it follows that the determinant of coefficient vanishes:}\notag
		D&=\left|\begin{array}{cc}
		2 k-m \omega^{2} & -k \notag\\
		-k & k-m \omega^{2}
		\end{array}\right|=0
	\intertext{	From this follows the determining equation for the eigenfrequencies,}\notag
		\omega^{4}-3 \frac{k}{m} \omega^{2}+\frac{k^{2}}{m^{2}}&=0\notag
		\intertext{with the positive solutions}\notag
		\omega_{1}=\frac{\sqrt{5}+1}{2} \sqrt{\frac{k}{m}} \text { and } \omega_{2}&=\frac{\sqrt{5}-1}{2} \sqrt{\frac{k}{m}}, \omega_{1}>\omega_{2} .\notag
		\intertext{By inserting the eigenfrequencies in (\ref*{so-05}) one sees that the higher frequency $\omega_{1}$ corresponds to the oppositephase mode, and the lower frequency $\omega_{2}$ to the equal-phase normal vibration:}\notag
		\text{with }\omega_{1}^{2}&=\frac{1}{2}(3+\sqrt{5}) \frac{k}{m}, \quad\text{ it follows from (\ref{so-05}) that }A_{2}=-\frac{\sqrt{5}-1}{2} A_{1}\notag\\ \text{with }\omega_{2}^{2}&=\frac{1}{2}(3-\sqrt{5}) \frac{k}{m},\text{ it follows from (\ref{so-05}) that }A_{2}=\frac{\sqrt{5}+1}{2} A_{1} .\notag
		\intertext{Since the two mass points are fixed in different ways, we find amplitudes of different magnitudes. The general solution is obtained as a superposition of the normal vibrations, using the calculated amplitudes ratios:}\notag
		x_{1}(t)&=C_{1} \cos \left(\omega_{1} t+\varphi_{1}\right)+C_{2} \cos \left(\omega_{2} t+\varphi_{2}\right)\notag\\
		x_{2}(t)&=-\frac{\sqrt{5}-1}{2} C_{1} \cos \left(\omega_{1} t+\varphi_{1}\right)+\frac{\sqrt{5}+1}{2} C_{2} \cos \left(\omega_{2} t+\varphi_{2}\right)\notag
		\end{align}
	\end{answer}
	\item  The Lagrangian of a system is given by $L=\frac{1}{2} m \dot{q}_{1}^{2}+2 m \dot{q}_{2}^{2}-k\left(\frac{5}{4} q_{1}^{2}+2 q_{2}^{2}-2 q_{1} q_{2}\right)$ where $m$ and $k$ are positive constants. The frequencies of its normal modes are
 \begin{tasks}(2)
	\task[\textbf{a.}]$\sqrt{\frac{k}{2 m}}, \sqrt{\frac{3 k}{m}}$
	\task[\textbf{b.}]$\sqrt{\frac{k}{2 m}}(13 \pm \sqrt{73})$
	\task[\textbf{c.}]$\sqrt{\frac{5 k}{2 m}}, \sqrt{\frac{k}{m}}$
	\task[\textbf{d.}]  $\sqrt{\frac{k}{2 m}}, \sqrt{\frac{6 k}{m}}$
\end{tasks}
	\begin{answer}
		\begin{align*}
		 L &=\frac{1}{2} m \dot{q}_{1}^{2}+2 m \dot{q}_{2}^{2}-k\left(\frac{5}{4} q_{1}^{2}+2 q_{2}^{2}-2 q_{1} q_{2}\right) \\ &=\frac{1}{2} m \dot{q}_{1}^{2}+\frac{1}{2} 4 m \dot{q}_{2}^{2}-\frac{1}{2} k\left(\frac{5}{2} q_{1}^{2}+4 q_{2}^{2}-4 q_{1} q_{2}\right) \\ \hat{T} &=\left(\begin{array}{cc}m & 0 \\ 0 & 4 m\end{array}\right), \hat{V}=\left(\begin{array}{cc}\frac{5}{2} k & -2 k \\ -2 k & 4 k\end{array}\right) 
		 \intertext{For frequencies of normal modes:}
		 \operatorname{det}\left|\omega^{2} \hat{T}-\hat{V}\right|&=0\\
		 \left|\begin{array}{cc}\left(m \omega^{2}-\frac{5}{2} k\right) & 2 k \\ 2 k & \left(4 m \omega^{2}-4 k\right)\end{array}\right|&=0 \Rightarrow 4\left(m \omega^{2}-k\right) \frac{\left(2 m \omega^{2}-5 k\right)}{2}-4 k^{2}=0\\
		 \Rightarrow 2 m^{2} \omega^{4}+5 k^{2}-7 k m \omega^{2}-2 k^{2}&=0\\
		 \Rightarrow 2\left(m \omega^{2}\right)^{2}-7 k\left(m \omega^{2}\right)+3 k^{2}&=0\\
		 \Rightarrow m \omega^{2}&=\frac{7 k \pm \sqrt{49 k^{2}-24 k^{2}}}{4}=\frac{7 k \pm \sqrt{49 k^{2}-24 k^{2}}}{4}\\&=\frac{7 k \pm 5 k}{4}=3 k, \frac{k}{2}\\
		 \therefore \omega&=\sqrt{\frac{3 k}{m}}, \sqrt{\frac{k}{2 m}}
		\end{align*}
		 Correct answer is option \textbf{(a)}
	\end{answer}
	
	
	
	
	
	
	
	
	
	
	
	
	
	
	
	
	
\end{enumerate}