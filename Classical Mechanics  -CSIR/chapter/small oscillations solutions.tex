\chapter{Small oscillations-solutions}
\begin{abox}
	Practice set 1
\end{abox}
\begin{enumerate}
		\item  A particle of unit mass moves in a potential $V(x)=a x^{2}+\frac{b}{x^{2}}$, where $a$ and $b$ are positive constants. The angular frequency of small oscillations about the minimum of the potential is
		{\exyear{NET JUNE 2011}}
	\begin{tasks}(2)
		\task[\textbf{A.}] $\sqrt{8 b}$
		\task[\textbf{B.}]$\sqrt{8 a}$
		\task[\textbf{C.}] $\sqrt{8 a / b}$
		\task[\textbf{D.}]$\sqrt{8 b / a}$
	\end{tasks}
	\begin{answer}
		\begin{align*}
		V(x)&=a x^{2}+\frac{b}{x^{2}} \Rightarrow \frac{\partial V}{\partial x}=0 \Rightarrow 2 a x-\frac{2 b}{x^{3}}=0 \Rightarrow a x^{4}-b=0 \Rightarrow x_{0}=\left(\frac{b}{a}\right)^{\frac{1}{4}}\\
		\text { Since } \omega&=\sqrt{\frac{k}{m}}, m=1\\
		k&=\left.\frac{\partial^{2} V}{\partial x^{2}}\right|_{x=x_{0}} \text { where } x_{0} \text { is stable equilibrium point. }\\
		\text { Hence } k&=\frac{\partial^{2} V}{\partial x^{2}}=2 a+\frac{6 b}{x_{0}^{4}}=2 a+\frac{6 b}{b /}=8 a \text { at } x=x_{0}=\left(\frac{b}{a}\right)^{\frac{1}{4}}\\
		\text { Thus, } \omega&=\sqrt{8 a} \text {. }
		\end{align*}
		The correct option is \textbf{(b)}
	\end{answer}
		\item Consider the motion of a classical particle in a one dimensional double-well potential $V(x)=\frac{1}{4}\left(x^{2}-2\right)^{2} .$ If the particle is displaced infinitesimally from the minimum on the $x$-axis (and friction is neglected), then
		{\exyear{NET JUNE 2012}}
	\begin{tasks}(1)
		\task[\textbf{A.}] the particle will execute simple harmonic motion in the right well with an angular frequency $\omega=\sqrt{2}$
		\task[\textbf{B.}]the particle will execute simple harmonic motion in the right well with an angular frequency $\omega=2$
		\task[\textbf{C.}]the particle will switch between the right and left wells
		\task[\textbf{D.}]the particle will approach the bottom of the right well and settle there
	\end{tasks}
	\begin{answer}
		\begin{align*}
		V(x)&=\frac{1}{4}\left(x^{2}-2\right)^{2} \Rightarrow \frac{\partial V}{\partial x}=\frac{2}{4}\left(x^{2}-2\right) \times 2 x=0 \Rightarrow x=0, x=\pm \sqrt{2}\\
		\frac{\partial^{2} V}{\partial x^{2}}&=3 x^{2}-2\\
		\text { At } x=0, \frac{\partial^{2} V}{\partial x^{2}}&<0 \text { so } V \text { is maximum. Thus it is unstable point }\\
		\left.\frac{\partial^{2} V}{\partial x^{2}}\right|_{x=\pm \sqrt{2}}&=4 \text { and it is stable equilibrium point with } \omega=\sqrt{\frac{\left.\frac{\partial^{2} V}{\partial x^{2}}\right|_{x=x_{0}}}{\mu}}=2 \quad \because \mu=1 \text {. }
		\end{align*}
		The correct option is \textbf{(b)}	
	\end{answer}
	
		\item Three particles of equal mass $(\mathrm{m})$ are connected by two identical massless springs of stiffness constant $(K)$ as shown in the figure\\
		\begin{figure}[H]
			\centering
			\includegraphics[height=1cm,width=5cm]{problem1}
		\end{figure}
		If $x_{1}, x_{2}$ and $x_{3}$ denote the horizontal displacement of the masses from their respective equilibrium positions the potential energy of the system is
		{\exyear{NET DEC 2012}}
	\begin{tasks}(2)
		\task[\textbf{A.}] $\frac{1}{2} K\left[x_{1}^{2}+x_{2}^{2}+x_{3}^{2}\right]$
		\task[\textbf{B.}]$\frac{1}{2} K\left[x_{1}^{2}+x_{2}^{2}+x_{3}^{2}-x_{2}\left(x_{1}+x_{3}\right)\right]$
		\task[\textbf{C.}]$\frac{1}{2} K\left[x_{1}^{2}+2 x_{2}^{2}+x_{3}^{2}-2 x_{2}\left(x_{1}+x_{3}\right)\right]$
		\task[\textbf{D.}]$\frac{1}{2} K\left[x_{1}^{2}+2 x_{2}^{2}-2 x_{2}\left(x_{1}+x_{3}\right)\right]$
	\end{tasks}
	\begin{answer}
		\begin{align*}
		V&=\frac{1}{2} K\left(x_{2}-x_{1}\right)^{2}+\frac{1}{2} K\left(x_{3}-x_{2}\right)^{2}\\
		V&=\frac{1}{2} K\left(x_{2}^{2}+x_{1}^{2}-2 x_{2} x_{1}\right)+\frac{1}{2} K\left(x_{3}^{2}+x_{2}^{2}-2 x_{3} x_{2}\right)\\
		V&=\frac{1}{2} K\left[x_{1}^{2}+2 x_{2}^{2}+x_{3}^{2}-2 x_{2}\left(x_{1}+x_{3}\right)\right]
		\end{align*}
		The correct option is \textbf{(c)}
	\end{answer}
		\item The time period of a simple pendulum under the influence of the acceleration due to gravity $g$ is $T$. The bob is subjected to an additional acceleration of magnitude $\sqrt{3} g$ in the horizontal direction. Assuming small oscillations, the mean position and time period of oscillation, respectively, of the bob will be
		{\exyear{NET JUNE 2014}}
	\begin{tasks}(2)
		\task[\textbf{A.}] $0^{\circ}$ to the vertical and $\sqrt{3} T$
		\task[\textbf{B.}]$30^{\circ}$ to the vertical and $T / 2$
		\task[\textbf{C.}]$60^{\circ}$ to the vertical and $T / \sqrt{2}$
		\task[\textbf{D.}]$0^{\circ}$ to the vertical and $T / \sqrt{3}$
	\end{tasks}
	\begin{answer}
		\begin{align*}
		T&=2 \pi \sqrt{\frac{l}{g}}\\
		g^{\prime}&=\sqrt{3 g^{2}+g^{2}}=\sqrt{4 g^{2}}=2 g\\
		T^{\prime}&=2 \pi \sqrt{\frac{l}{2 g}} \Rightarrow T^{\prime}=2 \pi \sqrt{\frac{l}{g}} \cdot \frac{1}{\sqrt{2}} \Rightarrow T^{\prime}=\frac{T}{\sqrt{2}}\\
		T \cos \theta&=m g, T \sin \theta=\sqrt{3} m g \Rightarrow \tan \theta=\sqrt{3} \Rightarrow \theta=60^{\circ}
		\end{align*}
		The correct option is \textbf{(c)}	
	\end{answer}
		\item A particle of mass $m$ is moving in the potential $V(x)=-\frac{1}{2} a x^{2}+\frac{1}{4} b x^{4}$ where $a, b$ are positive constants. The frequency of small oscillations about a point of stable equilibrium is
		{\exyear{NET DEC 2014}}
	\begin{tasks}(2)
		\task[\textbf{A.}] $\sqrt{a / m}$
		\task[\textbf{B.}]$\sqrt{2 a / m}$
		\task[\textbf{C.}]$\sqrt{3 a / m}$
		\task[\textbf{D.}]$\sqrt{6 a / m}$
	\end{tasks}
	\begin{answer}
		\begin{align*}
		\because V(x)&=-\frac{1}{2} a x^{2}+\frac{1}{4} b x^{4}	\\
		\frac{\partial V}{\partial x}&=0 \Rightarrow-a x+b x^{3}=0 \Rightarrow x\left[-a+b x^{2}\right]=0 \Rightarrow x=\pm\left(\frac{a}{b}\right)^{\frac{1}{2}}, 0\\
		\because \frac{\partial^{2} V}{\partial x^{2}}&=-a+3 b x^{2}\\
		\text { At } x=0, \frac{\partial^{2} V}{\partial x^{2}}&=-a \text { (Negative so it is unstable point) }\\
		\left.\frac{\partial^{2} V}{\partial x^{2}}\right|_{x=\pm\left(\frac{a}{b}\right)^{\frac{1}{2}}}&=-a+3 b \frac{a}{b}=2 a \text { (Positive so it is stable point) }\\
		\Rightarrow \omega&=\sqrt{\frac{\frac{\partial^{2} V}{\partial x^{2}}}{m}}=\sqrt{\frac{2 a}{m}}
		\end{align*}
		The correct option is \textbf{(b)}
	\end{answer}
		\item A particle of mass $m$, kept in potential $V(x)=-\frac{1}{2} k x^{2}+\frac{1}{4} \lambda x^{4}$ (where $k$ and $\lambda$ are positive constants), undergoes small oscillations about an equilibrium point. The frequency of oscillations is
		{\exyear{NET JUNE 2018}}
	\begin{tasks}(2)
		\task[\textbf{A.}] $\frac{1}{2 \pi} \sqrt{\frac{2 \lambda}{m}}$
		\task[\textbf{B.}]$\frac{1}{2 \pi} \sqrt{\frac{k}{m}}$
		\task[\textbf{C.}]$\frac{1}{2 \pi} \sqrt{\frac{2 k}{m}}$
		\task[\textbf{D.}]$\frac{1}{2 \pi} \sqrt{\frac{\lambda}{m}}$
	\end{tasks}
	\begin{answer}
		\begin{align*}
		V&=-\frac{1}{2} k x^{2}+\frac{1}{4} \lambda x^{4}\\
		\frac{d V}{d x}&=0 \quad-k x+\lambda x^{3}=0\\
		x=0, \quad x^{2}&=\frac{k}{\lambda} \Rightarrow x=x_{0}=\sqrt{\frac{k}{\lambda}}\\
		\frac{d^{2} V}{d x^{2}}&=-k \quad \text { at } x=0 \quad \text { so } x=0 \text { is unstable part }\\
		\frac{d^{2} V}{d x^{2}}&=2 k \text { at } x_{0}=\sqrt{\frac{k}{\lambda}} \text { so } x_{0}=\sqrt{\frac{k}{x}} \text { is stable equation point }\\
		\omega&=\sqrt{\frac{\left.\frac{d^{2} V}{d x^{2}}\right|_{x=x_{0}}}{m}}=\sqrt{\frac{2 k}{m}} \quad f=\frac{1}{2 \pi} \sqrt{\frac{2 k}{m}}
		\end{align*}
		The correct option is \textbf{(c)}
	\end{answer}
\end{enumerate}
\newpage
\begin{abox}
Practice set 2 solutions
\end{abox}
\begin{enumerate}
  \item A particle is placed in a region with the potential $V(x)=\frac{1}{2} k x^{2}-\frac{\lambda}{3} x^{3}$, where $k, \lambda>0$.
  Then,
	{\exyear{GATE 2010}}
\begin{tasks}(1)
	\task[\textbf{A.}] $x=0$ and $x=\frac{k}{\lambda}$ are points of stable equilibrium
	\task[\textbf{B.}]$x=0$ is a point of stable equilibrium and $x=\frac{k}{\lambda}$ is a point of unstable equilibrium
	\task[\textbf{C.}]$x=0$ and $x=\frac{k}{\lambda}$ are points of unstable equilibrium
	\task[\textbf{D.}]There are no points of stable or unstable equilibrium
\end{tasks}
\begin{answer}
\begin{align*}
V&=\frac{1}{2} k x^{2}-\frac{\lambda x^{3}}{3} \Rightarrow \frac{\partial V}{\partial x}=k x-\lambda x^{2}=0 \Rightarrow x=0, x=\frac{k}{\lambda}\\
&=\frac{\partial^{2} V}{\partial x^{2}}=k-2 \lambda x\\
A t x&=0, \frac{\partial^{2} V}{\partial x^{2}}=+v e(\text { Stable })\\
\text { at } x&=\frac{k}{\lambda}, \frac{\partial^{2} V}{\partial x^{2}}=-v e \text { (unstable) }
\end{align*}
The correct option is \textbf{(b)}	
\end{answer}
	\item Two bodies of mass $m$ and $2 m$ are connected by a spring constant $k$. The frequency of the normal mode is
	{\exyear{GATE 2011}}
\begin{tasks}(2)
	\task[\textbf{A.}] $\sqrt{3 k / 2 m}$
	\task[\textbf{B.}]$\sqrt{k / m}$
	\task[\textbf{C.}] $\sqrt{2 k / 3 m}$
	\task[\textbf{D.}]$\sqrt{k / 2 m}$
\end{tasks}
\begin{answer}
	\begin{align*}
\omega=\sqrt{\frac{k}{\mu}}&=\sqrt{\frac{k}{\frac{2 m}{3}}}=\sqrt{\frac{3 k}{2 m}}\\
\text { where reduce mass } \mu&=\frac{2 m m}{2 m+m}=\frac{2 m}{3} \text {. }
	\end{align*}
The correct option is \textbf{(a)}
\end{answer}
	\item A particle of unit mass moves along the $x$-axis under the influence of a potential, $V(x)=x(x-2)^{2}$. The particle is found to be in stable equilibrium at the point $x=2$. The time period of oscillation of the particle is
{	\exyear{GATE 2012}}
\begin{tasks}(2)
	\task[\textbf{A.}] $\frac{\pi}{2}$
	\task[\textbf{B.}]$\pi$
	\task[\textbf{C.}]$\frac{3 \pi}{2}$
	\task[\textbf{D.}]$2 \pi$
\end{tasks}
\begin{answer}
\begin{align*}
V(x)&=x(x-2)^{2} \Rightarrow \frac{\partial V}{\partial x}=(x-2)^{2}+2 x(x-2)=0 \Rightarrow x=2, x=\frac{2}{3}\\
\frac{\partial^{2} V}{\partial x^{2}}&=2(x-2)+2(x-2)+\left.2 x \Rightarrow \frac{\partial^{2} V}{\partial x^{2}}\right|_{x=2}=2 \times 2=4\\
\omega&=\sqrt{\left.\frac{\partial^{2} V}{\partial x^{2}}\right|_{x=2}} \Rightarrow \omega=\frac{2 \pi}{T}=2 \Rightarrow T=\pi
\end{align*}
The correct option is \textbf{(b)}	
\end{answer}
	\item Consider two small blocks, each of mass $M$, attached to two identical springs. One of the springs is attached to the wall, as shown in the figure. The spring constant of each spring is $k$. The masses slide along the surface and the friction is negligible. The frequency of one of the normal modes of the system is,
	{\exyear{GATE 2013}}
	\begin{figure}[H]
		\centering
		\includegraphics[height=3cm,width=5cm]{GATE1}
	\end{figure}
\begin{tasks}(2)
	\task[\textbf{A.}] $\sqrt{\frac{3+\sqrt{2}}{2}} \sqrt{\frac{k}{M}}$
	\task[\textbf{B.}]$\sqrt{\frac{3+\sqrt{3}}{2} \sqrt{\frac{k}{M}}}$
	\task[\textbf{C.}]$\sqrt{\frac{3+\sqrt{5}}{2}} \sqrt{\frac{k}{M}}$
	\task[\textbf{D.}]$\sqrt{\frac{3+\sqrt{6}}{2}} \sqrt{\frac{k}{M}}$
\end{tasks}
\begin{answer}
	\begin{align*}
		T&=\frac{1}{2} m \dot{x}_{1}^{2}+\frac{1}{2} m \dot{x}_{2}^{2}\\
		V&=\frac{1}{2} k x_{1}^{2}+\frac{1}{2} k\left(x_{2}-x_{1}\right)^{2}\\
		&=\frac{1}{2} k x_{1}^{2}+\frac{1}{2} k\left(x_{2}^{2}+x_{1}^{2}-2 x_{2} x_{1}\right)=\frac{1}{2} k\left(2 x_{1}^{2}+x_{2}^{2}-2 x_{2} x_{1}\right)\\
		T&=\left(\begin{array}{cc}
		m & 0 \\
		0 & m
		\end{array}\right) ; \quad V=\left(\begin{array}{cc}
		2 k & -k \\
		-k & k
		\end{array}\right)\\
		\left|\begin{array}{cc}
		2 k-\omega^{2} m & -k \\
		-k & k-\omega^{2} m
		\end{array}\right|&=0 \Rightarrow\left(2 k-\omega^{2} m\right)\left(k-\omega^{2} m\right)-k^{2}=0 \Rightarrow \omega=\sqrt{\frac{3+\sqrt{5}}{2}} \sqrt{\frac{k}{m}}
	\end{align*}
THe correct option is \textbf{(c)}
\end{answer}
	\item Two masses $m$ and $3 m$ are attached to the two ends of a massless spring with force constant $K$. If $m=100 \mathrm{~g}$ and $K=0.3 \mathrm{~N} / \mathrm{m}$, then the natural angular frequency of oscillation is $H z$.
	{\exyear{GATE 2014}}

\begin{answer}
	\begin{align*}
	f&=\frac{1}{2 \pi} \sqrt{\frac{k}{\mu}}\\
	\mu&=\frac{m_{1} \cdot m_{2}}{m_{1}+m_{2}}=\frac{3 m \cdot m}{4 m}=\frac{3 m}{4}\\
	\omega&=\sqrt{\frac{4 k}{3 m}}=2 \Rightarrow f=0.318 \mathrm{~Hz}
	\end{align*}
\end{answer}

	\item A particle of mass $m$ is in a potential given by
	$$
	V(r)=-\frac{a}{r}+\frac{a r_{0}^{2}}{3 r^{3}}
	$$
	where $a$ and $r_{0}$ are positive constants. When disturbed slightly from its stable equilibrium position it undergoes a simple harmonic oscillation. The time period of oscillation is
	{\exyear{GATE 2014}}
\begin{tasks}(2)
	\task[\textbf{A.}] $2 \pi \sqrt{\frac{m r_{0}^{3}}{2 a}}$
	\task[\textbf{B.}]$2 \pi \sqrt{\frac{m r_{0}{ }^{3}}{a}}$
	\task[\textbf{C.}]$2 \pi \sqrt{\frac{2 m r_{0}^{3}}{a}}$
	\task[\textbf{D.}]$4 \pi \sqrt{\frac{m r_{0}^{3}}{a}}$
\end{tasks}
\begin{answer}
\begin{align*}
V(r)&=-\frac{a}{r}+\frac{a r_{0}^{2}}{3 r^{3}}\\
\intertext{For equilibrium}
\frac{\partial V}{\partial r}&=\frac{a}{r^{2}}-\frac{3 a r_{0}^{2}}{3 r^{4}}=0, \quad r=\pm r_{0}\\
\frac{\partial^{2} V}{\partial r^{2}}&=-\frac{2 a}{r^{3}}+\left.\frac{4 a r_{0}^{2}}{r^{5}}\right|_{r_{0}}=-\frac{2 a}{r_{0}^{3}}+\frac{4 a r_{0}^{2}}{r_{0}^{5}}=\frac{2 a}{r_{0}^{3}}\\
\omega&=\sqrt{\frac{\left.\frac{\partial^{2} V}{\partial r^{2}}\right|_{r_{0}}}{m}} \Rightarrow T=2 \pi \sqrt{\frac{m r_{0}^{3}}{2 a}}
\end{align*}
The correct option id \textbf{(a)}
\end{answer}
	\item Two identical masses of $10 \mathrm{gm}$ each are connected by a massless spring of spring constant $1 \mathrm{~N} / \mathrm{m}$. The non-zero angular eigenfrequency of the system is. $. \mathrm{rad} / \mathrm{s} .$ (up to two decimal places)
	{\exyear{GATE 2017}}
\begin{answer}
\begin{align*}
\omega=\sqrt{\frac{k}{\mu}}, \quad \text { where } \mu=\frac{m}{2}=\frac{10}{2 \times 1000}=\frac{1}{200} \text { and } k=1 N / m, \quad \omega=14.14
\end{align*}
\end{answer}
	\item In the context of small oscillations, which one of the following does NOT apply to the normal coordinates?
{	\exyear{GATE 2018}}
\begin{tasks}(1)
	\task[\textbf{A.}] Each normal coordinate has an eigen-frequency associated with it
	\task[\textbf{B.}]The normal coordinates are orthogonal to one another
	\task[\textbf{C.}]The normal coordinates are all independent
	\task[\textbf{D.}]The potential energy of the system is a sum of squares of the normal coordinates with constant coefficients
\end{tasks}
\begin{answer}
 Normal co-ordinate must be independent. It is not necessary that it should orthogonal.\\
 The correct option is \textbf{(b)}	
\end{answer}

\end{enumerate}