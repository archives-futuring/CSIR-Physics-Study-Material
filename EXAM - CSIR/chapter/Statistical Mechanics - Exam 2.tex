\begin{abox}
	Statistical Mechanics
\end{abox}
\begin{enumerate}
	\item Consider a system of $N$ non-interacting spins, each of which has classical magnetic moment of magnitude $\mu$. The Hamiltonian of this system in an external magnetic field $\vec{H}$ is $\sum_{i=1}^{N} \vec{\mu}_{i} \cdot \vec{H}$, where $\vec{\mu}_{i}$ is the magnetic moment of the $i^{\text {th }}$ spin. The magnetization per spin at temperature $T$ is
\begin{tasks}(2)
\task[\textbf{A.}] $\frac{\mu^{2} H}{k_{B} T}$
\task[\textbf{B.}] $\mu\left[\operatorname{coth}\left(\frac{\mu H}{k_{B} T}\right)-\frac{k_{B} T}{\mu H}\right]$
\task[\textbf{C.}]  $\mu \sinh \left(\frac{\mu H}{k_{B} T}\right)$
\task[\textbf{D.}]  $\mu \tanh \left(\frac{\mu H}{k_{B} T}\right)$
\end{tasks}
\begin{answer}
\begin{align*}
\text{	For classical limit }M&=\frac{\int_{0}^{2 \pi} \int_{0}^{\pi} \mu \cos \theta \exp \frac{\mu H \cos \theta}{k T} \sin \theta d \theta d \phi}{\iint \exp \frac{\mu H \cos \theta}{k_{B} T} \sin \theta d \theta d \phi}\\
M&=\mu\left[\operatorname{coth}\left(\frac{\mu H}{k_{B} T}\right)-\frac{k_{B} T}{\mu H}\right]
\end{align*}
So the correct answer is \textbf{Option (B)}
\end{answer}
	\item  Consider a one-dimensional Ising model with $N$ spins, at very low temperatures when almost all spins are aligned parallel to each other. There will be a few spin flips with each flip costing an energy $2 J .$ In a configuration with $r$ spin flips, the energy of the system is $E=-N J+2 r J$ and the number of configuration is ${ }^{N} C_{r} ; r$ varies from 0 to $N$. The partition function is
\begin{tasks}(4)
\task[\textbf{A.}] $\left(\frac{J}{k_{B} T}\right)^{N}$
\task[\textbf{B.}]  $e^{-N J / k_{B} T}$
\task[\textbf{C.}] $\left(\sinh \frac{J}{k_{B} T}\right)^{N}$
\task[\textbf{D.}] $\left(\cosh \frac{J}{k_{B} T}\right)^{N}$
\end{tasks}
\begin{answer}
\begin{align*}
\intertext{ Let us consider only three energy levels, $E_{r}=-2 J+2 r J$ i.e. $E_{0}=-2 J, E_{1}=0$ and $E_{2}=2 J$ }
Q_{2}&=\frac{\left({ }^{2} C_{0} e^{-\beta E_{0}}+{ }^{2} C_{1} e^{-\beta E_{1}}+{ }^{2} C_{2} e^{-\beta E_{2}}\right)}{\sum_{r=0}^{2}{ }^{2} C_{r}}\\&=\frac{\left(e^{\beta 2 J}+2 e^{0}+e^{\beta 2 J}\right)}{4}=\frac{\left(e^{\beta J}+e^{\beta J}\right)^{2}}{4}\\
Q_{2}&=\left(\frac{e^{\beta J}+e^{\beta J}}{2}\right)^{2}=(\cosh \beta J)^{2} \Rightarrow(\cosh \beta J)^{2} \Rightarrow Q_{N}\\&=(\cosh \beta J)^{N}
\end{align*}
So the correct answer is \textbf{Option (D)}
\end{answer}
		Common Data for Questions 3 and 4: There are four energy levels $E, 2 E, 3 E$ and $4 E$ (where $E>0$ ). The canonical partition function of two particles is, if these particles are
	\item Two identical fermions
\begin{tasks}(1)
\task[\textbf{A.}] $e^{-2 \beta E}+e^{-4 \beta E}+e^{-6 \beta E}+e^{-8 \beta E}$
\task[\textbf{B.}] $e^{-3 \beta E}+e^{-4 \beta E}+2e^{-5 \beta E}+e^{-6 \beta E}+e^{-7 \beta E}$
\task[\textbf{C.}] $\left(e^{-\beta E}+e^{-2 \beta E}+e^{-3 \beta E}+e^{-4 \beta E}\right)^{2}$
\task[\textbf{D.}] $e^{-2 \beta E}-e^{-4 \beta E}+e^{-6 \beta E}-e^{-8 \beta E}$
\end{tasks}
\begin{answer}
\begin{align*}
\intertext{The possible value of Energy for two Fermions}
E_{1}&=3 E, E_{2}=4 E, E_{3}=5 E, E_{4}=6 E, E_{5}=7 E
\intertext{The partition function is $Z=e^{-3 \beta E}+e^{-4 \beta E}+2 e^{-5 \beta E}+e^{-6 \beta E}+e^{-7 \beta E}$, then the answer is option (B).}
\end{align*}
So the correct answer is \textbf{Option (B)}
\end{answer}
	\item Two distinguishable particles
\begin{tasks}(1)
\task[\textbf{A.}] $e^{-2 \beta E}+e^{-4 \beta E}+e^{-6 \beta E}+e^{-8 \beta E}$
\task[\textbf{B.}] $e^{-3 \beta E}+e^{-4 \beta E}+e^{-5 \beta E}+e^{-6 \beta E}+e^{-7 \beta E}$
\task[\textbf{C.}] $\left(e^{-\beta E}+e^{-2 \beta E}+e^{-3 \beta E}+e^{-4 \beta E}\right)^{2}$
\task[\textbf{D.}] $e^{-2 \beta E}-e^{-4 \beta E}+e^{-6 \beta E}-e^{-8 \beta E}$
\end{tasks}
\begin{answer}
\begin{align*}
\intertext{ When two particles are distinguishable then minimum value of Energy is $2 E$ and maximum value is $8 E$.
	So from checking all four options $\left(Z=e^{-\beta E}+e^{-2 \beta E}+e^{-3 \beta E}+e^{-4 \beta E}\right)^{2}$}
\end{align*}
So the correct answer is \textbf{Option (C)}
\end{answer}
	\item  A classical gas of molecules, each of mass $m$, is in thermal equilibrium at the absolute temperature $T .$ The velocity components of the molecules along the Cartesian axes are $v_{x}, v_{y}$ and $v_{z} .$ The mean value of $\left(v_{x}+v_{y}\right)^{2}$ is
\begin{tasks}(4)
\task[\textbf{A.}] $\frac{k_{B} T}{m}$
\task[\textbf{B.}] $\frac{3}{2} \frac{k_{B} T}{m}$
\task[\textbf{C.}] $\frac{1}{2} \frac{k_{B} T}{m}$
\task[\textbf{D.}] $\frac{2 k_{B} T}{m}$
\end{tasks}
\begin{answer}
\begin{align*}
\left\langle\left(V_{x}+V_{y}\right)^{2}\right\rangle&=\left\langle v_{x}^{2}\right\rangle+\left\langle v_{y}^{2}\right\rangle+2\left\langle v_{x} \cdot v_{y}\right\rangle\\&=\left\langle v_{x}^{2}\right\rangle+\left\langle v_{y}^{2}\right\rangle+2\left\langle v_{x}\right\rangle \cdot\left\langle v_{y}\right\rangle=\frac{2 \mathrm{k}_{\mathrm{B}} \mathrm{T}}{\mathrm{m}}\\
\because\left\langle v_{x}\right\rangle&=\left\langle v_{y}\right\rangle=0\text{ and} \left\langle V_{x}^{2}\right\rangle+\left\langle V_{y}^{2}\right\rangle=\frac{2 k_{B} T}{m}
\end{align*}
So the correct answer is \textbf{Option (D)}
\end{answer}
	\item For a free electron gas in two dimensions the variations of the density of states. $N(E)$ as a function of energy $E$, is best represented by
\begin{tasks}(2)
\task[\textbf{A.}] \begin{figure}[H]
	\centering
	\includegraphics[height=3.5cm,width=4.5cm]{SM-01}
\end{figure}
\task[\textbf{B.}] \begin{figure}[H]
	\centering
	\includegraphics[height=3.5cm,width=4.5cm]{SM-02}
\end{figure}
\task[\textbf{C.}]\begin{figure}[H]
	\centering
	\includegraphics[height=3.5cm,width=4.5cm]{SM-03}
\end{figure} 
\task[\textbf{D.}] \begin{figure}[H]
	\centering
	\includegraphics[height=3.5cm,width=4.5cm]{SM-04}
\end{figure}
\end{tasks}
\begin{answer}
\begin{align*}
N(E) \propto E^{0}
\end{align*}
So the correct answer is \textbf{Option (C)}
\end{answer}
	\item For a black body radiation in a cavity, photons are created and annihilated freely as a result of emission and absorption by the walls of the cavity. This is because
\begin{tasks}(1)
\task[\textbf{A.}] The chemical potential of the photons is zero
\task[\textbf{B.}] Photons obey Pauli exclusion principle
\task[\textbf{C.}] Photons are spin-1 particles
\task[\textbf{D.}] The entropy of the photons is very large
\end{tasks}
\begin{answer}
\begin{align*}
\text{The chemical potential of photon is zero}
\end{align*}
So the correct answer is \textbf{Option (A)}
\end{answer}
	\item Consider a system of $N$ non-interacting spin $-\frac{1}{2}$ particles, each having a magnetic moment $\mu$, is in a magnetic field $\vec{B}=B \hat{z} .$ If $E$ is the total energy of the system, then number of accessible microstates $\Omega$ is given by
\begin{tasks}(2)
\task[\textbf{A.}] $\Omega=\frac{N !}{\frac{1}{2}\left(N-\frac{E}{\mu B}\right) ! \frac{1}{2}\left(N+\frac{E}{\mu B}\right) !}$
\task[\textbf{B.}] $\Omega=\frac{\left(N-\frac{E}{\mu B}\right) !}{\left(N+\frac{E}{\mu B}\right) !}$
\task[\textbf{C.}] $\Omega=\frac{1}{2}\left(N-\frac{E}{\mu B}\right) ! \frac{1}{2}\left(N+\frac{E}{\mu B}\right) !$
\task[\textbf{D.}] $\Omega=\frac{N !}{\left(N+\frac{E}{\mu B}\right) !}$
\end{tasks}
\begin{answer}
\begin{align*}
\intertext{Number of microstate is ${ }^{N} C_{n_{1}}$, where $n_{1}$ is number of particle in $+\frac{1}{2}$ state and}
n_{2}&=\left(N-n_{1}\right)\text{ is number of state in }-\frac{1}{2}\text{ state.}\\
\text{	where }n_{1}&=\frac{1}{2}\left(N-\frac{E}{\mu B}\right), n_{2}=\frac{1}{2}\left(N+\frac{E}{\mu B}\right)\\
\text{So, number of microstate}=&\frac{N !}{\frac{1}{2}\left(N-\frac{E}{\mu B}\right) ! \frac{1}{2}\left(N+\frac{E}{\mu B}\right) !}
\end{align*}
So the correct answer is \textbf{Option (A)}
\end{answer}
	
\item 	Consider a system of two particles $A$ and $B$. Each particle can occupy one of three possible quantum states $|1\rangle,|2\rangle$ and $|3\rangle$. The ratio of the probability that the two particles
are in the same state to the probability that the two particles are in different states is calculated for bosons and classical (Maxwell-Boltzmann) particles. They are respectively
\begin{tasks}(4)
\task[\textbf{A.}] 1,0
\task[\textbf{B.}]  $\frac{1}{2}, 1$
\task[\textbf{C.}] $1, \frac{1}{2}$
\task[\textbf{D.}] $0, \frac{1}{2}$
\end{tasks}
\begin{answer}
For two particle in same state:\\
\begin{figure}[H]
	\centering
	\includegraphics[height=3cm,width=8cm]{CM-20}
\end{figure}
\begin{align*}
\text{Probability ratio: }&\frac{1 / 3}{1 / 3}=1
\end{align*}
For two particle in different state;
\begin{figure}[H]
	\centering
	\includegraphics[height=3.5cm,width=11cm]{CM-21}
\end{figure}
\begin{align*}
\text{Probability ratio: }&\frac{1 / 3}{2 / 3}=\frac{1}{2}
\end{align*}
So the correct answer is \textbf{Option (C)}
\end{answer}
	\item Consider three situations of 4 particles in one dimensional box of width $L$ with hard walls. In case (i), the particles are fermions, in case (ii) they are bosons, and in case (iii) they are classical. If the total ground state energy of the four particles in these three cases are $E_{F}, E_{B}$ and $E_{c l}$ respectively, which of the following is true?
\begin{tasks}(2)
\task[\textbf{A.}] $E_{F}=E_{B}=E_{c l}$
\task[\textbf{B.}] $E_{F}>E_{B}=E_{c l}$
\task[\textbf{C.}] $E_{F}<E_{B}<E_{c l}$
\task[\textbf{D.}] $E_{F}>E_{B}>E_{c l}$
\end{tasks}
\begin{answer}
\begin{align*}
\intertext{For fermions, in 1-D box of width $L$, the ground state energy for single particle is written as,}
\frac{\pi^{2} \hbar^{2}}{2 m l^{2}}&=\epsilon_{0}\\
\Rightarrow 1 \times \in_{0}+1 \times 4 \in_{0}+1 \times 9 \in_{0}+1 \times 16 \in_{0}&=30 \in_{0}\\
\text{ For Boson }&=4 \times \epsilon_{0},\\\text{ For Maxwell }&=4 \times \epsilon_{0}\\
E_{F}>E_{B}&=E_{c l}
\end{align*}
So the correct answer is \textbf{Option (B)}
\end{answer}
	\item A monoatomic gas consists of atoms with two internal energy levels, ground state $E_{0}=0$ and an excited state $E_{1}=E$. The specific heat of the gas is given by
\begin{tasks}(2)
\task[\textbf{A.}] $\frac{3}{2} k$
\task[\textbf{B.}] $\frac{E^{2} e^{E / k T}}{k T^{2}\left(1+e^{E / k T}\right)^{2}}$
\task[\textbf{C.}] $\frac{3}{2} k+\frac{E^{2} e^{E / k T}}{k T^{2}\left(1+e^{E / k T}\right)^{2}}$
\task[\textbf{D.}] $\frac{3}{2} k-\frac{E^{2} e^{E / k T}}{k T^{2}\left(1+e^{E / k T}\right)^{2}}$
\end{tasks}
\begin{answer}
\begin{align*}
E_{0}&=0, \quad E_{1}=E\\
\text{Then partition function is}\\
z&=\sum e^{-\beta E_{i}} \Rightarrow z\\&=e^{-\beta \times 0}+e^{-\beta E} \Rightarrow \ln z\\&=\ln \left(1+e^{-\beta E_{1}}\right)\\
U&=\langle E\rangle=\frac{-\partial}{\partial \beta} \ln z\\&=-\frac{\partial}{\partial \beta} \ln \left(1+e^{-\beta E}\right)\\&=-\frac{1}{\left(1+e^{-\beta E}\right)}(-E) e^{-\beta E}\\&=\frac{E e^{-\beta E}}{1+e^{-\beta E}} \quad\left[\because \beta=k_{B} T\right]\\
\left(\frac{\partial U}{\partial T}\right)_{v}=C_{V}&=\frac{\left(1+e^{-\frac{E}{k_{B} T}}\right) E . e^{-\frac{E}{k_{B} T}} \cdot\left(\frac{E}{k_{B} T^{2}}\right)-E e^{-\frac{E}{k_{B} T}} \cdot e^{-\frac{E}{k_{B} T}}\left(\frac{E}{k_{B} T^{2}}\right)}{\left(1+e^{-\frac{E}{k_{B} T}}\right)^{2}}\\
C_{V}&=\frac{\frac{E^{2}}{k_{B} T^{2}} e^{-\frac{E}{k_{\mathrm{B}} T}}+\frac{E^{2}}{k_{B} T^{2}} e^{-\frac{2 E}{k_{\mathrm{B}} T}}-\frac{E^{2}}{k_{B} T^{2}} e^{-\frac{2 E}{k_{\mathrm{B}} T}}}{\left(1+e^{-\frac{E}{k_{\mathrm{B}} T}}\right)^{2}}\\&=\frac{E^{2} e^{-\frac{E}{k_{\mathrm{B}} T}}}{k_{B} T^{2}\left(1+e^{-\frac{E}{k_{\mathrm{B}} T}}\right)^{2}}\\&=\frac{E^{2} e^{\frac{E}{k_{\mathrm{B}} T}}}{k_{B} T^{2}\left(1+e^{\frac{E}{k_{\mathrm{B}} T}}\right)^{2}}\\
\text{If gas will classically allowed, then }C_{V}&=\frac{3}{2} k_{B}\\
\text{	and quantum mechanically, }C_{V}&=\frac{E^{2} e^{\frac{E}{k_{B} T}}}{k_{B} T^{2}\left(1+e^{\frac{E}{k_{B} T}}\right)^{2}}\\
\therefore \quad C_{V}&=\frac{3}{2} k_{B}+\frac{E^{2} e^{E / k T}}{k T^{2}\left(1+e^{E / k T}\right)^{2}}
\end{align*}
So the correct answer is \textbf{Option (C)}
\end{answer}
\item 	 Which of the following atoms cannot exhibit Bose-Einstein condensation, even in principle?
{\exyear{GATE 2010}}
\begin{tasks}(4)
\task[\textbf{A.}]  ${ }^{1} \mathrm{H}_{1}$
\task[\textbf{B.}] ${ }^{4} \mathrm{H}_{2}$
\task[\textbf{C.}] ${ }^{23} \mathrm{Na}_{11}$
\task[\textbf{D.}] ${ }^{30} \mathrm{~K}_{19}$
\end{tasks}
\begin{answer}
For Bose-Einstein condensation:\\
Number of electron + number of proton + number of neutron = Even\\
For ${ }^{30} \mathrm{~K}_{19}$\\
Number of proton $=19$, Number of electron $=19$, Number of neutron $=11$\\
$19+19+11=49$ this is odd. So it will not exhibit Bose-Einstein condensation.\\\\
So the correct answer is \textbf{Option (D)}
\end{answer}
	
	\item Consider a system of 3 fermions which can occupy any of the 4 available energy states with equal probability. The entropy of the system is
{	\exyear{GATE 2014}}
\begin{tasks}(4)
\task[\textbf{A.}] $k_{B} \ln 2$
\task[\textbf{B.}] $2 k_{B} \ln 2$
\task[\textbf{C.}]  $2 k_{B} \ln 4$
\task[\textbf{D.}]  $3 k_{B} \ln 4$
\end{tasks}
\begin{answer}
\begin{align*}
\intertext{Solution: Number of ways that 3 fermions will adjust in 4 available energy is ${ }^{4} C_{3}=4$ so entropy is $k_{B} \ln 4=2 k_{B} \ln 2$}
\end{align*}
So the correct answer is \textbf{Option (B)}
\end{answer}
	\item The Hamiltonian of a classical nonlinear one dimensional oscillator is $H=\frac{1}{2 m} p^{2}+\lambda x^{4}$, where $\lambda>0$ is a constant. The specific heat of a collection of a collection of $N$ independent such oscillators is
\begin{tasks}(4)
	\task[\textbf{A.}] $\frac{3 N k_{B}}{2}$
	\task[\textbf{B.}] $\frac{3 N k_{B}}{4}$
	\task[\textbf{C.}] $N k_{B}$
	\task[\textbf{D.}] $\frac{N k_{B}}{2}$
\end{tasks}
\begin{answer}
	\begin{align*}
	H&=\frac{p^{2}}{2 m}+\lambda x^{4}, \quad \lambda>0\\
	\langle H\rangle&=\left\langle\frac{p^{2}}{2 m}\right\rangle+\langle V\rangle=\frac{1}{2} k_{B} T+2 \lambda \frac{\int_{0}^{\infty} x^{4} e^{-\beta x x^{4}} d x}{2 \int_{0}^{\infty} e^{-\beta x^{4}} d x}\\&=\frac{1}{2} k_{B} T+2 \lambda \frac{\frac{5 / 4}{4(\lambda \beta)^{5 / 4}}}{2 \frac{\sqrt{5 / 4}}{(\lambda \beta)^{1 / 4}}}\\
	\Rightarrow\langle H\rangle&=\frac{1}{2} k_{B} T+\lambda \frac{(\lambda \beta)^{1 / 4}}{4(\lambda \beta)^{5 / 4}}=\frac{1}{2} k_{B} T+\frac{\lambda}{4} \frac{1}{\lambda \beta}=\frac{1}{2} k_{B} T+\frac{k_{B} T}{4}\\&=\frac{3}{4} k_{B} T=\frac{3}{4} k_{B} T\\
	\Rightarrow C_{V}&=\frac{3}{4} N k_{B}
	\end{align*}
	So the correct answer is \textbf{Option (B)}
\end{answer}	
	\item  The rotational energy levels of a molecule are $E_{\ell}=\frac{\hbar^{2}}{2 I_{0}} \ell(\ell+1)$, where $\ell=0,1,2, \ldots$ and $I_{0}$ is its moment of inertia. The contribution of the rotational motion to the Helmholtz free energy per molecule, at low temperatures in a dilute gas of these molecules, is approximately
\begin{tasks}(2)
	\task[\textbf{A.}]  $-k_{B} T\left(1+\frac{\hbar^{2}}{I_{0} k_{B} T}\right)$
	\task[\textbf{B.}] $-k_{B} T e^{\frac{\hbar^{2}}{I_{0} k_{B} T}}$
	\task[\textbf{C.}] $-k_{B} T$
	\task[\textbf{D.}] $-3 k_{B} T e^{-\frac{\hbar^{2}}{I_{0} k_{B} T}}$
\end{tasks}
\begin{answer}
	\begin{align*}
	E_{\ell}&=\frac{\hbar^{2}}{2 I_{0}} \ell(\ell+1) \quad \ell=0,1,2, \ldots\\
	z&=\sum_{\ell=0}^{\infty}(2 \ell+1) e^{\frac{-\beta \hbar^{2} \ell(\ell+1)}{2 I_{0}}}\\
	z&=1+\sum_{\ell=0}^{\infty}(2 \ell+1) e^{\frac{-\hbar^{2} \ell(\ell+1)}{2 I_{0} k_{B} T}}\\
	F&=-k_{B} T \ln z=-k_{B} T \ln \left(1+\sum_{\ell=1}^{\infty}(2 \ell+1) e^{\frac{-\hbar^{2} \ell(\ell+1)}{2 I_{0} k_{B} T}}\right)\\
	\ln (1+x)&=x-\frac{x^{2}}{2}+\ldots
	\intertext{For low temperature, higher temperature can be neglected}
	F&=-k_{B} T \sum_{\ell=1}^{\infty}(2 \ell+1) e^{-\frac{-\hbar^{2} \ell(\ell+1)}{2 I_{0} k_{B} T}}\\&=-k_{B} T\left[3 e^{\frac{-\hbar^{2}}{I_{0} k_{B} T}}+\ldots\right]=-3 k_{B} T e^{-\frac{\hbar^{2}}{I_{0} k_{B} T}}
	\end{align*}
	So the correct answer is \textbf{Option (D)}
\end{answer}	
	
\item The number of ways of distributing 11 indistinguishable bosons in 3 different energy levels is
\begin{tasks}(4)
	\task[\textbf{A.}]  $3^{11}$
	\task[\textbf{B.}] $11^{3}$
	\task[\textbf{C.}] $\frac{(13) !}{2 !(11) !}$
	\task[\textbf{D.}]  $\frac{(11) !}{3 ! 8 !}$
\end{tasks}
\begin{answer}
	\begin{align*}
	n&=11 \quad g=3
	\end{align*}
	So the correct answer is \textbf{Option (C)}
\end{answer}	
	
	\item Partition function for a gas of photons is given as, $\ln Z=\frac{\pi^{2} V\left(k_{B} T\right)^{3}}{45 \hbar^{3} C^{3}}$. The specific heat of the photon gas varies with temperature as
\begin{tasks}(2)
	\task[\textbf{A.}] \begin{figure}[H]
		\centering
		\includegraphics[height=3cm,width=4.5cm]{SP-2}
	\end{figure}
	\task[\textbf{B.}] \begin{figure}[H]
		\centering
		\includegraphics[height=3cm,width=4.5cm]{SP-3}
	\end{figure}
	\task[\textbf{C.}] \begin{figure}[H]
		\centering
		\includegraphics[height=3cm,width=4.5cm]{SP-4}
	\end{figure}
	\task[\textbf{D.}] \begin{figure}[H]
		\centering
		\includegraphics[height=3cm,width=4.5cm]{SP-5}
	\end{figure}
\end{tasks}
\begin{answer}
	\begin{align*}
	\mathrm{U}=\mathrm{K}_{\mathrm{B}} \mathrm{T}^{2} \frac{\partial \ln \mathrm{z}}{\partial \mathrm{T}}, \quad \mathrm{C}_{\mathrm{v}}=\left(\frac{\partial \mathrm{U}}{\partial \mathrm{T}}\right)_{\mathrm{v}} \Rightarrow \mathrm{C}_{\mathrm{v}} \propto \mathrm{T}^{3}
	\end{align*}
	So the correct answer is \textbf{Option (A)}
\end{answer}	
\item 	For an ideal Fermi gas in three dimensions, the electron velocity $V_{F}$ at the Fermi surface is related to electron concentration $n$ as,
\begin{tasks}(4)
	\task[\textbf{A.}] $V_{F} \propto n^{2 / 3}$
	\task[\textbf{B.}]  $V_{F} \propto n$
	\task[\textbf{C.}] $V_{F} \propto n^{1 / 2}$
	\task[\textbf{D.}] $V_{F} \propto n^{1 / 3}$
\end{tasks}
\begin{answer}
	\begin{align*}
	E_{F}&=\frac{1}{2} m V_{F}^{2} \quad \because E_{F} \propto n^{2 / 3} \Rightarrow V_{F}^{2} \propto n^{2 / 3} \Rightarrow V_{F} \propto n^{1 / 3}
	\end{align*}
	So the correct answer is \textbf{Option (D)}
\end{answer}
	\item Let $N_{M B}, N_{B E}, N_{F D}$ denote the number of ways in which two particles can be distributed in two energy states according to Maxwell-Boltzmann, Bose-Einstein and Fermi-Dirac statistics respectively. Then $N_{M B}: N_{B E}: N_{F D}$ is
\begin{tasks}(4)
	\task[\textbf{A.}]  $4: 3: 1$
	\task[\textbf{B.}]  $4: 2: 3$
	\task[\textbf{C.}] $4: 3: 3$
	\task[\textbf{D.}] $4: 3: 2$
\end{tasks}
\begin{answer}
	For Maxwell, Boltzmann, $$W=\frac{\underline{N}}{\lfloor n}$$ $$g^{n}=\frac{[2}{\lfloor 2} 2^{2}=4$$
	For Boson, $$N=2, g=2, n=2$$ 
	$$ W=\frac{\lfloor n+g-1}{\lfloor\underline{n} \mid g-1}=\frac{\mid 2+2-1}{\lfloor 2 \mid 2-1}=3$$
	For Fermion, $$N=2, g=2, n=2$$
	$$ W=\frac{\mid g}{|g| g-n}=\frac{\lfloor 2}{\lfloor 2 \mid 1}=1$$
	Correct answer is \textbf{option (A)}
\end{answer}
\item In 1 - dimension, an ensemble of $N$ classical particles has energy of the form $E=\frac{P_{x}^{2}}{2 m}+\frac{1}{2} k x^{2}$. The average internal energy of the system at temperature $T$ is 
\begin{tasks}(4)
	\task[\textbf{A.}] $\frac{3}{2} N k_{B} T$
	\task[\textbf{B.}] $\frac{1}{2} N k_{B} T$ 
	\task[\textbf{C.}] $3 N k_{B} T$
	\task[\textbf{D.}] $N k_{B} T$
\end{tasks}
\begin{answer}
	Since, $E=\left(\frac{P_{x}^{2}}{2 m}+\frac{1}{2} k x^{2}\right)$\\
	Now, $\langle E\rangle=\frac{1}{2 m}\left\langle P_{x}^{2}\right\rangle+\frac{1}{2} k\left\langle x^{2}\right\rangle=\frac{k T}{2}+\frac{k T}{2}=k T$\\
	And for $N$ - classical particle, $\langle E\rangle=N k T$\\
	Correct answer is \textbf{option (D)}
\end{answer}		
\end{enumerate}