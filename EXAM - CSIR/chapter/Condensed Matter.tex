\begin{abox}
	Condensed Matter Physics -Full Test
	\end{abox}
\begin{enumerate}
	\item A substance with face-centred cubic lattice has density $6250 \mathrm{~kg} / \mathrm{m}^{3}$ and molecular weight ${6 0} \cdot 2$. Calculate the lattice constant $a$. Given Avogadro number $=6 \cdot 02 \times 10^{26}(\mathrm{~kg}-$ mole) ${ }^{-1}$.
	\begin{tasks}(2)
		\task[\textbf{a.}]3.14   \AA
		\task[\textbf{b.}]4   \AA
		\task[\textbf{c.}]5  \AA 
		\task[\textbf{d.}]2.5   \AA 
	\end{tasks}
	\begin{answer}
		We have lattice constant
		$$\begin{aligned}
		a&=\left(\frac{n \mathrm{M}}{\mathrm{N} \rho}\right)^{1 / 3}\\
		\text{Here $n=$ number  }&\text{of molecules per unit cell in f.c.c. Lattice} \\
		n&=4\\
		 \mathrm{M}&=60 \cdot 2\\
		  N&=6 \cdot 02 \times 10^{26}(\mathrm{~kg}- mole )^{-1}\\
		 \rho&=6250 \mathrm{~kg} / \mathrm{m}^{3}\\
		\therefore \quad a &=\left[\frac{4 \times 6.02}{6250 \times 6.02 \times 10^{26}}\right] \\
		&=4 \times 10^{-10} \mathrm{~m} \\
		&=4 \text{\AA}
		\end{aligned}$$
		Option(B)
	\end{answer}
 \item A beam of $X$-rays is incident on a sodium chloride crystal (lattice spacing = $\left.2 \cdot 82 \times 10^{-10} \mathrm{~m}\right)$. The first order Bragg reflection is observed at a glancing angle of $8^{\circ} 35^{\prime}$. What is the wavelength of $X$-rays ? At what angles would the second and the third order Bragg reflectings occur?
\begin{tasks}(2)
	\task[\textbf{a.}]0.84   \AA
	\task[\textbf{b.}]1   \AA
	\task[\textbf{c.}]3.2  \AA 
	\task[\textbf{d.}]2.5   \AA 
\end{tasks}
\begin{answer}
According to Bragg's law
$$
\begin{aligned}
2 d \sin \theta &=n \lambda \\
\lambda &=\frac{2 d \sin \theta}{n}
\\\text{Substituting the values,} d&=2 \cdot 82 \times 10^{-10} \mathrm{~m}\\
 \theta&=8^{\circ} 35^{\prime} \text{ and $n=1$, we get,}\\
 \lambda &=\frac{2 \times 2.82 \times 10^{-10} \times \sin 8^{\circ} 35}{1} \\
 &=2 \times 2.82 \times 10^{-10} \times 0.1492 \\
 &=0 \cdot 842 \times 10^{-10} \mathrm{~m} \\
 &=0 \cdot 842 \text{\AA}
\end{aligned}
$$
Option(A)
\end{answer}
\item The potential energy function for the force between two atoms in diatomic molecule can be expressed as follows :
$$
U(x)=\frac{a}{x^{12}}-\frac{b}{x^{6}}
$$
Here $a$ and $b$ are positive constants and $x$ is the distance between atoms. At what value of $x$ is $\mathrm{U}(x)$ a minimum.
\begin{tasks}(2)
	\task[\textbf{a.}]$(\frac{2 a}{b})^{1 / 2}$  
	\task[\textbf{b.}]$(\frac{ a}{b})^{1 / 2}$  
	\task[\textbf{c.}] $(\frac{ a}{b})^{1 / 6}$  
	\task[\textbf{d.}] $(\frac{2 a}{b})^{1 / 6}$  
\end{tasks}
\begin{answer}
	$$\begin{aligned}
		\mathrm{U}(x)&=\frac{a}{x^{12}}-\frac{b}{x^{6}}\\
		 \text{For minimum value of $ \mathrm{U}(x)$, }  \frac{d \mathrm{U}}{d x}&=0 \\
		 \text{i.e.,} \frac{d}{d x}\left\{\frac{a}{x^{12}}-\frac{b}{x^{6}}\right\}&=0\\
		 \frac{1}{x^{7}}\left\{6 b-\frac{12 a}{x^{6}}\right\}&=0\\
		 x&=(\frac{2 a}{b})^{1 / 6}
	\end{aligned}$$
	Option(D)
\end{answer}
\item Mobilities of electrons and holes in a sample of intrinsic germanium at room temperature are $0.36 \mathrm{~m}^{2} / \mathrm{V}$-sec and $0.17 \mathrm{~m}^{2} / \mathrm{V}$-sec respectively. If the electron and hole densities are each equal to $2.5 \times 10^{19} \mathrm{~m}^{-3}$, calculate the electrical conductivity and resistivity of germanium.

\begin{tasks}(2)
	\task[\textbf{a.}]$2.12 \mathrm{mho}$  
	\task[\textbf{b.}]$2.5 \mathrm{mho}$
	\task[\textbf{c.}] $3 \mathrm{mho}$
	\task[\textbf{d.}] $2.82 \mathrm{mho}$
\end{tasks}
\begin{answer}
	$$
	\begin{aligned}
	\text{The conductivity of}&\text{ an intrinsic semiconductor is given by}\\
	\sigma_{\text {int }}=q n_{i} &\left(\mu_{n}+\mu_{p}\right) \mathrm{mho} / \mathrm{metre} \\
	n &=p=n_{i} \\
	&=2.5 \times 10^{19} \mathrm{~m}^{-3} \\
	\mu_{n} &=0.36 \mathrm{~m}^{2} / \mathrm{v} . \mathrm{s} . \\
	\mu_{p} &=0.17 \mathrm{~m}^{2} / \text { v.s. }\\
	\text{Hence, electrical }&\text { conductivity,}\\
	\sigma_{\text {int }} &=1.6 \times 10^{-19} \times 2.5 \times 10^{19}(0.36+0 \cdot 17) \\
	&=2.12 \mathrm{mho} / \text { metre }
	\end{aligned}
	$$
	Option(A)
\end{answer}
 \item In an $\mathrm{N}$-type semiconductor, the Fermi-level lies $0.5 \mathrm{eV}$ below the conduction band. If the concentration of donor atoms is tripled, find the new position of the Fermi level, taking the value of $k T=0.03 \mathrm{eV}$.
 \begin{tasks}(2)
 	\task[\textbf{a.}] $1.468 \mathrm{eV}$ 
 	\task[\textbf{b.}]$0.468 \mathrm{eV}$ 
 	\task[\textbf{c.}]$3.124 \mathrm{eV}$  
 	\task[\textbf{d.}]$2.468 \mathrm{eV}$  
 \end{tasks}
 \begin{answer}
 	$$\begin{aligned}
 	\text{The donor atom density $\mathrm{N}_{d}$}&\text{ is given by}\\
 	\mathrm{N}_{d}&=\mathrm{N}_{c} e^{-\left(\mathrm{E}_{\mathrm{C}}-\mathrm{E}_{\mathrm{F}}\right) / k \mathrm{~T}}\\
 \text{	Where $\mathrm{E}_{C}$ is the energy }&\text{corresponding to the bottom of}\\
 \text{ conduction band and $\mathrm{E}_{\mathrm{F}}$ is the Fermi}&\text{ level energy. Let initially }\\
 	\mathrm{N}_{d} &=\mathrm{N}_{d_{1}} \text { and } \mathrm{E}_{\mathrm{F}}=\mathrm{E}_{\mathrm{F}_{1}} \\
 	\mathrm{~N}_{d_{1}} &=\mathrm{N}_{c} e^{-\left(\mathrm{E}_{\mathrm{C}}-\mathrm{E}_{\mathrm{F} 1}\right) / k \mathrm{~T}} --------(1)\\
 	\text{After tripling }&\text{the donor concentration}\\
 	3\mathrm{~N}_{d_{1}}&=\mathrm{N}_{c}e^{-\left(\mathrm{E}_{\mathrm{C}}-\mathrm{E}_{\mathrm{F}_{2}}\right) / k \mathrm{~T}}---------(2)\\
 	\text{Dividing (2)}&\text{ by (1)}\\
 	3&=e^{-\left[\left(\mathrm{E}_{\mathrm{C}}-\mathrm{E}_{\mathrm{F}_{2}}\right) / k \mathrm{~T}-\left(\mathrm{E}_{\mathrm{C}}-\mathrm{E}_{\mathrm{F}_{1}}\right) / k \mathrm{~T}\right]}
 	\log _{e} 3\\ &=-\frac{\left(E_{C}-E_{\mathrm{F}_{2}}\right)}{0 \cdot 03}-\frac{0 \cdot 5}{0 \cdot 03} \\
 	&=-\frac{\left(E_{C}-E_{\mathrm{F}_{2}}\right)-0.5}{0 \cdot 03}\\
 	\text{Or }
 	\mathrm{E}_{\mathrm{C}}-\mathrm{E}_{\mathrm{F}_{2}} &=0.5-0.03 \ln 3 \\
 	&=0.5-0.03 \times \ln 3 \\
 	&=0.5-0.032 \\
 	&=0.468 \mathrm{eV}
 	\end{aligned}
 	$$
 	Thus, after tripling the impurity, Fermi-level will lie $0.468 \mathrm{eV}$ below the conduction band.\\
 	Option(B)
 \end{answer}

\item What would be the electron velocity at the zone edge of a solid crystal of atomic spacing $5 \AA$ ? Can this velocity be realised in practice?
\begin{tasks}(2)
	\task[\textbf{a.}]$0 . 245 \times 10^{6} \mathrm{~m} / \mathrm{s}$  
	\task[\textbf{b.}]$0 . 712 \times 10^{6} \mathrm{~m} / \mathrm{s}$ 
	\task[\textbf{c.}] $3. 712 \times 10^{6} \mathrm{~m} / \mathrm{s}$ 
	\task[\textbf{d.}]$1 . 712 \times 10^{6} \mathrm{~m} / \mathrm{s}$  
\end{tasks}
\begin{answer}
$$
\begin{aligned}
m v&=\hbar k\\
\text{or }
v&=\frac{\hbar k}{m}\\
\text{Here}
k&=\frac{\pi}{a}\\
\text{For the first}&\text{ Brillouin zone}\\
\therefore \quad v&=\frac{h}{2 \pi} \cdot \frac{1}{m} \cdot \frac{\pi}{a}=\frac{h}{2 m a}\\
\text{Here, }
a &=5 \AA=5 \times 10^{-10} \mathrm{~m} \\
h &=6 \cdot 63 \times 10^{-34} \mathrm{~J}-\mathrm{s}, \\
m &=9.31 \times 10^{-31} \mathrm{~kg} 
\end{aligned}
$$
$$
\begin{aligned}
 \therefore \quad v&=\frac{6 \cdot 63 \times 10^{-34}}{2 \times 9.31 \times 10^{-31} \times 5 \times 10^{-10}} \mathrm{~m} / \mathrm{s} \\
& =\frac{6 \cdot 63}{9.31} \times \frac{10^{-34}}{10^{-31} \times 10^{-10} \times 10} \\
& =0 . 712 \times 10^{6} \mathrm{~m} / \mathrm{s}
\end{aligned}
$$
This velocity cannot be achieved in practice and because above $10^{5} \mathrm{~m} / \mathrm{s}$ hot electron effect sets in.
Option(B)
\end{answer}
\item In a one dimensional crystal with atomic spacing $2 \cdot 5 \AA$, calculate the free electron energy at which the first Bragg reflection occurs.
\begin{tasks}(2)
	\task[\textbf{a.}]  $11 . 32 \times 10^{-19} \mathrm{~J}$
	\task[\textbf{b.}]$9 . 44 \times 10^{-19} \mathrm{~J}$
	\task[\textbf{c.}] $8 .20\times 10^{-19} \mathrm{~J}$
	\task[\textbf{d.}] $2. 55 \times 10^{-19} \mathrm{~J}$
\end{tasks}
\begin{answer}
$$
\begin{aligned}
v &=\frac{h}{2 m a} \\
\mathrm{E} &=\frac{1}{2} m v^{2} \\
& =\frac{h^{2}}{4 m^{2} a^{2}} \times \frac{m}{2}\\&=\frac{h^{2}}{8 m a^{2}} \\
& =\frac{(6 \cdot 63)^{2} \times 10^{-68}}{8 \times 9 \cdot 31 \times 10^{-31} \times(2 \cdot 5) \times 10^{-20}} \\
& =0.0944 \times 10^{-17} \mathrm{~J} \\
& =9 . 44 \times 10^{-19} \mathrm{~J}
\end{aligned}
$$
Option(B)
\end{answer}
  \item A unit cell has the dimensions $a=b=4 $\AA, $c=8 \AA, \alpha=\beta=90^{\circ}$ and $\gamma=120^{\circ}$. The spacing between the $(210)$ planes is-
  \begin{tasks}(2)
  	\task[\textbf{a.}]  $1.54 $ \AA
  	\task[\textbf{b.}]$1 \cdot 313 $\AA
  	\task[\textbf{c.}]$2 \cdot 14 $ \AA
  	\task[\textbf{d.}] $0 \cdot 94 $ \AA
  \end{tasks}

\begin{answer}
	$$
	\begin{aligned}
	\text{For hexagonal}&\text{ Structure,  }\\
	\frac{1}{d_{h k l}^{2}}&=\frac{4}{3} \frac{h^{2}+h k+k^{2}}{a^{2}}+\frac{l^{2}}{c^{2}}\\
	d_{h k l} &=\frac{\frac{4}{3}\left(h^{2}+h k+k^{2}\right)}{a^{2}}+\frac{l^{2}}{c^{2}} \\
	&=\left[\frac{4}{3 a^{2}}\left(2^{2}+2 \times 1+1\right)\right]^{-1 / 2} \\
	&=a \times\left[\frac{4}{3} \times 7\right]^{-1 / 2} \\
	&=\frac{4 \AA}{\sqrt{28 / 3}} \\
	&=\frac{4}{3.05} \text{\AA} \\
	d_{h k l} &=1.313 \text{\AA}
	\end{aligned}
	$$
	Option(B)
\end{answer}
	\item A narrow beam of $X$ - rays with wavelength $1.5 \text{\AA}$ is reflected from an ionic crystal with an $f c c$ lattice structure with a density of $3.32 \mathrm{gcm}^{-3}$. The molecular weight is $108 \mathrm{amu}$ $\left(1 a m u=1.66 \times 10^{-24} g\right)$\\
\textbf{	A.} The lattice constant is
{\exyear{NET/JRF(JUNE-2011)}}

\begin{tasks}(4)
\task[\textbf{A.}] $6.00 \text{\AA}$
\task[\textbf{B.}]  $4.56 \text{\AA}$
\task[\textbf{C.}] $4.00 \text{\AA}$
\task[\textbf{D.}] $2.56 \text{\AA}$
\end{tasks}
\begin{answer}
$$\begin{aligned}
\text{Given }n_{\text {eff }}&=4, M=108 \mathrm{~kg}, \rho=3.32 \mathrm{gm} \mathrm{cm}^{-3}=3320 \mathrm{kgm}^{-3}\\
N_{A}&=6.023 \times 10^{+26}\text{ atoms }/ \mathrm{kmol}\\
a^{3}&=\frac{n_{e f f} \times M}{N_{A} \times \rho}=\frac{4 \times 108}{6.023 \times 10^{26} \times 3320}\\&=6.00 \times 10^{-30} \mathrm{~m}^{3}=6.00 \times 10^{-10}=6.00 A^{0}
\end{aligned}$$
So the correct answer is \textbf{Option (A)}
\end{answer}
\item If the number density of a free electron gas in three dimensions is increased eight times, its Fermi temperature will
{\exyear{NET/JRF(DEC-2011)}}

\begin{tasks}(2)
\task[\textbf{A.}] Increase by a factor of 4
\task[\textbf{B.}]  Decrease by a factor of 4
\task[\textbf{C.}] Increase by a factor of 8
\task[\textbf{D.}]  Decrease by a factor of 8
\end{tasks}
\begin{answer}
$$\begin{aligned}
\text{The relation between Fermi}&\text{ energy and electron density is} \\
	E_{F}&=\frac{\hbar^{2}}{2 m}\left(3 \pi^{2} n\right)^{2 / 3}\\
\Rightarrow E_{F}^{\prime}&=\frac{\hbar^{2}}{2 m}\left(3 \pi^{2} \times 8 n\right)^{2 / 3}=4 E_{F} \Rightarrow T_{F}^{\prime}=\frac{4 E_{F}}{E_{F}} T_{F}=4 T_{F}
\end{aligned}$$
So the correct answer is \textbf{Option (A)}
\end{answer}
	\question The dispersion relation of phonons in a solid is given by
$$
\omega^{2}(k)=\omega_{0}^{2}\left(3-\cos k_{x} a-\cos k_{y} a-\cos k_{z} a\right)
$$
The velocity of the phonons at large wavelength is
{\exyear{NET/JRF(JUNE-2012)}}

\begin{tasks}(4)
\task[\textbf{A.}] $\omega_{0} a / \sqrt{3}$
\task[\textbf{B.}] $\omega_{0} a$
\task[\textbf{C.}] $\sqrt{3} \omega_{0} a$
\task[\textbf{D.}] $\omega_{0} a / \sqrt{2}$
\end{tasks}
\begin{answer}
\begin{align*}
\intertext{For large $\lambda,\left(k_{x} a, k_{y} a, k_{z} a\right)$ are small.}\\
\omega^{2}(k)&=\omega_{0}^{2}\left[3-\left(1-\frac{k_{x}^{2} a^{2}}{2}\right)-\left(1-\frac{k_{y}^{2} a^{2}}{2}\right)-\left(1-\frac{k_{z}^{2} a^{2}}{2}\right)\right]\\&=\frac{\omega_{0}^{2} a^{2}}{2}\left(k_{x}^{2}+k_{y}^{2}+k_{z}^{2}\right)\\
\omega^{2}(k)&=\frac{\omega_{0}^{2} a^{2}}{2} k^{2} \Rightarrow \omega=\frac{\omega_{0} a}{\sqrt{2}} k \Rightarrow v_{g}=\frac{d \omega}{d k}=\frac{\omega_{0} a}{\sqrt{2}}
\end{align*}
So the correct answer is \textbf{Option (D)}
\end{answer}
\item The dispersion relation of phonons in a solid is given by
$$
\omega^{2}(k)=\omega_{0}^{2}\left(3-\cos k_{x} a-\cos k_{y} a-\cos k_{z} a\right)
$$
The velocity of the phonons at large wavelength is
{\exyear{NET/JRF(JUNE-2012)}}

\begin{tasks}(4)
\task[\textbf{A.}] $\omega_{0} a / \sqrt{3}$
\task[\textbf{B.}] $\omega_{0} a$
\task[\textbf{C.}] $\sqrt{3} \omega_{0} a$
\task[\textbf{D.}] $\omega_{0} a / \sqrt{2}$
\end{tasks}
\begin{answer}
$$\begin{aligned}
\text{For large }&\text{$\lambda,\left(k_{x} a, k_{y} a, k_{z} a\right)$ are small.}\\
\omega^{2}(k)&=\omega_{0}^{2}\left[3-\left(1-\frac{k_{x}^{2} a^{2}}{2}\right)-\left(1-\frac{k_{y}^{2} a^{2}}{2}\right)-\left(1-\frac{k_{z}^{2} a^{2}}{2}\right)\right]\\&=\frac{\omega_{0}^{2} a^{2}}{2}\left(k_{x}^{2}+k_{y}^{2}+k_{z}^{2}\right)\\
\omega^{2}(k)&=\frac{\omega_{0}^{2} a^{2}}{2} k^{2} \\ \omega&=\frac{\omega_{0} a}{\sqrt{2}} k \\ v_{g}&=\frac{d \omega}{d k}=\frac{\omega_{0} a}{\sqrt{2}}
\end{aligned}$$
So the correct answer is \textbf{Option (D)}
\end{answer}
\item For $T$ much less than the Debye temperature of copper, the temperature dependence of the specific heat at constant volume of copper, is given by (in the following $a$ and $b$ are positive constants)
{\exyear{NET/JRF(DEC-2019)}}

\begin{tasks}(4)
\task[\textbf{A.}] $a T^{3}$
\task[\textbf{B.}] $a T+b T^{3}$
\task[\textbf{C.}] $a T^{2}+b T^{3}$
\task[\textbf{D.}] $\exp \left(-\frac{a}{k_{B} T}\right)$
\end{tasks}
\begin{answer}
$$\begin{aligned}
\text{The specific heat of}&\text{ model is sum of electric and phonon specific heat}\\
C&=C_{e}+C_{p h}\\
\text{For }T \ll \theta_{0}: C_{P h}&=b T^{3}\text{ and }C_{e}=a T \therefore C=a T+b T^{3}
\end{aligned}$$
So the correct answer is \textbf{Option (B)}
\end{answer}	\item Hard disc of radius $R$ are arranged in a two-dimensional triangular lattice. What is the fractional area occupied by the discs in the closest possible packing?
{\exyear{NET/JRF(JUNE-2018)}}

\begin{tasks}(4)
\task[\textbf{A.}] $\frac{\pi \sqrt{3}}{6}$
\task[\textbf{B.}]  $\frac{\pi}{3 \sqrt{2}}$
\task[\textbf{C.}] $\frac{\pi \sqrt{2}}{5}$
\task[\textbf{D.}] $\frac{2 \pi}{7}$
\end{tasks}
\begin{answer}
\begin{align*}
P . F&=\frac{n_{e f f} \times \pi r^{2}}{A}\\
\text{where }n_{e f f}&=\frac{1}{3} \times 6+1=3\text{ and }A=6 \times \frac{\sqrt{3}}{4} a^{2}\\&=\frac{3 \sqrt{3}}{2} a^{2}=\frac{3 \sqrt{3}}{2}(2 r)^{2}=6 \sqrt{3} r^{2}\\
P . F .&=\frac{3 \times \pi r^{2}}{6 \sqrt{3} r^{2}}=\frac{\pi}{2 \sqrt{3}}=\frac{\pi \sqrt{3}}{6}
\end{align*}
So the correct answer is \textbf{Option (A)}
\end{answer}
	\item In order to estimate the specific heat of phonons, the appropriate method to apply would be
{\exyear{GATE 2019}}

\begin{tasks}(1)
	\task[\textbf{A.}] Einstein model for acoustic phonons and Debye model for optical phonons
	\task[\textbf{B.}] Einstein model for optical phonons and Debye model for acoustic phonons
	\task[\textbf{C.}]  Einstein model for both optical and acoustic phonons
	\task[\textbf{D.}] Debye model for both optical and acoustic phonons
\end{tasks}
\begin{answer}
	At low temperature, the optical branch phonons have energies higher than $k_{B} T$ and therefore, optical branch waves are not excited. And Debye model is not suitable for optical branch instead it is suitable for acoustical branch. Whereas Einstein model is useful for high temperature and therefore can be applied to optical branch.\\\\
	So the correct answer is \textbf{Option (B)}
\end{answer}
\item The relative magnetic permeability of a type-I super conductor is
{\exyear{GATE 2019}}

\begin{tasks}(4)
	\task[\textbf{A.}] 0
	\task[\textbf{B.}] $-1$
	\task[\textbf{C.}] $2 \pi$
	\task[\textbf{D.}] $\frac{1}{4 \pi}$
\end{tasks}
\begin{answer}
	\begin{align*}
	B&=\mu_{0}(H+M)=\mu_{0}(H+\chi H)\\&=\mu_{0}(1+\chi) H=\mu H\\
	\therefore \quad \mu&=\mu_{0}(1+x) \Rightarrow \mu_{r}=\frac{\mu}{\mu_{0}}=1+\chi\\
	\text{For type-I superconductor: }\chi&=-1\\
	\therefore \mu_{r}&=1-1=0
	\end{align*}
	So the correct answer is \textbf{Option (A)}
\end{answer}
\item Consider a one dimensional lattice with a weak periodic potential $U(x)=U_{0} \cos \left(\frac{2 \pi x}{a}\right)$ The gap at the edge of the Brillouin zone $\left(k=\frac{\pi}{a}\right)$ is:
{\exyear{GATE 2017}}

\begin{tasks}(4)
	\task[\textbf{A.}] $U_{0}$
	\task[\textbf{B.}] $\frac{U_{0}}{2}$
	\task[\textbf{C.}] $2 U_{0}$
	\task[\textbf{D.}] $\frac{U_{0}}{4}$
\end{tasks}
\begin{answer}
	\begin{align*}
	U&=U_{0} \cos \left(\frac{2 \pi}{a} x\right)\\
	\text{	Energy at the edge of Brillouin Zone is }U_{t}&=U_{0} \cos \left(\frac{2 \pi}{a} \cdot \frac{a}{\pi}\right)\\
	\text{Energy at the }k&=0\text{ is }U_{b}=U_{0} \\\therefore\text{ Band gap }\Delta U&=U_{t}-U_{b}=-2 U_{0}
	\end{align*}
	So the correct answer is \textbf{Option (C)}
\end{answer}
	\item The dispersion relation for phonons in a one dimensional monoatomic Bravais lattice with lattice spacing $a$ and consisting of ions of masses $M$ is given by $\omega(k)=\sqrt{\frac{2 c}{M}[1-\cos (k a)]}$, where $\omega$ is the frequency of oscillation, $k$ is the wavevector and $C$ is the spring constant. For the long wavelength modes $(\lambda>>a)$, the ratio of the phase velocity to the group velocity is---------------
{\exyear{GATE 2015}}
\begin{tasks}(2)
	\task[\textbf{a.}] 1 
	\task[\textbf{b.}]$\omega_{0}$
	\task[\textbf{c.}]a 
	\task[\textbf{d.}]ka 
\end{tasks}
\begin{answer}
\begin{align*}
\omega(k)&=\sqrt{\frac{2 C}{M}[1-\cos (k a)]}
\intertext{For long wavelength modes $(\lambda>>a)$}
\because \cos (k a) \cong 1-\frac{(k a)^{2}}{2} \Rightarrow \omega(k)&=\sqrt{\frac{2 C}{M}\left[1-1+\frac{(k a)^{2}}{2}\right]}=a \sqrt{\frac{C}{M}} k\\
\text{Phase velocity }v_{P}&=\frac{\omega}{k}=a \sqrt{\frac{C}{M}}\text{ and Group velocity} \\v_{g}&=\frac{d \omega}{d k}=a \sqrt{\frac{C}{M}} \Rightarrow \frac{v_{P}}{v_{g}}=1
\end{align*}
Option(A)
\end{answer}
	\item A conventional type-I superconductor has a critical temperature of $4.7 \mathrm{~K}$ at zero magnetic field and a critical magnetic field of $0.3$ Tesla at $0 K$. The critical field in Tesla at $2 K$ (rounded off to three decimal places) is-----------
{\exyear{GATE 2019}}
\begin{tasks}(2)
	\task[\textbf{a.}] 0.125
	\task[\textbf{b.}]1.25
	\task[\textbf{c.}] 0.246
	\task[\textbf{d.}] 3.24
\end{tasks}
\begin{answer}
\begin{align*}
H_{c}(T)&=H_{0}\left[1-\left(\frac{T}{T_{c}}\right)^{2}\right]\\&=0.3\left[1-\left(\frac{2}{4.7}\right)^{2}\right]=0.3\left[1-(0.426)^{2}\right]\\
&=0.3[1-0.181]=0.3 \times 0.819=0.246 \mathrm{Atm}
\end{align*}
Option(C)
\end{answer}
	\item Consider a metal which obeys the Sommerfield model exactly. If $E_{F}$ is the Fermi energy of the metal at $T=0 K$ and $R_{H}$ is its Hall coefficient, which of the following statements is correct?
{\exyear{GATE 2016}}

\begin{tasks}(2)
\task[\textbf{A.}] $R_{H} \propto E_{F}^{\frac{3}{2}}$
\task[\textbf{B.}] $R_{H} \propto E_{F}^{\frac{2}{3}}$
\task[\textbf{C.}] $R_{H} \propto E_{F}^{\frac{-3}{2}}$
\task[\textbf{D.}] $R_{H}$ is independent of $E_{F}$.
\end{tasks}
\begin{answer}
\begin{align*}
R_{H}&=\frac{1}{n e}, \quad\text{ where }E_{F}=\frac{\hbar^{2}}{2 m}\left(3 \pi^{2} n\right)^{2 / 3} \Rightarrow n\\&=\left(\frac{2 m}{\hbar^{2}}\right)^{3 / 2} \cdot \frac{\left(E_{F}\right)^{3 / 2}}{3 \pi^{2}} \Rightarrow R_{H} \propto E_{F}^{-3 / 2}
\end{align*}
So the correct answer is \textbf{Option (C)}
\end{answer}
	\item The energy required to create a lattice vacancy in a crystal is equal to $1 \mathrm{eV}$. The ratio of the number densities of vacancies $n(1200 \mathrm{~K}) / n(300 \mathrm{~K})$ when the crystal is at equilibrium at $1200 \mathrm{~K}$ and $300 \mathrm{~K}$, respectively, is approximately
\exyear{NET/JRF(JUNE-2012)}

\begin{tasks}(4)
\task[\textbf{A.}] $\exp (-30)$
\task[\textbf{B.}] $\exp (-15)$
\task[\textbf{C.}] $\exp (15)$
\task[\textbf{D.}] $\exp (30)$
\end{tasks}
\begin{answer}
$$\begin{aligned}
\text{ The equation }&\text{for number density of vacancies}\\
 n&=N e^{-E / 2 k_{B} T}\\
\text{ where $E$} :& \text{Energy required to form vacancies, $\mathrm{N}$ : density of lattice sites}\\
\therefore \frac{n_{1}}{n_{2}}&=\frac{e^{-E / 2 k_{B} T_{1}}}{e^{-E / 2 k_{B} T_{2}}}=e^{\frac{+E}{2 k_{B}}\left[\frac{1}{T_{2}}-\frac{1}{T_{1}}\right]}, \frac{n(1200 K)}{n(300 K)}\\&=e^{\frac{E}{2 k_{B}}\left[\frac{1}{300}-\frac{1}{1200}\right]}=e^{\frac{E}{2 k_{n}}\left[\frac{1}{400}\right]}=e^{30}
\end{aligned}$$
So the correct answer is \textbf{Option (D)}
\end{answer}
\end{enumerate}