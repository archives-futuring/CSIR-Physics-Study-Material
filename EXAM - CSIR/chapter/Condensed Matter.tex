\begin{abox}
	Condensed Matter Physics -Full Test
	\end{abox}
\begin{enumerate}
	\item A substance with face-centred cubic lattice has density $6250 \mathrm{~kg} / \mathrm{m}^{3}$ and molecular weight ${6 0} \cdot 2$. Calculate the lattice constant $a$. Given Avogadro number $=6 \cdot 02 \times 10^{26}(\mathrm{~kg}-$ mole) ${ }^{-1}$.
	\begin{answer}
		We have lattice constant
		$$\begin{aligned}
		a&=\left(\frac{n \mathrm{M}}{\mathrm{N} \rho}\right)^{1 / 3}\\
		\text{Here $n=$ number  }&\text{of molecules per unit cell in f.c.c. Lattice} \\
		n&=4\\
		 \mathrm{M}&=60 \cdot 2\\
		  N&=6 \cdot 02 \times 10^{26}(\mathrm{~kg}- mole )^{-1}\\
		 \rho&=6250 \mathrm{~kg} / \mathrm{m}^{3}\\
		\therefore \quad a &=\left[\frac{4 \times 6.02}{6250 \times 6.02 \times 10^{26}}\right] \\
		&=4 \times 10^{-10} \mathrm{~m} \\
		&=4 \text{\AA}
		\end{aligned}$$
	\end{answer}
 \item A beam of $X$-rays is incident on a sodium chloride crystal (lattice spacing = $\left.2 \cdot 82 \times 10^{-10} \mathrm{~m}\right)$. The first order Bragg reflection is observed at a glancing angle of $8^{\circ} 35^{\prime}$. What is the wavelength of $X$-rays ? At what angles would the second and the third order Bragg reflectings occur?

\begin{answer}
According to Bragg's law
$$
\begin{aligned}
2 d \sin \theta &=n \lambda \\
\lambda &=\frac{2 d \sin \theta}{n}
\\\text{Substituting the values,} d&=2 \cdot 82 \times 10^{-10} \mathrm{~m}\\
 \theta&=8^{\circ} 35^{\prime} \text{ and $n=1$, we get,}\\
 \lambda &=\frac{2 \times 2.82 \times 10^{-10} \times \sin 8^{\circ} 35}{1} \\
 &=2 \times 2.82 \times 10^{-10} \times 0.1492 \\
 &=0 \cdot 842 \times 10^{-10} \mathrm{~m} \\
 &=0 \cdot 842 \text{\AA}
\end{aligned}
$$
\end{answer}
\item The potential energy function for the force between two atoms in diatomic molecule can be expressed as follows :
$$
U(x)=\frac{a}{x^{12}}-\frac{b}{x^{6}}
$$
Here $a$ and $b$ are positive constants and $x$ is the distance between atoms.

(a) At what value of $x$ is $\mathrm{U}(x)$ equal to zero.

(b) At what value of $x$ is $\mathrm{U}(x)$ a minimum.

(c) Derive an expression for the force between two atoms and show that the two atoms repel each other for $x$ less than $x_{0}$ and attract each other for $x$ greater than $x_{0}$. What is the value of $x$ ?

Solution : (a) Given
$$
\mathrm{U}(x)=\frac{a}{x^{12}}-\frac{b}{x^{6}}
$$
If $\mathrm{U}(x)=0, \frac{a}{x^{12}}-\frac{b}{x^{6}}=0$
$$
\text { or } \quad \frac{1}{x^{6}}\left\{\frac{a}{x^{6}}-b\right\}=0
$$
This gives

or
$$
\begin{aligned}
&x=\left(\frac{a}{b}\right)^{1 / 6} \\
&x=\infty
\end{aligned}
$$
(b) For minimum value of $\mathrm{U}(x), \frac{d \mathrm{U}}{d x}=0$ i.e., $\quad \frac{d}{d x}\left\{\frac{a}{x^{12}}-\frac{b}{x^{6}}\right\}=0$
$$
\frac{1}{x^{7}}\left\{6 b-\frac{12 a}{x^{6}}\right\}=0
$$
This gives $x=\infty$ or $x=(2 a / b)^{1 / 6}$ by

(c) The force $\mathrm{F}$ between two atoms is given or
$$
\begin{aligned}
F &=-\frac{d U}{d x} \\
&=\frac{d}{d x}\left\{\frac{a}{x^{12}}-\frac{b}{x^{6}}\right\} \\
&=\frac{12 a}{x^{13}}-\frac{6 b}{x^{7}}
\end{aligned}
$$
The force $\mathrm{F}$ is repulsive

$\therefore \quad \frac{12 a}{x^{13}}-\frac{6 b}{x^{7}}>6 b$
$$
\begin{aligned}
\frac{1}{x^{7}}\left\{\frac{12 a}{x^{6}}-6 b\right\} &>0 \\
\frac{12 a}{x^{6}} &>6 b \\
x &<\left(\frac{2 a}{b}\right)^{1 / 6}
\end{aligned}
$$
Hence, $x<x_{0}$ where $x_{0}=\left(\frac{2 a}{b}\right)^{1 / 6}$

The force $\mathrm{F}$ is attractive if $\mathrm{F}$ is negative, i.e.,
$$
\frac{12 a}{x^{13}}-\frac{6 b}{x^{7}}<0
$$
This gives
$$
\begin{aligned}
&x>\left(\frac{2 a}{b}\right)^{1 / 6} \\
&x>x_{0}
\end{aligned}
$$
Thus, the two atoms repel each other for $x$ less than $x_{0}$ and attract each other for $x$ greater than $x_{0}$ where
$$
x_{0}=\left(\frac{2 a}{b}\right)^{1 / 2}
$$
\end{enumerate}