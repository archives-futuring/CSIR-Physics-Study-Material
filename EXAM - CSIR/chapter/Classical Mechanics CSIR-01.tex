\chapter{Classical Mechanics-1}
\begin{enumerate}
	\item The number of degrees of freedom of a rigid body in $d$ space-dimensions is

	 \begin{tasks}(2)
		\task[\textbf{a.}]$2 d$
		\task[\textbf{b.}]6
		\task[\textbf{c.}]$d(d+1) / 2$
		\task[\textbf{d.}]  $d$ !
	\end{tasks}
	\begin{answer}
	So the correct answer is \textbf{Option (c)}
	\end{answer}
	\item The Lagrangian of a particle moving in a plane s given in Cartesian coordinates as
	$$
	L=\dot{x} \dot{y}-x^{2}-y^{2}
	$$
	In polar coordinates the expression for the canonical momentum $p_{r}$ (conjugate to the radial coordinate $r$ ) is

	 \begin{tasks}(2)
		\task[\textbf{a.}]$\dot{r} \sin \theta+r \dot{\theta} \cos \theta$
		\task[\textbf{b.}]$\dot{r} \cos \theta+r \dot{\theta} \sin \theta$
		\task[\textbf{c.}]$2 \dot{r} \cos \theta-r \dot{\theta} \sin 2 \theta$
		\task[\textbf{d.}] $\dot{r} \sin 2 \theta+r \dot{\theta} \cos 2 \theta$
	\end{tasks}
\begin{answer}
	$$
	\begin{aligned}
	L&=\dot{x} \dot{y}-x^{2}-y^{2}=\dot{x} \dot{y}-\left(x^{2}+y^{2}\right)\\
	x&=r \cos \theta, y=r \sin \theta \Rightarrow \dot{x}=\dot{r} \cos \theta-r \sin \theta \dot{\theta}, \quad \dot{y}=\dot{r} \sin \theta+r \cos \theta \dot{\theta}\\
	&\text{Therefore, option
	(d) is correct.}\\
L&=\dot{r}^{2} \sin \theta \cos \theta-r^{2} \sin \theta \cos \theta \dot{\theta}^{2}+\dot{r} r \cos ^{2} \theta \dot{\theta}-\dot{r} r \sin ^{2} \theta \dot{\theta}\\
P_{r}&=\frac{\partial L}{\partial \dot{r}} \Rightarrow 2 \dot{r} \sin \theta \cos \theta+r \dot{\theta}\left(\cos ^{2} \theta-\sin ^{2} \theta\right)\\
\Rightarrow P_{r}&=\dot{r} \sin 2 \theta+r \dot{\theta} \cos 2 \theta
\end{aligned}
$$
	So the correct answer is \textbf{Option (d)}
\end{answer}
	\item The parabolic coordinates $(\xi, \eta)$ are related to the Cartesian coordinates $(x, y)$ by $x=\xi \eta$ and $y=\frac{1}{2}\left(\xi^{2}-\eta^{2}\right)$. The Lagrangian of a two-dimensional simple harmonic oscillator of mass $m$ and angular frequency $\omega$ is

	 \begin{tasks}(1)
		\task[\textbf{a.}]$\frac{1}{2} m\left[\dot{\xi}^{2}+\dot{\eta}^{2}-\omega^{2}\left(\xi^{2}+\eta^{2}\right)\right]$
		\task[\textbf{b.}]$\frac{1}{2} m\left(\xi^{2}+\eta^{2}\right)\left[\left(\dot{\xi}^{2}+\dot{\eta}^{2}\right)-\frac{1}{4} \omega^{2}\left(\xi^{2}+\eta^{2}\right)\right]$
		\task[\textbf{c.}] $\frac{1}{2} m\left(\xi^{2}+\eta^{2}\right)\left[\dot{\xi}^{2}+\dot{\eta}^{2}-\frac{1}{2} \omega^{2} \xi \eta\right]$
		\task[\textbf{d.}] $\frac{1}{2} m\left(\xi^{2}+\eta^{2}\right)\left[\dot{\xi}^{2}+\dot{\eta}^{2}-\frac{1}{4} \omega^{2}\right]$
	\end{tasks}
\begin{answer}
	$$
	\begin{aligned}
	&\text { For two dimensional Harmonic oscillation }\\
	L&=\frac{1}{2} m\left(\dot{x}^{2}+\dot{y}^{2}\right)-\frac{1}{2} m \omega^{2}\left(x^{2}+y^{2}\right) \\
	x&=\xi \eta, \quad y=\frac{1}{2}\left(\xi^{2}-\eta^{2}\right) \\
	\dot{x}&=\dot{\xi} \eta+\xi \dot{\eta}, \quad \dot{y}=\xi \dot{\xi}-\eta \dot{\eta}\\
	L&=\frac{1}{2} m\left[(\dot{\xi} \eta+\xi \dot{\eta})^{2}+(\xi \dot{\xi}-\eta \dot{\eta})^{2}\right]-\frac{1}{2} m \omega^{2}\left[\xi^{2} \eta^{2}+\frac{1}{4}\left(\xi^{2}-\eta^{2}\right)^{2}\right] \\
	L&=\frac{1}{2} m\left(\dot{\xi}^{2} \eta^{2}+\xi^{2} \dot{\eta}^{2}+\xi^{2} \dot{\xi}^{2}+\eta^{2} \dot{\eta}^{2}\right)-\frac{1}{8} m \omega^{2}\left(\xi^{4}+\eta^{4}+2 \xi^{2} \eta^{2}\right) \\
	&=\frac{1}{2} m\left(\xi^{2}+\eta^{2}\right)\left(\dot{\eta}^{2}+\dot{\xi}^{2}\right)-\frac{1}{8} m \omega^{2}\left(\xi^{2}+\eta^{2}\right)^{2} \\
	&=\frac{1}{2} m\left(\xi^{2}+\eta^{2}\right)\left[\dot{\eta}^{2}+\dot{\xi}^{2}-\frac{1}{4} \omega^{2}\left(\xi^{2}+\eta^{2}\right)\right]
\end{aligned}
$$
So the correct answer is \textbf{Option (b)}
\end{answer}
	\item A particle of mass $m$ moves inside a bowl. If the surface of the bowl is given by the equation $z=\frac{1}{2} a\left(x^{2}+y^{2}\right)$, where $a$ is a constant, the Lagrangian of the particle is

	 \begin{tasks}(1)
		\task[\textbf{a.}]$\frac{1}{2} m\left(\dot{r}^{2}+r^{2} \dot{\phi}^{2}-g a r^{2}\right)$
		\task[\textbf{b.}]$\frac{1}{2} m\left[\left(1+a^{2} r^{2}\right) \dot{r}^{2}+r^{2} \dot{\phi}^{2}\right]$
		\task[\textbf{c.}]$\frac{1}{2} m\left(\dot{r}^{2}+r^{2} \dot{\theta}^{2}+r^{2} \sin ^{2} \theta \dot{\phi}^{2}-g a r^{2}\right)$
		\task[\textbf{d.}] $\frac{1}{2} m\left[\left(1+a^{2} r^{2}\right) \dot{r}^{2}+r^{2} \dot{\phi}^{2}-g a r^{2}\right]$
	\end{tasks}
\begin{answer}
	$$
	\begin{aligned}
	L&=\frac{1}{2} m\left(\dot{x}^{2}+\dot{y}^{2}+\dot{z}^{2}\right)-m g z, \text { where } z=\frac{1}{2} a\left(x^{2}+y^{2}\right)\\
	&\text { It has cylindrical symmetry. Thus } x=r \cos \phi, y=r \sin \phi, z=\frac{1}{2} a\left(r^{2}\right) \text {. }\\
	\dot{x}&=\dot{r} \cos \phi-r \sin \phi \dot{\phi}, \dot{y}=\dot{r} \sin \phi+r \cos \phi \dot{\phi} \text { and } \dot{z}=a(r \dot{r})\\
	\text { So, } L&=\frac{1}{2} m\left[\left(1+a^{2} r^{2}\right) \dot{r}^{2}+r^{2} \dot{\phi}^{2}-g a r^{2}\right] \text {. }
\end{aligned}
$$
So the correct answer is \textbf{Option (d)}
\end{answer}
	\item The Lagrangian of a particle of mass $m$ moving in one dimension is given by
	$$
	L=\frac{1}{2} m \dot{x}^{2}-b x
	$$
	where $b$ is a positive constant. The coordinate of the particle $x(t)$ at time $t$ is given by: (in following $c_{1}$ and $c_{2}$ are constants)

	 \begin{tasks}(2)
		\task[\textbf{a.}] $-\frac{b}{2 m} t^{2}+c_{1} t+c_{2}$
		\task[\textbf{b.}]$c_{1} t+c_{2}$
		\task[\textbf{c.}]$c_{1} \cos \left(\frac{b t}{m}\right)+c_{2} \sin \left(\frac{b t}{m}\right)$
		\task[\textbf{d.}] $c_{1} \cosh \left(\frac{b t}{m}\right)+c_{2} \sinh \left(\frac{b t}{m}\right)$
	\end{tasks}
\begin{answer}
	$$
	\begin{aligned}
	\text { Equation of motion } \frac{d}{d t}\left(\frac{\partial L}{\partial \dot{x}}\right)-\frac{\partial L}{\partial x}&=0 \Rightarrow \frac{d}{d t}(m \dot{x})+b=0 \Rightarrow m \ddot{x}+b=0 \Rightarrow m \ddot{x}=-b\\
	\frac{d^{2} x}{d t^{2}}&=-\frac{b}{m} \Rightarrow \frac{d x}{d t}=-\frac{b}{m} t+c_{1} \Rightarrow x=-\frac{b}{m} \frac{t^{2}}{2}+c_{1} t+c_{2}
\end{aligned}
$$
So the correct answer is \textbf{Option (a)}
\end{answer}
	\item Which of the following terms, when added to the Lagrangian $L(x, y, \dot{x}, \dot{y})$ of a system with two degrees of freedom will not change the equations of motion?

	 \begin{tasks}(4)
		\task[\textbf{a.}]$x \ddot{x}-y \ddot{y}$
		\task[\textbf{b.}]$x \ddot{y}-y \ddot{x}$
		\task[\textbf{c.}]$x \dot{y}-y \dot{x}$
		\task[\textbf{d.}] $y \dot{x}^{2}+x \dot{y}^{2}$
	\end{tasks}
	\begin{answer}
		$$
		\begin{aligned}
		&L(x, y, \dot{x}, \dot{y})\\
		&L^{\prime}=L(x, y, \dot{x}, \dot{y})+x \ddot{y}-y \ddot{x}\\
		&\frac{d^{\prime}}{d t^{\prime}}\left(\frac{\partial L^{\prime}}{\partial \dot{x}}\right)-\frac{\partial L^{\prime}}{\partial x}=\frac{d}{d t}\left(\frac{\partial L}{\partial \dot{x}}\right)-\frac{\partial L}{\partial x}+\ddot{y}=0=0+\ddot{y}=0 \\ \dot{y}&=c_{1}\\
		&\frac{d}{d t}\left(\frac{\partial L}{\partial y}\right)-\frac{\partial L^{\prime}}{\partial y}=\frac{d}{d t}\left(\frac{\partial L}{\partial \dot{y}}\right)-\frac{\partial L}{\partial y}+\ddot{x}=0=0-\ddot{x}=0 \\ \dot{x}&=c_{2}
	\end{aligned}
	$$
	So the correct answer is \textbf{Option (b)}
	\end{answer}
\item If the Lagrangian of a particle moving in one dimensions is given by $L=\frac{\dot{x}^{2}}{2 x}-V(x)$ the Hamiltonian is

 \begin{tasks}(2)
	\task[\textbf{a.}]$\frac{1}{2} x p^{2}+V(x)$
	\task[\textbf{b.}] $\frac{\dot{x}^{2}}{2 x}+V(x)$
	\task[\textbf{c.}]$\frac{1}{2} \dot{x}^{2}+V(x)$
	\task[\textbf{d.}] $\frac{p^{2}}{2 x}+V(x)$
\end{tasks}
\begin{answer}
	$$
	\begin{aligned}
	\text { Since } H&=p_{x} \dot{x}-L \text { and } \frac{\partial L}{\partial \dot{x}}=p_{x} \Rightarrow \frac{\dot{x}}{x}=p_{x} \Rightarrow \dot{x}=p_{x} x \text {. }\\
	H&=p_{x} \dot{x}-\frac{\dot{x}^{2}}{2 x}+V(x) \Rightarrow H=p_{x}\left(p_{x} x\right)-\frac{\left(p_{x} x\right)^{2}}{2 x}+V(x) \Rightarrow H=\frac{p_{x}^{2} x}{2}+V(x)
\end{aligned}
$$
	So the correct answer is \textbf{Option (a)}
\end{answer}
\item A particle of mass $m$ and coordinate $q$ has the Lagrangian $L=\frac{1}{2} m \dot{q}^{2}-\frac{\lambda}{2} q \dot{q}^{2}$, where $\lambda$ s a constant. The Hamiltonian for the system is given by 

 \begin{tasks}(2)
	\task[\textbf{a.}]$\frac{p^{2}}{2 m}+\frac{\lambda q p^{2}}{2 m^{2}}$
	\task[\textbf{b.}] $\frac{p^{2}}{2(m-\lambda q)}$
	\task[\textbf{c.}]$\frac{p^{2}}{2 m}+\frac{\lambda q p^{2}}{2(m-\lambda q)^{2}}$
	\task[\textbf{d.}] $\frac{p \dot{q}}{2}$
\end{tasks}
\begin{answer}
	$$
	\begin{aligned}
	H&=\sum \dot{q} p-L \text { where } L=\frac{1}{2} m \dot{q}^{2}-\frac{\lambda}{2} q \dot{q}^{2}\\
	\frac{\partial L}{\partial \dot{q}}&=p=m \dot{q}-\lambda q \dot{q} \Rightarrow p=\dot{q}(m-\lambda q) \Rightarrow \dot{q}=\frac{p}{m-\lambda q} \\
	\Rightarrow H&=\dot{q} p-L=\frac{p^{2}}{(m-\lambda q)}-\frac{1}{2} m \frac{\left(p^{2}\right)}{(m-\lambda q)^{2}}+\frac{\lambda}{2} q \cdot \frac{p^{2}}{(m-\lambda q)^{2}} \\
	\Rightarrow H&=\dot{q} p-L=\frac{p^{2}}{(m-\lambda q)}-\frac{p^{2}}{2(m-\lambda q)^{2}}(m-\lambda q) \\
	\Rightarrow H&=\dot{q} p-L=\frac{p^{2}}{(m-\lambda q)}-\frac{p^{2}}{2(m-\lambda q)} \Rightarrow H=\frac{p^{2}}{2(m-\lambda q)}
\end{aligned}
$$
	So the correct answer is \textbf{Option (b)}
\end{answer}
\item If the Lagrangian of a dynamical system in two dimensions is $L=\frac{1}{2} m \dot{x}^{2}+m \dot{x} \dot{y}$, then its Hamiltonian is

 \begin{tasks}(2)
	\task[\textbf{a.}] $H=\frac{1}{m} p_{x} p_{y}+\frac{1}{2 m} p_{y}^{2}$
	\task[\textbf{b.}] $H=\frac{1}{m} p_{x} p_{y}+\frac{1}{2 m} p_{x}^{2}$
	\task[\textbf{c.}] $H=\frac{1}{m} p_{x} p_{y}-\frac{1}{2 m} p_{y}^{2}$
	\task[\textbf{d.}]  $H=\frac{1}{m} p_{x} p_{y}-\frac{1}{2 m} p_{x}^{2}$
\end{tasks}
\begin{answer}
	$$
	\begin{aligned}
	L&=\frac{1}{2} m \dot{x}^{2}+m \dot{x} \dot{y} \Rightarrow \frac{\partial L}{\partial \dot{x}}=m \dot{x}+m \dot{y}=p_{x}\\
	\Rightarrow \quad \frac{\partial L}{\partial \dot{y}}&=m \dot{x}=p_{y} \quad \text { or } \quad \dot{x}=\frac{p_{y}}{m}\\
	\text{put }\dot{x}&=\frac{p_{y}}{m}\text{ in equation (i) }\Rightarrow p_{y}+m \dot{y}=p_{x} \Rightarrow \dot{y}=\frac{p_{x}-p_{y}}{m}\\
	H&=p_{x} \dot{x}+p_{y} \dot{y}-L=p_{x} \dot{x}+p_{y} \dot{y}-\frac{1}{2} m \dot{x}^{2}-m \dot{x} \dot{y}\\
	\text { put value of }& \dot{x} \text { and } \dot{y} \Rightarrow H=\frac{p_{x} p_{y}}{m}-\frac{p_{y}^{2}}{2 m}
\end{aligned}
$$
	So the correct answer is \textbf{Option (c)}
\end{answer}
\item The Hamiltonian of a system with generalized coordinate and momentum $(q, p)$ is $H=p^{2} q^{2} . A$ solution of the Hamiltonian equation of motion is (in the following $A$ and $B$ are constants)

 \begin{tasks}(2)
	\task[\textbf{a.}]$p=B e^{-2 A t}, \quad q=\frac{A}{B} e^{2 A t}$
	\task[\textbf{b.}]$p=A e^{-2 A t}, \quad q=\frac{A}{B} e^{-2 A t}$
	\task[\textbf{c.}]$p=A e^{A t}, \quad q=\frac{A}{B} e^{-A t}$
	\task[\textbf{d.}]  $p=2 A e^{-A^{2} t}, \quad q=\frac{A}{B} e^{A^{2} t}$
\end{tasks}
\begin{answer}
	$$
	\begin{aligned}
	H&=p^{2} q^{2}\\
	&\text { From Hamilton's equation }\\
	\frac{\partial H}{\partial q}&=-\dot{p} \Rightarrow \frac{d p}{d t}=-2 p^{2} q \hspace{4cm}(i)\\
	\frac{\partial H}{\partial p}&=\dot{q} \Rightarrow \frac{d q}{d t}=2 p q^{2}(ii)\\
	&\text { from equations (i) and \hspace{4cm}(ii) }\\
	\frac{d p}{p}&=-\frac{d q}{q}\\
	&\text { Integrating both sides, } \ln p=-\ln q+\ln A\\
	p q&=A\hspace{2cm}(iii)\\
	&\text { from equation (i) }\\
	\frac{d p}{d t}&=-2 p^{2} q=-2 p A\\
	\Rightarrow \quad \int \frac{d p}{p}&=-\int 2 A d t+\ln B \Rightarrow \ln \frac{p}{B}=-2 A t \Rightarrow p=B e^{-2 A t}
\end{aligned}
$$
Putting this value of $p$ in equation (iii) gives $q=\frac{A}{B} e^{2 A t}$\\ Hence, the correct option is (a)\\
	So the correct answer is \textbf{Option (a)}
\end{answer}
\item A point mass $m$, is constrained to move on the inner surface of a paraboloid of revolution $x^{2}+y^{2}=a z$ (where $a>0$ is a constant). When it spirals down the surface, under the influence of gravity (along $-z$ direction), the angular speed about the $z$ - axis is proportional to

 \begin{tasks}(2)
	\task[\textbf{a.}]1 (independent of $z$ )
	\task[\textbf{b.}]$z$
	\task[\textbf{c.}]$z^{-1}$
	\task[\textbf{d.}]  $z^{-2}$
\end{tasks}
\begin{answer}
	$$
	\begin{aligned}
	&\text { Using Lagrangian in cylindrical coordinate }\\
	L&=\frac{1}{2} m\left(\dot{r}^{2}+r^{2} \dot{\theta}^{2}+\dot{z}^{2}\right)-m g z\\
	\text { with constraint } x^{2}+y^{2}&=a z \Rightarrow r^{2}=a z \Rightarrow \dot{z}=\frac{2 r \dot{r}}{a}\\
	L&=\frac{1}{2} m\left(\dot{r}^{2}+r^{2} \dot{\theta}^{2}+\left(\frac{2 r \dot{r}}{a}\right)^{2}\right)-\frac{m g r^{2}}{a}\\
	\theta \text { is cyclic coordinate so } &\frac{\partial L}{\partial \theta}=0 \Rightarrow \frac{\partial L}{\partial \dot{\theta}}=J \Rightarrow m r^{2} \dot{\theta}=J \Rightarrow \dot{\theta} \propto \frac{1}{r^{2}} \propto \frac{1}{z}
\end{aligned}
$$
	So the correct answer is \textbf{Option (c)}
\end{answer}
\item The Poisson bracket $\{|\vec{r}|,|\vec{p}|\}$ has the value

 \begin{tasks}(4)
	\task[\textbf{a.}]$|\vec{r}||\vec{p}|$
	\task[\textbf{b.}] $\hat{r} \cdot \hat{p}$
	\task[\textbf{c.}]3
	\task[\textbf{d.}] 1
\end{tasks}
\begin{answer}
	$$
	\begin{aligned}
	\vec{r}&=x \hat{i}+y \hat{j}+z \hat{k},|\vec{r}|=\left(x^{2}+y^{2}+z^{2}\right)^{1 / 2}, p=p_{x} \hat{i}+p_{y} \hat{j}+p_{z} \hat{k}\\
	|\vec{p}|&=\left(p_{x}^{2}+p_{y}^{2}+p_{z}^{2}\right)^{1 / 2}\\
	\{|\vec{r}|,|\vec{p}|\}&=\left(\frac{\partial|\vec{r}|}{\partial x} \cdot \frac{\partial|\vec{p}|}{\partial p_{x}}-\frac{\partial|\vec{r}|}{\partial p_{x}} \cdot \frac{\partial|\vec{p}|}{\partial x}\right)+\left(\frac{\partial|\vec{r}|}{\partial y} \cdot \frac{\partial|\vec{p}|}{\partial p_{y}}-\frac{\partial|\vec{r}|}{\partial p_{y}} \cdot \frac{\partial|\vec{p}|}{\partial y}\right)+\left(\frac{\partial|\vec{r}|}{\partial z} \cdot \frac{\partial|\vec{p}|}{\partial p_{z}}-\frac{\partial|\vec{r}|}{\partial p_{z}} \cdot \frac{\partial|\vec{p}|}{\partial y}\right)\\
	&=\frac{x}{|\vec{r}|} \frac{p_{x}}{|\vec{p}|}+\frac{y}{|\vec{r}|} \frac{p_{y}}{|\vec{p}|}+\frac{z}{|\vec{r}|} \frac{p_{z}}{|\vec{p}|}=\frac{\vec{r} \cdot \vec{p}}{|\vec{r}||\vec{p}|}=(\hat{r} \cdot \hat{p})
\end{aligned}
$$
So the correct answer is \textbf{Option (b)}
\end{answer}
\item A system is governed by the Hamiltonian
$$
H=\frac{1}{2}\left(p_{x}-a y\right)^{2}+\frac{1}{2}\left(p_{x}-b x\right)^{2}
$$
where $a$ and $b$ are constants and $p_{x}, p_{y}$ are momenta conjugate to $x$ and $y$ respectively.
For what values of $a$ and $b$ will the quantities $\left(p_{x}-3 y\right)$ and $\left(p_{y}+2 x\right)$ be conserved?

 \begin{tasks}(2)
	\task[\textbf{a.}]$a=-3, b=2$
	\task[\textbf{b.}]$a=3, b=-2$
	\task[\textbf{c.}]$a=2, b=-3$
	\task[\textbf{d.}] $a=-2, b=3$
\end{tasks}
\begin{answer}
	$$
	\begin{aligned}
	&\text { Poisson bracket }\left[p_{x}-3 y, H\right]=0 \text { and }\left[p_{y}+2 y, H\right]=0\\
	&p_{y}(b-3)+x\left(3 b-b^{2}\right)=0 \text { and } p_{x}(a+2)-y\left(2 a+a^{2}\right)=0\\
	&\Rightarrow \quad a=-2, b=3
\end{aligned}
$$
So the correct answer is \textbf{Option (d)}
\end{answer}
\item The Hamiltonian of a classical one-dimensional harmonic oscillator is $H=\frac{1}{2}\left(p^{2}+x^{2}\right)$, in suitable units. The total time derivative of the dynamical variable $(p+\sqrt{2} x)$ is

 \begin{tasks}(2)
	\task[\textbf{a.}]$\sqrt{2} p-x$
	\task[\textbf{b.}]$p-\sqrt{2} x$
	\task[\textbf{c.}] $p+\sqrt{2} x$
	\task[\textbf{d.}] $x+\sqrt{2} p$
\end{tasks}
\begin{answer}
	$$
	\begin{aligned}
	H&=\frac{p^{2}}{2}+\frac{x^{2}}{2}\qquad \text { Let say dynamical variable } A=(p+\sqrt{2} x)\\
	\frac{d A}{d t}&=[A, H]+\frac{\partial A}{\partial t}\\
	\text { It is given } &\frac{\partial A}{\partial t}=0 \Rightarrow \frac{d A}{d t}=[A, H]\\
	\frac{d A}{d t}&=\left[p+\sqrt{2} x, \frac{p^{2}}{2}+\frac{x^{2}}{2}\right]=\left[p, \frac{x^{2}}{2}\right]+\left[\sqrt{2} x, \frac{p^{2}}{2}\right]\\&=\frac{-2 x}{2}+\frac{\sqrt{2} 2 p}{2}=-x+\sqrt{2} p=\sqrt{2} p-x
\end{aligned}
$$
So the correct answer is \textbf{Option (a)}
\end{answer}
\item The coordinates and momenta $x_{i}, p_{i}(i=1,2,3)$ of a particle satisfy the canonical Poisson bracket relations $\left\{x_{i}, p_{j}\right\}=\delta_{i j}$. If $C_{1}=x_{2} p_{3}+x_{3} p_{2}$ and $C_{2}=x_{1} p_{2}-x_{2} p_{1}$ are constants of motion, and if $C_{3}=\left\{C_{1}, C_{2}\right\}=x_{1} p_{3}+x_{3} p_{1}$, then

 \begin{tasks}(2)
	\task[\textbf{a.}]$\left\{C_{2}, C_{3}\right\}=C_{1}$ and $\left\{C_{3}, C_{1}\right\}=C_{2}$
	\task[\textbf{b.}] $\left\{C_{2}, C_{3}\right\}=-C_{1}$ and $\left\{C_{3}, C_{1}\right\}=-C_{2}$
	\task[\textbf{c.}] $\left\{C_{2}, C_{3}\right\}=-C_{1}$ and $\left\{C_{3}, C_{1}\right\}=C_{2}$
	\task[\textbf{d.}] $\left\{C_{2}, C_{3}\right\}=C_{1}$ and $\left\{C_{3}, C_{1}\right\}=-C_{2}$
\end{tasks}
\begin{answer}
	$$
	\begin{aligned}
	C_{1}&=x_{2} p_{3}+x_{3} p_{2}, \quad C_{2}=x_{1} p_{2}-x_{2} p_{1}, C_{3}=x_{1} p_{3}+x_{3} p_{1}\\
	\left\{C_{2}, C_{3}\right\}&=\left(\frac{\partial C_{2}}{\partial x_{1}} \frac{\partial C_{3}}{\partial p_{1}}-\frac{\partial C_{2}}{\partial p_{1}} \frac{\partial C_{3}}{\partial x_{1}}\right)+\left(\frac{\partial C_{2}}{\partial x_{2}} \frac{\partial C_{3}}{\partial p_{2}}-\frac{\partial C_{2}}{\partial p_{2}} \frac{\partial C_{3}}{\partial x_{2}}\right)+\left(\frac{\partial C_{2}}{\partial x_{3}} \frac{\partial C_{3}}{\partial p_{3}}-\frac{\partial C_{2}}{\partial p_{3}} \frac{\partial C_{3}}{\partial x_{3}}\right) \\
	\left\{C_{2}, C_{3}\right\}&=\left(p_{2} x_{3}-\left(-x_{2}\right) p_{3}\right)+\left(0-x_{1} \cdot 0\right)+\left(0 \cdot x_{1}-0 \cdot p_{1}\right)=\left(p_{2} x_{3}+x_{2} p_{3}\right)=C_{1}\\
	\left\{C_{3}, C_{1}\right\}&=\left(\frac{\partial C_{3}}{\partial x_{1}} \frac{\partial C_{1}}{\partial p_{1}}-\frac{\partial C_{3}}{\partial p_{1}} \frac{\partial C_{1}}{\partial x_{1}}\right)+\left(\frac{\partial C_{3}}{\partial x_{2}} \frac{\partial C_{1}}{\partial p_{2}}-\frac{\partial C_{3}}{\partial p_{2}} \frac{\partial C_{1}}{\partial x_{2}}\right)+\left(\frac{\partial C_{3}}{\partial x_{3}} \frac{\partial C_{1}}{\partial p_{3}}-\frac{\partial C_{3}}{\partial p_{3}} \frac{\partial C_{1}}{\partial x_{3}}\right) \\
	\left\{C_{3}, C_{1}\right\}&=\left(p_{3} \cdot 0-x_{3} \cdot 0\right)+\left(0 \cdot x_{3}-0 \cdot p_{3}\right)+\left(p_{1} x_{2}-x_{1} p_{2}\right)=-\left(x_{1} p_{2}-x_{2} p_{1}\right)=-C_{2}
\end{aligned}
$$
So the correct answer is \textbf{Option (d)}
\end{answer}
\end{enumerate}