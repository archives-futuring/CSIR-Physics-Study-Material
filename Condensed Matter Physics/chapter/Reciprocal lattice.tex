\chapter{RECIPROCAL LATTICE}


\section{Reciprocal Lattice}

In certain saturation involving the presence of a number of sets of parallel planes with different orientations, it becomes difficult to visualize all such plane because of their two-dimensional nature.

The problem was simplified by P.P. Ewald by developing a new type of lattice known as the reciprocal lattice.

A Reciprocal lattice to a direct lattice is constructed using the following procedure:

(i). Take origin at some arbitrary point and draw normals to every set of parallel planes of the direct lattice.

(ii). Take length of each normal equal to the reciprocal of the interplanar spacing for the corresponding set of planes. The terminal points of these normal form the reciprocal lattice.

The concept of reciprocal lattice is useful for redefining the Bragg's condition and introducing the concept of Brillouin Zone.

\subsection{Reciprocal lattice to $\mathrm{SC}$ lattice}

The primitive translation vectors $f$ a simple cubic lattice may be written as

The reciprocal lattice vectors the $\mathrm{SC}$ lattice are obtained

$$
\begin{aligned}
&a^{*}=2 \pi \frac{\vec{b} \times \vec{c}}{|a \cdot b \times c|}=2 \pi \frac{a \hat{j} \times a \hat{k}}{a^{3}}=\frac{2 \pi}{a} \hat{i} \\
&b^{*}=2 \pi \frac{\vec{c} \times \vec{a}}{|\vec{a} \cdot \vec{b} \times \vec{c}|}=\frac{2 \pi}{a} \hat{J} \\
&c^{*}=2 \pi \frac{\vec{a} \times \vec{b}}{\mid \vec{a} \cdot(\vec{b} \times \vec{c} \mid}=\frac{2 \pi}{a} \vec{K}
\end{aligned}
$$

$$
\begin{aligned}
& \left.\vec{a}=a \hat{i} \quad \vec{b}=b \hat{j} \quad \vec{c}=a \hat{k} ; \quad V=|\vec{a} \cdot \vec{b} \times \vec{c}|=a^{3} \quad \ldots \quad \begin{array}{l}i \cdot \hat{i}=\hat{j} \cdot \hat{j}=\hat{k} \cdot \hat{k}=1 \\\ldots\end{array}\right\} \\
& \hat{i} \cdot \hat{j}=\hat{j} \cdot \hat{k}=\hat{k} \cdot \hat{i}=0\} 
\end{aligned}
$$



Equation (1), (2) \& (3) indicate that all the three reciprocal lattice vectors are equal in magnitude which means that the reciprocal lattice to SC lattice is also simple cubic but with lattice constant equal to $2 \pi /$ a.

\subsection{Reciprocal lattice to $\mathrm{BCC}$ lattice}

the primitive translation vectors of a body centred cubic lattice

$$
\begin{aligned}
&a^{\prime}=a / 2(\hat{i}+\hat{j}-\hat{k}) \\
&b^{\prime}=a / 2(-\hat{i}+\hat{j}+\hat{k}) \quad \text { use } \vec{b} \times \vec{c}=\left[\begin{array}{l}
\hat{i} \\
b_{x} \\
c_{x}
\end{array}\right. \\
&c^{\prime}=a / 2(\hat{i}-\hat{j}+\hat{k}) \\
&V=(\vec{a} \cdot \vec{b} \times \vec{c})=a / 2(\hat{i}+\hat{j}-\hat{k}) \cdot\left[\frac{a^{2}}{4}(-\hat{i}+\hat{j}+\hat{k}) \times(\hat{i}-\hat{j}+\hat{k})\right] \\
&\left.=\hat{i}\left[b_{y}\left(c_{z}-c_{z} b_{z}\right)\right]-\hat{j}\left[b_{x} c_{z}-c_{x} b_{z}\right)\right]+\hat{k}\left[b_{x} c_{y}-c_{x} b_{y}\right]
\end{aligned}
$$

$$
\begin{aligned}
& \text { use } \vec{b} \times \vec{c}=\left[\begin{array}{lll}\hat{i} & \hat{j} & \hat{k} \\b_{x} & b_{y} & b_{z} \\c_{x} & c_{y} & c_{z}\end{array}\right]
\end{aligned}
$$

Reciprocal lattice vector $=\mathrm{a}^{3} / 2$

$a^{*}=2 \pi \frac{b^{\prime} \times c^{\prime}}{(\vec{a} \cdot \vec{b} \times \vec{c})}=\frac{2 \pi\left(a^{2} / 2\right)}{a^{3} / 2}(\hat{l}+\hat{j})=\frac{2 \pi}{a}(\vec{l}+\hat{j})$

$b^{*}=2 \pi \frac{c^{\prime} \times a^{\prime}}{(\vec{a} \cdot \vec{b} \times \vec{c})}=\frac{2 \pi}{a}(\hat{j}+\hat{k})$

$c^{*} 2 \pi \frac{a^{\prime} \times b^{\prime}}{\left(a: b^{\prime} \times \vec{c}\right)}=\frac{2 \pi}{a}(\hat{k}+\hat{l})$

Thus reciprocal lattice to a bcc lattice is fcc lattice

\subsection{Reciprocal lattice to $\mathrm{FCC}$ lattice:}

The primitive translations vectors of an fcc lattice are

$$
\begin{aligned}
&a^{\prime}=a / 2(\hat{l}+\hat{j}), \quad b^{\prime}=a / 2(\hat{j}+\hat{k}), \quad c^{\prime}=a / \\
&V=|\vec{a} \cdot \vec{b} \times \vec{c}|=a^{3} / 4 \\
&a^{*}=2 \pi \frac{b^{\prime} \times \vec{c}}{\left(a^{\prime} \cdot b^{\prime} \times \vec{c}\right)}=2 \pi \frac{\left(a^{2} / 4\right)(\hat{l}+\hat{j}-\hat{k})}{a^{3} / 4}=\frac{2 \pi}{a}(\hat{l}+\hat{j}-\hat{k}) \\
&b^{\prime}=2 \pi \frac{c^{\prime} \times a^{\prime}}{a^{\prime} \cdot b^{\prime} \times c^{\prime}}=\frac{2 \pi}{a}(-\hat{l}+\hat{j}+\hat{k})
\end{aligned}
$$

$$
\begin{aligned}
& c^{\prime}=a / 2(\hat{k}+\hat{l}) 
\end{aligned}
$$


$$
c^{\prime}=2 \pi \frac{a^{\prime} \times b^{\prime}}{a \cdot\left(b^{\prime} \times \vec{c}\right)}=\frac{2 \pi}{a}(\hat{l}-\hat{j}+\hat{k})
$$

thus reciprocal lattice to an fcc lattice is a bcc lattice

\section{Properties of Reciprocal lattice}

(1) Each point in a reciprocal lattice corresponding to particular set of parallel planes of the direct lattice.

(2) The distance of a reciprocal lattice point from an arbitrarily fixed origin is inversely proportional to the interplanar spacing of the corresponding parallel planes of the direct lattice.

(3) The volume of a unit cell of the reciprocal lattice is inversely proportional to the corresponding unit cell of the direct lattice.

(4) The unit cell of the reciprocal lattice need not be a parallelepiped. It is customary to deal with Winger-Feitz cell of reciprocal lattice which constitutes the Brillouin zone.

Example: Two-dimensional lattice has the basis vectors $a=2 \hat{x} \quad b=\hat{x}+2 \hat{y}$. Find the reciprocal lattice vector.

Ans: $\pi=\left(\hat{x}-\frac{\hat{y}}{2}\right), \pi \hat{y}$

\section{Bragg's Law in Reciprocal Lattice:}

The Bragg's diffraction condition obtained earlier by considering reflection from parallel lattice plane can be used to express geometrical relationship between the vectors in the reciprocal lattice such a using a Ewald construction.



The Bragg's law itself takes a diffraction from in the reciprocal lattice to obtain the modified form of the Bragg's law, we redraw the vector $\overrightarrow{\mathrm{AO}}, \overrightarrow{O \mathrm{~B}}, \overrightarrow{\mathrm{AB}}$ such that each is magnified by a constant factor of $2 \pi$. 



Let the new vector be $\overrightarrow{\mathrm{AO}}, \overrightarrow{O \mathrm{~B}}$ and $\overrightarrow{\mathrm{AB}}$ respectively as shown in figure, since $\quad \mathrm{AO}=\frac{2 \pi}{\lambda}$, we can represent the wave vector $\mathrm{K}$ by the vector $\overrightarrow{\mathrm{AO}}$. The vector $\overrightarrow{O \mathrm{~B}}$ is the reciprocal vector and is written as $\mathrm{G}$. Thus, according to vector algebra $\overrightarrow{\mathrm{AB}}$ must be equal to $(\mathrm{K}+\mathrm{G})$. For diffraction to occur, the point B' must be on the sphere. i.e.

$$
\begin{aligned}
&|\overrightarrow{\mathrm{AB}}|=|\overrightarrow{\mathrm{AO}}| \\
&(K+G)^{2}=K^{2} \\
&K^{2}+2 K \cdot G+G^{2}=K^{2} \\
&2 K \cdot G+G^{2}=0
\end{aligned}
$$

This is the vector form of Bragg's law and is used in construction of the Brillouin zones.

The vector $\overrightarrow{\mathrm{A}^{\prime} \mathrm{B}^{\prime}}$ represents the direction of reflected or scattered beam denting it by $\mathrm{K}^{\prime}$, we get $K^{\prime}=K+G$

Which gives $K^{2}=K^{2}$ or $K^{\prime}-K=n a$

And $K^{\prime}-K=K=G$

This indicates that the scattering does not change the magnitude of wave vector $\mathrm{K}$, only its direction is changed, the scattered wave differs from the incident by a reciprocal lattice vector $G$. 



\subsection{Brillouin Zones:}

"The $1^{\text {st }}$ Brillouin Zone is the smallest volume entired by planes that are the perpendicular bisector of the reciprocal lattice vectors drawn from the origin".

A Brillouin Zone is the locus of all those $\mathrm{K}$-values in the reciprocal lattice which are Bragg's reflected.

\section{(i) Brillouin zones for $\mathrm{SC}$ :}

$a=a \hat{i}, b=a \hat{j}, c=a \hat{k}$

$a^{*}=\left(\frac{2 \pi}{a}\right) i, \quad b^{*}=\left(\frac{2 \pi}{a}\right) j, c^{*}=\left(\frac{2 \pi}{a}\right) \hat{k}$

Therefore, the reciprocal lattice vector is written as

$\mathrm{G}=\left(\frac{2 \pi}{a}\right)(h \hat{i}+k \hat{j}+l \hat{k})$

where $\mathrm{h}, \mathrm{k}$ and $l$ are integers. The wave vector $\mathrm{k}$ can be expressed as

$K=K_{x} \hat{i}+K_{y} \hat{j}+K_{z} \hat{k}$

From the Bragg's condition we have $2 \mathrm{~K} \cdot \mathrm{G}+\mathrm{G}^{2}=0$

$\frac{4 \pi}{a}\left[\left(K_{x} \hat{i}+K_{y} \hat{j}+K_{z} \hat{k}\right) \cdot(h \hat{i}+k \hat{j}+l \hat{k})\right]+\frac{4 \pi}{a^{2}}\left(h^{2}+k^{2}+l^{2}\right)=0$

$h K_{x}+k K_{y}+l K_{z}=-\left(\frac{\pi}{a}\right)\left(h^{2}+k^{2}+l^{2}\right)$

the $\mathrm{K}$-values which are Bragg reflected are obtained by considering all possible combination of $\mathrm{h}$ and $\mathrm{K}$.

for $\mathrm{h}=\pm 1$ and $\mathrm{k}=l=0, K_{x}=\pm \pi / \mathrm{a}$ and $K_{y} \& K_{z}$ is arbitrary.

for $\mathrm{h}=l=0$ and $\mathrm{k}=\pm 1, K_{y}=\pm \pi / \mathrm{a}$ and $K_{x} \& K_{z}$ is arbitrary

for $\mathrm{h}=\mathrm{k}=0$ and $\mathrm{l}=\pm 1, K_{z}=\pm \pi / \mathrm{a}$ and $K_{x} \& K_{y}$ is arbitrary

These six planes construct a cube of length $2 \pi / \mathrm{a}$, thus the $1^{\text {st }}$ B.Z. of the simple cubic is also a simple cubic with volume $(2 \pi / a)^{3}$. 



(ii) Brillouin Zone of $\mathrm{BCC}$ lattice:

$a=\frac{a}{2}(\hat{i}+\hat{j}-\hat{k}) \quad a^{*}=\left(\frac{2 \pi}{a}\right)(\hat{i}+\hat{j})$

$b=\frac{a}{2}(-\hat{i}+\hat{j}+\hat{k}) \quad b^{*}=\left(\frac{2 \pi}{a}\right)(\hat{j}+\hat{k})$



$c^{*}=\left(\frac{2 \pi}{a}\right)(\hat{k}+\hat{i})$

$c=\frac{a}{2}(\hat{i}-\hat{j}+\hat{k})$

The G-type vector is

$\mathrm{G}=\mathrm{ha}^{*}+\mathrm{kb}^{*}+\mathrm{lc}^{*}$

$\left(\frac{2 \pi}{a}\right)[(h+k) \hat{i}+(k+l) \hat{j}+(h+l) \hat{k}]$

shortest non-zero $\mathrm{G}^{\mathrm{s}}$ are the following eight vector

$\frac{2 \pi}{a}(\pm \hat{i} \pm \hat{j}) ; \frac{2 \pi}{a}(\pm \hat{j} \pm \hat{k}) ; \frac{2 \pi}{a}(\pm \hat{k} \pm \hat{i})$

The first B.Z. is the region enclosed by the normal bisector planes to these 12 vectors. This zone has the sphere of a regular 12 faced solid and is called Rhombic dodecahedron.

(iii) Brillouin Zone of $\mathbf{F C C}$ lattice:

$a=\frac{a}{2}(\hat{i}+\hat{j}) \quad a^{*}=\left(\frac{2 \pi}{a}\right)(\hat{i}+\hat{j}-\hat{k})$



$b=\frac{a}{2}(\hat{j}+\hat{k}) \quad b^{*}=\left(\frac{2 \pi}{a}\right)(-\hat{i}+\hat{j}+\hat{k})$

$c=\frac{a}{2}(\hat{k}+\hat{i}) \quad c^{*}=\left(\frac{2 \pi}{a}\right)(\hat{i}-\hat{j}+\hat{k})$

The G-type vector is

$\mathrm{G}=\mathrm{ha}^{*}+\mathrm{kb}^{*}+\mathrm{lc}^{*}$

$\left(\frac{2 \pi}{a}\right)[(h-k+l) \hat{i}+(h+k-l) \hat{j}+(-h+k+l) \hat{k}]$

shortest non-zero $\mathrm{G}^{\mathrm{ss}}$ are the following eight vector

$$
\frac{2 \pi}{a}(\hat{i} \pm \hat{j} \pm \hat{k})
$$



The boundaries of the $1^{\text {st }}$ B.Z. are determined mostly by the normal bisector planes to the above 8 vectors.

However, the corners of the octahedral obtained in this manner are truncated by the planes which are normal bisectors to the following 6 reciprocal lattice vectors.

$$
\left(\frac{2 \pi}{a}\right)(\pm 2 \hat{i}) ;\left(\frac{2 \pi}{a}\right)(\pm 2 \hat{j}) ;\left(\frac{2 \pi}{a}\right)(\pm 2 \hat{k})
$$

The $1^{\text {st }}$ B.Z. has the shape of Truncated Octahedron 



\section{Solved Example}

Example:. A hypothetical two-dimensional lattice has basis vector $\vec{a}_{1}=a \hat{x}$ and $\vec{a}_{2}=\frac{a}{2}(\hat{x}+\sqrt{3} \hat{y})$.

Find the area of its reciprocal lattice

Solution: $\quad \vec{a}_{1}=a \hat{x}$

$$
\vec{a}_{2}=\frac{a}{2}(\hat{x}+\sqrt{3} \hat{y})
$$

Assume, $\vec{a}_{3}=\hat{z}$

Now, $\vec{a}_{1} \cdot\left(\vec{a}_{2} \times \vec{a}_{3}\right)=a \hat{x} \cdot\left[\frac{a}{2}(\hat{x}+\sqrt{3} \hat{y}) \times \hat{z}\right]=\frac{a^{2}}{2} \hat{x} \cdot[-\hat{y}+\sqrt{3} \hat{x}]=\frac{a^{2}}{2}[0+\sqrt{3}]=\frac{\sqrt{3} a^{2}}{2}$

$\vec{a}_{1}^{*}=2 \pi \frac{\vec{a}_{2} \times \vec{a}_{3}}{\vec{a}_{1} \cdot\left(\vec{a}_{2} \times \vec{a}_{3}\right)}=2 \pi \frac{(\sqrt{3} \hat{x}-\hat{y}) \frac{a}{2}}{\frac{\sqrt{3}}{2} a^{2}}=\frac{2 \pi}{a}\left(\hat{x}-\frac{\hat{y}}{\sqrt{3}}\right)$ and $\vec{a}_{2}^{*}=2 \pi \frac{\vec{a}_{3} \times \vec{a}_{1}}{\vec{a}_{1} \cdot\left(\vec{a}_{2} \times \vec{a}_{3}\right)}=\frac{4 \pi}{\sqrt{3} a} \hat{y}$

Area of reciprocal cell is

$A^{*}=\vec{a}_{1}^{*} \times \vec{a}_{2}^{*}=\frac{2 \pi}{a}\left(\hat{x}-\frac{\hat{y}}{\sqrt{3}}\right) \times \frac{4 \pi}{\sqrt{3} a} \hat{y}=\frac{8 \pi^{2}}{\sqrt{3} a^{2}}(\hat{z})$

Example: The two-dimensional lattice of Grapheme is arranged in a Honeycomb lattice where carbon occupy the vertices as shown below. The positive vectors are $\vec{a}$ and $\vec{a}_{2}$. The lattice spacing is $a$. Find the area of the Brilliouin zone


Solution: The primitive vectors are written as

$$
\begin{aligned}
&\vec{a}_{1}=\sqrt{3} a \cos 30^{\circ} \hat{i}+\sqrt{3} a \cos 60^{\circ} \hat{j}=\frac{\sqrt{3}}{2} a(\sqrt{3} \hat{i}+\hat{j}) \\
&\vec{a}_{2}=\sqrt{3} a \cos 30^{\circ} \hat{i}-\sqrt{3} a \cos 60^{\circ} \hat{j}=\frac{\sqrt{3}}{2} a(\sqrt{3} \hat{i}-\hat{j})
\end{aligned}
$$





area of the primitive cell is

$$
A=\left|\vec{a}_{1} \times \vec{a}_{2}\right|=\frac{3 \sqrt{3}}{2} a^{2}
$$

The reciprocal lattice vectors are (assume $\vec{a}_{3}=\hat{k}$ )

$$
\begin{aligned}
&\vec{a}_{1}^{*}=\frac{2 \pi}{a} \frac{\vec{a}_{2} \times \vec{a}_{3}}{V}=\frac{2 \pi}{3 a}(\hat{i}+\sqrt{3} \hat{j}) \\
&\vec{a}_{2}^{*}=\frac{2 \pi}{a} \frac{\vec{a}_{3} \times \vec{a}_{1}}{V}=\frac{2 \pi}{3 a}(\hat{i}-\sqrt{3} \hat{j})
\end{aligned}
$$

area of the Brilliouin zone is

$$
A^{*}=\left|\vec{a}_{1}^{*} \times \vec{a}_{2}^{*}\right|=\frac{4 \pi^{2}}{9 a^{2}}|(\hat{i}+\sqrt{3} \hat{j}) \times(\hat{i}-\sqrt{3} \hat{j})|=\frac{4 \pi^{2}}{9 a^{2}} \cdot 2 \sqrt{3}=\frac{8 \pi^{2}}{3 \sqrt{3} a^{2}}
$$

Example: The lattice vector of graphene can be written as

$$
\vec{a}=\frac{3 a}{2}\left(\hat{x}+\frac{1}{\sqrt{3}} \hat{y}\right): \vec{b}=\frac{3 a}{2}\left(\hat{x}-\frac{1}{\sqrt{3}} \hat{y}\right)
$$

where the carbon-carbon distance is $a=1 \cdot 42 \AA$. Find the area of the Brillouin zone is

Solution: Given, $\vec{a}=\frac{3 a}{2}\left(\hat{x}+\frac{1}{\sqrt{3}} \hat{y}\right) ; \quad \vec{b}=\frac{3 a}{2}\left(\hat{x}-\frac{1}{\sqrt{3}} \hat{y}\right)$

Assume, $\vec{c}=\hat{k}$

The reciprocal lattice vectors are

$$
\overrightarrow{a^{*}}=2 \pi \frac{\vec{b} \times \vec{c}}{\vec{a} \cdot(\vec{b} \times \vec{c})}=\frac{2 \pi}{3 a}(\hat{x}+\sqrt{3} \hat{y}) \text { and } \overrightarrow{b^{*}}=2 \pi \frac{\vec{c} \times \vec{a}}{\vec{a} \cdot(\vec{b} \times \vec{c})}=\frac{2 \pi}{3 a}(\hat{x}+\sqrt{3} \hat{y})
$$

The area of the Brillouin zone is

$$
\begin{aligned}
&A=\overrightarrow{a^{*} \times b^{*}}=\left(\frac{2 \pi}{3 a}\right)^{2}[(\hat{x}-\sqrt{3} \hat{y}) \times(\hat{x}-\sqrt{3} \hat{y})] \\
&=\left(\frac{2 \pi}{3 a}\right)^{2}[0-\sqrt{3} \hat{z}-\sqrt{3} \hat{z}]=\left(\frac{2 \pi}{3 a}\right)^{2}[-2 \sqrt{3} \hat{z}] \\
&\text { Area, }|A|=\left(\frac{2 \pi}{3 a}\right)^{2}(2 \sqrt{3})=\frac{8 \pi^{2}}{3 \sqrt{3} a^{2}}=7 \cdot 54(\hat{A})^{-2}
\end{aligned}
$$