\chapter{Nuclear force-solutions}
\begin{abox}
	Practice set 1 solutions
	\end{abox}
\begin{enumerate}
	\item The range of the nuclear force between two nucleons due to the exchange of pions is $1.40 \mathrm{fm}$. If the mass of pion is $140 \mathrm{MeV} / \mathrm{c}^{2}$ and the mass of the rho-meson is $770 \mathrm{MeV} / \mathrm{c}^{2}$, then the range of the force due to exchange of rho-mesons is
	{\exyear{NET JUNE 2017}}
	\begin{tasks}(2)
		\task[\textbf{A.}](a) $1.40 \mathrm{fm}$
		\task[\textbf{B.}]$7.70 \mathrm{fm}$
		\task[\textbf{C.}]$0.25 \mathrm{fm}$
		\task[\textbf{D.}]$0.18 \mathrm{fm}$
	\end{tasks}
\begin{answer}
	Range for nuclear force between nucleon will be $R=c \Delta t=\frac{\hbar c}{m c^{2}}$ and $\hbar c=199 \mathrm{MeVfm}$\\
	$\Rightarrow R=\frac{199 \mathrm{MeVfm}}{770 \frac{\mathrm{MeV}}{c^{2}} \times c^{2}} \approx 0.25 \mathrm{fm}$
\end{answer}
\end{enumerate}
\newpage
\begin{abox}
	Practice set 2 solutions
	\end{abox}
\begin{enumerate}
\item The ground state wavefunction of deuteron is in a superposition of $s$ and $d$ states. Which of the following is NOT true as a consequence?
{\exyear{GATE 2010}}
\begin{tasks}(2)
	\task[\textbf{A.}] It has a non-zero quadruple moment 
	\task[\textbf{B.}]The neutron-proton potential is non-central
	\task[\textbf{C.}] The orbital wavefunction is not spherically symmetric
	\task[\textbf{D.}]The Hamiltonian does not conserve the total angular momentum
\end{tasks}
\begin{answer}
	The correct option is \textbf{(d)}
\end{answer}
\item Deuteron has only one bound state with spin parity $1^{+}$, isospin 0 and electric quadrupole moment $0.286 \mathrm{efm}^{2}$. These data suggest that the nuclear forces are having
{\exyear{GATE 2012}}
\begin{tasks}(2)
	\task[\textbf{A.}] only spin and isospin dependence
	\task[\textbf{B.}] no spin dependence and no tensor components
	\task[\textbf{C.}]spin dependence but no tensor components
	\task[\textbf{D.}]spin dependence along with tensor components
\end{tasks}
\begin{answer}
The correct option is \textbf{(d)}	
\end{answer}
\item Which of the following statements is NOT correct?
{\exyear{GATE 2016}}
\begin{tasks}(2)
	\task[\textbf{A.}] A deuteron can be disintegrated by irradiating it with gamma rays of energy $4 \mathrm{MeV}$.
	\task[\textbf{B.}] A deuteron has no excited states.
	\task[\textbf{C.}] A deuteron has no electric quadrupole moment.
	\task[\textbf{D.}] The ${ }^{1} S_{0}$ state of deuteron cannot be formed.
\end{tasks}
\begin{answer}
	The correct option is \textbf{(c)}
\end{answer}
\end{enumerate}