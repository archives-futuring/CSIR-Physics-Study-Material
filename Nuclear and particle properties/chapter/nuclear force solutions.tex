\chapter{Nuclear Force-Solutions}
\begin{abox}
	Practice Set 1 Solutions
	\end{abox}
\begin{enumerate}
	\item The range of the nuclear force between two nucleons due to the exchange of pions is $1.40 \mathrm{fm}$. If the mass of pion is $140 \mathrm{MeV} / \mathrm{c}^{2}$ and the mass of the rho-meson is $770 \mathrm{MeV} / \mathrm{c}^{2}$, then the range of the force due to exchange of rho-mesons is
	{\exyear{NET JUNE 2017}}
	\begin{tasks}(2)
		\task[\textbf{A.}] $1.40 \mathrm{fm}$
		\task[\textbf{B.}]$7.70 \mathrm{fm}$
		\task[\textbf{C.}]$0.25 \mathrm{fm}$
		\task[\textbf{D.}]$0.18 \mathrm{fm}$
	\end{tasks}
\begin{answer}
	Range for nuclear force between nucleon will be $R=c \Delta t=\frac{\hbar c}{m c^{2}}$ and $\hbar c=199 \mathrm{MeVfm}$\\
	$\Rightarrow R=\frac{199 \mathrm{MeVfm}}{770 \frac{\mathrm{MeV}}{c^{2}} \times c^{2}} \approx 0.25 \mathrm{fm}$
\end{answer}
	\item  The reaction ${ }_1^2 D+{ }_1^2 D \rightarrow{ }_2^4 H e+\pi^0$ cannot proceed via strong interactions because it violates the conservation of
	{\exyear{ NET/JRF (JUNE-2015)}}
	\begin{tasks}(2)
		\task[\textbf{a.}]Angular momentum
		\task[\textbf{b.}]Electric charge
		\task[\textbf{c.}]Baryon number
		\task[\textbf{d.}]Isospin 
	\end{tasks}
	\begin{answer}
		\begin{align*}
		&{ }_1 D^2+{ }_1 D^2 \rightarrow{ }_2 \mathrm{He}^4+\pi^0\quad
		\text{(Not conserved)}\\
		&I: 0 \qquad 0 \rightarrow\quad 0 \qquad 1\\
		&\text{This isopin is not conserved in above reaction.}
		\end{align*}
	\end{answer}
	\item  A deuteron $d$ captures a charged pion $\pi^{-}$in the $l=1$ state, and subsequently decays into a pair of neutrons $(n)$ via strong interaction. Given that the intrinsic parities of $\pi^{-}, d$ and $n$ are $-1,+1$ and $+1$ respectively, the spin wavefunction of the final state neutrons is
	{\exyear{NET/JRF (JUNE-2018)}}
	\begin{tasks}(2)
		\task[\textbf{a.}]Linear combination of a singlet and a triplet
		\task[\textbf{b.}]Singlet
		\task[\textbf{c.}]Triplet
		\task[\textbf{d.}]Doublet 
	\end{tasks}
	\begin{answer}
		\begin{align*}
		&\text{ Parity must conserve intersections}\\
		&\pi+d \rightarrow n+n\\
		&\text{The parity of the initial state is}\\
		&(-1)^l P_\pi P_d=(-1)^1(-1)(+1)=+1\\
		&\text{The parity of the final state is}\\
		&(-1)^l P_n P_n=(-1)^l(+1)(+1)=(-1)^l=1\qquad \because l=0,2, \ldots .\\
		&\text{word or phrase}
		\end{align*}
		So the correct answer is \textbf{Option (b)}
	\end{answer}
	\item  The strong nuclear force between a neutron and a proton in a zero orbital angular momentum state is denoted by $F_{n p}(r)$, where $r$ is the separation between them. Similarly, $F_{n n}(r)$ and $F_{p p}(r)$ denote the forces between a pair of neutrons and protons, respectively, in zero orbital momentum state. Which of the following is true on average if the inter-nucleon distance is $0.2 \mathrm{fm}<r<2 \mathrm{fm}$ ?
	\begin{tasks}(1)
		\task[\textbf{a.}]$F_{n p}$ is attractive for triplet spin state, and $F_{n n}, F_{p p}$ are always repulsive
		\task[\textbf{b.}]$F_{n n}$ and $F_{n p}$ are always attractive and $F_{p p}$ is repulsive in the triplet spin state
		\task[\textbf{c.}]$F_{p p}$ and $F_{n p}$ are always attractive and $F_{n n}$ is always repulsive
		\task[\textbf{d.}]All three forces are always attractive 
	\end{tasks}
	\begin{answer}
		Inside the nucleus the interaction between neutron neutron and newtran-proton is always attractive due to nuclear force whereas between proton-proton it is repulusive due to coulombic interaction:
		Thus $F_{n n}$ and $F_{n p}$ are always attractive and $F_{p p}$ is repulsive\\
		So the correct answer is \textbf{Option (b)}
	\end{answer}
\end{enumerate}
\colorlet{ocre1}{ocre!70!}
\colorlet{ocrel}{ocre!30!}
\setlength\arrayrulewidth{1pt}
\begin{table}[H]
	\centering
	\arrayrulecolor{ocre}
	\begin{tabular}{|p{1.5cm}|p{1.5cm}||p{1.5cm}|p{1.5cm}|}
		\hline
		\multicolumn{4}{|c|}{\textbf{Answer key}}\\\hline\hline
		\rowcolor{ocrel}Q.No.&Answer&Q.No.&Answer\\\hline
		1&\textbf{0.25} &2&\textbf{}\\\hline 
		3&\textbf{b} &4&\textbf{b} \\\hline
		5&\textbf{} &6&\textbf{} \\\hline
		
	\end{tabular}
\end{table}















\newpage
\begin{abox}
	Practice Set-2 Solutions
\end{abox}
\begin{enumerate}
	\item The ground state wavefunction of deuteron is in a superposition of $s$ and $d$ states. Which of the following is NOT true as a consequence?
	{\exyear{GATE 2010}}
	\begin{tasks}(1)
		\task[\textbf{A.}] It has a non-zero quadruple moment 
		\task[\textbf{B.}]The neutron-proton potential is non-central
		\task[\textbf{C.}] The orbital wavefunction is not spherically symmetric
		\task[\textbf{D.}]The Hamiltonian does not conserve the total angular momentum
	\end{tasks}
	\begin{answer}
		So the correct answer is \textbf{Option (d)}
	\end{answer}
	\item Deuteron has only one bound state with spin parity $1^{+}$, isospin 0 and electric quadrupole moment $0.286 \mathrm{efm}^{2}$. These data suggest that the nuclear forces are having
	{\exyear{GATE 2012}}
	\begin{tasks}(1)
		\task[\textbf{A.}] Only spin and isospin dependence
		\task[\textbf{B.}] No spin dependence and no tensor components
		\task[\textbf{C.}]Spin dependence but no tensor components
		\task[\textbf{D.}]Spin dependence along with tensor components
	\end{tasks}
	\begin{answer}
		So the correct answer is \textbf{Option (d)}
	\end{answer}
	\item Which of the following statements is NOT correct?
	{\exyear{GATE 2016}}
	\begin{tasks}(1)
		\task[\textbf{A.}] A deuteron can be disintegrated by irradiating it with gamma rays of energy $4 \mathrm{MeV}$.
		\task[\textbf{B.}] A deuteron has no excited states.
		\task[\textbf{C.}] A deuteron has no electric quadrupole moment.
		\task[\textbf{D.}] The ${ }^{1} S_{0}$ state of deuteron cannot be formed.
	\end{tasks}	
	\begin{answer}
		So the correct answer is \textbf{Option (c)}
	\end{answer}
\end{enumerate}
\colorlet{ocre1}{ocre!70!}
\colorlet{ocrel}{ocre!30!}
\setlength\arrayrulewidth{1pt}
\begin{table}[H]
	\centering
	\arrayrulecolor{ocre}
	\begin{tabular}{|p{1.5cm}|p{1.5cm}||p{1.5cm}|p{1.5cm}|}
		\hline
		\multicolumn{4}{|c|}{\textbf{Answer key}}\\\hline\hline
		\rowcolor{ocrel}Q.No.&Answer&Q.No.&Answer\\\hline
		1&\textbf{d} &2&\textbf{d}\\\hline 
		3&\textbf{c} &4&\textbf{} \\\hline
		5&\textbf{} &6&\textbf{} \\\hline
		
	\end{tabular}
\end{table}