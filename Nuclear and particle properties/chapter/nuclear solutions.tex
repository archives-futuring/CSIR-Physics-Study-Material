\chapter{Practice set solutions-nuclear}
\begin{abox}
	Practice set 1 solutions
	\end{abox}
\begin{enumerate}
	\item The radius of a ${ }_{29}^{64} \mathrm{Cu}$ nucleus is measured to be $4.8 \times 10^{-13} \mathrm{~cm}$.
	{\exyear{NET JUNE 2011}}\\
(A) The radius of a ${ }_{12}^{27} \mathrm{Mg}$ nucleus can be estimated to be
\begin{tasks}(2)
	\task[\textbf{A.}] $2.86 \times 10^{-13} \mathrm{~cm}$
	\task[\textbf{B.}]$5.2 \times 10^{-13} \mathrm{~cm}$
	\task[\textbf{C.}] $3.6 \times 10^{-13} \mathrm{~cm}$
	\task[\textbf{D.}]$8.6 \times 10^{-13} \mathrm{~cm}$
\end{tasks}
\begin{answer}
\begin{align*}
R&=R_{0}(A)^{1 / 3} \Rightarrow \frac{R_{M g}}{R_{C u}}=\left(\frac{A_{M g}}{A_{C u}}\right)^{1 / 3}=\left(\frac{27}{64}\right)^{1 / 3}\\
\Rightarrow \frac{R_{M g}}{R_{C u}}&=\frac{3}{4} \Rightarrow R_{M g}=\frac{3}{4} \times 4.8 \times 10^{-13}=3.6 \times 10^{-13} \mathrm{~cm}
\end{align*}
The correct option is \textbf{(c)}	
\end{answer}
(B) The root-mean-square (r.m.s) energy of a nucleon in a nucleus of atomic number $A$ in its ground state varies as:
\begin{tasks}(2)
	\task[\textbf{A.}] $A^{4 / 3}$
	\task[\textbf{B.}]$A^{1 / 3}$
	\task[\textbf{C.}] $A^{-1 / 3}$
	\task[\textbf{D.}] $A^{-2 / 3}$
\end{tasks}
\begin{answer}
	The correct option is \textbf{(c)}
\end{answer}
	\item According to the shell model the spin and parity of the two nuclei ${ }_{51}^{125} S b$ and ${ }_{38}^{89} \mathrm{Sr}$ are, respectively,
	{\exyear{NET DEC 2011}}\\
\begin{tasks}(2)
	\task[\textbf{A.}] $\left(\frac{5}{2}\right)^{+}$and $\left(\frac{5}{2}\right)^{+}$
	\task[\textbf{B.}]$\left(\frac{5}{2}\right)^{+}$and $\left(\frac{7}{2}\right)^{+}$
	\task[\textbf{C.}]$\left(\frac{7}{2}\right)^{+}$and $\left(\frac{5}{2}\right)^{+}$
	\task[\textbf{D.}]$\left(\frac{7}{2}\right)^{+}$and $\left(\frac{7}{2}\right)^{+}$
\end{tasks}
\begin{answer}
	${ }_{51}^{125} \mathrm{Sb} ; Z=51$ and $N=74$
	$$
	Z=51
	$$
	$$
	\left(s_{1 / 2}\right)^{2}\left(p_{3 / 2}\right)^{4}\left(p_{1 / 2}\right)^{2}\left(d_{5 / 2}\right)^{6}\left(s_{1 / 2}\right)^{2}\left(d_{3 / 2}\right)^{4}\left(f_{7 / 2}\right)^{8}\left(p_{3 / 2}\right)^{4}\left(f_{5 / 2}\right)^{6}\left(p_{1 / 2}\right)^{2}\left(g_{9 / 2}\right)^{10}\left(g_{7 / 2}\right)^{1}
	$$
	$\Rightarrow j=\frac{7}{2}$ and $l=4$. Thus spin and parity $=\left(\frac{7}{2}\right)^{+}$\\
	\begin{align*}
		&{ }_{38}^{89} S r ; Z=38 \text { and } N=51 \\
		&N=51: \\
		&\left(s_{1 / 2}\right)^{2}\left(p_{3 / 2}\right)^{4}\left(p_{1 / 2}\right)^{2}\left(d_{5 / 2}\right)^{6}\left(s_{1 / 2}\right)^{2}\left(d_{3 / 2}\right)^{4}\left(f_{7 / 2}\right)^{8}\left(p_{3 / 2}\right)^{4}\left(f_{5 / 2}\right)^{6}\left(p_{1 / 2}\right)^{2}\left(g_{9 / 2}\right)^{10}\left(g_{7 / 2}\right)^{1} \\
		&\Rightarrow j=\frac{7}{2} \text { and } l=4 \text {. Thus spin and parity }=\left(\frac{7}{2}\right)^{+}
	\end{align*}
	The correct option is \textbf{(d)}
\end{answer}
	\item The difference in the Coulomb energy between the mirror nuclei ${ }_{24}^{49} \mathrm{Cr}$ and ${ }_{25}^{49} \mathrm{Mn}$ is 6.0 MeV. Assuming that the nuclei have a spherically symmetric charge distribution and that $e^{2}$ is approximately $1.0 \mathrm{MeV}-\mathrm{fm}$, the radius of the ${ }_{25}^{49} \mathrm{Mn}$ nucleus is
{\exyear{NET DEC 2011}}\\
\begin{tasks}(2)
	\task[\textbf{A.}] $4.9 \times 10^{-13} \mathrm{~m}$
	\task[\textbf{B.}]$4.9 \times 10^{-15} \mathrm{~m}$
	\task[\textbf{C.}]$5.1 \times 10^{-13} \mathrm{~m}$
	\task[\textbf{D.}]$5.1 \times 10^{-15} \mathrm{~m}$
\end{tasks}
\begin{answer}
	$$
	R=\frac{3 e^{2}}{5 \cdot \Delta W}\left(Z_{1}^{2}-Z_{2}^{2}\right)=\frac{3 \times 1 \times 10^{-15}}{5 \times 6}\left(25^{2}-24^{2}\right)=4.9 \times 10^{-15} \mathrm{~m}
	$$
	The correct option is \textbf{(b)}
\end{answer}
	\item The ground state of ${ }_{12}^{207} \mathrm{~Pb}$ nucleus has spin-parity $J^{p}=\frac{1^{-}}{2}$, while the first excited state has $J^{p}=\frac{5^{-}}{2}$.The electromagnetic radiation emitted when the nucleus makes a transition from the first excited state to ground state are
{\exyear{NET JUNE 2012}}\\
\begin{tasks}(2)
	\task[\textbf{A.}] E2 and E3
	\task[\textbf{B.}] M2 or E3
	\task[\textbf{C.}] E2 or M3
	\task[\textbf{D.}] M2 or M3
\end{tasks}
\begin{answer}
 No parity change; $\Delta J=2,3$\\
For $E_{l}$ type, $\Delta \pi=(-1)^{l}$, (for no parity change $l=2$ )\\
For $M_{l}$ type, $\Delta \pi=(-1)^{l+1}$, (for no parity change $l=3$ )\\
$\Delta J=2$, No parity change $\rightarrow E 2 ; \Delta J=3$, No parity change $\rightarrow M 3$\\
The correct option is \textbf{(c)}	
\end{answer}
\item The binding energy of a light nucleus $(Z, A)$ in $\mathrm{MeV}$ is given by the approximate formula
$$
B(A, Z) \approx 16 A-20 A^{2 / 3}-\frac{3}{4} Z^{2} A^{-1 / 3}+30 \frac{(N-Z)^{2}}{A}
$$
where $N=A-Z$ is the neutron number. The value of $Z$ of the most stable isobar for a given $A$ is
{\exyear{NET JUNE 2013}}
\begin{tasks}(2)
	\task[\textbf{A.}] $\frac{A}{2}\left(1-\frac{A^{2 / 3}}{160}\right)^{-1} \quad$  
	\task[\textbf{B.}]$\frac{A}{2}$
	\task[\textbf{C.}]$\frac{A}{2}\left(1-\frac{A^{2 / 3}}{120}\right)^{-1}$
	\task[\textbf{D.}]$\frac{A}{2}\left(1+\frac{A^{4 / 3}}{64}\right)^{-1}$
\end{tasks}
\begin{answer}
	$$\left.\frac{\partial B}{\partial Z}\right|_{Z=Z^{\prime}}=0 \Rightarrow Z^{\prime}=\frac{A}{2}\left(1-\frac{A^{2 / 3}}{160}\right)^{-1}$$
	The correct option is \textbf{(a)}
\end{answer}
\item The intrinsic electric dipole moment of a nucleus ${ }_{Z}^{A} X$
{\exyear{NET DEC 2013}}
\begin{tasks}(2)
	\task[\textbf{A.}] increases with $Z$, but independent of $A$
	\task[\textbf{B.}]decreases with $Z$, but independent of $A$
	\task[\textbf{C.}]is always zero
	\task[\textbf{D.}]increases with $Z$ and $A$
\end{tasks}
\begin{answer}
	The correct option is \textbf{(d)}
\end{answer}
\item According to the shell model, the total angular momentum (in units of $\hbar$ ) and the parity of the ground state of the ${ }_{3}^{7} L i$ nucleus is
{\exyear{NET DEC 2013}}
\begin{tasks}(2)
	\task[\textbf{A.}] $\frac{3}{2}$ with negative parity
	\task[\textbf{B.}] $\frac{3}{2}$ with positive parity
	\task[\textbf{C.}]$\frac{1}{2}$ with positive parity
	\task[\textbf{D.}]$\frac{7}{2}$ with negative parity
\end{tasks}
\begin{answer}
 $Z=3, N=4$\\
	For odd $Z=3 ;\left(s_{1 / 2}^{2}\right)\left(p_{3 / 2}^{1}\right) \Rightarrow j=3 / 2, l=1$ and parity $=(-1)^{1}=-1$.
\end{answer}
\item If the binding energy $B$ of a nucleus (mass number $A$ and charge $Z$ ) is given by
$$
B=a_{V} A-a_{S} A^{2 / 3}-a_{s y m} \frac{(2 Z-A)^{2}}{A}+\frac{a_{C} Z^{2}}{A^{1 / 3}}
$$
where $a_{V}=16 \mathrm{MeV}, a_{S}=16 \mathrm{MeV}, a_{s y m}=24 \mathrm{MeV}$ and $a_{C}=0.75 \mathrm{MeV}$, then for the most stable isobar for a nucleus with $A=216$ is
{\exyear{NET DEC 2014}}
\begin{tasks}(2)
	\task[\textbf{A.}] 68 
	\task[\textbf{B.}]72
	\task[\textbf{C.}]84
	\task[\textbf{D.}]92
\end{tasks}
\begin{answer}
	\begin{align*}
	&\text { For the most stable isobar for a nucleus }\\
	\frac{d B}{d Z}&=0 \Rightarrow-a_{s y m} \frac{2(2 Z-A) \times 2}{A}+\frac{2 a_{C} Z}{A^{1 / 3}}=0\\
	&\Rightarrow 24 \frac{2(2 Z-216) \times 2}{216}+0.75 \frac{2 Z}{(216)^{1 / 3}}=0 \Rightarrow \frac{4(2 Z-216)}{9}+\frac{3}{4} \frac{2 Z}{6}=0\\
	&\Rightarrow \frac{4(2 Z-216)}{9}+\frac{Z}{4}=0 \Rightarrow 16(2 Z-216)+9 Z=0 \Rightarrow 41 Z=216 \times 16 \Rightarrow Z=82.3
	\end{align*}
	The correct option is \textbf{(c)}
\end{answer}
\item Let us approximate the nuclear potential in the shell model by a three dimensional isotropic harmonic oscillator. Since the lowest two energy levels have angular momenta $l=0$ and $l=1$ respectively, which of the following two nuclei have magic numbers of protons and neutrons?
{\exyear{NET JUNE 2015}}
\begin{tasks}(2)
	\task[\textbf{A.}] ${ }_{2}^{4} \mathrm{He}$ and ${ }_{8}^{16} \mathrm{O}$
	\task[\textbf{B.}]${ }_{1}^{2} D$ and ${ }_{4}^{8} B e$
	\task[\textbf{C.}]${ }_{2}^{4} \mathrm{He}$ and ${ }_{4}^{8} \mathrm{Be}$
	\task[\textbf{D.}]${ }_{2}^{4} \mathrm{He}$ and ${ }_{6}^{12} \mathrm{C}$
\end{tasks}
\begin{answer}
	${ }_{2} H e^{4}$ has $Z=2, N=2$\\
	${ }_{8} O^{16}$ has $Z=8, N=8$ magic numbers $(2,8,20,28,50,82,126)$\\
	The correct option is \textbf{(a)}
\end{answer}
\item Of the nuclei of mass number $A=125$, the binding energy calculated from the liquid drop model (given that the coefficients for the Coulomb and the asymmetry energy are $a_{c}=0.7 \mathrm{MeV}$ and $a_{s y m}=22.5 \mathrm{MeV}$ respectively) is a maximum for
{\exyear{NET DEC 2015}}
\begin{tasks}(2)
	\task[\textbf{A.}] ${ }_{54}^{125} \mathrm{Xe}$
	\task[\textbf{B.}] ${ }_{53}^{124} I$
	\task[\textbf{C.}]${ }_{52}^{125} \mathrm{Te}$
	\task[\textbf{D.}]${ }_{51}^{125} \mathrm{Sb}$
\end{tasks}
\begin{answer}
	\begin{align*}
	Z_{0}&=\frac{4 a_{a}+a_{c} A^{-1 / 3}}{2 a_{c} A^{-1 / 3}+8 a_{a} A^{-1}}=\frac{4 a_{a} A+a_{c} A^{2 / 3}}{8 a_{a}+2 a_{c} A^{2 / 3}}\\
	Z_{0}&=\frac{4 \times 22.5 \times 125+0.7\left(5^{3}\right)^{2 / 3}}{8 \times 22.5+2 \times 0.7\left(5^{3}\right)^{2 / 3}}\\
	&=\frac{11250+17.5}{180+35}=\frac{11267.5}{215}=52.4 \Rightarrow Z_{0} \approx 52
	\end{align*}
	The correct option is \textbf{(c)}
\end{answer}
\item A radioactive element $X$ decays to $Y$, which in turn decays to a stable element $Z$. The decay constant from $X$ to $Y$ is $\lambda_{1}$, and that from $Y$ to $Z$ is $\lambda_{2}$. If, to begin with, there are only $N_{0}$ atoms of $X$, at short times $\left(t \ll \frac{1}{\lambda_{1}}\right.$ as well as $\left.\frac{1}{\lambda_{2}}\right)$ the number of atoms of $Z$ will be
{\exyear{NET JUNE 2016}}
\begin{tasks}(2)
	\task[\textbf{A.}] $\frac{1}{2} \lambda_{1} \lambda_{2} N_{0} t^{2}$
	\task[\textbf{B.}]$\frac{\lambda_{1} \lambda_{2}}{2\left(\lambda_{1}+\lambda_{2}\right)} N_{0} t$
	\task[\textbf{C.}]$\left(\lambda_{1}+\lambda_{2}\right)^{2} N_{0} t^{2}$
	\task[\textbf{D.}] $\left(\lambda_{1}+\lambda_{2}\right) N_{0} t$
\end{tasks}
\begin{answer}
	\begin{align*}
	X \stackrel{\lambda_{1}}{\longrightarrow} Y \stackrel{\lambda_{2}}{\longrightarrow} Z\\
	\begin{array}{cccc}
	t=0 & N_{0} & 0 & 0 \\
	t & N_{1} & N_{2} & N_{3}
	\end{array}
	\end{align*}
	\begin{align*}
	&\text { Rate equations }\\
	N_{1}&=N_{0} e^{-\lambda_{1} t}, \frac{d N_{2}}{d t}=\lambda_{1} N_{1}-\lambda_{2} N_{2}, \frac{d N_{3}}{d t}=\lambda_{2} N_{2}\\
	N_{3}&=N_{0}\left[1+\frac{\lambda_{1} e^{-\lambda_{2} t}}{\left(\lambda_{2}-\lambda_{1}\right)}-\frac{\lambda_{2} e^{-\lambda_{1} t}}{\left(\lambda_{2}-\lambda_{1}\right)}\right]\\
	&=N_{0}\left[1+\frac{\lambda_{1}}{\left(\lambda_{2}-\lambda_{1}\right)}\left(1-\lambda_{2} t+\frac{\lambda_{2}^{2} t^{2}}{2}\right)-\frac{\lambda_{2}}{\left(\lambda_{2}-\lambda_{1}\right)}\left(1-\lambda_{1} t+\frac{\lambda_{1}^{2} t^{2}}{2}\right)\right]\\
	&=N_{0}\left[1+\frac{\lambda_{1}}{\left(\lambda_{2}-\lambda_{1}\right)}-\frac{\lambda_{1} \lambda_{2} t}{\left(\lambda_{2}-\lambda_{1}\right)}+\frac{\lambda_{1}}{\left(\lambda_{2}-\lambda_{1}\right)} \frac{\lambda_{2}^{2} t^{2}}{2}\right.\\
	&\left.-\frac{\lambda_{2}}{\left(\lambda_{2}-\lambda_{1}\right)}+\frac{\lambda_{2} \lambda_{1} t}{\left(\lambda_{2}-\lambda_{1}\right)}-\frac{\lambda_{2}}{\left(\lambda_{2}-\lambda_{1}\right)} \frac{\lambda_{1}^{2} t^{2}}{2}\right]\\
	&=N_{0}\left[\frac{\lambda_{1}}{\left(\lambda_{2}-\lambda_{1}\right)} \times \frac{\lambda_{2}^{2} t^{2}}{2}-\frac{\lambda_{2}}{\left(\lambda_{2}-\lambda_{1}\right)} \times \frac{\lambda_{1}^{2} t^{2}}{2}\right]\\
	&=\frac{\lambda_{1} \lambda_{2} t^{2}}{2} N_{0}\left[\frac{\lambda_{2}}{\lambda_{2}-\lambda_{1}}-\frac{\lambda_{1}}{\lambda_{2}-\lambda_{1}}\right]=\frac{1}{2} \lambda_{1} \lambda_{2} N_{0} t^{2}
	\end{align*}
	The correct option is \textbf{(a)}
\end{answer}
\item Let $E_{S}$ denotes the contribution of the surface energy per nucleon in the liquid drop model. The ratio $E_{S}\left({ }_{13}^{27} \mathrm{Al}\right): E_{S}\left({ }_{30}^{64} \mathrm{Zn}\right)$ is
{\exyear{NET JUNE 2016}}
\begin{tasks}(2)
	\task[\textbf{A.}](a) $2: 3$
	\task[\textbf{B.}] $4: 3$
	\task[\textbf{C.}] $5: 3$
	\task[\textbf{D.}] $3: 2$
\end{tasks}
\begin{answer}
	$$E_{S}=\frac{B}{A}=\frac{A^{\frac{2}{3}}}{A} \propto A^{-\frac{1}{3}} \Rightarrow \frac{E_{S}(A l)}{E_{S}\left(Z_{n}\right)}=\frac{(27)^{-\frac{1}{3}}}{(64)^{-\frac{1}{3}}}=\frac{(64)^{\frac{1}{3}}}{(27)^{\frac{1}{3}}}=\frac{4}{3}$$
	The correct option is \textbf{(b)}
\end{answer}
\item According to the shell model, the nuclear magnetic moment of the ${ }_{13}^{27} \mathrm{Al}$ nucleus is (Given that for a proton $g_{l}=1, g_{s}=5.586$, and for a neutron $g_{l}=0, g_{s}=-3.826$ )
{\exyear{NET JUNE 2016}}
\begin{tasks}(2)
	\task[\textbf{A.}] $-1.913 \mu_{N}$
	\task[\textbf{B.}]$14.414 \mu_{N}$
	\task[\textbf{C.}]$4.793 \mu_{N}$
	\task[\textbf{D.}]0
\end{tasks}
\begin{answer}
$${ }_{13} A l^{27}: Z=13, N=14 \text { for } Z=13, S_{1 / 2}^{2}, P_{3 / 2}^{4}, P_{1 / 2}^{2}, d_{5 / 2}^{5} \Rightarrow j=\frac{5}{2}, l=2$$
$$
\text { Magnetic moment, } \mu=\frac{1}{2}\left[2 j-1+g_{S}\right] \mu_{N}=\frac{1}{2}\left[2 \times \frac{5}{2}-1+5.586\right] \mu_{N} \Rightarrow \mu=4.793 \mu_{N}
$$
The correct option is \textbf{(c)}	
\end{answer}
\item The spin-parity assignments for the ground and first excited states of the isotope ${ }_{28}^{57} N i$, in the single particle shell model, are
{\exyear{NET DEC 2017}}
\begin{tasks}(2)
	\task[\textbf{A.}] $\left(\frac{1}{2}\right)^{-}$and $\left(\frac{3}{2}\right)^{-}$
	\task[\textbf{B.}]$\left(\frac{5}{2}\right)^{+}$and $\left(\frac{7}{2}\right)^{+}$
	\task[\textbf{C.}]$\left(\frac{3}{2}\right)^{+}$and $\left(\frac{5}{2}\right)^{+}$
	\task[\textbf{D.}]$\left(\frac{3}{2}\right)^{-}$and $\left(\frac{5}{2}\right)^{-}$
\end{tasks}
\begin{answer}
	For ${ }_{28} N i^{57}: \quad P=28, N=29 \rightarrow$ will decide the $j^{P}$\\
	So, for $N=29$, ground state configuration,\\\\
	$1 s_{1 / 2}^{2} 1 p_{3 / 2}^{4} 1 p_{1 / 2}^{2} 1 d_{5 / 2}^{6} 2 s_{1 / 2}^{2} 1 d_{3 / 2}^{4} 1 f_{7 / 2}^{8} 2 p_{3 / 2}^{1}$\\\\
	So, $j=\frac{3}{2}, l=1$\\
	Spin parity for ground state of ${ }_{28} N i^{57} \rightarrow\left(\frac{3}{2}\right)^{-}$\\
	For first excited state,\\
	\begin{align*}
		&1 s_{1 / 2}^{2} 1 p_{3 / 2}^{4} 1 p_{1 / 2}^{2} 1 d_{5 / 2}^{6} 2 s_{1 / 2}^{2} 1 d_{3 / 2}^{4} 1 f_{7 / 2}^{8} 2 p_{3 / 2}^{1} \rightarrow 1 f_{5 / 2} \\
		&P=\frac{5}{2}, l=3 \Rightarrow \text { spin parity } \rightarrow\left(\frac{5}{2}\right)^{-}
	\end{align*}
The correct option is \textbf{(d)}
\end{answer}
\end{enumerate}
\newpage
\begin{abox}
	Practice set 2 solutions
	\end{abox}
\begin{enumerate}
\item In the nuclear shell model the spin parity of ${ }_{7}^{15} N$ is given by
{\exyear{GATE 2010}}
\begin{tasks}(2)
	\task[\textbf{A.}] $\frac{1^{-}}{2}$
	\task[\textbf{B.}]$\frac{1^{+}}{2}$
	\task[\textbf{C.}]$\frac{3^{-}}{2}$
	\task[\textbf{D.}]$\frac{3^{+}}{2}$
\end{tasks}
\begin{answer}
	$$
	Z=7 ;\left(s_{1 / 2}\right)^{2}\left(p_{3 / 2}\right)^{4}\left(p_{1 / 2}\right)^{1} \text { and } N=8
	$$
	$$
	l=1, J=\frac{1}{2} \Rightarrow \text { parity }=(-1)^{1}=-1, \quad \text { spin }-\text { parity }=\left(\frac{1}{2}\right)^{-}
	$$
	The correct option is \textbf{(a)}
\end{answer}
\item The semi-empirical mass formula for the binding energy of nucleus contains a surface correction term. This term depends on the mass number $A$ of the nucleus as
{\exyear{GATE 2011}}
\begin{tasks}(2)
	\task[\textbf{A.}] $A^{-1 / 3}$
	\task[\textbf{B.}]$A^{1 / 3}$
	\task[\textbf{C.}]$A^{2 / 3}$
	\task[\textbf{D.}]$A$
\end{tasks}
\begin{answer}
	The correct option is \textbf{(c)}
\end{answer}
\item According to the single particles nuclear shell model, the spin-parity of the ground state of ${ }_{8}^{17} O$ is
{\exyear{GATE 2011}}
\begin{tasks}(2)
	\task[\textbf{A.}] $\frac{1}{2}$
	\task[\textbf{B.}]$\frac{3}{2}$
	\task[\textbf{C.}]$\frac{3}{2}^{+}$
	\task[\textbf{D.}] $\frac{5^{+}}{2}$
\end{tasks}
\begin{answer}
	$$Z=8 \text { and } N=9 ;\left(s_{1 / 2}\right)^{2}\left(p_{3 / 2}\right)^{4}\left(p_{1 / 2}\right)^{2}\left(d_{5 / 2}\right)^{1}$$
	$$
	l=2, J=\frac{5}{2} \Rightarrow \text { parity }=(-1)^{2}=+1, \quad \text { spin - parity }=\left(\frac{5}{2}\right)^{+}
	$$
	The correct option is \textbf{(d)}
\end{answer}
\item In the nuclear shell model, the potential is modeled as $V(r)=\frac{1}{2} m \omega^{2} r^{2}-\lambda \vec{L} \cdot \vec{S}, \lambda>0$ The correct spin-parity and isospin assignments for the ground state of ${ }_{6}^{13} \mathrm{C}$ is
{\exyear{GATE 2015}}
\begin{tasks}(2)
	\task[\textbf{A.}] $\frac{1^{-}}{2} ; \frac{-1}{2}$
	\task[\textbf{B.}]$\frac{1^{+}}{2} ; \frac{-1}{2}$
	\task[\textbf{C.}] $\frac{3^{+}}{2} ; \frac{1}{2}$
	\task[\textbf{D.}]$\frac{3^{-}}{2} ; \frac{-1}{2}$
\end{tasks}
\begin{answer}
	$$
	{ }^{13} C_{6}, \quad N=7, Z=6, \text { for } N=7 ; \quad\left(1 S_{\frac{1}{2}}\right)^{2}\left(1 P_{\frac{3}{2}}\right)^{4}\left(P_{\frac{1}{2}}\right)^{1} \Rightarrow j=\frac{1}{2} \text { and } l=1
	$$
	$$
	\text { Thus spin- parity is }\left(\frac{1}{2}\right)^{-} \text {. }
	$$
\end{answer}
\item According to the nuclear shell model, the respective ground state spin-parity values of ${ }_{8}^{15} O$ and ${ }_{8}^{17} O$ nuclei are
{\exyear{GATE 2016}}
\begin{tasks}(2)
	\task[\textbf{A.}] $\frac{1^{+}}{2}, \frac{1^{-}}{2}$
	\task[\textbf{B.}]$\frac{1}{2}^{-}, \frac{5^{+}}{2}$
	\task[\textbf{C.}]$\frac{3^{-}}{2}, \frac{5^{+}}{2}$
	\task[\textbf{D.}]$\frac{3^{-}}{2}, \frac{1^{-}}{2}$
\end{tasks}
\begin{answer}
	${ }_{8}^{15} O ; Z=8$ and $N=7 ; \quad N=7:\left(s_{1 / 2}\right)^{2}\left(p_{3 / 2}\right)^{4}\left(p_{1 / 2}\right)^{1}$\\
	$\Rightarrow j=\frac{1}{2}$ and $l=1$. Thus spin and parity $=\left(\frac{1}{2}\right)^{-}$\\
	${ }_{8}^{17} O ; Z=8$ and $N=9 ; \quad N=9:\left(s_{1 / 2}\right)^{2}\left(p_{3 / 2}\right)^{4}\left(p_{1 / 2}\right)^{2}\left(d_{5 / 2}\right)^{1}$\\
	 $\Rightarrow j=\frac{5}{2}$ and $l=2$. Thus spin and parity $=\left(\frac{5}{2}\right)^{+}$
\end{answer}
\item An $\alpha$ particle is emitted by ${ }_{90}^{230} T h$ nucleus. Assuming the potential to be purely Coulombic beyond the point of separation, the height of the Coulomb barrier is $\mathrm{MeV}$ (up to two decimal places).
$$
\left(\frac{e^{2}}{4 \pi \epsilon_{0}}=1.44 \mathrm{MeV}-\mathrm{fm}, r_{0}=1.30 \mathrm{fm}\right)
$$
{\exyear{GATE 2018}}
\begin{answer}
The height of coulomb barrier for $\alpha$ particle from \\
${ }_{90} T h^{230} \rightarrow_{88} X^{226}+2 \mathrm{He}^{4}(\alpha$ - particle $)$
$$
V_{C}=\frac{1}{4 \pi \epsilon_{0}}\left(\frac{2 z e^{2}}{R}\right)
$$
Here, $R_{0}=1.3 \mathrm{fm}, \frac{e^{2}}{4 \pi \in_{0}}=1.44 \mathrm{MeV} \mathrm{fm}$
And $R=R_{0} A^{1 / 3}$\\
Here, we consider pure Coulombic interection
\begin{align*}
	&A_{T h}^{1 / 3}=A_{X}^{1 / 3}+A_{\alpha}^{1 / 3}=(226)^{1 / 3}+(4)^{1 / 3}=(6.09+1.58)=7.67 \\
	&R=R_{0} A_{T h}^{1 / 3}=1.3(7.67)
\end{align*}
Hence, $V_{C}=\left(\frac{e^{2}}{4 \pi \in_{0}}\right) \frac{2 \times 90}{1.3(7.67)}=\frac{180 \times 1.44}{1.3 \times 7.67} \frac{\mathrm{MeV}}{f \mathrm{fm}}$
$$
V_{C}=25.995 \mathrm{MeV}
$$	
\end{answer}
\item The nuclear spin and parity of ${ }_{20}^{40} \mathrm{Ca}$ in its ground state is
{\exyear{GATE 2019}}
\begin{tasks}(2)
	\task[\textbf{A.}] $0^{+}$
	\task[\textbf{B.}] $0^{-}$
	\task[\textbf{C.}] $1^{+}$
	\task[\textbf{D.}] $1^{-}$
\end{tasks}
\begin{answer}
	${ }_{20}^{40} \mathrm{Ca}$ is an even-even nuclei, therefore $I=0, P=+v e$\\
	Spin-parity $=0^{+}$\\
	The correct option is \textbf{(a)}
\end{answer}
\item A radioactive element $X$ has a half-life of 30 hours. It decays via alpha, beta and gamma emissions with the branching ratio for beta decay being $0.75$. The partial half-life for beta decay in unit of hours is
{\exyear{GATE 2019}}
\begin{answer}
 Branching ratio is the fraction of particles (here $\beta$ ) which decays by an individual decay mode with respect to the total number of particles which decays
 $$B R=\frac{\left(\frac{d N}{d t}\right)_{x}}{\left(\frac{d t}{d t}\right)_{\beta}}=\frac{\left(T_{1 / 2}\right)_{x}}{\left(T_{1 / 2}\right)_{\beta}} \Rightarrow\left(T_{1 / 2}\right)_{\beta}=\frac{\left(T_{1 / 2}\right)_{x}}{B R}=\frac{30}{0.75}=40 \text { hours }$$	
\end{answer}
\end{enumerate}
