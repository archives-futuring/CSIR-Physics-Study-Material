\chapter{Nuclear physics Solutions}
\begin{abox}
	Practise set-1
\end{abox}
\begin{enumerate}
	\item  The radius of a ${ }_{29}^{64} \mathrm{Cu}$ nucleus is measured to be $4.8 \times 10^{-13} \mathrm{~cm}$. The radius of a ${ }_{12}^{27} \mathrm{Mg}$ nucleus can be estimated to be
	\begin{tasks}(2)
		\task[\textbf{a.}]$2.86 \times 10^{-13} \mathrm{~cm}$
		\task[\textbf{b.}]$5.2 \times 10^{-13} \mathrm{~cm}$
		\task[\textbf{c.}] $3.6 \times 10^{-13} \mathrm{~cm}$
		\task[\textbf{d.}] $8.6 \times 10^{-13} \mathrm{~cm}$
	\end{tasks}
\begin{answer}
	\begin{align*}
\text{Since }R&=R_0(A)^{1 / 3} \Rightarrow \frac{R_{M g}}{R_{C u}}=\left(\frac{A_{M g}}{A_{C u}}\right)^{1 / 3}=\left(\frac{27}{64}\right)^{1 / 3}\\
	\Rightarrow \frac{R_{M g}}{R_{C u}}&=\frac{3}{4} \Rightarrow R_{M g}=\frac{3}{4} \times 4.8 \times 10^{-13}=3.6 \times 10^{-13} \mathrm{~cm} .
	\end{align*}
	So the correct answer is \textbf{Option (c)}
\end{answer}
	\item  The intrinsic electric dipole moment of a nucleus ${ }_Z^A X$
	\begin{tasks}(1)
		\task[\textbf{a.}]increases with $Z$, but independent of $A$
		\task[\textbf{b.}]decreases with $Z$, but independent of $A$
		\task[\textbf{c.}] is always zero
		\task[\textbf{d.}] increases with $Z$ and $A$
	\end{tasks}
\begin{answer}
So the correct answer is \textbf{Option (c)}
\end{answer}
	\item  In deep inelastic scattering electrons are scattered off protons to determine if a proton has any internal structure. The energy of the electron for this must be at least
	\begin{tasks}(2)
		\task[\textbf{a.}]$1.25 \times 10^9 \mathrm{eV}$
		\task[\textbf{b.}]$1.25 \times 10^{12} \mathrm{eV}$
		\task[\textbf{c.}] $1.25 \times 10^6 \mathrm{eV}$
		\task[\textbf{d.}] $1.25 \times 10^8 \mathrm{eV}$
	\end{tasks}
\begin{answer}
	\begin{align*}
	\intertext{ The internal structure of proton can only be determined if the wavelength of the incoming electron is nearly equal to the size of the proton}
&\text{	i.e. $\lambda=R=1.2 A^{1 / 3}(\mathrm{fm})=1.2 \mathrm{fm}=1.2 \times 10^{-15} \mathrm{~m}$}\\
	&\text { According to de-Broglie relation, } \lambda=\frac{h}{p}=\frac{h}{\sqrt{2 m E}}\\
&\text { This can be also written as } E^2=h^2 \lambda^2 / c^2+m_0^2 c^4
	\end{align*}
	So the correct answer is \textbf{Option (b)}
\end{answer}
	\item  The difference in the Coulomb energy between the mirror nuclei ${ }_{24}^{49} \mathrm{Cr}$ and ${ }_{25}^{49} \mathrm{Mn}$ is $6.0 \mathrm{MeV}$. Assuming that the nuclei have a spherically symmetric charge distribution and that $\frac{e^2}{4 \pi \varepsilon_0}$ is approximately $1.0 \mathrm{MeV}-f m$, the radius of the ${ }_{25}^{49} \mathrm{Mn}$ nucleus is
	(a) $4.9 \times 10^{-13} \mathrm{~m}$
	(b) $4.9 \times 10^{-15} \mathrm{~m}$
	(c) $5.1 \times 10^{-13} \mathrm{~m}$
	(d) $5.1 \times 10^{-15} \mathrm{~m}$
	\begin{answer}
		\begin{align*}
		R=\frac{3 e^2}{5 \cdot \Delta W}\left(Z_1^2-Z_2^2\right)=\frac{3 \times 1 \times 10^{-15}}{5 \times 6}\left(25^2-24^2\right)=4.9 \times 10^{-15} \mathrm{~m}
		\end{align*}
			So the correct answer is \textbf{Option (b)}
	\end{answer}
\end{enumerate}
\newpage
\begin{abox}
	Practise set-2
\end{abox}
\begin{enumerate}
	\item  Inside a large nucleus, a nucleon with mass $939 \mathrm{MeVc}^{-2}$ has Fermi momentum $1.40 \mathrm{fm}^{-1}$ at absolute zero temperature. Its velocity is $X c$, where the value of $X$ is ---------(up to two decimal places).
	$
	\text { ( } \hbar c=197 \mathrm{MeV}-\mathrm{fm})
	$
	\begin{answer}
		\begin{align*}
	&\text{ Here, Fermi-momentum or fermi radius, $k_F=1.40 \mathrm{fm}^{-1}$ and $\hbar c=197 \mathrm{Mev}-\mathrm{fm}$ }\\
&\text{	Now, Fermi velocity -}\\
	&	V_F=\frac{P}{m}=\frac{\hbar k_F}{m}=\frac{(\hbar c) k_F \cdot c}{m c^2}=\frac{(197) \times 1 \cdot 40 \times c}{939}=\frac{275 \cdot 8 c}{939}=0.29 c
		\end{align*}
		So the correct answer is \textbf{0.29}
	\end{answer}
	\item  The mean kinetic energy of a nucleon in a nucleus of atomic weight $A$ varies as $A^n$, where $n$ is ---------(upto two decimal places)
	\begin{answer}
		\begin{align*}
		\langle T\rangle&=\frac{\int_0^R-\frac{\hbar^2}{2 m}\left(\frac{d^2}{d r^2}+\frac{1}{r} \frac{d}{d r}\right) 4 \pi r^2 d r}{\int_0^R 4 \pi r^2 d r}=\frac{-\frac{\hbar^2}{2 m} 4 \pi \int_0^R(2+2) d r}{\int_0^R 4 \pi r^2 d r}=\frac{-\frac{\hbar^2}{2 m} 4 \pi \times 4 R}{4 \pi R^3 / 3}\\
		\Rightarrow\langle T\rangle \propto \frac{1}{R^2}&=\frac{1}{\left(R_0 A^{\frac{1}{3}}\right)^2}=\frac{1}{A^{\frac{2}{3}}}=A^{-\frac{2}{3}} \Rightarrow n=-\frac{2}{3}=-0.667=-0.67
		\end{align*}
			So the correct answer is \textbf{-0.67}
	\end{answer}
	\item  According to the Fermi gas model of nucleus, the nucleons move in a spherical volume of radius $R\left(=R_0 A^{\frac{1}{3}}\right.$, where $A$ is the mass number and $R_0$ is an empirical constant with the dimensions of length). The Fermi energy of the nucleus $E_F$ is proportional to
	\begin{tasks}(4)
		\task[\textbf{a.}]$R_0^2$
		\task[\textbf{b.}]$\frac{1}{R_0}$
		\task[\textbf{c.}]$\frac{1}{R_0^2}$
		\task[\textbf{d.}] $\frac{1}{R_0^3}$
	\end{tasks}
\begin{answer}
	\begin{align*}
	\text { Fermi energy } E_F&=\frac{\hbar^2}{2 m}\left(3 \pi^2 \frac{N}{V}\right)^{2 / 3}\\
	V&=\frac{4 \pi}{3} R^3=\frac{4 \pi}{3}\left(R_0 A^{1 / 3}\right)^3=\frac{4 \pi}{3} R_0^3 A \\
	\therefore\quad E_F&=\frac{\hbar^2}{2 m}\left(\frac{3 \pi^2 N}{\frac{4 \pi}{3} R_0^3 A}\right)^{2 / 3}=\frac{\hbar^2}{2 m}\left(\frac{9 \pi N}{4 A} \cdot \frac{1}{R_0^3}\right)^{2 / 3} \quad \Rightarrow E_F \propto \frac{1}{R_0^2}
	\end{align*}
	So the correct answer is \textbf{Option (c)}
\end{answer}
\item  The stable nucleus that has $\frac{1}{3}$ the radius of ${ }^{189} O s$ nucleus is,
\begin{tasks}(4)
	\task[\textbf{a.}]${ }^7 \mathrm{Li}$
	\task[\textbf{b.}] ${ }^{16} \mathrm{O}$
	\task[\textbf{c.}]${ }^4 \mathrm{He}$
	\task[\textbf{d.}] ${ }^{14} N$
\end{tasks}
\begin{answer}
	\begin{align*}
	R=\frac{1}{3} R_{O s} \Rightarrow R_0(A)^{1 / 3}=\frac{1}{3} R_0(189)^{1 / 3} \Rightarrow A=7
	\end{align*}
	So the correct answer is \textbf{Option (a)}
\end{answer}
\item  The binding energy of the $k$-shell electron in a Uranium atom $(Z=92, A=238)$ will be modified due to (i) screening caused by other electrons and (ii) the finite extent of the nucleus as follows:
\begin{tasks}(1)
	\task[\textbf{a.}]Increases due to (i), remains unchanged due to (ii)
	\task[\textbf{b.}]Decreases due to (i), decreases due to (ii)
	\task[\textbf{c.}]Increases due to (i), increases due to (ii)
	\task[\textbf{d.}] Decreases due to (i), remains unchanged due to (ii)
\end{tasks}
\begin{answer}
	So the correct answer is \textbf{Option (b)}
\end{answer}
\end{enumerate}