\chapter{MATHEMATICAL TOOLS FOR QUANTUM MECHANICS}
\section{The Linear Vector Space and Hilbert Space}
In quantum mechanics the state of a physical system is a vector in a complex vector space. Observables(eg: position, velocity, momentum,..etc.)are linear operators, in fact, Hermitian operators acting on this complex vector space
\subsection{The Linear Vector Space}
A linear vector space consists of two set of elements and two algebraic rules :
\begin{itemize}
	\item A set of vectors $\psi,\phi,\chi, \cdots $ and a set of scalars a,b,c....;
	\item A rule of vector addition and a rule of scalar multiplication
\end{itemize}
\textbf{(a) Addition Rule}\\
The addition rule has the properties and the structure of an abelian group:
\begin{itemize}
	\item If $\psi$ and $\phi$ are vectors of a space,their sum $\psi+\phi$ is also a vector of the same space.
	\item Commutativity :$\psi+\phi=\phi+\psi$
	\item Associativity:$(\psi+\phi)+\chi=\psi+(\phi+\chi)$
	\item The existance of a zero or neutral vector: for each vector $\psi$, there must be zero vector 0 such that : 0+$\psi=\psi+0=\psi$
	\item Existance of a symmetric or inverse vector : each vector $\psi$ must have a symmetric vector (-$\psi$) such that ($\psi$)+(-$\psi$)=(-$\psi$)+$\psi$=0.
\end{itemize}
\textbf{(b) Multiplication Rule}\\
The multiplication of a vector by scalars (scalars can be real or complex numbers) has these properties:
\begin{itemize}
	\item The product of a scalar with vector gives another vector.In general if $\psi$ and $\phi$ are the two vectors of the space,any linear combination $a\psi+b\phi$ is also a vector of the space. a and b being scalars.
	\item Distributivity with respect to addition\\
	$a(\psi+\phi)=a\psi+a\phi$\\
	$(a+b)\psi=a\psi+b\psi$
	\item Associativity with respect to multiplication of scalars:\\
	$a(b\psi)=(ab)\psi$
	\item For each element $\psi$ there must exist a unitary scalar I and a zero scalar "0" such that \\
	$I\psi=\psi I =\psi$ and $0\psi=\psi0=0$
\end{itemize}
\subsection{The Hilbert Space}
A Hilbert space $\mathcal{H}$ consists of a set of vectors $\psi,\phi,\chi,\cdots$ and set of scalars a,b,c which satisfies the following four properties.\\
\textbf{(a)} $\mathcal{H}$ \textbf{is a linear space.}\\
\textbf{(b)} $\mathcal{H}$ \textbf{has a defind scalar product that is strictly positive}.\\
\textbf{Scalar Product}
  \par The scalar product of an element $\psi$ with another element $\phi$ is in general a complex number,denoted by ($\psi,\phi$) \\
  ($\psi,\phi$)=complex number
  \begin{note}
  	Since the scalar product is a complex number ,the quandity ($\psi,\phi$) is generally not equal to ($\phi,\psi$)\\
  	($\psi,\phi$)=($\psi*,\phi$)
  \end{note}
\textbf{Properties of Scalar Product}
\begin{itemize}
	\item The scalar product of $\psi$ with $\phi$ is equal to the complex conjugate of the scalar product of $\phi$ with $\psi$\\
	($\psi,\phi$)=($\phi,\psi$)*
	\item The scalar product of $\phi$ with $\psi$ is linear with respect to the second factor if $\psi=$ $a \psi_{1}+b \psi_{2}$ :
	$$
	\left(\phi, a \psi_{1}+b \psi_{2}\right)=a\left(\phi, \psi_{1}\right)+b\left(\phi, \psi_{2}\right),
	$$
	and antilinear with respect to the first factor if $\phi=a \phi_{1}+b \phi_{2}:$
	$$
	\left(a \phi_{1}+b \phi_{2}, \psi\right)=a^{*}\left(\phi_{1}, \psi\right)+b^{*}\left(\phi_{2}, \psi\right) .
	$$
	\item - The scalar product of a vector $\psi$ with itself is a positive real number:
	$$
	(\psi, \psi)=\|\psi\|^{2} \geq 0
	$$
	where the equality holds only for $\psi=O$.
\end{itemize}
\textbf{(c)}$\mathcal{H}$ \textbf{is separable}\\
There exists a Cauchy sequence $\psi_{n} \in \mathcal{H}(n=1,2, \ldots)$ such that for every $\psi$ of $\mathcal{H}$ and $\varepsilon>0$, there exists at least one $\psi_{n}$ of the sequence for which
$$
\left\|\psi-\psi_{n}\right\|<\varepsilon .
$$
\textbf{(d)} $\mathcal{H}$ \textbf{is complete}\\
Every Cauchy sequence $\psi_{n} \in \mathcal{H}$ converges to an element of $\mathcal{H}$. That is, for any $\psi_{n}$, the relation
$$
\lim _{n, m \rightarrow \infty}\left\|\psi_{n}-\psi_{m}\right\|=0,
$$
defines a unique limit $\psi$ of $\mathcal{H}$ such that
$$
\lim _{n \rightarrow \infty}\left\|\psi-\psi_{n}\right\|=0 .
$$
\subsection{Linear Vector Space That Are Hilbert Spaces}
\begin{itemize}
	\item The first one is the three diamenensional Euclidean vector space which have finite(descrete) set of base vectors.
	\item The second example is the space of the entire complex functions $\psi(x)$ which hace infinite(contineous) basis.
\end{itemize}
\section{Dimension and Basis of a Vector Space}
A set of $N$ nonzero vectors $\phi_{1}, \phi_{2}, \ldots, \phi_{N}$ is said to be linearly independent if and only if the solution of the equation
$$
\sum_{i=1}^{N} a_{i} \phi_{i}=0
$$
is $a_{1}=a_{2}=\cdots=a_{N}=0 .$ But if there exists a set of scalars, which are not all zero, so that one of the vectors (say $\phi_{n}$ ) can be expressed as a linear combination of the others,
$$
\phi_{n}=\sum_{i=1}^{n-1} a_{i} \phi_{i}+\sum_{i=n+1}^{N} a_{i} \phi_{i},
$$
$\text { the set }\left\{\phi_{i}\right\} \text { is said to be linearly dependent. }$\\
\textbf{Dimension:} The dimension of a vector space is given by the maximum number of linearly independent vectors the space can have. For instance, if the maximum number of linearly independent vectors a space has is $N$ (i.e.. $\phi_{1}, \phi_{2}, \ldots, \phi_{N}$ ). this space is said to be $N$-dimensional. In this $N$-dimensional vector space, any vector $\psi$ can be expanded as a linear combination:
$$
\psi=\sum_{i=1}^{N} a_{i} \phi_{t} .
$$
\textbf{Basis:} The basis of a vector space consists of a set of the maximum possible number of linearly independent vectors belonging to that space. This set of vectors, $\phi_{1}, \phi_{2}, \ldots, \phi_{N}$, to be denoted in short by $\left\{\phi_{1}\right\}$, is called the basis of the vector space, while the vectors $\phi_{1}, \phi_{2}, \ldots, \phi_{N}$ are called the base vectors.
\section{Dirac Notation}
The physical state of a sysytem is represented in quantum mechanics by elements of a Hilbert space; These elments are called state vectors.We can represents the state vectors in different bases by means of function expansions.\\\\
\textbf{Kets: Elements of a Vector Space}\\
Dirac denoted the state vector $\psi $ by the symbol $|\psi\rangle$, which he called a ket vector, or simply a ket. Kets belong to the Hilbert (vector) space $\mathcal{H}$, or, in short, to the ket-space.\\\\
\textbf{Bras: Elements of a Dual Space}\\
We know from linear algebra that a dual space can be associated with every vector space. Dirac denoted the elements of a dual space by the symbol $\langle 1 \mid$, which he called a bra vector, or simply a bra; for instance, the element $\langle\psi \mid$ represents a bra. Note: For every ket $|\psi\rangle$ there exists a unique bra $\langle\psi|$ and vice versa. Again, while kets belong to the Hilbert space $\mathcal{H}$, the corresponding bras belong to its dual (Hilbert) space $\mathcal{H}_{d}$.\\\\
 \textbf{Bra-Ket: Dirac Notation for the Scalar Product}\\
 Dirac denoted the scalar (inner) product by the symbol $\langle\mid\rangle$, which he called a a bra-ket. For instance, the scalar product $(\phi, \psi)$ is denoted by the bra-ket $\langle\phi \mid \psi\rangle$ :
 $$
 (\phi, \psi) \quad \longrightarrow \quad\langle\phi \mid \psi\rangle \text {. }
 $$
 \begin{note}
 	\begin{itemize}
 		\item When a ket (or bra) is multiplied by a complex number, we also get a ket (or bra).
 		\item In the coordinate representation,the scalar product $\langle\phi \mid \psi\rangle$ is given by\\
 		$$\langle\phi \mid \psi\rangle=\int \phi^*(r,t)\psi(r,t)d^3r$$
 	\end{itemize}
 \end{note}
\newpage
\textbf{ Properties of Kets, Bras, and Bra-Kets}  \\
\begin{itemize}
	\item \textbf{ Every Ket Has a Corresponding Bra}\\
	To every $k e t|\psi\rangle$, there corresponds a unique bra $\langle\psi|$ and vice versa:
	$|\psi\rangle \quad \longleftrightarrow \quad\langle\psi|$.
	There is a one-to-one correspondence between bras and kets:
	$$
	a|\psi\rangle+b|\phi\rangle \longleftrightarrow a^{*}\langle\psi|+b^{*}\langle\phi| \text {. }
	$$
	where $a$ and $b$ are complex numbers. The following is a common notation:
	$$
	|a \psi\rangle=a|\psi\rangle, \quad\langle a \psi|=a^{*}\langle\psi| 
	$$
	\item \textbf{Properties of the Scalar Product}
	\begin{enumerate}
		\item $\langle\phi \mid \psi\rangle^{*}=\langle\psi \mid \phi\rangle $ .
		\item $\left\langle\psi \mid a_{1} \psi_{1}+a_{2} \psi_{2}\right\rangle= a_{1}\left\langle\psi \mid \psi_{1}\right\rangle+a_{2}\left\langle\psi \mid \psi_{2}\right\rangle $
		\item $\left\langle a_{1} \phi_{1}+a_{2} \phi_{2} \mid \psi\right\rangle= a_{1}^{*}\left\langle\phi_{1} \mid \psi\right\rangle+a_{2}^{*}\left\langle\phi_{2} \mid \psi\right\rangle $
		\item $\left\langle a_{1} \phi_{1}+a_{2} \phi_{2} \mid b_{1} \psi_{1}+b_{2} \psi_{2}\right\rangle= a_{1}^{*} b_{1}\left\langle\phi_{1} \mid \psi_{1}\right\rangle+a_{1}^{*} b_{2}\left\langle\phi_{1} \mid \psi_{2}\right\rangle 
		+a_{2}^{*} b_{1}\left\langle\phi_{2} \mid \psi_{1}\right\rangle+a_{2}^{*} b_{2}\left\langle\phi_{2} \mid \psi_{2}\right\rangle$ 
	\end{enumerate}
\item \textbf{ The Norm Is Real and Positive}\\
For any state vector $|\psi\rangle$ of the Hilbert space $\mathcal{H}$, the norm $\langle\psi \mid \psi\rangle$ is real and positive; $\langle\psi \mid \psi\rangle$ is equal to zero only for the case where $|\psi\rangle=O$, where $O$ is the zero vector. If the state $|\psi\rangle$ is normalized then $\langle\psi \mid \psi\rangle=1$.
\item \textbf{Schwarz Inequality}
For any two states $|\psi\rangle$ and $|\phi\rangle$ of the Hilbert space, we can show that
$$
|\langle\psi \mid \phi\rangle|^{2} \leq\langle\psi \mid \psi\rangle\langle\phi \mid \phi\rangle .
$$
If $|\psi\rangle$ and $|\phi\rangle$ are linearly dependent (i.e., proportional: $|\psi\rangle=\alpha|\phi\rangle$, where $\alpha$ is a scalar), this relation becomes an equality. The Schwarz inequality  is analogous to the following relation of the real Euclidean space
$$
|\vec{A} \cdot \vec{B}|^{2} \leq|\vec{A}|^{2}|\vec{B}|^{2}
$$
\item \textbf{Triangle Inequality}\\
$$
\sqrt{\langle\psi+\phi \mid \psi+\phi\rangle} \leq \sqrt{\langle\psi \mid \psi\rangle}+\sqrt{\langle\phi \mid \phi\rangle} .
$$
If $|\psi\rangle$ and $|\phi\rangle$ are linearly dependent, $|\psi\rangle=\alpha|\phi\rangle$, and if the proportionality scalar $\alpha$ is real and positive, the triangle inequality becomes an equality. The counterpart of this inequality in Euclidean space is given by $|\vec{A}+\vec{B}| \leq|\vec{A}|+|\vec{B}|$.
\item \textbf{Orthogonal States}\\
Two kets, $|\psi\rangle$ and $|\phi\rangle$, are said to be orthogonal if they have a vanishing scalar product:
$$
\langle\psi \mid \phi\rangle=0 .
$$
\item \textbf{Orthonormal States}\\
Two kets, $|\psi\rangle $ and $|\phi \rangle $ are said to be orthonormal and if each one of them has a unit norm:\\
$$ \langle \psi \mid \phi \rangle =0$$ 
$$ \langle \psi \mid \psi \rangle =1$$ 
$$ \langle \phi \mid \phi \rangle =1$$ 
\end{itemize}
\begin{exercise}
	Consider the following two kets\\
	|$\psi\rangle=\left(\begin{array}{c}
		-3 i \\
		2+i \\
		4
	\end{array}\right), \quad|\phi\rangle=\left(\begin{array}{c}
		2 \\
		-i \\
		2-3 i
	\end{array}\right)$\\
	(a) Find the bra $\langle \phi |$ \\
	(b) Evaluate the scalar product $\langle \phi \mid \psi \rangle $
\end{exercise}
\begin{answer}
(a) the bra $\langle\phi|$ can be obtained by simply taking the complex conjugate of the transpose of the ket $|\phi\rangle$ :
	$$
	\langle\phi|=\left(\begin{array}{lll}
	2 & i & 2+3 i
	\end{array}\right) .
	$$
	(b) The scalar product $\langle\phi \mid \psi\rangle$ can be calculated as follows:
	$$
	\begin{aligned}
	\langle\phi \mid \psi\rangle &=\left(\begin{array}{lll}
	2 & i & 2+3 i
	\end{array}\right)\left(\begin{array}{c}
	-3 i \\
	2+i \\
	4
	\end{array}\right) \\
	&=2(-3 i)+i(2+i)+4(2+3 i) \\
	&=7+8 i .
	\end{aligned}
	$$
\end{answer}
\begin{exercise}
	Consider the states $|\psi\rangle=3 i\left|\phi_{1}\right\rangle-7 i\left|\phi_{2}\right\rangle$ and $|\chi\rangle=-\left|\phi_{1}\right\rangle+2 i\left|\phi_{2}\right\rangle$, where $\left|\phi_{1}\right\rangle$ and $\left|\phi_{2}\right\rangle$ are orthonormal.\\
	(a) Calculate $|\psi+\chi\rangle$ and $\langle\psi+\chi|$.\\
	(b) Calculate the scalar products $\langle\psi \mid \chi\rangle$ and $\langle\chi \mid \psi\rangle$. Are they equal?
\end{exercise}
\begin{answer}
	(a) The calculation of $|\psi+\chi\rangle$ is straightforward:
	$$
	\begin{aligned}
	|\psi+\chi\rangle &=|\psi\rangle+|\chi\rangle=\left(3 i\left|\phi_{1}\right\rangle-7 i\left|\phi_{2}\right\rangle\right)+\left(-\left|\phi_{1}\right\rangle+2 i\left|\phi_{2}\right\rangle\right) \\
	&=(-1+3 i)\left|\phi_{1}\right\rangle-5 i\left|\phi_{2}\right\rangle .
	\end{aligned}
	$$
	This leads at once to the expression of $\langle\psi+\chi|$ :
	$$
	\langle\psi+\chi|=(-1+3 i)^{*}\left\langle\phi_{1}\right|+(-5 i)^{*}\left\langle\phi_{2}\right|=(-1-3 i)\left\langle\phi_{1}\right|+5 i\left\langle\phi_{2}\right| \text {. }
	$$
	(b) Since $\left\langle\phi_{1} \mid \phi_{1}\right\rangle=\left\langle\phi_{2} \mid \phi_{2}\right\rangle=1,\left\langle\phi_{1} \mid \phi_{2}\right\rangle=\left\langle\phi_{2} \mid \phi_{1}\right\rangle=0$, and since the bras corresponding to the kets $|\psi\rangle=3 i\left|\phi_{1}\right\rangle-7 i\left|\phi_{2}\right\rangle$ and $|\chi\rangle=-\left|\phi_{1}\right\rangle+2 i\left|\phi_{2}\right\rangle$ are given by $\langle\psi|=-3 i\left\langle\phi_{1}\right|+7 i\left\langle\phi_{2}\right|$ and $\langle\chi|=-\left\langle\phi_{1}\right|-2 i\left\langle\phi_{2}\right|$, the scalar products are\\
	$$\begin{aligned}
		\langle\psi \mid \chi\rangle &=\left(-3 i\left\langle\phi_{1}\left|+7 i\left\langle\phi_{2}\right|\right)\left(-\left|\phi_{1}\right\rangle+2 i\left|\phi_{2}\right\rangle\right)\right.\right.\\
		&=(-3 i)(-1)\left\langle\phi_{1} \mid \phi_{1}\right\rangle+(7 i)(2 i)\left\langle\phi_{2} \mid \phi_{2}\right\rangle \\
		&=-14+3 i \\
		\langle\chi \mid \psi\rangle &=\left(-\left\langle\phi_{1}\left|-2 i\left\langle\phi_{2}\right|\right)\left(3 i\left|\phi_{1}\right\rangle-7 i\left|\phi_{2}\right\rangle\right)\right.\right.\\
		&=(-1)(3 i)\left\langle\phi_{1} \mid \phi_{1}\right\rangle+(-2 i)(-7 i)\left\langle\phi_{2} \mid \phi_{2}\right\rangle \\
		&=-14-3 i .
	\end{aligned}$$
	$\text { We see that }\langle\psi \mid \chi\rangle \text { is equal to the complex conjugate of }\langle\chi \mid \psi\rangle \text {. }$
\end{answer}
\section{Operators}
\subsection{General Definitions}
 Definition of an operator: An operator $ \hat{A}$ is a mathematical rule that when applied to a ket $|\psi\rangle$ transforms it into another ket $\left|\psi^{\prime}\right\rangle$ of the same space and when it acts on a bra $\langle\phi|$ transforms it into another bra $\left\langle\phi^{\prime}\right|$ :
 $$
 \hat{A}|\psi\rangle=\left|\psi^{\prime}\right\rangle, \quad\langle\phi| \hat{A}=\left\langle\phi^{\prime}\right| .
 $$
 A similar definition applies to wave functions:
 $$
 \hat{A} \psi(\vec{r})=\psi^{\prime}(\vec{r}), \quad \phi(\vec{r}) \hat{A}=\phi^{\prime}(\vec{r})
 $$
 \textbf{Examples of Operators}
 \begin{itemize}
 	\item $\text { Unity operator: it leaves any ket unchanged, } \hat{I}|\psi\rangle=|\psi\rangle \text {. }$
 	\item $\text { The gradient operator: } \vec{\nabla} \psi(\vec{r})=(\partial \psi(\vec{r}) / \partial x) \vec{i}+(\partial \psi(\vec{r}) / \partial y) \vec{j}+(\partial \psi(\vec{r}) / \partial z) \vec{k} \text {. }$
 	\item The linear momentum operator:$\vec{P}\psi(r)=-i\hbar \nabla \psi(r)$ 
 	\item The parity operator :$\mathcal{P}\psi(r)=\psi(-r)$
 \end{itemize}
\textbf{Product of Operators}\\
The product of two operators is generally not commutative:\\
$$\hat{A}\hat{B}\neq \hat{B}\hat{A}$$
But they are associative\\
$$\hat{A}\hat{B}\hat{C}=\hat{A}(\hat{B}\hat{C})=(\hat{A}\hat{B})\hat{C}$$
\textbf{Linear Operators}\\
An operator $\hat{A}$ is said to be linear if it obeys the distributive law and like, all operators ,it commute with constants.That is an operator $\hat{A}$ is linear if ,for any vector $|\psi_{1}\rangle $ and $|\psi_{2}\rangle $ and any complex numbers $a_1$ and $a_2$ we have\\
$$\hat{A}(a_1\mid \psi_{1}\rangle +a_2\mid \psi_{2}\rangle)=a_1 \hat{A} \mid \psi_{1}\rangle +a_2\hat{A}\mid \psi_{2}\rangle $$
and
$$\left(\left\langle\psi_{1}\right| a_{1}+\left\langle\psi_{2}\right| a_{2}\right) \hat{A}=a_{1}\left\langle\psi_{1}\right| \hat{A}+a_{2}\left\langle\psi_{2}\right| \hat{A}$$
\textbf{Expectation Value of an Operator}\\
The expectation or mean value $\langle\hat{A}\rangle$ of an operator $\hat{A}$ with respect to a state $|\psi\rangle$ is defined by
$$
\langle\hat{A}\rangle=\frac{\langle\psi|\hat{A}| \psi\rangle}{\langle\psi \mid \psi\rangle} .
$$
\subsection{Hermitian Operator}
\textbf{Hermitian adjoint}\\
The Hermitian adjoint or conjugate$ \alpha^{\dagger}$, of a complex number $\alpha$ is the complex conjugate of this number: $\alpha^{\dagger}=\alpha^{*}$. The Hermitian adjoint, or simply the adjoint, $\hat{A}^{\dagger}$, of an operator $\hat{A}$ is defined by this relation:
$$
\left\langle\psi\left|\hat{A}^{\dagger}\right| \phi\right\rangle=\langle\phi|\hat{A}| \psi\rangle^{*} .
$$
\textbf{Properties}\\
To obtain the Hermitian adjoint of any expression, we must cyclically reverse the order of the factors and make three replacements:
\begin{itemize}
	\item $\text { Replace constants by their complex conjugates: } \alpha^{\dagger}=\alpha^{*} \text {. }$
	\item $\text { Replace kets (bras) by the corresponding bras (kets): } \left.(|\psi\rangle)^{\dagger}=\langle\psi| \text { and }(\langle\psi|)^{\dagger}=|\psi\rangle\right\rangle \text {. }$
	\item  $\text { Replace operators by their adjoints. }$
\end{itemize}
Following these rules, we can write
$$
\begin{aligned}
\left(\hat{A}^{\dagger}\right)^{\dagger} &=\hat{A}, \\
(a \hat{A})^{\dagger} &=a^{*} \hat{A}^{\dagger}, \\
\left(\hat{A}^{n}\right)^{\dagger} &=\left(\hat{A}^{\dagger}\right)^{n}, \\
(\hat{A}+\hat{B}+\hat{C}+\hat{D})^{\dagger} &=\hat{A}^{\dagger}+\hat{B}^{\dagger}+\hat{C}^{\dagger}+\hat{D}^{\dagger}, \\
(\hat{A} \hat{B} \hat{C} \hat{D})^{\dagger} &=\hat{D}^{\dagger} \hat{C}^{\dagger} \hat{B}^{\dagger} \hat{A}^{\dagger}, \\
(\hat{A} \hat{B} \hat{C}|\psi\rangle)^{\dagger} &=\langle\psi| D^{\dagger} C^{\dagger} B^{\dagger} A^{\dagger} .
\end{aligned}
$$
\textbf{Hermitian and Skew Hermitian Operators}\\
An operator $\hat{A}$ is said to be Hermitian if it is equal to its adjoint $\hat{A}^{\dagger}$ :
$$
\hat{A}=\hat{A}^{\dagger} \quad \text { or } \quad\langle\psi|\hat{A}| \phi\rangle=\langle\phi|\hat{A}| \psi\rangle^{*} .
$$
On the other hand ,an operator $\hat{B}$ is said to be skew Hermitian or anti Hermitian if \\
$$ \hat{B}^{\dagger}=-B $$
or $$ \langle \psi \mid \hat{B} \mid \phi \rangle =-\langle \phi \mid \hat{B} \mid \psi\rangle ^*$$
\textbf{Examples}\\
Hermitian:$\hat{A}+\hat{A}^{\dagger}$ and $i(\hat{A}-\hat{A}^{\dagger})$ etc\\
Anti Hermitian:$i(\hat{A}+\hat{A}^{\dagger})$
\begin{exercise}
	$\text { Prove that the operators } i(d / d x) \text { and } d^{2} / d x^{2} \text { are Hermitian. }$
\end{exercise}
\begin{answer}
	\begin{align*}
	\int_{-\infty}^{\infty} \psi_{m}^{*}\left(i \frac{d}{d x}\right) \psi_{n} d x&=i\left[\psi_{m}^{*} \psi_{n}\right]_{-\infty}^{\infty}-i \int_{-\infty}^{\infty} \psi_{n} \frac{d}{d x} \psi_{m}^{*} d x=\int_{-\infty}^{\infty}\left(i \frac{d}{d x} \psi_{m}\right)^{*} \psi_{n} d x\\
	\therefore \quad &i d / d x\text{ is Hermitian.}\\
	\int_{-\infty}^{\infty} \psi_{m}^{*} \frac{d^{2} \psi_{n}}{d x^{2}} d x&=\left[\psi_{m}^{*} \frac{d \psi_{n}}{d x}\right]_{-m}^{\infty}-\int_{-}^{\infty} \frac{d \psi_{n}}{d x} \frac{d \psi_{m}^{*}}{d x} d x\\
	&=\left[\frac{d \psi_{m}^{*}}{d x} \psi_{n}\right]_{-\infty}^{\infty}+\int_{-\infty}^{\infty} \psi_{n} \frac{d^{2} \psi_{m}^{*}}{d x^{2}} d x=\int_{-\infty}^{\infty} \frac{d^{2} \psi_{m}^{*}}{d x^{2}} \psi_{n} d x\\
	\text{Thus, }&d^{2} / d x^{2}\text{ is Hermitian.	}
	\end{align*}	
\end{answer}
\begin{exercise}
$\text { Prove that operator } P_{x}=-i \hbar \frac{d}{d x} \text { is Hermitian but } D_{x}=\frac{d}{d x} \text { is not Hermitian. }$
\end{exercise}
\begin{answer}
 The given operator $-i \hbar \frac{d}{d x}$ is the same as $P_{x}$. Let $\psi_{1}(x)$ and $\psi_{2}(x)$ be two arbitrary functions.
$$
\begin{aligned}
\left\langle\psi_{1} \mid P_{x} \psi_{2}\right\rangle &=\int_{-\infty}^{x} \psi_{1}^{*}(x)\left[-i \hbar \frac{d}{d x} \psi_{2}(x)\right] d x \\
&=-i \hbar \int_{-\infty}^{\infty} \psi_{1}^{*}(x) \frac{d}{d x} \psi_{2}(x) d x
\end{aligned}
$$
Integrating by parts, this is equal to
$$
-i \hbar\left[\left.\psi_{1}^{*}(x) \psi_{2}(x)\right|_{-\infty} ^{\infty}-\int_{-\infty}^{\infty} \frac{d \psi_{1}^{*}(x)}{d x} \psi_{2}(x) d x\right]
$$
For the wave function to be square integrable, it must go to zero as $x$ goes to $-\infty$ or $+\infty$.
Thus, the first term in the square bracket is zero. So,
$$
\begin{aligned}
\left\langle\psi_{1}\left|P_{x}\right| \psi_{2}\right\rangle&=i \hbar \int_{-\infty}^{\infty} \frac{d \psi_{1}^{*}(x)}{d x} \psi_{2}(x) d x\\
\text { Also, } 
\left\langle P_{x} \psi_{1} \mid \psi_{2}\right\rangle &=\int_{-\infty}^{\infty}\left[P_{x} \psi_{1}(x)\right]^{*} \psi_{2}(x) d x \\
&=\int_{-\infty}^{\infty}\left[-i \hbar \frac{d \psi_{1}(x)}{d x}\right]^{*} \psi_{2}(x) d x=i \hbar \int_{-\infty}^{\infty} \frac{d \psi_{1}^{*}(x)}{d x} \psi_{2}(x) d x
\end{aligned}
$$
From (i) and (ii), $\left\langle\psi_{1} \mid P_{x} \psi_{2}\right\rangle=\left\langle P_{x} \psi_{1} \mid \psi_{2}\right\rangle$\\
Hence $P_{x}=-i \hbar \frac{d}{d x}$ is Hermitian. Similar calculation with $A=\frac{d}{d x}$ will give,
\begin{align*}
\left\langle\psi_{1} \mid A \psi_{2}\right\rangle&=-\int_{-\infty}^{\infty} \frac{d \psi_{1}^{*}(x)}{d x} \psi_{2}(x) d x\\
\text{And, }\quad\left\langle A \psi_{1} \mid \psi_{2}\right\rangle&=\int_{-\infty}^{\infty} \frac{d \psi_{1}^{*}(x)}{d x} \psi_{2}(x) d x\\
\text{Hence, }\left\langle\psi_{1} \mid A \psi_{2}\right\rangle &\neq\left\langle A \psi_{1} \mid \psi_{2}\right\rangle\text{ and so, }A=\frac{d}{d x}\text{ is not Hermitian.	}
\end{align*}



\end{answer}
\begin{exercise}
	$\text { Show that operator } O=(1+i) A B+(1-i) B A \text { is Hermitian if } A \text { and } B \text { is Hermitian. }$
\end{exercise}
\begin{answer}
$$[(1+i) A B+(1-i) B A]^{\dagger}=[(1+i) A B]^{\dagger}+[(1-i) B A]^{\dagger}$$	
$$
\begin{aligned}
&=(1+i)^{*}(A B)^{\dagger}+(1-i)^{*}(B A)^{\dagger} \\
&=(1-i) B^{\dagger} A^{\dagger}+(1+i) A^{\dagger} B^{\dagger} \quad \text { if } A \text { and } B \text { is hermitian } \\
&=(1-i) B A+(1+i) A B \\
&(1+i) A B+(1-i) B A
\end{aligned}
$$
Thus the given operator is Hermitian
\end{answer}
\subsection{Projection Operators}
An operator $\hat{P}$ is said to be projection operator if it is Hermitian and equal to its own square:\\
$$\hat{P}^{\dagger}=\hat{P} \quad \quad \hat{P}^2=\hat{P}$$
THe unit operator $\hat{I}$ is a simple example of a projection operator Since $$\hat{I}^{\dagger}=\hat{I}, \hat{I}^2=\hat{I}$$
\textbf{Properties}\\
\begin{itemize}
	\item The product of two commuting projection operators, $\hat{P}_{1}$ and $\hat{P}_{2}$, is also a projection operator, since
	$$
	\left(\hat{P}_{1} \hat{P}_{2}\right)^{\dagger}=\hat{P}_{2}^{\dagger} \hat{P}_{1}^{\dagger}=\hat{P}_{2} \hat{P}_{1}=\hat{P}_{1} \hat{P}_{2} \text { and }\left(\hat{P}_{1} \hat{P}_{2}\right)^{2}=\hat{P}_{1} \hat{P}_{2} \hat{P}_{1} \hat{P}_{2}=\hat{P}_{1}^{2} \hat{P}_{2}^{2}=\hat{P}_{1} \hat{P}_{2} \text {. }
	$$
	\item  The sum of two projection operators is generally not a projection operator.
	\item  Two projection operators are said to be orthogonal if their product is zero.
	\item For a sum of projection operators $\hat{P}_{1}+\hat{P}_{2}+\hat{P}_{3}+\cdots$ to be a projection operator, it is necessary and sufficient that these projection operators be mutually orthogonal (i.e., the cross-product terms must vanish).
\end{itemize}
\subsection{Commutator Algebra}
The commutator of two operators $\hat{A}$ and $\hat{B}$, denoted by $[\hat{A}, \hat{B}]$, is defined by
$$
[\hat{A}, \hat{B}]=\hat{A} \hat{B}-\hat{B} \hat{A}
$$
and the anticommutator $\{\hat{A}, \hat{B}\}$ is defined by
$$
\{\hat{A}, \hat{B}\}=\hat{A} \hat{B}+\hat{B} \hat{A}
$$
Two operators are said to commute if their commutator is equal to zero and hence $\hat{A} \hat{B}=\hat{B} \cdot \hat{A}$. Any operator commutes with itself:
$$
[\hat{A}, \hat{A}]=0
$$
Note that if two operators are Hermitian and their product is also Hermitian, these operators commute:
$$
(\hat{A} \hat{B})^{\dagger}=\hat{B}^{\dagger} \hat{A}^{\dagger}=\hat{B} \hat{A}
$$
and since $(\hat{A} \hat{B})^{\dagger}=\hat{A} \hat{B}$ we have $\hat{A} \hat{B}=\hat{B} \hat{A}$.\\
\textbf{Example}\\
$$\left[\hat{X}, \hat{P}_{x}\right]=i \hbar \hat{I}, \quad\left[\hat{Y}, \hat{P}_{y}\right]=i \hbar \hat{I}, \quad\left[\hat{Z}, \hat{P}_{z}\right]=i \hbar \hat{I}$$
Where $\hat{P_x}=-i\hbar \frac{\partial}{\partial x}$, $\hat{P_y}=-i\hbar \frac{\partial}{\partial y}$, $\hat{P_z}=-i\hbar \frac{\partial}{\partial z}$ and $\hat{I}$ is the unit operator.\\
\begin{note}
	\begin{enumerate}
		\item $\left[\hat{X}, \hat{P}_{x}\right]=i \hbar, \quad\left[\hat{Y}, \hat{P}_{Y}\right]=i \hbar, \quad\left[\hat{Z}, \hat{P}_{Z}\right]=i \hbar$
		\item $\left[\hat{X}, \hat{P}_{y}\right]=\left[\hat{X}, \hat{P}_{z}\right]=\left[\hat{Y}, \hat{P}_{x}\right]=\left[\hat{Y}, \hat{P}_{z}\right]=\left[\hat{Z}, \hat{P}_{x}\right]=\left[\hat{Z}, \hat{P}_{y}\right]=0$
		\item $\left[\hat{R}_{j}, \hat{P}_{k}\right]=i \hbar \delta_{j k}, \quad\left[\hat{R}_{j}, \hat{R}_{k}\right]=0, \quad\left[\hat{P}_{j}, \hat{P}_{k}\right]=0 \quad(j, k=x, y, z)$
		\item $\left[\hat{X}^{n}, \hat{P}_{x}\right]=i \hbar n \hat{X}^{n-1}, \quad\left[\hat{X}, \hat{P}_{x}^{n}\right]=i \hbar n \hat{P}_{x}^{n-1}$
		\item $\left.\left[  f(\hat{X}), \hat{P}_{x}\right]=i \hbar \frac{d f(\hat{X})}{d \hat{X}} \Longrightarrow[\hat{P}, F(\hat{\vec{R}})]=-i \hbar \vec{\nabla} F(\hat{\vec{R}})\right)$\\
		Where F is a function of the operator $\hat{R}$
	\end{enumerate}
\end{note}
\textbf{Properties}\\
\begin{itemize}
	\item Antisymmetry:
	$$[\hat{A}, \hat{B}]=-[\hat{B}, \hat{A}]$$
	\item Linearity:
	 $$[\hat{A}, \hat{B}+\hat{C}+\hat{D}+\cdots]=[\hat{A}, \hat{B}]+[\hat{A}, \hat{C}]+[\hat{A}, \hat{D}]+\cdots$$
	 \item  Hermitian conjugate of a commutator: 
	 $$[\hat{A}, \hat{B}]^{\dagger}=\left[\hat{B}^{\dagger}, \hat{A}^{\dagger}\right]$$
	 \item  Distributivity: 
	 $$\begin{aligned}
	 	&{[\hat{A}, \hat{B} \hat{C}]=[\hat{A}, \hat{B}] \hat{C}+\hat{B}[\hat{A}, \hat{C}]} \\
	 	&{[\hat{A} \hat{B}, \hat{C}]=\hat{A}[\hat{B}, \hat{C}]+[\hat{A}, \hat{C}] \hat{B}}
	 \end{aligned}$$
	 \item jacobi identity:
	 $$[\hat{A},[\hat{B}, \hat{C}]]+[\hat{B},[\hat{C}, \hat{A}]]+[\hat{C},[\hat{A}, \hat{B}]]=0$$
	 \item Operators commute with scalars: an operator $\hat{A}$ commutes with any scalar $b$ :
	 $$
	 [\hat{A}, b]=0
	 $$
\end{itemize}
\begin{note}
	Uncertainity relation between two operator A  and B is \\
	$$\Delta A \Delta B \geq \frac{1}{2}\left| \langle \left[ \hat{A},\hat{B}\right] \rangle \right| $$
\end{note}
\begin{exercise}
	(a)Show that the commutator of two Hermitian operator is anti Hermitian.\\
	(b)Evaluate the commutator $\left[ \hat{A},\left[ \hat{B},\hat{C}\right] \hat{D}\right] $
\end{exercise}
\begin{answer}
	(a) If $\hat{A}$ and $\hat{B}$ are Hermitian, we can write
	$$
	[\hat{A}, \hat{B}]^{\dagger}=(\hat{A} \hat{B}-\hat{B} \hat{A})^{\dagger}=\hat{B}^{\dagger} \hat{A}^{\dagger}-\hat{A}^{\dagger} \hat{B}^{\dagger}=\hat{B} \hat{A}-\hat{A} \hat{B}=-[\hat{A}, \hat{B}] 
	$$
	that is, the commutator of $\hat{A}$ and $\hat{B}$ is anti-Hermitian: $[\hat{A}, \hat{B}]^{\dagger}=-[\hat{A}, \hat{B}]$.\\
	(b) Using the distributivity relation we have
	$$
	\begin{aligned}
	{[\hat{A},{[\hat{B},\hat{C}]}\hat{D}]} &=\hat{A}[\hat{B},\hat{C}]\hat{D}-[\hat{B},\hat{C}]\hat{D}\hat{A} \\
	&=\hat{A}(\hat{B} \hat{C}-\hat{C} \hat{B})\hat{D}-(\hat{B} \hat{C}-\hat{C} \hat{B})\hat{D} \hat{A}  \\
	&=\hat{A} \hat{B} \hat{C} \hat{D}-\hat{A} \hat{C} \hat{B} \hat{D}-\hat{B} \hat{C} \hat{D} \hat{A}+\hat{C} \hat{B} \hat{D} \hat{A}
	\end{aligned}
	$$
\end{answer}
\begin{exercise}
 Find the following commutation relations:
	(i) $\left[\frac{\partial}{\partial x}, \frac{\partial^{2}}{\partial x^{2}}\right]$
	(ii) $\left[\frac{\partial}{\partial x}, F(x)\right]$
\end{exercise}
\begin{answer}
	\begin{align*}
	\text{(i) }\left[\frac{\partial}{\partial x}, \frac{\partial^{2}}{\partial x^{2}}\right] \psi&=\left(\frac{\partial}{\partial x} \frac{\partial^{2}}{\partial x^{2}}-\frac{\partial^{2}}{\partial x^{2}} \frac{\partial}{\partial x}\right) \psi=\left(\frac{\partial^{3}}{\partial x^{3}}-\frac{\partial^{3}}{\partial x^{3}}\right) \psi=0\\
	\text{(ii)} \left[\frac{\partial}{\partial x}, F(x)\right] \psi&=\frac{\partial}{\partial x}(F \psi)-F \frac{\partial}{\partial x} \psi=\frac{\partial F}{\partial x} \psi+F \frac{\partial \psi}{\partial x}-F \frac{\partial \psi}{\partial x}=\frac{\partial F}{\partial x} \psi\\
	\text{Thus, }\left[\frac{\partial}{\partial x}, F(x)\right]&=\frac{\partial F}{\partial x}
	\end{align*}
\end{answer}
\begin{exercise}
	If Hamiltonian of system is $H=\frac{p_{x}^{2}}{2 m}+V(x)$ then Find commutation $[H, x]$ and $[[H, x], x]$
\end{exercise}
\begin{answer}
	\begin{align*}
	\text{ As, }H&=p^{2} / 2 m+V(x)\\
	\text{We have, }[H, x]&=\frac{1}{2 m}\left[p^{2}, x\right]=-i \hbar p / m \\
\text{	and so, }[[H, x], x]&=\frac{i \hbar}{m}[p, x]=-\hbar^{2} / m\\
	\text{Hence, }\langle m|[[H, x] x]| m\rangle&=-\frac{\hbar^{2}}{m}
	\end{align*}
\end{answer}
\subsection{Functions of Operators}
\textbf{Commutators Involving Function Operators}\\
If $\hat{A}$ commutes with another operator $\hat{B}$, then $\hat{B}$ commutes with any operator function that depends on $\hat{A}$ :
$$
[\hat{A}, \hat{B}]=0 \quad \Longrightarrow \quad[\hat{B}, F(\hat{A})]=0 ;
$$
in particular, $F(\hat{A})$ commutes with $\hat{A}$ and with any other function, $G(\hat{A})$, of $\hat{A}$ :
$$
[\hat{A}, F(\hat{A})]=0, \quad\left[\hat{A}^{n}, F(\hat{A})\right]=0, \quad[F(\hat{A}), G(\hat{A})]=0 .
$$
\subsection{Inverse and Unitary Operators}
\textbf{Inverse of an Operator}\\
 Assuming it exists the inverse $\hat{A}^{-1}$ of a linear operator $A$ is defined by the relation
$$
\hat{A}^{-1} \hat{A}=\hat{A} \hat{A}^{-1}=\hat{I},
$$
where $\hat{I}$ is the unit operator, the operator that leaves any state $|\psi\rangle$ unchanged.\\
\textbf{Unitary Operator}\\
 A linear operator $\hat{U}$ is said to be unitary if its inverse $\hat{U}^{-1}$ is equal to its adjoint $\hat{U}^{\dagger}$ :
$$
\hat{U}^{\dagger}=\hat{U}^{-1} \quad \text { or } \quad \hat{U} \hat{U}^{\dagger}=\hat{U}^{\dagger} \hat{U}=\hat{I} .
$$
The product of two unitary operators is also unitary, since
$$
(\hat{U} \hat{V})(\hat{U} \hat{V})^{\dagger}=(\hat{U} \hat{V})\left(\hat{V}^{\dagger} \hat{U}^{\dagger}\right)=\hat{U}\left(\hat{V} \hat{V}^{\dagger}\right) \hat{U}^{\dagger}=\hat{U} \hat{U}^{\dagger}=\hat{I},
$$
or $(\hat{U} \hat{V})^{\dagger}=(\hat{U} \hat{V})^{-1}$. This result can be generalized to any number of operators; the product of a number of unitary operators is also unitary, since
$$
\begin{aligned}
(\hat{A} \hat{B} \hat{C} \hat{D} \cdots)(\hat{A} \hat{B} \hat{C} \hat{D} \cdots)^{\dagger} &=\hat{A} \hat{B} \hat{C} \hat{D}(\cdots) \hat{D}^{\dagger} \hat{C}^{\dagger} \hat{B}^{\dagger} \hat{A}^{\dagger}=\hat{A} \hat{B} \hat{C}\left(\hat{D} \hat{D}^{\dagger}\right) \hat{C}^{\dagger} \hat{B}^{\dagger} \hat{A}^{\dagger} \\
&=\hat{A} \hat{B}\left(\hat{C} \hat{C}^{\dagger}\right) \hat{B}^{\dagger} \hat{A}^{\dagger}=\hat{A}\left(\hat{B} \hat{B}^{\dagger}\right) \hat{A}^{\dagger} \\
&=\hat{A} \hat{A}^{\dagger}=\hat{l}
\end{aligned}
$$
\subsection{Eigen Value and Eigen Vector of an Operator}
A state vector $|\psi\rangle$ is said to be an eigenvector (also called an eigenket or eigenstate) of an operator $\hat{A}$ if the application of $\hat{A}$ to $|\psi\rangle$ gives
$$
\hat{A}|\psi\rangle=a|\psi\rangle,
$$
where $a$ is a complex number, called an eigenvalue of $\hat{A}$. This equation is known as the eigenvalue equation, or eigenvalue problem, of the operator $\hat{A}$. Its solutions yield the eigenvalues and eigenvectors of $\hat{A}$.
\begin{exercise}
	$\text { Show that if } \hat{A}^{-1} \text { exists, the eigenvalues of } \hat{A}^{-1} \text { are just the inverses of those of } \hat{A} \text {. }$
\end{exercise}
\begin{answer}
	\begin{align*}
	\text{Since }\hat{A}^{-1} \hat{A}=\hat{I}\text{ we have on the one hand}\\
	\hat{A}^{-1} \hat{A}|\psi\rangle&=|\psi\rangle,\\
\text{	and on the other hand}\\
	\hat{A}^{-1} \hat{A}|\psi\rangle&=\hat{A}^{-1}(\hat{A}|\psi\rangle)=a \hat{A}^{-1}|\psi\rangle .\\
\text{	Combining the previous two equations, we obtain}\\
	a \hat{A}^{-1}|\psi\rangle&=|\psi\rangle,\\
\text{	hence}\quad
	\hat{A}^{-1}|\psi\rangle&=\frac{1}{a}|\psi\rangle
\intertext{	This means that $|\psi\rangle$ is also an eigenvector of $\hat{A}^{-1}$ with eigenvalue $1 / a$. That is, if $\hat{A}^{-1}$ exists, then}
	\hat{A}|\psi\rangle&=a|\psi\rangle \quad \Longrightarrow \quad \hat{A}^{-1}|\psi\rangle=\frac{1}{a}|\psi\rangle .
	\end{align*}
\end{answer}
\section{Theorems} 
\begin{theorem}
	\begin{align*}
\intertext{	For an Hermitian operator ,all of its eigen values are real and the eigen vectors corresponding to different eigen values are orthogonal.}	
\text{	If }\hat{A}^{\dagger}&=\hat{A}, \hat{A}|\phi_{n}\rangle =a_n |\phi_{n}\rangle \implies a_n=\text{real number }\\
\text{	and}\quad 
	\langle \phi_{m}\mid \phi_{n}\rangle&=\delta_{mn}
	\end{align*}
\end{theorem}
\begin{theorem}
If two Hermitian operators, $\hat{A}$ and $\hat{B}$, commute and if $\hat{A}$ has no degenerate eigenvalue, then each eigenvector of $\hat{A}$ is also an eigenvector of $\hat{B}$. In addition, we can construct a common orthonormal basis that is made of the joint eigenvectors of $\hat{A}$ and $\hat{B}$.
\end{theorem}
\begin{theorem}
 The eigenvalues of an anti-Hermitian operator are either purely imaginary or equal to zero.
\end{theorem}
\begin{theorem}
	 The eigenvalues of a unitary operator are complex numbers of moduli equal to one; the eigenvectors of a unitary operator that has no degenerate eigenvalues are mutually orthogonal.
\end{theorem}
\section{Wavefunction in Coordinate and Momentum Representations}
We have $$xp-px=i\hbar$$
So $$\langle x\mid xp-px\mid x^{\prime}\rangle=i\hbar\langle x\mid x^{\prime}\rangle$$
$$x\langle x |p|x^{\prime}\rangle -\langle x |p|x^{\prime}\rangle x^{\prime}=i\hbar \delta(x-x^{\prime})$$
$$\langle x|p|x^{\prime}\rangle =i\hbar \frac{\delta(x-x^{\prime})}{x-x^{\prime}}$$
We have $-\frac{\delta(x)}{x}=\delta^{\prime}(x)$\\
$$\langle x|p|x^{\prime}\rangle=-i\hbar \frac{\partial}{\partial x}\delta(x-x^{\prime})$$
So similarly $$\langle x|p|\psi(t)\rangle=-i\hbar\frac{\partial}{\partial x} \langle|\psi(t)\rangle$$
$$\langle x|p|\psi(t)\rangle=-i\hbar \frac{\partial }{\partial x}\psi(x,t)$$
So the momentum operator in position basis \\
$$\hat{P}=-i\hbar \frac{\partial}{\partial x}$$
Now let us find the expression of position operator in the momentum basis \\
$$\langle p|xp-px|p^{\prime}\rangle =i\hbar \delta (p-p^{\prime})$$
$$p|px-xp|p^{\prime}\rangle =-i\hbar \delta(p-p^{\prime})$$\\
$$p\langle p|x|p^{\prime}\rangle -\langle p|x|p^{\prime}\rangle p^{\prime}=-i\hbar\delta(p-p^{\prime})$$
$$\langle p|x|p^{\prime}\rangle=+i\hbar\frac{\partial}{\partial x}\delta(p-p^{\prime})$$
So $$\langle p|x|\psi(t)\rangle =i\hbar\frac{\partial}{\partial p} \langle p|\psi(t)\rangle=i\hbar\frac{\partial}{\partial p}\psi (p,t)$$
	$$
	\begin{aligned}
	\text{ie }\hat{x}=i\hbar\frac{\partial}{\partial p}&\text{ in momentum space}\\
	\text{So in three dimension position space}
	\hat{p}&=-i\hbar \nabla_{r}, 
	\hat{x}=x\\
\text{	Momentum space}
	\hat{p}&=p, 
	\hat{x}=i\hbar \nabla_{p}\\
\text{	Now let us check whether momentum }&\text{is self adjoint}\\
	\int_{-\infty}^{+\infty} f^*(x)\left( -i\hbar \frac{\partial}{\partial x}g(x)\right)dx&=-i\hbar\left\lbrace  \left[ f^{*}(x)g(x) \right]_{-\infty}^{+\infty}-\int_{-\infty}^{+\infty} g(x)\frac{\partial}{\partial x}f^*(x)dx\right\rbrace 
	=i\hbar \int \frac{\partial}{\partial x} f^*(x) g(x)dx\\
	&=\int\left( -i\hbar \frac{\partial}{\partial x}f(x)\right) ^{\dagger}g(x)dx\\
	&=\int(pf)^{\dagger}g dx,\text{  So momentum is hermitian .}\\
\text{	Since -i}\hbar\implies i\hbar;\frac{\partial}{\partial x} \implies -\frac{\partial}{\partial x}&\text{ to keep momentum operator hermitian,so } \frac{\partial}{\partial x} \text{is anti hermitian.}\\
	\text{So }\left( \frac{d^n}{dx^n}\right) ^{\dagger}&=(-1)^n \left( \frac{d^n}{dx^n}\right)\\
\end{aligned}
$$
	Its a good position to discuss the time evolution of a state vector .The time evolution of the state vector is prescribed by the rule called schrodinger equation which is given as\\
$$i\hbar \frac{d}{dt}|\psi(t)\rangle =\hat{H}|\psi(t)\rangle $$
This is not an eigen value equation since $\hat{H}$ is an operator.\\
Since we are only dealing with autonomous system $\hat{H}$ is explicitly time independant.So we can write the formal solution \\
$$|\psi(t) \rangle =e^{-\frac{i}{\hbar}\hat{H(t)}} |\psi(o)\rangle $$
or $$|\psi(t) \rangle =e^{\frac{-i}{\hbar}\hat{H}(t-t_0)} |\psi(t_0)\rangle$$
Let's denote $e^{\frac{-i}{\hbar}\hat{H}(t-t_0)}$ as U.
Now $$(U^{\dagger})=e^{\frac{i}{\hbar}\hat{H}(t-t_0)}=U^{-1}$$
ie $$U^{\dagger}U^{-1}=U^{-1}U=I\implies \text{ unitary operator }$$
It is interesting thing is that even in non-automonous system time evolution operator is unitary operator.,but formal solution went be this \\
If the system changes from $t_0$ to $  t_2$\\
Then, $$U(t_2,t_0)=U(t_2,t_1)U(t_1,t_0)$$
Product of unitary operator is unitary \\
If we normalized $|\psi(0)\rangle$ would it be preserved in time evolution.
$$
\begin{aligned}
	|\psi(t) \rangle &=e^{-\frac{i}{\hbar}\hat{H(t)}} |\psi(o)\rangle \\
	\text{So }\langle \psi(t)|&=\langle \psi(0)|e^{\frac{i}{\hbar}
		H^{\dagger}t}\\
\text{	So }\langle \psi(t)\mid \psi(t)\rangle&=\langle \psi(o)|e^{\frac{i}{\hbar}H^{\dagger}t}e^{\frac{-i}{\hbar}H^{\dagger}t}\mid \psi(o)\rangle\\
\text{	Remember }\quad e^{\hat{A}}e^{\hat{B}}&=e^{\hat{A}+\hat{B}}
\text{	If }\left[ A,B\right] =0\\
	 \left[ H^{\dagger},H\right] &=0 \quad\text{ Since H is hermitian.}\\
\text{	So }\langle \psi(t)\mid \psi(t)\rangle&=\langle \psi_{0}|\psi_{0}\rangle
\end{aligned}
$$
Ie probability is conserved in time evolution as in classical mechanics.(liouville's thoerem).The preservation of probality follows from the unitary of the time evolution operator.\\
Now we have studied the schrodinger equation in the abstract basis .Then how does it look like in position basis.\\
we have \\
$$
\begin{aligned}
i\hbar \frac{d}{dt}|\psi(t)\rangle &=\hat{H}|\psi(t)\rangle 
\text{with} \hat{H}=\frac{p^2}{2m}+V(r)\\
\text{in position basis (3-D)}\\
\langle r|i\hbar \frac{d}{dt}|\psi(t)\rangle&=\langle r|\hat{H}|\psi(t)\rangle\\
i\hbar \frac{d}{dt}\langle |\psi(t)\rangle&=\langle r|\frac{p^2}{2m}+V(r)|\psi(t)\rangle\\
i\hbar\frac{\partial}{\partial t}\psi(r,t)&=-\frac{\hbar^2}{2m} \nabla^2\langle r|\psi(t)\rangle+\langle r|v(r)|\psi(t)\rangle\\
&=-\frac{\hbar^2}{2m}\nabla^2\psi(r,t)+v(r)\psi(r,t)\\
i\hbar \frac{\partial}{\partial t}\psi(r,t)&=-\frac{\hbar^2}{2m} \nabla^2 \psi(r,t)+v(r)\psi(r,t)\\
\text{This is partial differential }&\text{equation,first order in time.}
\end{aligned}
$$
Let's calculate
$$
\begin{aligned}
\langle x|\hat{P}|x^{\prime}&=\langle x|-i\hbar \frac{\partial}{\partial x}|x^{\prime}\\
&=-i\hbar \frac{\partial}{\partial x} \langle x|x^{\prime} \rangle\\
\text{similarly}\\
\langle x|\hat{P}|\hat{P}\rangle&=-i\hbar \frac{\partial}{\partial x}\langle x|p\rangle\\
p\langle x|\hat{P}\rangle&=-i\hbar\frac{\partial}{\partial x} \langle x|p\rangle\\
\frac{\partial \langle x|p\rangle}{\langle x|\hat{p}\rangle}&=\frac{p}{-i\hbar}\partial x\\
\langle x|p\rangle &\propto e^{\frac{ipx}{\hbar}}\\
\langle p|x\rangle &\propto e^{\frac{-ipx}{\hbar}}\\
\text{so we get}
\psi(x,t)&=\int_{-\infty}^{+\infty} dp\langle x|p\rangle \hat{\psi}(p,t)\\
\psi(x,t)&=\frac{1}{\sqrt{2\pi \hbar}}\int_{-\infty}^{+\infty} dp e^{ipx} \vec{\psi}(p,t)\\
\vec{\psi}(p,t)&=\frac{1}{\sqrt{2\pi \hbar}} \int_{-\infty}^{+\infty}dxe^{-ipx} \psi(x,t)\\
\end{aligned}
$$
In 3-D
$$
\begin{aligned}
\psi(r,t)&=\frac{1}{(2\pi \hbar)^{3/2}}\int_{-\infty}^{+\infty}d^3r e^{\frac{i\vec{p}\cdot \vec{r}}{\hbar}} \vec{\psi}(p,t)\\
\vec{\psi}(p,t)&=\frac{1}{(2\pi \hbar)^{3/2}}\int_{-\infty}^{+\infty}d^3p e^{\frac{-i\vec{p}\cdot \vec{r}}{\hbar}} \psi(x,t)
\end{aligned}
$$
Turns out that momentum space wave function is the fourier transform of the position space wave function.\\
Parsevals theorem guarantee that\\
$$\int_{-\infty}^{+\infty} d^3p |\psi(p,t)|^2=\int_{-\infty}^{+\infty} d^3r |\psi(r,t)|^2$$
ie if wavefunction is normalized in position space it will be normalized in momentum space.
\section{Parity Operator}
The space reflection about the origin of the coordinate system is called inversion or a parity operation.This transformation is descrete. The parity operator $\hat{P}$ is defined by its action on the $|\vec{r}\rangle$ of the position space:
$$\hat{P}|\vec{r}\rangle =|\vec{-r}\rangle$$
such that 
 $$\hat{P}\psi(\vec{r})=\psi(-\vec{r})$$
 \begin{note}
 	\begin{itemize}
 		\item Parity operator is Hermitian $\vec{P}^{\dagger}=\vec{P}$
 		\item From the definition we have\\
 		$$\vec{P}^2 \psi(\vec{r})=\vec{P}\psi(-\vec{r})=\psi(\vec{r})$$
 		Hence $\vec{P}^2$ is equal to the unity operator\\\\
 		$\vec{P}^2=I$ or $\vec{P}=\vec{P}^{-1}$\\
 		\item The parity operator is therefore unitary, since its Hermitian adjoint is equal to its inverse.\\
 		$$\vec{P}^{\dagger}=\vec{P}^{-1}$$ 
 		\item Now, since $\hat{P}^{2}=\hat{I}$, the eigenvalues of $\hat{P}$ are $+1$ or $-I$ with the corresponding eigenstates
 		$$
 		\hat{P} \psi_{+}(\vec{r})=\psi_{+}(-\vec{r})=\psi_{+}(\vec{r}), \quad \hat{P} \psi_{-}(\vec{r})=\psi_{-}(-\vec{r})=-\psi_{-}(\vec{r}) .
 		$$
 		The eigenstate $\left|\psi_{+}\right\rangle$is said to be even and $\left|\psi_{-}\right\rangle$is odd. Therefore, the eigenfunctions of the parity operator have definite parity: they are either even or odd.\\
 		\item \textbf{Even and Odd Operators}\\
 		An operator $\hat{A}$ is said to be $e v e n$ if it obeys the condition
 		$$
 		\hat{P} \hat{A} \hat{P}=\hat{A}
 		$$
 		and an operator $\hat{B}$ is odd if
 		$$
 		\hat{P} \hat{B} \hat{P}=-\hat{B}
 		$$
 		We can easily verify that even operators commute with the parity operator $\hat{P}$ and that odd operators anticommute with $\hat{P}$ :
 		$$
 		\begin{aligned}
 		\hat{A} \hat{P} &=(\hat{P} \hat{A} \hat{P}) \hat{P}=\hat{P} \hat{A} \hat{P}^{2}=\hat{P}\hat{A} \\
 		\hat{B} \hat{P} &=-(\hat{P} \hat{B} \hat{P}) \hat{P}=-\hat{P} \hat{B} \hat{P}^{2}=-\hat{P} \hat{B}
 		\end{aligned}
 		$$
 	\end{itemize}
 \end{note}

\newpage 
\begin{abox}
	Practice set 1
	\end{abox}
\begin{enumerate}
	\begin{minipage}{\textwidth}
	\item Consider a particle in a one dimensional potential that satisfies $V(x)=V(-x)$. Let $\left|\psi_{0}\right\rangle$ and $\left|\psi_{1}\right\rangle$ denote the ground and the first excited states, respectively, and let $|\psi\rangle=\alpha_{0}\left|\psi_{0}\right\rangle+\alpha_{1}\left|\psi_{1}\right\rangle$ be a normalized state with $\alpha_{0}$ and $\alpha_{1}$ being real constants. The expectation value $\langle x\rangle$ of the position operator $x$ in the state $|\psi\rangle$ is given by
	\exyear{NET DEC 2011}
\end{minipage}
\begin{tasks}(2)
	\task[\textbf{A.}] $\alpha_{0}^{2}\left\langle\psi_{0}|x| \psi_{0}\right\rangle+\alpha_{1}^{2}\left\langle\psi_{1}|x| \psi_{1}\right\rangle$
	\task[\textbf{B.}]$\alpha_{0} \alpha_{1}\left[\left\langle\psi_{0}|x| \psi_{1}\right\rangle+\left\langle\psi_{1}|x| \psi_{0}\right\rangle\right]$
	\task[\textbf{C.}]$\alpha_{0}^{2}+\alpha_{1}^{2}$
	\task[\textbf{D.}]$2 \alpha_{0} \alpha_{1}$
\end{tasks}
\begin{minipage}{\textwidth}
	\item The wave function of a particle at time $t=0$ is given by $|\psi(0)\rangle=\frac{1}{\sqrt{2}}\left(\left|u_{1}\right\rangle+\left|u_{2}\right\rangle\right)$, where
	$\left|u_{1}\right\rangle$ and $\left|u_{2}\right\rangle$ are the normalized eigenstates with eigenvalues $E_{1}$ and $E_{2}$ respectively, $\left(E_{2}>E_{1}\right)$. The shortest time after which $|\psi(t)\rangle$ will become orthogonal to $|\psi(0)\rangle$ is
	\exyear{NET DEC 2011}
\end{minipage}
\begin{tasks}(2)
	\task[\textbf{A.}] $\frac{-\hbar \pi}{2\left(E_{2}-E_{1}\right)}$
	\task[\textbf{B.}]$\frac{\hbar \pi}{E_{2}-E_{1}}$
	\task[\textbf{C.}]$\frac{\sqrt{2} \hbar \pi}{E_{2}-E_{1}}$
	\task[\textbf{D.}]$\frac{2 \hbar \pi}{E_{2}-E_{1}}$
\end{tasks}
\begin{minipage}{\textwidth}
	\item $\text { The commutator }\left[x^{2}, p^{2}\right] \text { is }$
	\exyear{NET JUNE 2012}
\end{minipage}
\begin{tasks}(2)
	\task[\textbf{A.}] $2 i \hbar x p$
	\task[\textbf{B.}]$2 i \hbar(x p+p x)$
	\task[\textbf{C.}]$2 i \hbar p x$
	\task[\textbf{D.}]$2 i \hbar(x p-p x)$
\end{tasks}
\begin{minipage}{\textwidth}
	\item Which of the following is a self-adjoint operator in the spherical polar coordinate system $(r, \theta, \phi)$ ?
	\exyear{NET JUNE 2012}
\end{minipage}
\begin{tasks}(2)
	\task[\textbf{A.}] $-\frac{i \hbar}{\sin ^{2} \theta} \frac{\partial}{\partial \theta}$
	\task[\textbf{B.}]$-i \hbar \frac{\partial}{\partial \theta}$
	\task[\textbf{C.}] $-\frac{i \hbar}{\sin \theta} \frac{\partial}{\partial \theta}$
	\task[\textbf{D.}] $-i \hbar \sin \theta \frac{\partial}{\partial \theta}$
\end{tasks}
\begin{minipage}{\textwidth}
	\item Given the usual canonical commutation relations, the commutator $[A, B]$ of $A=i\left(x p_{y}-y p_{x}\right)$ and $B=\left(y p_{z}+z p_{y}\right)$ is
	\exyear{NET DEC 2012}
\end{minipage}
\begin{tasks}(2)
	\task[\textbf{A.}] $\hbar\left(x p_{z}-p_{x} z\right)$
	\task[\textbf{B.}]$-\hbar\left(x p_{z}-p_{x} z\right)$
	\task[\textbf{C.}]$\hbar\left(x p_{z}+p_{x} z\right)$
	\task[\textbf{D.}]$-\hbar\left(x p_{z}+p_{x} z\right)$
\end{tasks}
\begin{minipage}{\textwidth}
	\item If the operators $A$ and $B$ satisfy the commutation relation $[A, B]=I$, where $I$ is the identity operator, then
	\exyear{NET JUNE 2013}
\end{minipage}
\begin{tasks}(2)
	\task[\textbf{A.}] $\left[e^{A}, B\right]=e^{A}$
	\task[\textbf{B.}]$\left[e^{A}, B\right]=\left[e^{B}, A\right]$
	\task[\textbf{C.}]$\left[e^{A}, B\right]=\left[e^{-B}, A\right]$
	\task[\textbf{D.}]$\left[e^{A}, B\right]=I$
\end{tasks}
\begin{minipage}{\textwidth}
	\item Suppose Hamiltonian of a conservative system in classical mechanics is $H=\omega x p$, where $\omega$ is a constant and $x$ and $p$ are the position and momentum respectively. The corresponding Hamiltonian in quantum mechanics, in the coordinate representation, is
	\exyear{NET DEC 2014}
\end{minipage}
\begin{tasks}(2)
	\task[\textbf{A.}] $-i \hbar \omega\left(x \frac{\partial}{\partial x}-\frac{1}{2}\right)$
	\task[\textbf{B.}]$-i \hbar \omega\left(x \frac{\partial}{\partial x}+\frac{1}{2}\right)$
	\task[\textbf{C.}] $-i \hbar \omega x \frac{\partial}{\partial x}$
	\task[\textbf{D.}]$-\frac{i \hbar \omega}{2} \times \frac{\partial}{\partial x}$
\end{tasks}
\begin{minipage}{\textwidth}
	\item Let $x$ and $p$ denote, respectively, the coordinate and momentum operators satisfying the canonical commutation relation $[x, p]=i$ in natural units $(\hbar=1)$. Then the commutator $\left[x, p e^{-p}\right]$ is
	\exyear{NET DEC 2014}
\end{minipage}
\begin{tasks}(2)
	\task[\textbf{A.}] $i(1-p) e^{-p}$
	\task[\textbf{B.}]$i\left(1-p^{2}\right) e^{-p}$
	\task[\textbf{C.}]$i\left(1-e^{-p}\right)$
	\task[\textbf{D.}]ipe $^{-p}$
\end{tasks}
\begin{minipage}{\textwidth}
	\item The wavefunction of a particle in one-dimension is denoted by $\psi(x)$ in the coordinate representation and by $\phi(p)=\int \psi(x) e^{\frac{-i p x}{\hbar}} d x$ in the momentum representation. If the action of an operator $\hat{T}$ on $\psi(x)$ is given by $\hat{T} \psi(x)=\psi(x+a)$, where $a$ is a constant then $\hat{T} \phi(p)$ is given by
	\exyear{NET JUNE 2015}
\end{minipage}
\begin{tasks}(2)
	\task[\textbf{A.}] $-\frac{i}{\hbar} \operatorname{ap} \phi(p)$
	\task[\textbf{B.}]$e^{\frac{-i a p}{\hbar}} \phi(p)$
	\task[\textbf{C.}]$e^{\frac{+i a p}{\hbar}} \phi(p)$
	\task[\textbf{D.}]$\left(1+\frac{i}{\hbar} a p\right) \phi(p)$
\end{tasks}
\begin{minipage}{\textwidth}
	\item Two different sets of orthogonal basis vectors $\left\{\left(\begin{array}{l}1 \\ 0\end{array}\right),\left(\begin{array}{l}0 \\ 1\end{array}\right)\right\}$ and $\left\{\frac{1}{\sqrt{2}}\left(\begin{array}{l}1 \\ 1\end{array}\right), \frac{1}{\sqrt{2}}\left(\begin{array}{c}1 \\ -1\end{array}\right)\right\}$ are given for a two dimensional real vector space. The matrix representation of a linear operator $\hat{A}$ in these basis are related by a unitary transformation. The unitary matrix may be chosen to be
	\exyear{NET JUNE 2015}
\end{minipage}
\begin{tasks}(2)
	\task[\textbf{A.}] $\left(\begin{array}{cc}0 & -1 \\ 1 & 0\end{array}\right)$
	\task[\textbf{B.}]$\left(\begin{array}{ll}0 & 1 \\ 1 & 0\end{array}\right)$
	\task[\textbf{C.}]$\frac{1}{\sqrt{2}}\left(\begin{array}{cc}1 & 1 \\ 1 & -1\end{array}\right)$
	\task[\textbf{D.}] $\frac{1}{\sqrt{2}}\left(\begin{array}{ll}1 & 0 \\ 1 & 1\end{array}\right)$
\end{tasks}
\begin{minipage}{\textwidth}
	\item A Hermitian operator $\hat{O}$ has two normalized eigenstates $|1\rangle$ and $|2\rangle$ with eigenvalues 1 and 2 , respectively. The two states $|u\rangle=\cos \theta|1\rangle+\sin \theta|2\rangle$ and $|v\rangle=\cos \phi|1\rangle+\sin \phi|2\rangle$ are such that $\langle v|\hat{O}| v\rangle=7 / 4$ and $\langle u \mid v\rangle=0$. Which of the following are possible values of $\theta$ and $\phi$ ?
	\exyear{NET DEC 2015}
\end{minipage}
\begin{tasks}(2)
	\task[\textbf{A.}] $\theta=-\frac{\pi}{6}$ and $\phi=\frac{\pi}{3}$
	\task[\textbf{B.}]$\theta=\frac{\pi}{6}$ and $\phi=\frac{\pi}{3}$
	\task[\textbf{C.}]$\theta=-\frac{\pi}{4}$ and $\phi=\frac{\pi}{4}$
	\task[\textbf{D.}]$\theta=\frac{\pi}{3}$ and $\phi=-\frac{\pi}{6}$
\end{tasks}
\begin{minipage}{\textwidth}
	\item If $\hat{L}_{x}, \hat{L}_{y}, \hat{L}_{z}$ are the components of the angular momentum operator in three dimensions the commutator $\left[\hat{L}_{x}, \hat{L}_{x} \hat{L}_{y} \hat{L}_{z}\right]$ may be simplified to
	\exyear{NET JUNE 2016}
\end{minipage}
\begin{tasks}(2)
	\task[\textbf{A.}] $i \hbar L_{x}\left(\hat{L}_{z}^{2}-\hat{L}_{y}^{2}\right)$
	\task[\textbf{B.}]$i \hbar \hat{L}_{z} \hat{L}_{y} \hat{L}_{x}$
	\task[\textbf{C.}]$i \hbar L_{x}\left(2 \hat{L}_{z}^{2}-\hat{L}_{y}^{2}\right)$
	\task[\textbf{D.}]0
\end{tasks}
\begin{minipage}{\textwidth}
	\item Consider the operator, $a=x+\frac{d}{d x}$ acting on smooth function of $x$. Then commutator $[\alpha, \cos x]$ is
	\exyear{NET DEC 2016}
\end{minipage}
\begin{tasks}(2)
	\task[\textbf{A.}] $-\sin x$
	\task[\textbf{B.}]$\cos x$
	\task[\textbf{C.}]$-\cos x$
	\task[\textbf{D.}]0
\end{tasks}
\begin{minipage}{\textwidth}
	\item Consider the operator $\vec{\pi}=\vec{p}-q \vec{A}$, where $\vec{p}$ is the momentum operator, $\vec{A}=\left(A_{x}, A_{y}, A_{z}\right)$ is the vector potential and $q$ denotes the electric charge. If $\vec{B}=\left(B_{x}, B_{y}, B_{z}\right)$ denotes the magnetic field, the $z$-component of the vector operator $\vec{\pi} \times \vec{\pi}$ is
	\exyear{NET DEC 2016}
\end{minipage}
\begin{tasks}(2)
	\task[\textbf{A.}] $i q \hbar B_{z}+q\left(A_{x} p_{y}-A_{y} p_{x}\right)$
	\task[\textbf{B.}]$-i q \hbar B_{z}-q\left(A_{x} p_{y}-A_{y} p_{x}\right)$
	\task[\textbf{C.}]$-i q \hbar B_{2}$
	\task[\textbf{D.}] $i q \hbar B_{z}$
\end{tasks}
\begin{minipage}{\textwidth}
	\item $\text { The two vectors }\left(\begin{array}{l}
	a \\
	0
	\end{array}\right) \text { and }\left(\begin{array}{l}
	b \\
	c
	\end{array}\right) \text { are orthonormal if }$
	\exyear{NET JUNE 2017}
\end{minipage}
\begin{tasks}(2)
	\task[\textbf{A.}] $a=\pm 1, b=\pm 1 / \sqrt{2}, c=\pm 1 / \sqrt{2}$
	\task[\textbf{B.}] $a=\pm 1, b=\pm 1, c=0$
	\task[\textbf{C.}]$a=\pm 1, b=0, c=\pm 1$
	\task[\textbf{D.}] $a=\pm 1, b=\pm 1 / 2, c=1 / 2$
\end{tasks}
\begin{minipage}{\textwidth}
	\item Let $x$ denote the position operator and $p$ the canonically conjugate momentum operator of a particle. The commutator
	$$
	\left[\frac{1}{2 m} p^{2}+\beta x^{2}, \frac{1}{m} p^{2}+\gamma x^{2}\right]
	$$
	where $\beta$ and $\gamma$ are constants, is zero if
	\exyear{NET DEC 2017}
\end{minipage}
\begin{tasks}(2)
	\task[\textbf{A.}] $\gamma=\beta$
	\task[\textbf{B.}]$\gamma=2 \beta$
	\task[\textbf{C.}]$\gamma=\sqrt{2} \beta$
	\task[\textbf{D.}]$2 \gamma=\beta$
\end{tasks}
\begin{minipage}{\textwidth}
	\item Consider the operator $A_{x}=L_{y} p_{z}-L_{z} p_{y}$, where $L_{i}$ and $p_{i}$ denote, respectively, the components of the angular momentum and momentum operators. The commutator $\left[A_{x}, x\right]$ where $x$ is the $x$ - component of the position operator, is
	\exyear{NET DEC 2018}
\end{minipage}
\begin{tasks}(2)
	\task[\textbf{A.}] $-i \hbar\left(z p_{z}+y p_{y}\right)$
	\task[\textbf{B.}]$-i \hbar\left(z p_{z}-y p_{y}\right)$
	\task[\textbf{C.}]$i \hbar\left(z p_{z}+y p_{y}\right)$
	\task[\textbf{D.}]$i \hbar\left(z p_{z}-y p_{y}\right)$
\end{tasks}
\end{enumerate}


\colorlet{ocre1}{ocre!70!}
\colorlet{ocrel}{ocre!30!}
\setlength\arrayrulewidth{1pt}
\begin{table}[H]
	\centering
	\arrayrulecolor{ocre}
	
	\begin{tabular}{|p{1.5cm}|p{1.5cm}||p{1.5cm}|p{1.5cm}|}
		\hline
		\multicolumn{4}{|c|}{\textbf{Answer key}}\\\hline\hline
		\rowcolor{ocrel}Q.No.&Answer&Q.No.&Answer\\\hline
		1&\textbf{b}&2&\textbf{b}\\\hline
		3&\textbf{b}&4&\textbf{c}\\\hline
		5&\textbf{c}&6&\textbf{a}\\\hline
		7&\textbf{b}&8&\textbf{a}\\\hline
		9&\textbf{c}&10&\textbf{c}\\\hline
		11&\textbf{a}&12&\textbf{a}\\\hline
		13&\textbf{a}&14&\textbf{d}\\\hline
		15&\textbf{c}&16&\textbf{b}\\\hline
		17&\textbf{a}&&\\\hline
	\end{tabular}
\end{table}

























\newpage
\begin{abox}
	Practice set 2
	\end{abox}
\begin{enumerate}
	\begin{minipage}{\textwidth}
		\item The quantum mechanical operator for the momentum of a particle moving in one dimension is given by
		\exyear{GATE 2011}
	\end{minipage}
	\begin{tasks}(2)
		\task[\textbf{A.}] $i \hbar \frac{d}{d x}$
		\task[\textbf{B.}]$-i \hbar \frac{d}{d x}$
		\task[\textbf{C.}]$i \hbar \frac{\partial}{\partial t}$
		\task[\textbf{D.}]$-\frac{\hbar^{2}}{2 m} \frac{d^{2}}{d x^{2}}$
	\end{tasks}
\begin{minipage}{\textwidth}
	\item If $L_{x}, L_{y}$ and $L_{z}$ are respectively the $x, y$ and $z$ components of angular momentum operator $L$. The commutator $\left[L_{x} L_{y}, L_{z}\right]$ is equal to
	\exyear{GATE 2011}
\end{minipage}
\begin{tasks}(2)
	\task[\textbf{A.}] $i \hbar\left(L_{x}^{2}+L_{y}^{2}\right)$
	\task[\textbf{B.}]$2 i \hbar L_{z}$
	\task[\textbf{C.}]$i \hbar\left(L_{x}^{2}-L_{y}^{2}\right)$
	\task[\textbf{D.}] 0
\end{tasks}
\textbf{common data questions 3 and 4 }\\
In a one-dimensional harmonic oscillator, $\varphi_{0}, \varphi_{1}$ and $\varphi_{2}$ are respectively the ground, first and the second excited states. These three states are normalized and are orthogonal to one another $\psi_{1}$ and $\psi_{2}$ are two states defined by
$$
\psi_{1}=\varphi_{0}-2 \varphi_{1}+3 \varphi_{2}, \psi_{2}=\varphi_{0}-\varphi_{1}+\alpha \varphi_{2}, \psi_{2}=\varphi_{0}-\varphi_{1}+\alpha \varphi_{2}
$$
where $\alpha$ is a constant\\
\begin{minipage}{\textwidth}
	\item $\text { The value of } \alpha \text { which } \psi_{2} \text { is orthogonal to } \psi_{1} \text { is }$
	\exyear{GATE 2011}
\end{minipage}
\begin{tasks}(4)
	\task[\textbf{A.}]2
	\task[\textbf{B.}]1
	\task[\textbf{C.}]-1
	\task[\textbf{D.}]-2
\end{tasks}
\begin{minipage}{\textwidth}
	\item For the value of $\alpha$ determined in $\mathrm{Q} 3$, the expectation value of energy of the oscillator in the state $\psi_{2}$ is
\end{minipage}
\begin{tasks}(4)
	\task[\textbf{A.}] $\hbar \omega$
	\task[\textbf{B.}]$3 \hbar \omega / 2$ 
	\task[\textbf{C.}]$3 \hbar \omega$
	\task[\textbf{D.}]$9 \hbar \omega / 2$
\end{tasks}
\begin{minipage}{\textwidth}
	\item Which one of the following commutation relations is NOT CORRECT? Here, symbols have their usual meanings.
	\exyear{GATE 2013}
\end{minipage}
\begin{tasks}(2)
	\task[\textbf{A.}] $\left[L^{2}, L_{z}\right]=0$
	\task[\textbf{B.}]$\left\lfloor L_{x}, L_{y}\right\rfloor=i \hbar L_{z}$
	\task[\textbf{C.}]$\left[L_{z}, L_{+}\right]=\hbar L_{+}$
	\task[\textbf{D.}] $\left[L_{z}, L_{-}\right]=\hbar L_{-}$
\end{tasks}
\begin{minipage}{\textwidth}
	\item Let $\vec{L}$ and $\vec{p}$ be the angular and linear momentum operators, respectively, for a a particle. The commutator $\left\lfloor L_{x}, p_{y}\right\rfloor$ gives
	\exyear{GATE 2015}
\end{minipage}
\begin{tasks}(4)
	\task[\textbf{A.}] $-i \hbar p_{z}$
	\task[\textbf{B.}]0
	\task[\textbf{C.}]$i \hbar p_{x}$
	\task[\textbf{D.}]$i \hbar p_{z}$
\end{tasks}
\begin{minipage}{\textwidth}
	\item $\text { Which of the following operators is Hermitian? }$
	\exyear{GATE 2016}
\end{minipage}
\begin{tasks}(4)
	\task[\textbf{A.}] $\frac{d}{d x}$
	\task[\textbf{B.}]$\frac{d^{2}}{d x^{2}}$
	\task[\textbf{C.}]$i \frac{d^{2}}{d x^{2}}$
	\task[\textbf{D.}]$\frac{d^{3}}{d x^{3}}$
\end{tasks}
\begin{minipage}{\textwidth}
	\item If $x$ and $p$ are the $x$ components of the position and the momentum operators of a particle respectively, the commutator $\left[x^{2}, p^{2}\right]$ is
	\exyear{GATE 2016}
\end{minipage}
\begin{tasks}(2)
	\task[\textbf{A.}] $i \hbar(x p-p x)$
	\task[\textbf{B.}]$2 i \hbar(x p-p x)$
	\task[\textbf{C.}]$i \hbar(x p+p x)$
	\task[\textbf{D.}]$2 i \hbar(x p+p x)$
\end{tasks}
\begin{minipage}{\textwidth}
	\item $\text { For the parity operator } P, \text { which of the following statements is NOT true? }$
	\exyear{GATE 2016}
\end{minipage}
\begin{tasks}(2)
	\task[\textbf{A.}] $P^{\dagger}=P$
	\task[\textbf{B.}] $P^{2}=-P$
	\task[\textbf{C.}] $P^{2}=I$
	\task[\textbf{D.}]$P^{\dagger}=P^{-1}$
\end{tasks}
\begin{minipage}{\textwidth}
	\item $\text { Which one of the following operators is Hermitian? }$
	\exyear{GATE 2017}
\end{minipage}
\begin{tasks}(2)
	\task[\textbf{A.}] $i \frac{\left(p_{x} x^{2}-x^{2} p_{x}\right)}{2}$
	\task[\textbf{B.}]$i \frac{\left(p_{x} x^{2}+x^{2} p_{x}\right)}{2}$
	\task[\textbf{C.}]$e^{i p_{x} a}$
	\task[\textbf{D.}]$e^{-i p_{x} a}$
\end{tasks}
\end{enumerate}
\colorlet{ocre1}{ocre!70!}
\colorlet{ocrel}{ocre!30!}
\setlength\arrayrulewidth{1pt}
\begin{table}[H]
	\centering
	\arrayrulecolor{ocre}
	
	\begin{tabular}{|p{1.5cm}|p{1.5cm}||p{1.5cm}|p{1.5cm}|}
		\hline
		\multicolumn{4}{|c|}{\textbf{Answer key}}\\\hline\hline
		\rowcolor{ocrel}Q.No.&Answer&Q.No.&Answer\\\hline
		1&\textbf{b}&2&\textbf{c}\\\hline
		3&\textbf{c}&4&\textbf{b}\\\hline
		5&\textbf{d}&6&\textbf{d}\\\hline
		7&\textbf{b}&8&\textbf{d}\\\hline
		9&\textbf{b}&10&\textbf{a}\\\hline
	\end{tabular}
\end{table}
\newpage
\begin{abox}
	Practice set 3
	\end{abox}
\begin{enumerate}
	\begin{minipage}{\textwidth}
	\item If $\left|\phi_{1}\right\rangle$ and $\left|\phi_{2}\right\rangle$ be two orthonormal state vectors such that $A=\left|\phi_{1}\right\rangle\left(\phi_{2}|+| \phi_{2}\right\rangle\langle\phi|$, then If $\left|\phi_{1}\right\rangle$ and $\left|\phi_{2}\right\rangle$ be two orthonormal state vectors such that $A=\left|\phi_{1}\right\rangle\left\langle\phi_{2}|+| \phi_{2}\right\rangle\left\langle\phi_{1}\right|$, then\\
	(a) Prove that $A$ is Hermitian\\
	(b) Find the value of $A^{2}$.
\end{minipage}
\begin{answer}
(a) For $A$ to be a projection operator, $A$ should be Hermitian and $A^{2}$ should be equal to $A$. The Hermitian adjoint of $\left|\phi_{1}\right\rangle\left\langle\phi_{2}\right|$ is $\left|\phi_{2}\right\rangle\left\langle\phi_{1}\right|$ and that of $\left|\phi_{2}\right\rangle\left\langle\phi_{1}\right|$ is $\left|\phi_{1}\right\rangle\left\langle\phi_{2}\right|$. So
	\begin{align*}
	&A^{\dagger}=\left[\left|\phi_{1}\right\rangle\left\langle\phi_{2}|+| \phi_{2}\right\rangle\left\langle\phi_{1}\right|\right]^{\dagger}=\left[\left|\phi_{1}\right\rangle\left\langle\phi_{2}\right|\right]^{\dagger}+\left[\left|\phi_{2}\right\rangle\left\langle\phi_{1}\right|\right]^{\dagger} \\
	&=\left|\phi_{2}\right\rangle\left\langle\phi_{1}|+| \phi_{1}\right\rangle\left\langle\phi_{2}\right|=A
	\end{align*}
	Hence $A$ is Hermitian.\\
\begin{align*}
	&\text { Now, } \quad A^{2}=\left[\left|\phi_{1}\right\rangle\left\langle\phi_{2}|+| \phi_{2}\right\rangle\left\langle\phi_{1}\right|\right]\left[\left|\phi_{1}\right\rangle\left\langle\phi_{2}|+| \phi_{2}\right\rangle\left\langle\phi_{1}\right|\right] \\
	&=\left|\phi_{1}\right\rangle\left\langle\phi_{2}\left|\left[\left|\phi_{1}\right\rangle\left\langle\phi_{2}|+| \phi_{2}\right\rangle\left\langle\phi_{1}\right|\right]+\right| \phi_{2}\right\rangle\left\langle\phi_{1}\right|\left[\left|\phi_{1}\right\rangle\left\langle\phi_{2}|+| \phi_{2}\right\rangle\left\langle\phi_{1}\right|\right] \\
	&=\left[\left|\phi_{1}\right\rangle\left\langle\phi_{2} \mid \phi_{1}\right\rangle\left\langle\phi_{2}|+| \phi_{1}\right\rangle\left\langle\phi_{2} \mid \phi_{2}\right\rangle\left\langle\phi_{1}\right|\right]+\left[\left|\phi_{2}\right\rangle\left\langle\phi_{1} \mid \phi_{1}\right\rangle\left\langle\phi_{2}|+| \phi_{2}\right\rangle\left\langle\phi_{1} \mid \phi_{2}\right\rangle\left\langle\phi_{1}\right|\right] \\
	&\text { Since }\left|\phi_{1}\right\rangle \text { and }\left|\phi_{2}\right\rangle \text { are orthonormal, } \\
	&\qquad A^{2}==\left|\phi_{1}\right\rangle\left\langle\phi_{1}|+| \phi_{2}\right\rangle\left\langle\phi_{2}\right|
\end{align*}	
\end{answer}
	\begin{minipage}{\textwidth}
	\item (a) Find the Eigen State of momentum operator $P_{x}=-i \hbar \frac{d}{d x}$. If eigen value is $\lambda$ by relation $P_{x} \phi=\lambda \phi$ where $\frac{\lambda}{\hbar}=k$.\\
	(b) Expand the wave function $\psi(x)=A \sin k x \sin 2 k x$ in basis of Eigen functions of momentum operator $P_{x}$
\end{minipage}
\begin{answer}
	$P_{x} \phi=\lambda \phi \text { where } \frac{\lambda}{\hbar}=k$\\\\
	Case 1: If $\lambda$ is positive $P_{x} \phi=\hbar k \phi \Rightarrow-i \frac{d \phi}{d x}=k \phi \Rightarrow \frac{d \phi}{\phi}=i k d x \Rightarrow \ln \phi=i k x+C \Rightarrow \phi=e^{i k x}$\\
	Case 2 : If $\lambda$ is negative\\
	$P_{x} \phi=\hbar k \phi \Rightarrow-i \frac{d \phi}{d x}=-k \phi \Rightarrow \frac{d \phi}{\phi}=-i k d x \Rightarrow \ln \phi=-i k x+C \Rightarrow \phi=e^{-i k x}$\\\\
	(b) Expand the function $\psi(x)=A \sin k x \sin 2 k x$ as a linear combination of eigenfunctions of the momentum operator $P_{x}$.
	\begin{align*}
	&\psi(x)=A \sin k x \sin 2 k x=A\left(\frac{e^{i k x}-e^{-i k x}}{2 i}\right)\left(\frac{e^{2 i k x}-e^{-2 i k x}}{2 i}\right) \\
	&\frac{A}{4}\left(-e^{-3 i k x}+e^{-i k x}+e^{i k x}-e^{3 i k x}\right)
	\end{align*}
\end{answer}
	\begin{minipage}{\textwidth}
	\item If $\left|\phi_{1}\right\rangle=A\left(\begin{array}{l}1 \\ 0 \\ 0\end{array}\right)\left|\phi_{2}\right\rangle=B\left(\begin{array}{l}0 \\ i \\ i\end{array}\right)\left|\phi_{3}\right\rangle=C\left(\begin{array}{c}0 \\ i \\ -i\end{array}\right)$\\
	(a) Find normalization constant $A, B, C$ for ket $\left|\phi_{1}\right\rangle\left|\phi_{2}\right\rangle\left|\phi_{3}\right\rangle$\\
	(b) Prove that $\left|\phi_{1}\right\rangle,\left|\phi_{2}\right\rangle$ and $\left|\phi_{3}\right\rangle$ are orthogonal\\
	(c) Check whether $\left|\phi_{1}\right\rangle,\left|\phi_{2}\right\rangle$ and $\left|\phi_{3}\right\rangle$ are linearly independent or not.
\end{minipage}
\begin{answer}
	(a) $\left\langle\phi_{1} \mid \phi_{1}\right\rangle=1 \Rightarrow A^{*}\left(\begin{array}{lll}1 & 0 & 0\end{array}\right) A\left(\begin{array}{l}1 \\ 0 \\ 0\end{array}\right)=1 \Rightarrow A=1$
	$$
	\left\langle\phi_{2} \mid \phi_{2}\right\rangle=1 \Rightarrow B^{*}\left(\begin{array}{lll}
	0 & -i & -i
	\end{array}\right) B\left(\begin{array}{l}
	0 \\
	i \\
	i
	\end{array}\right)=1 \Rightarrow B=\frac{1}{\sqrt{2}}
	$$
	$$
	\left\langle\phi_{3} \mid \phi_{3}\right\rangle=1 \Rightarrow C^{*}\left(\begin{array}{lll}
	0 & -i & i
	\end{array}\right) C\left(\begin{array}{c}
	0 \\
	i \\
	-i
	\end{array}\right)=1 \Rightarrow C=\frac{1}{\sqrt{2}}
	$$
	(b) $\left\langle\phi_{1} \mid \phi_{2}\right\rangle=A^{*}\left(\begin{array}{lll}1 & 0 & 0\end{array}\right) B\left(\begin{array}{l}0 \\ i \\ i\end{array}\right)=0$
	$$
	\left\langle\phi_{1} \mid \phi_{3}\right\rangle=A^{*}\left(\begin{array}{lll}
	1 & 0 & 0
	\end{array}\right) C\left(\begin{array}{c}
	0 \\
	i \\
	-i
	\end{array}\right)=0
	$$
	$$
	\left\langle\phi_{2} \mid \phi_{3}\right\rangle=A^{*}\left(\begin{array}{lll}
	0 & -i & -i
	\end{array}\right) C\left(\begin{array}{c}
	0 \\
	i \\
	-i
	\end{array}\right)=A^{*} C(0+1-1)=0
	$$
(c)	$c_{1}\left|\phi_{1}\right\rangle+c_{2}\left|\phi_{2}\right\rangle+c_{3}\left|\phi_{3}\right\rangle=0 \Rightarrow c_{1}\left(\begin{array}{c}
		1 \\
		0 \\
		0
	\end{array}\right)+c_{2}\left(\begin{array}{c}
		0 \\
		i \\
		i
	\end{array}\right)+c_{3}\left(\begin{array}{c}
		0 \\
		i \\
		-i
	\end{array}\right)=0$\\
	$$
	c_{1}=0 \quad c_{2}+c_{3}=0 \text { and } c_{2}-c_{3}=0 \Rightarrow c_{1}=0, c_{2}=0, c_{3}=0
	$$
	So $\left|\phi_{1}\right\rangle,\left|\phi_{2}\right\rangle$ and $\left|\phi_{3}\right\rangle$ are linearly independent
\end{answer}
	\begin{minipage}{\textwidth}
	\item If Hamiltonian of system is $H=\frac{p_{x}^{2}}{2 m}+V(x)$ then Find commutation $[H, x]$ and $[[H, x], x]$
\end{minipage}
\begin{answer}
	$\text { As, } H=p^{2} / 2 m+V(x)$\\\\
	We have, $[H, x]=\frac{1}{2 m}\left[p^{2}, x\right]=-i \hbar p / m$\\\\
	 and so, $[[H, x], x]=\frac{i \hbar}{m}[p, x]=-\hbar^{2} / m$\\\\
	  Hence, $\langle m|[[H, x], x]| m\rangle=-\frac{\hbar^{2}}{m}$.
\end{answer}
\end{enumerate}

