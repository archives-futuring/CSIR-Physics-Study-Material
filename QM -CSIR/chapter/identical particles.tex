\chapter{Identical particles}
\section{Introduction}
\textbf{Identical particles}\\
Particle having same intrinsic properties (mass, charge, spin) as known as identical particles. There are two types of identical particles i.e.
\begin{enumerate}
	\item Classical identical particle(Distiquishable particles)\\
	Volume available for a gas molecule at NTP is $\sim 10^{-25} \mathrm{~m}^{3}$ and volume of single gas molecule $\sim 10^{-30} \mathrm{~m}^{3}$ So, we can identify every molecule of the gas. Hence, gas molecules are distinguishable.
	\item Quantum identical particle(indistiquishable)
	Volume available for each electron taking part in conduction within the metals is $\sim 10^{-28} \mathrm{~m}^{3}$ but the volume of a single electron is $\sim 10^{-27} \mathrm{~m}^{3}$. This shows that the electron wave functions overlap considerably and hence cannot be identified separately, i.e. they are indistinguishable.
\end{enumerate}
\begin{note}
	For a system of 'n' indistinguishable particles say electrons, we may be able to specify their positions, but it is not possible to say which electron is at a particular position. Therfore, interchanging the coordinates of any two electrons does not change the Hamiltonian. In other words, the Hamiltonian of the system is symmetric in coordinates of the particles i.e.
	$$
	\hat{H}\left(x_{1}, x_{2}, x_{3} \ldots \ldots x_{i}, x_{i+1}, \ldots x_{n}\right)=\hat{H}\left(x_{1}, x_{2}, x_{3} \ldots \ldots x_{i+1}, x_{i}, \ldots x_{n}\right)
	$$
	\end{note}
\textbf{Particle exchange operator}\\
$\text { This operator exchanges the particle in pair (both position and spin co-ordinates). }$
\begin{note}
	$\text { (i) Particle exchange operator is a hermitian operator. }$\\
	(ii) Since particles are indistinguishable, any operator representing a physical property of the sys tem must be symmetric w.r.t the particle exchange. Therefore, any operator representing a physical property commute with the particle exchange operator.
\end{note}
\subsection{ Symmetric wave function:}
A wave function is said to be symmetric if interchange of any pair of particles leaves the wave function unchanged.
$$
\psi_{\mathrm{s}}(1,2, \ldots . \mathrm{i}, \mathrm{j} \ldots \mathrm{N})=\psi_{\mathrm{s}}(1,2 \ldots \ldots . \mathrm{j}, \mathrm{i}, \ldots . . \mathrm{N})
$$
The particles having symmetric wave function obey BE statistics and known as bosons.
Example: Photons (spin 1), neutral helium (spin 0), $\alpha$-particle (spin 0), deutron (spin 1)\\
\subsection{ Anti-symmetric wave function:}
A wave function is said to be anti-symmetric if interchange of any pair of particle changes the signof the wave function.
$$
\psi_{\mathrm{a}}(1,2, \ldots \ldots . \mathrm{i}, \mathrm{j}, \ldots \ldots \mathrm{N})=-\psi_{\mathrm{a}}(1,2, \ldots \ldots . \mathrm{j}, \mathrm{i} \ldots . . \mathrm{N})
$$
The particles having anti-symmetric wave functions are known as fermions and follow FD statistics. Example: Electron, protons, neutrons, muons (spin $1 / 2$ ), $\Omega$ particle (spin $3 / 2$ )
\section{Symmetrization postulate}
The state of the system containing $\mathrm{N}$ identical particles, are either totally symmetric or totally anti symmetric under the exchange of any pair of particles and the states with mixed symmetry do not exist.
\begin{note}
	(i) Bosons have symmetric wave functions and Fermions have anti-symmetric wave functions\\
	(ii) Wave function of a system of identical bosons is totally symmetric and of a system of identical fermions is totally anti-symmetric.
	
	The total wave function corresponding to a system containing of $\mathrm{N}$ identical particles consisting of two parts (a) space part and (b) spin part, i.e.
	Therefore,
	$$
	\psi_{\text {total }}=\psi_{\text {space }} \times \chi_{\text {spin }}
	$$
	Therefore\\
	(i) Both space part and spin part of total wave function of a system of identical bosons will be either symmetric or anti-symmetric.
	$$
	\psi_{s, f e r m i o n}=\left\{\begin{array}{l}
	\psi_{s}(\text { space }) \times \chi_{a}(\text { spin }) \\
	\psi_{a}(\text { space }) \times \chi_{s}(\text { spin })
	\end{array}\right.
	$$
	(ii) any one of the space part and spin part of total wave function of a system of identical fermions will be symmetric and other one is anti-symmetric.
	$$
	\psi_{s, j e r m i o n}=\left\{\begin{array}{l}
	\psi_{s}(\text { space }) \times \chi_{a}(\text { spin }) \\
	\psi_{a}(\text { space }) \times \chi_{s}(\text { spin })
	\end{array}\right.
	$$
	\end{note}
	\textbf{Composite particle:}\\
Consider a system of identical composite particles where each particle is composed of two or more identical particles. Spin of each composite particles can be obtained by adding the spin of its constituents. If total spin of the composite particles is half-integer, then the particle will behave as a fermion. If total spin of the composite particles is integer, then the particle will behave as a boson.\\
\textbf{example} \\
(i) A system of $\mathrm{N}$ identical $\mathrm{H}$ atoms. Each $\mathrm{H}$ atom consists of 1 proton and 1 electron (spin $\mathrm{I} / 2$) Therefore each H atom has spin 1 and it will behave as a boson.So the system of N identical H atoms will have symmetric wavefunction\\
(ii) ${ }^{4} \mathrm{He}$ (two protons and two meutrons) has integer spin and will behave as boson.\\
(iii)$^3He$(2 proton and 1 neutron) has half integral spin and will behave as fermion
\section{Symmetric and anti symmetric wavefunction(space part)}
For a system of ' $n$ ' non-interacting distinguishable particles, the wave function can be written as
$$
\psi\left(\vec{r}_{1}, \vec{r}_{2}, \vec{r}_{3} \ldots \ldots \ldots \vec{r}_{n}\right)=\psi_{1}\left(\vec{r}_{1}\right) \cdot \psi_{2}\left(\vec{r}_{2}\right) \cdot \psi_{3}\left(\vec{r}_{3}\right) \ldots \ldots \ldots \psi_{n}\left(\vec{r}_{n}\right)
$$
But, for a system of ' $n$ ' non-interacting indistinguishable particles, the wave function will be different.
Example:
Consider a system of 2 non-interacting indistinguishable particles, in which one particle is in state 1 and the other particle is in state 2. Since, the particles are distinguishable, therefore the possible eigenfunctions are $\psi_{1}\left(\vec{r}_{1}\right) \psi_{2}\left(\vec{r}_{2}\right)$ and $\psi_{1}\left(\vec{r}_{2}\right) \psi_{2}\left(\vec{r}_{1}\right)$ and both are equal probable. The symmetric and anti-symmetric combinations for the wave function of the system of 2 particles will be
$$
\begin{aligned}
&\psi_{s}\left(\vec{r}_{1}, \vec{r}_{2}\right)=\frac{1}{\sqrt{2}}\left[\psi_{1}\left(\vec{r}_{1}\right) \psi_{2}\left(\vec{r}_{2}\right)+\psi_{1}\left(\vec{r}_{2}\right) \psi_{2}\left(\vec{r}_{1}\right)\right] \\
&\psi_{a}\left(\vec{r}_{1}, \vec{r}_{2}\right)=\frac{1}{\sqrt{2}}\left[\psi_{1}\left(\vec{r}_{1}\right) \psi_{2}\left(\vec{r}_{2}\right)-\psi_{1}\left(\vec{r}_{2}\right) \psi_{2}\left(\vec{r}_{1}\right)\right]
\end{aligned}
$$
If both the particles are in the same state, then $\psi_{a}\left(\vec{r}_{1}, \vec{r}_{2}\right)=0$ i.e. two identical fermions cannot occupy the same state. This is known as 'Pauli's exclusion principle'.\\
In general, if we take a system of $\mathrm{N}$ identical non-interacting indistinguishable particles and each particle is in a different state, then there will be $N !$ different type possible eiegenfunctions having same value of energy. The degenaracy arises due to the exchange of indistingushable particles and known as exchange degenaracy. The symmetric and anti-symmetric combinations for the wave function of the system of $\mathrm{N}$ particles will be\\
$$\begin{aligned}
	&\psi_{S}\left(\vec{r}_{1}, \vec{r}_{2}, \ldots \ldots . \vec{r}_{N}\right)=\frac{1}{\sqrt{N !}} \sum_{P} \hat{P}\left[\psi_{1}\left(\vec{r}_{1}\right) \psi_{2}\left(\vec{r}_{2}\right) \psi_{3}\left(\vec{r}_{3}\right) \ldots \ldots \ldots \psi_{n}\left(\vec{r}_{N}\right)\right] \\
	&\psi_{a}\left(\vec{r}_{1}, \vec{r}_{2}, \ldots \ldots . \vec{r}_{N}\right)=\frac{1}{\sqrt{N !}}\left|\begin{array}{llll}
		\psi_{1}\left(\vec{r}_{1}\right) & \psi_{1}\left(\vec{r}_{2}\right) & \ldots \ldots \ldots \ldots & \psi_{1}\left(\vec{r}_{N}\right) \\
		\psi_{2}\left(\vec{r}_{1}\right) & \psi_{2}\left(\vec{r}_{2}\right) & \ldots \ldots \ldots \ldots & \psi_{2}\left(\vec{r}_{N}\right) \\
		\vdots & & & \\
		\psi_{N}\left(\vec{r}_{1}\right) & \psi_{N}\left(\vec{r}_{2}\right) & \ldots \ldots \ldots \ldots & \psi_{N}\left(\vec{r}_{N}\right)
	\end{array}\right|
\end{aligned}$$
\section{Symmetric and anti symmetric wavefunction(spin part)}
Spin functions for two-electron system:
Two electrons each having spin $1 / 2$ i.e. $s_{1}=\frac{1}{2}, s_{2}=\frac{1}{2}$, then the total spin of two electrons will be
$$
s=\left|s_{1}+s_{2}\right| \ldots \ldots\left|s_{1}-s_{2}\right|=1,0
$$
and corresponding magnetic spin quantum number will be $m_{s}=1,0,-1,0$
Therefore, the possible spin states will be $\left|s, m_{s}\right\rangle \equiv|0,0\rangle,|1,1\rangle,|1,0\rangle,|1,-1\rangle$ and given as follows:\\
\textbf{Symmetric Triplet state:}\\
$$
\begin{aligned}
|1,1\rangle &=|\uparrow \uparrow\rangle=|\alpha \alpha\rangle=\left|\frac{1}{2}, \frac{1}{2}\right\rangle\left|\frac{1}{2}, \frac{1}{2}\right\rangle \\
|1,-1\rangle &=|\downarrow \downarrow\rangle=|\beta \beta\rangle=\left|\frac{1}{2},-\frac{1}{2}\right\rangle\left(\frac{1}{2},-\frac{1}{2}\right\rangle \\
|1,0\rangle &=\frac{1}{\sqrt{2}}[|\uparrow \downarrow\rangle+|\downarrow \uparrow\rangle]=\frac{1}{\sqrt{2}}[|\alpha \beta\rangle+|\beta \alpha\rangle] \\
&=\frac{1}{\sqrt{2}}\left[\left|\frac{1}{2}, \frac{1}{2}\right\rangle\left\langle\frac{1}{2},-\frac{1}{2}\right\rangle+\left|\frac{1}{2},-\frac{1}{2}\right\rangle\left\langle\frac{1}{2}, \frac{1}{2}\right\rangle\right]
\end{aligned}
$$
\textbf{Anti-symmetric Singlet state:}
$$
\begin{aligned}
|0,0\rangle &=\frac{1}{\sqrt{2}}[|\uparrow \downarrow\rangle-|\downarrow \uparrow\rangle]=\frac{1}{\sqrt{2}}[|\alpha \beta\rangle-|\beta \alpha\rangle] \\
&\left.=\frac{1}{\sqrt{2}}\left[\left|\frac{1}{2}, \frac{1}{2}\right\rangle\left|\frac{1}{2},-\frac{1}{2}\right\rangle-\left|\frac{1}{2},-\frac{1}{2}\right\rangle\left|\frac{1}{2}, \frac{1}{2}\right\rangle\right]\right]
\end{aligned}
$$
For a system of $N$ identical non-interacting indistinguishable particles each having spin 's', total number of possible spin states is $(2 s+1)^{N}$.
\begin{exercise}
Example 1. $N$ non interacting bosons are in an infinite potential well defined by $V(x)=0$ for $0<x<a$; $V(x)=\infty$ for $x<0$ and for $x>a$. Find the ground state energy of the system. What would be the ground state energy if the particles are fermions.
	\end{exercise}
\begin{answer}
	The energy eigenvalue of a particle in the infinite square well is given by
	$$
	E_{n}=\frac{\pi^{2} \hbar^{2} n^{2}}{2 m a^{2}}, \quad n=1,2,3, \ldots
	$$
	As the particles are bosons, all the $\mathrm{N}$ particles will be in the $n=1$ state in the ground state configuration. Hence, the ground state energy of the configuration will be
	$$
	E=\frac{N \pi^{2} \hbar^{2}}{2 m a^{2}}
	$$
	If the particles are fermions, a state can have only two of them, one spin up and the other spin down.\\
	xilaal Part Therefore, the lowest $\mathrm{N} / 2$ states will be filled. The total ground state energy will be
	$$
	\begin{aligned}
	E &=2 \frac{\pi^{2} \hbar^{2}}{2 m a^{2}}\left[1^{2}+2^{2}+3^{3}+\ldots+(N / 2)^{2}\right] \\
	&=\frac{\pi^{2} \hbar^{2}}{m a^{2}} \frac{1}{6}\left[\frac{N}{2}\left(\frac{N}{2}+1\right)\left(2 \frac{N}{2}+1\right)\right] \\
	&=\frac{\pi^{2} \hbar^{2}}{24 m a^{2}} N(N+1)(N+2)
	\end{aligned}
	$$
\end{answer}
\begin{exercise}
 Consider two noninteracting electrons described by the Hamiltonian
	$$
	H=\frac{p_{1}^{2}}{2 m}+\frac{p_{2}^{2}}{2 m}+V\left(x_{1}\right)+V\left(x_{2}\right)
	$$
	where $V(x)=0$ for $0<x<a ; V(x)=\infty$ for $x<0$ and for $x>a$. If both the electrons are in the same spin state, what is the lowest energy and eigenfunction of the two-electron system?
	\end{exercise}
\begin{answer}
	As both the electrons are in the same spin state, the possible combinations of spin part of the wave functions will be $|\uparrow \uparrow\rangle$ or $|\downarrow \downarrow\rangle$, both being symmetric. Since, the system contains two electrons (fermions), then the total wave function of the sytem will be anti-symmetric. Hence the space function must be antisymmetric. Therefore, both electrons cannot be in the same state. So, ground state energy will corresponds to $n_{1}=1, n_{2}=2$\\
	Ground state energy $\left(n_{1}=1, n_{2}=2\right)=\frac{\pi^{2} \hbar^{2}}{2 m a^{2}}+\frac{4 \pi^{2} \hbar^{2}}{2 m a^{2}}=\frac{5 \pi^{2} \hbar^{2}}{2 m a^{2}}$\\
	Since, the electrons are indistinguishable, therefore the antisymmetric combination for the space part is
	$$
	\frac{1}{\sqrt{2}}\left[\psi_{1}\left(\vec{r}_{1}\right) \psi_{2}\left(\vec{r}_{2}\right)-\psi_{1}\left(\vec{r}_{2}\right) \psi_{2}\left(\vec{r}_{1}\right)\right]
	$$
\end{answer}
\begin{exercise}
	Example 3. Sixteen noninteracting electrons are confined in a potential $V(x)=\infty$ for $x<0$ and $x>0 ; V(x)=0$, for $0<x<a$.\\
	(i) What is the energy of the least energetic electron in the ground state?\\
	(ii) What is the energy of the most energetic electron in the ground state?\\
	(iii) What is the Fermi energy $E_{f}$ of the system?
	\end{exercise}
\begin{answer}
	(i) The least energetic electron in the ground state is given by $E_{1}=\frac{\pi^{2} \hbar^{2}}{2 m a^{2}}$.\\
	(ii) In the given potential, the energy eigenvalue
	$$
	E_{n}=\frac{\pi^{2} \hbar^{2} n^{2}}{2 m a^{2}}, \quad n=1,2,3, \ldots
	$$
	As two electrons can go into each of the states $n=1,2,3, \ldots$, the highest filled level will have $n=8$ and its energy will be
	$$
	E_{8}=\frac{\pi^{2} \hbar^{2} 8^{2}}{2 m a^{2}}=\frac{32 \pi^{2} \hbar^{2}}{m a^{2}}
	$$
	(iii)The energy of the highest filled state is the Fermi energy $E_{F}$ Hence,
	$$
	E_{F}=\frac{32 \pi^{2} \hbar^{2}}{m a^{2}}
	$$
\end{answer}


\newpage
\begin{abox}
	Practice set 1
	\end{abox}
\begin{enumerate}
	\begin{minipage}{\textwidth}
		\item Consider a particle in a one dimensional potential that satisfies $V(x)=V(-x) .$ Let $\left|\psi_{0}\right\rangle$ and $\left|\psi_{1}\right\rangle$ denote the ground and the first excited states, respectively, and let $|\psi\rangle=\alpha_{0}\left|\psi_{0}\right\rangle+\alpha_{1}\left|\psi_{1}\right\rangle$ be a normalized state with $\alpha_{0}$ and $\alpha_{1}$ being real constants. The expectation value $\langle x\rangle$ of the position operator $x$ in the state $|\psi\rangle$ is given by
		\exyear{NET DEC 2011}
	\end{minipage}
	\begin{tasks}(2)
		\task[\textbf{A.}] $\alpha_{0}^{2}\left\langle\psi_{0}|x| \psi_{0}\right\rangle+\alpha_{1}^{2}\left\langle\psi_{1}|x| \psi_{1}\right\rangle$
		\task[\textbf{B.}]$\alpha_{0} \alpha_{1}\left[\left\langle\psi_{0}|x| \psi_{1}\right\rangle+\left\langle\psi_{1}|x| \psi_{0}\right\rangle\right]$
		\task[\textbf{C.}]$\alpha_{0}^{2}+\alpha_{1}^{2}$
		\task[\textbf{D.}]$2 \alpha_{0} \alpha_{1}$
	\end{tasks}
\begin{minipage}{\textwidth}
	\item Consider a system of two non-interacting identical fermions, each of mass $m$ in an infinite square well potential of width $a$. (Take the potential inside the well to be zero and ignore spin). The composite wavefunction for the system with total energy $E=\frac{5 \pi^{2} \hbar^{2}}{2 m a^{2}}$ is
	\exyear{NET JUNE 2014}
\end{minipage}
\begin{tasks}(2)
	\task[\textbf{A.}] $\frac{2}{a}\left[\sin \left(\frac{\pi x_{1}}{a}\right) \sin \left(\frac{2 \pi x_{2}}{a}\right)-\sin \left(\frac{2 \pi x_{1}}{a}\right) \sin \left(\frac{\pi x_{2}}{a}\right)\right]$
	\task[\textbf{B.}]$\frac{2}{a}\left[\sin \left(\frac{\pi x_{1}}{a}\right) \sin \left(\frac{2 \pi x_{2}}{a}\right)+\sin \left(\frac{2 \pi x_{1}}{a}\right) \sin \left(\frac{\pi x_{2}}{a}\right)\right]$
	\task[\textbf{C.}]$\frac{2}{a}\left[\sin \left(\frac{\pi x_{1}}{a}\right) \sin \left(\frac{3 \pi x_{2}}{2 a}\right)-\sin \left(\frac{3 \pi x_{1}}{2 a}\right) \sin \left(\frac{\pi x_{2}}{a}\right)\right]$
	\task[\textbf{D.}]$\frac{2}{a}\left[\sin \left(\frac{\pi x_{1}}{a}\right) \cos \left(\frac{\pi x_{2}}{a}\right)-\sin \left(\frac{\pi x_{2}}{a}\right) \cos \left(\frac{\pi x_{2}}{a}\right)\right]$
\end{tasks}
\begin{minipage}{\textwidth}
	\item The state vector of a one-dimensional simple harmonic oscillator of angular frequency $\omega$, at time $t=0$, is given by $|\psi(0)\rangle=\frac{1}{\sqrt{2}}[|0\rangle+|2\rangle]$, where $|0\rangle$ and $|2\rangle$ are the normalized ground state and the second excited state, respectively. The minimum time $t$ after which the state vector $|\psi(t)\rangle$ is orthogonal to $|\psi(0)\rangle$, is
	\exyear{NET DEC 2017}
\end{minipage}
\begin{tasks}(2)
	\task[\textbf{A.}] $\frac{\pi}{2 \omega}$
	\task[\textbf{B.}]$\frac{2 \pi}{\omega}$
	\task[\textbf{C.}]$\frac{\pi}{\omega}$
	\task[\textbf{D.}]$\frac{4 \pi}{\omega}$
\end{tasks}
\end{enumerate}
\colorlet{ocre1}{ocre!70!}
\colorlet{ocrel}{ocre!30!}
\setlength\arrayrulewidth{1pt}
\begin{table}[H]
	\centering
	\arrayrulecolor{ocre}
	
	\begin{tabular}{|p{1.5cm}|p{1.5cm}||p{1.5cm}|p{1.5cm}|}
		\hline
		\multicolumn{4}{|c|}{\textbf{Answer key}}\\\hline\hline
		\rowcolor{ocrel}Q.No.&Answer&Q.No.&Answer\\\hline
		1&\textbf{b}&2&\textbf{a}\\\hline
		3&\textbf{a}&&\\\hline
	\end{tabular}
\end{table}
\newpage
\begin{abox}
	Practice set 2
	\end{abox}
\begin{enumerate}
\begin{minipage}{\textwidth}
	\item Consider the wavefunction $\psi=\psi\left(\vec{r}_{1}, \vec{r}_{2}\right) \chi_{s}$ for a fermionic system consisting of two spinhalf particles. The spatial part of the wavefunction is given by
	$$
	\psi\left(\vec{r}_{1}, \vec{r}_{2}\right)=\frac{1}{\sqrt{2}}\left[\phi_{1}\left(\vec{r}_{1}\right) \phi_{2}\left(\vec{r}_{2}\right)+\phi_{2}\left(\vec{r}_{1}\right) \phi_{1}\left(\vec{r}_{2}\right)\right]
	$$
	where $\phi_{1}$ and $\phi_{2}$ are single particle states. The spin part $\chi_{s}$ of the wavefunction with spin states $\alpha(+1 / 2)$ and $\beta(-1 / 2)$ should be
	\exyear{GATE 2013}
\end{minipage}
\begin{tasks}(2)
	\task[\textbf{A.}]$\frac{1}{\sqrt{2}}(\alpha \beta+\beta \alpha)$
	\task[\textbf{B.}]$\frac{1}{\sqrt{2}}(\alpha \beta-\beta \alpha)$
	\task[\textbf{C.}]$\alpha \alpha$
	\task[\textbf{D.}] $\beta \beta$
\end{tasks}
\begin{minipage}{\textwidth}
	\item The ground state and first excited state wave function of a one dimensional infinite potential well are $\psi_{1}$ and $\psi_{2}$ respectively. When two spin-up electrons are placed in this potential which one of the following with $x_{1}$ and $x_{2}$ denoting the position of the two electrons correctly represents the space part of the ground state wave function of the system?
	\exyear{GATE 2014}
\end{minipage}
\begin{tasks}(2)
	\task[\textbf{A.}] $\frac{1}{\sqrt{2}}\left[\psi_{1}\left(x_{1}\right) \psi_{2}\left(x_{1}\right)-\psi_{1}\left(x_{2}\right) \psi_{2}\left(x_{2}\right)\right]$
	\task[\textbf{B.}]$\frac{1}{\sqrt{2}}\left[\psi_{1}\left(x_{1}\right) \psi_{2}\left(x_{2}\right)+\psi_{1}\left(x_{2}\right) \psi_{2}\left(x_{1}\right)\right]$
	\task[\textbf{C.}]$\frac{1}{\sqrt{2}}\left[\psi_{1}\left(x_{1}\right) \psi_{2}\left(x_{1}\right)+\psi_{1}\left(x_{2}\right) \psi_{2}\left(x_{2}\right)\right]$
	\task[\textbf{D.}]$ \frac{1}{\sqrt{2}}\left[\psi_{1}\left(x_{1}\right) \psi_{2}\left(x_{2}\right)-\psi_{1}\left(x_{2}\right) \psi_{2}\left(x_{1}\right)\right]$
\end{tasks}
\begin{minipage}{\textwidth}
	\item $\psi_{1}$ and $\psi_{2}$ are two orthogonal states of a spin $\frac{1}{2}$ system. It is given that
	$\psi_{1}=\frac{1}{\sqrt{3}}\left(\begin{array}{l}1 \\ 0\end{array}\right)+\sqrt{\frac{2}{3}}\left(\begin{array}{l}0 \\ 1\end{array}\right)$, where $\left(\begin{array}{l}1 \\ 0\end{array}\right)$ and $\left(\begin{array}{l}0 \\ 1\end{array}\right)$ represent the spin-up and spin-down states, respectively. When the system is in the state $\psi_{2}$ its probability to be in the spin-up state is
	\exyear{GATE 2014}
\end{minipage}
\begin{minipage}{\textwidth}
	\item For a spin $\frac{1}{2}$ particle, let $|\uparrow\rangle$ and $|\downarrow\rangle$ denote its spin up and spin down states respectively. If $|a\rangle=\frac{1}{\sqrt{2}}(|\uparrow\rangle|\downarrow\rangle+|\downarrow\rangle|\uparrow\rangle)$ and $|b\rangle=\frac{1}{\sqrt{2}}(|\uparrow\rangle|\downarrow\rangle-|\downarrow\rangle|\uparrow\rangle)$ are composite states of two such particles, which of the following statements is true for their total spin $S ?$
	\exyear{GATE 2018}
\end{minipage}
\begin{tasks}(2)
	\task[\textbf{A.}] $S=1$ for $|a\rangle$ and $|b\rangle$ is not an eigenstate of the operator $\hat{S}^{2}$
	\task[\textbf{B.}]$|a\rangle$ is not an eigenstate of the operator $\hat{S}^{2}$ and $S=0$ for $|b\rangle$
	\task[\textbf{C.}]$S=0$ for $|a\rangle$, and $S=1$ for $|b\rangle$
	\task[\textbf{D.}]$S=1$ for $|a\rangle$, and $S=0$ for $|b\rangle$
\end{tasks}
\begin{minipage}{\textwidth}
	\item Let $\left|\psi_{2}\right\rangle=\left(\begin{array}{c}1 \\ 0\end{array}\right),\left|\psi_{2}\right\rangle=\left(\begin{array}{c}0 \\ 1\end{array}\right)$ represent two possible states of a two-level quantum system. The state obtained by the incoherent superposition of $\left|\psi_{1}\right\rangle$ and $\left|\psi_{2}\right\rangle$ is given by a density matrix that is defined as $\rho \equiv c_{1}\left|\psi_{1}\right\rangle\left\langle\psi_{1}\left|+c_{2}\right| \psi_{2}\right\rangle\left\langle\psi_{2}\right| .$ If $c_{1}=0.4$ and $c_{2}=0.6$, the matrix element $\rho_{22}$ (rounded off to one decimal place) is
	\exyear{GATE 2018}
\end{minipage}
\end{enumerate}
\colorlet{ocre1}{ocre!70!}
\colorlet{ocrel}{ocre!30!}
\setlength\arrayrulewidth{1pt}
\begin{table}[H]
	\centering
	\arrayrulecolor{ocre}
	
	\begin{tabular}{|p{1.5cm}|p{1.5cm}||p{1.5cm}|p{1.5cm}|}
		\hline
		\multicolumn{4}{|c|}{\textbf{Answer key}}\\\hline\hline
		\rowcolor{ocrel}Q.No.&Answer&Q.No.&Answer\\\hline
		1&\textbf{b}&2&\textbf{d}\\\hline
		3&\textbf{2/3}&4&\textbf{d}\\\hline
		5&\textbf{0.6}&&\\\hline
	\end{tabular}
\end{table}
