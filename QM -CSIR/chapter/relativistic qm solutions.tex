\chapter{Relativistic qm solutions}
\begin{abox}
	Practice set 1 solutions
	\end{abox}
\begin{enumerate}
	\begin{minipage}{\textwidth}
		\item The Dirac Hamiltonian $H=c \vec{\alpha} \cdot \vec{p}+\beta m c^{2}$ for a free electron corresponds to the classical relation $E^{2}=p^{2} c^{2}+m^{2} c^{4}$. The classical energy-momentum relation of a piratical of charge $q$ in a electromagnetic potential $(\phi, \vec{A})$ is $(E-q \phi)^{2}=c^{2}\left(\vec{p}-\frac{q}{c} \vec{A}\right)^{2}+m^{2} c^{4}$.
		Therefore, the Dirac Hamiltonian for an electron in an electromagnetic field is
		\exyear{NET JUNE 2015}
	\end{minipage}
	\begin{tasks}(2)
		\task[\textbf{A.}] $c \vec{\alpha} \cdot p+\frac{e}{C} \vec{A} \cdot \vec{A}+\beta m c^{2}-e \phi$
		\task[\textbf{B.}]$c \vec{\alpha} \cdot\left(\vec{p}+\frac{e}{c} \vec{A}\right)+\beta m c^{2}+e \phi$
		\task[\textbf{C.}]$c\left(\vec{\alpha} \cdot \vec{p}+e \phi+\frac{e}{c}|\vec{A}|\right)+\beta m c^{2}$
		\task[\textbf{D.}]$c \vec{\alpha} \cdot\left(\vec{p}+\frac{e}{c} \vec{A}\right)+\beta m c^{2}-e \phi$
	\end{tasks}
\begin{answer}
	Electromagnetic interaction of Dirac particle
	$$
	H=\left[\left(\vec{P}-\frac{q \vec{A}}{c}\right)^{2} c^{2}+m^{2} c^{4}\right]^{\frac{1}{2}}+q \phi
	$$
	Quantum mechanical Hamiltonian
	$$
	i \hbar \frac{\partial \psi}{\partial t}=\left[c \vec{\alpha}\left(P-\frac{q \vec{A}}{c}\right)+\beta m c^{2}+q \phi\right] \psi
	$$
	put $q=-e$
	$$
	H=\left[c \vec{\alpha} \cdot\left(\vec{P}+\frac{e}{c} \vec{A}\right)+\beta m c^{2}-e \phi\right]
	$$
	The correct option is \textbf{(d)}
\end{answer}
\begin{minipage}{\textwidth}
	\item The dynamics of a free relativistic particle of mass $m$ is governed by the Dirac Hamiltonian $H=c \vec{\alpha} \cdot \vec{p}+\beta m c^{2}$, where $\vec{p}$ is the momentum operator and $\vec{\alpha}=\left(\alpha_{x}, \alpha_{y}, \alpha_{z}\right)$ and $\beta$ are four $4 \times 4$ Dirac matrices. The acceleration operator can be expressed as
	\exyear{NET DEC 2016}
\end{minipage}
\begin{tasks}(2)
	\task[\textbf{A.}] $\frac{2 i c}{\hbar}(c \vec{p}-\vec{\alpha} H)$
	\task[\textbf{B.}]$2 \mathrm{ic}^{2} \vec{\alpha} \beta$
	\task[\textbf{C.}]$\frac{i c}{\hbar} H \vec{\alpha}$
	\task[\textbf{D.}]$-\frac{2 i c}{\hbar}(c \vec{p}+\vec{\alpha} H)$
\end{tasks}
\begin{answer}
	$H=c \vec{\alpha} \cdot \vec{p}+\beta m c^{2}$
	If $v_{x}$ velocity of $x$ direction
	From the Ehrenfest theorem\\
	$$
	\begin{aligned}
	&v_{x}=\frac{d x}{d t}=\frac{1}{i \hbar}[x, H]+\frac{\partial x}{\partial t}=\frac{1}{i \hbar}\left[x, c \alpha_{x} p_{x}+c \alpha_{y} p_{y}+c \alpha_{z} p_{z}+\beta m c^{2}\right]+0 \\
	&=\frac{c}{i \hbar}\left[x, \alpha_{x} p_{x}\right]=c \alpha_{x}
	\end{aligned}
	$$
	Similarly, acceleration is given by
	$$
	a_{x}=\frac{d v_{x}}{d t}=\frac{1}{i \hbar}\left[c \alpha_{x}, H\right]=\frac{c}{i \hbar}\left[\alpha_{x}, c \alpha_{x} p_{x}+c \alpha_{y} p_{y}+c \alpha_{z} p_{z}+\beta m c^{2}\right]
	$$
	Using relation $\alpha_{i} \alpha_{j}+\alpha_{j} \alpha_{i}=0, \alpha_{i} \beta+\beta \alpha_{i}=0$ and $\left[\alpha_{i}, p_{j}\right]=0$\\
	\begin{align*}
		&{\left[\alpha_{x}, c \alpha_{x} p_{x}\right]=0} \\
		&{\left[\alpha_{x}, c \alpha_{y} p_{y}\right]=c\left[\alpha_{x} \alpha_{y}-\alpha_{y} \alpha_{x}\right] p_{y}+\alpha_{y} c\left[\alpha_{x}, p_{y}\right]=\left[c \alpha_{x} \alpha_{y}-\left(-c \alpha_{x} \alpha_{y}\right)\right] p_{y}+0=2 c \alpha_{x} \alpha_{y} p_{y}} \\
		&{\left[\alpha_{x}, c \alpha_{z} p_{z}\right]=\left[c \alpha_{x} \alpha_{z}-c \alpha_{z} \alpha_{x}\right] p_{z}+\alpha_{z}\left[c \alpha_{x}, p_{z}\right]=\left[c \alpha_{x} \alpha_{z}-\left(-c \alpha_{x} \alpha_{z}\right)\right] p_{z}+0=2 c \alpha_{x} \alpha_{z} p_{z}} \\
		&{\left[\alpha_{x}, \beta m c^{2}\right]=\left[\alpha_{x} \beta-\beta \alpha_{x}\right] m c^{2}=2 m c^{2} \alpha_{x} \beta}\\
		&a_{x}=\frac{c}{i \hbar}\left[2 c \alpha_{x} \alpha_{y} p_{y}+2 c \alpha_{x} \alpha_{z} p_{z}+2 \alpha_{x} \beta m c^{2}\right] \\
		&a_{x}=\frac{2 \alpha_{x} c}{i \hbar}\left[c \alpha_{y} p_{y}+c \alpha_{z} p_{z}+\beta m c^{2}+c \alpha_{x} p_{x}-c \alpha_{x} p_{x}\right] \\
		&a_{x}=\frac{2 \alpha_{x} c}{i \hbar}\left[c \alpha_{x} p_{x}+c \alpha_{y} p_{y}+c \alpha_{z} p_{z}+\beta m c^{2}-c \alpha_{x} p_{x}\right]\\
		&a_{x}=\frac{2 c}{i \hbar}\left[\alpha_{x} \cdot H-c \alpha_{x} \alpha_{x} p_{x}\right]=\frac{2 i c}{\hbar}\left[c \alpha_{x} \alpha_{x} p_{x}-\alpha_{x} \cdot H\right],\left(\alpha_{x}^{2}=x\right) \\
		&\vec{a}=a_{x} \hat{i}+a_{y} \hat{j}+a_{z} \hat{k}=\left[\frac{2 i c}{\hbar}(c \vec{p}-\vec{\alpha} \cdot \vec{H})\right]
	\end{align*}
	The correct option is \textbf{(a)}
\end{answer}
\end{enumerate}
\newpage
\begin{abox}
	Practice set 2
	\end{abox}
\begin{enumerate}
	\begin{minipage}{\textwidth}
	\item (a) Write the expression of momentum operator in relativistic quantum mechanics.\\
	(b) The relativistic energy momentum relation for particle with rest mass $m_{0}$ is given by $\frac{E^{2}}{c^{2}}-p^{2}=m_{0}^{2} c^{2}$, derive down Klein-Gordon equation and find the wave function.\\
	(c) The value of energy is $E=\pm c \sqrt{m_{0}^{2} c^{2}+p^{2}}$, what is significant of positive energy and negative energy
\end{minipage}
\begin{answer}
(a) The four momentum vector is given by $\vec{p}=\left(\frac{E}{c}, p_{x}, p_{y}, p_{z}\right)$ where $\frac{E}{c}$ is the time coordinate of momentum (where $E$ is total energy) and $p_{x}, p_{y}, p_{z}$ are space component of momentum vector.\\
In quantum mechanics the total Energy is given by,
$$
E=i \hbar \frac{\partial}{\partial t} \text { and } p_{x}=-i \hbar \frac{\partial}{\partial x}, p_{y}=-i \hbar \frac{\partial}{\partial y} p_{z}=-i \hbar \frac{\partial}{\partial z}
$$
So momentum operator in relativistic form is given by
$$
p=\frac{i \hbar}{c} \frac{\partial}{\partial t}+\left(-i \hbar \frac{\partial}{\partial x}\right) \hat{i}+\left(-i \hbar \frac{\partial}{\partial y}\right) \hat{j}+\left(-i \hbar \frac{\partial}{\partial z}\right) \hat{k}
$$
So, momentum operator $\vec{p}=i \hbar\left(\frac{\partial}{c \partial t}-\vec{\nabla}\right)$ where $\vec{\nabla}=\hat{i} \frac{\partial}{\partial x}+\hat{j} \frac{\partial}{\partial y}+\hat{k} \frac{\partial}{\partial z}$\\
$\text { (b) } \frac{E^{2}}{c^{2}}-p^{2}=m_{0}^{2} c^{2} \Rightarrow\left(\frac{i \hbar \frac{\partial}{\partial t} \cdot i \hbar \frac{\partial}{\partial t}}{c^{2}}\right)-(-i \hbar \vec{\nabla}) \cdot(-i \hbar \vec{\nabla})=m_{0}^{2} c^{2}$\\
\begin{align*}
&\Rightarrow\left(-\frac{\hbar^{2}}{c^{2}} \frac{\partial^{2}}{\partial t^{2}}\right)-\left(-\hbar^{2} \nabla^{2}\right)=m_{0}^{2} c^{2} \Rightarrow-\frac{1}{c^{2}} \frac{\partial^{2}}{\partial t^{2}}+\nabla^{2}=\frac{m_{0}^{2} c^{2}}{\hbar^{2}} \\
&\Rightarrow\left(\frac{1}{c^{2}} \frac{\partial^{2}}{\partial t^{2}}-\nabla^{2}+\frac{m_{0}^{2} c^{2}}{\hbar^{2}}\right) \psi=0 \ldots \ldots \ldots \ldots \ldots . .(\mathrm{A})
\end{align*}
Equation (A) is identifies as wave equation and its general solution is $\psi=\exp \left[\frac{i}{\hbar}(\vec{p} \cdot \vec{r}-E t)\right]$, where $E=+c \sqrt{p^{2}+m_{0}^{2} c^{2}}$ and $E=-c \sqrt{p^{2}+m_{0}^{2} c^{2}}$\\
(c) There exist solution both for positive $E=+c \sqrt{p^{2}+m_{0}^{2} c^{2}}$ as well as $E=-c \sqrt{p^{2}+m_{0}^{2} c^{2}}$ energy respectively. The positive energy $E=+c \sqrt{p^{2}+m_{0}^{2} c^{2}}$ corresponds to particle and negative energy $E=-c \sqrt{p^{2}+m_{0}^{2} c^{2}}$ corresponds for antiparticle.
\end{answer}
	\begin{minipage}{\textwidth}
	\item (a) Write Dirac's equation for a free particle.\\
	(b) Find the form of probability current density in Dirac's formalization.
\end{minipage}
\begin{answer}
	$\text { (a) Dirac's Hamiltonian is given by } H=c \cdot \vec{\alpha} \cdot \vec{p}+\beta m c^{2} \text { Where } H=i \hbar \frac{d}{d t} \text { and } \vec{p}=-i \hbar \vec{\nabla}$\\
	(b) $i \hbar \frac{\partial \psi}{\partial t}=-i \hbar c \vec{\alpha} \cdot \vec{\nabla} \psi+\beta m c^{2} \psi$ $\cdots$ equation A\\
	$-i \hbar \frac{\partial \psi^{*}}{\partial t}=i \hbar c \vec{\alpha} \cdot \vec{\nabla} \psi^{*}+\beta m c^{2} \psi^{*}$ $\cdots$ equation B\\
	$\text { Multiplying equation (A) with } \psi^{*} \text { and Multiplying equation (B) with } \psi, \text { we have }$\\
		$i \hbar \psi^{*} \frac{\partial \psi}{\partial t}=-i \hbar c \vec{\alpha} \cdot \psi^{*} \vec{\nabla} \psi+\beta m c^{2} \psi^{*} \psi$ $\cdots$ equation C\\
		$-i \hbar \psi \frac{\partial \psi^{*}}{\partial t}=i \hbar c \psi \vec{\alpha} \cdot \vec{\nabla} \psi^{*}+\beta m c^{2} \psi \psi^{*}$ $\cdots$ equation D\\
		Subtracting (D) from (C)
		\begin{align*}
		&i \hbar\left(\psi^{*} \frac{\partial \psi}{\partial t}+\psi \frac{\partial \psi^{*}}{\partial t}\right)=-i c \hbar\left(\psi^{*} \vec{\alpha} \cdot \vec{\nabla} \psi+\vec{\nabla} \psi^{*} \cdot \vec{\alpha} \psi\right) \\
		&\frac{\partial}{\partial t}\left(\psi^{*} \psi\right)+\vec{\nabla} \cdot\left(c \psi^{*} \alpha \psi\right)=0 \Rightarrow \frac{\partial \rho}{\partial t}+\vec{\nabla} \cdot \vec{J}=0 \\
		&\vec{J}(r, t)=c \psi^{*} \alpha \psi, \quad \rho(r, t)=\psi^{*} \psi
		\end{align*}
\end{answer}
\end{enumerate}