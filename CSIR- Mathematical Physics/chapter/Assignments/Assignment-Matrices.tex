\chapter{Assignment-Matrices}
\section{MCQ}
\begin{enumerate}
	\item The eigen value of matrix $A=\left(\begin{array}{lll}1 & 0 & 1 \\ 0 & 1 & 0 \\ 1 & 0 & 1\end{array}\right)$ is
	 \begin{tasks}(2)
		\task[\textbf{a.}] $\lambda=1,0,2$
		\task[\textbf{b.}]$\lambda=-1,2,2$
		\task[\textbf{c.}] $\lambda=0,0,3$
		\task[\textbf{d.}] $\lambda=1,1,1$
	\end{tasks}
	\item The eigen value of matrix $A=\left(\begin{array}{ccc}1 & \sqrt{8} & 0 \\ \sqrt{8} & 1 & \sqrt{8} \\ 0 & \sqrt{8} & 1\end{array}\right)$ is
	 \begin{tasks}(2)
		\task[\textbf{a.}]$\lambda=1,0,2$
		\task[\textbf{b.}]$\lambda=-1,2,2$
		\task[\textbf{c.}]$\lambda=-3,1,5$
		\task[\textbf{d.}]$\lambda=1,1,1$
	\end{tasks}
	\item The eigen value of matrix $A=\left(\begin{array}{lll}0 & 1 & 0 \\ 1 & 0 & 1 \\ 0 & 1 & 0\end{array}\right)$ is
	 \begin{tasks}(2)
		\task[\textbf{a.}]$\lambda=-1,0,1$
		\task[\textbf{b.}]$\lambda=0,-2,2$
		\task[\textbf{c.}]$\lambda=0,0,0$
		\task[\textbf{d.}]  $\lambda=-\sqrt{2}, 0, \sqrt{2}$
	\end{tasks}
	\item The eigen value of matrix $A=\left(\begin{array}{lll}0 & 1 & 1 \\ 1 & 0 & 1 \\ 1 & 1 & 0\end{array}\right)$ is
	 \begin{tasks}(2)
		\task[\textbf{a.}] $\lambda=-1,-1,2$
		\task[\textbf{b.}]$\lambda=0,-2,2$
		\task[\textbf{c.}] $\lambda=0,0,0$
		\task[\textbf{d.}] $\lambda=-\sqrt{2}, 0, \sqrt{2}$
	\end{tasks}
	\item The degenerate eigen value of matrix $A=\left(\begin{array}{lll}1 & 1 & 1 \\ 1 & 1 & 1 \\ 1 & 1 & 1\end{array}\right)$ is
	 \begin{tasks}(2)
		\task[\textbf{a.}]0
		\task[\textbf{b.}]1
		\task[\textbf{c.}]2
		\task[\textbf{d.}] 3
	\end{tasks}
	\item The eigen vector of matrix $A=\left(\begin{array}{lll}1 & 0 & 1 \\ 0 & 1 & 0 \\ 1 & 0 & 1\end{array}\right)$ corresponding to eigen value 0 is
	 \begin{tasks}(2)
		\task[\textbf{a.}] $\left[\begin{array}{l}1 \\ 0 \\ 0\end{array}\right]$
		\task[\textbf{b.}]$\left[\begin{array}{c}1 \\ 0 \\ -1\end{array}\right]$
		\task[\textbf{c.}] $\left[\begin{array}{l}1 \\ 0 \\ 1\end{array}\right]$
		\task[\textbf{d.}]  $\left[\begin{array}{c}1 \\ -1 \\ 0\end{array}\right]$
	\end{tasks}
	\item The eigen vector of matrix $A=\left(\begin{array}{ccc}1 & \sqrt{8} & 0 \\ \sqrt{8} & 1 & \sqrt{8} \\ 0 & \sqrt{8} & 1\end{array}\right)$ corresponding to eigen value 5 is
	 \begin{tasks}(2)
		\task[\textbf{a.}]$\left[\begin{array}{c}1 \\ \sqrt{2} \\ 1\end{array}\right]$
		\task[\textbf{b.}]$\left[\begin{array}{c}1 \\ -\sqrt{2} \\ 1\end{array}\right]$
		\task[\textbf{c.}] $\left[\begin{array}{c}\sqrt{2} \\ 1 \\ 1\end{array}\right]$
		\task[\textbf{d.}]  $\left[\begin{array}{c}-\sqrt{2} \\ 1 \\ 1\end{array}\right]$
	\end{tasks}
	\item The eigen vectors of matrix $A=\left(\begin{array}{ccc}0 & 1 & 0 \\ 1 & 0 & 1 \\ 0 & 1 & 0\end{array}\right)$ corresponding to eigen values $-\sqrt{2}, 0, \sqrt{2}$ are respectively
	 \begin{tasks}(2)
		\task[\textbf{a.}] $\left[\begin{array}{c}1 \\ -\sqrt{2} \\ 1\end{array}\right],\left[\begin{array}{c}1 \\ 0 \\ -1\end{array}\right],\left[\begin{array}{c}1 \\ \sqrt{2} \\ 1\end{array}\right]$
		\task[\textbf{b.}]$\left[\begin{array}{c}1 \\ \sqrt{2} \\ 1\end{array}\right],\left[\begin{array}{c}1 \\ 0 \\ -1\end{array}\right],\left[\begin{array}{c}1 \\ -\sqrt{2} \\ 1\end{array}\right]$
		\task[\textbf{c.}]$\left[\begin{array}{c}1 \\ -\sqrt{2} \\ 1\end{array}\right],\left[\begin{array}{c}1 \\ \sqrt{2} \\ 1\end{array}\right],\left[\begin{array}{c}1 \\ 0 \\ -1\end{array}\right]$
		\task[\textbf{d.}]  $\left[\begin{array}{c}1 \\ 0 \\ -1\end{array}\right],\left[\begin{array}{c}1 \\ -\sqrt{2} \\ 1\end{array}\right],\left[\begin{array}{c}1 \\ \sqrt{2} \\ 1\end{array}\right]$
	\end{tasks}
	\item If one of the eigen vector of matrix $A=\left(\begin{array}{lll}0 & 1 & 1 \\ 1 & 0 & 1 \\ 1 & 1 & 0\end{array}\right)$ corresponding to eigen value $-1$ is $\left[\begin{array}{c}-2 \\ 1 \\ 1\end{array}\right]$ then other orthogonal eigen vector for same eigen value is
	 \begin{tasks}(2)
		\task[\textbf{a.}]$\left[\begin{array}{c}0 \\ 1 \\ -1\end{array}\right] \quad$
		\task[\textbf{b.}]$\left[\begin{array}{c}-4 \\ 2 \\ 2\end{array}\right]$
		\task[\textbf{c.}]$\left[\begin{array}{l}1 \\ 0 \\ 1\end{array}\right]$
		\task[\textbf{d.}] $\left[\begin{array}{c}1 \\ -1 \\ 0\end{array}\right]$
	\end{tasks}
	\item If one of the eigen vector of matrix $A=\left(\begin{array}{lll}1 & 1 & 1 \\ 1 & 1 & 1 \\ 1 & 1 & 1\end{array}\right)$ corresponding to eigen value 0 is $\left[\begin{array}{c}0 \\ 1 \\ -1\end{array}\right]$ then other orthogonal eigen vector for same eigen value is
	 \begin{tasks}(2)
		\task[\textbf{a.}] $\left[\begin{array}{c}-2 \\ 1 \\ 1\end{array}\right]$
		\task[\textbf{b.}]$\left[\begin{array}{c}0 \\ 2 \\ -2\end{array}\right]$
		\task[\textbf{c.}]$\left[\begin{array}{l}1 \\ 0 \\ 1\end{array}\right]$
		\task[\textbf{d.}]  $\left[\begin{array}{c}1 \\ -1 \\ 0\end{array}\right]$
	\end{tasks}
	\item A square matrix $3 \times 3$ is given by $A=\left[\begin{array}{lll}2 & 0 & 0 \\ 0 & 1 & 1 \\ 0 & 1 & 1\end{array}\right]$ is diagonalized in eigenvector of
	matrix $S=\left[\begin{array}{ccc}1 & 0 & 0 \\ 0 & \frac{1}{\sqrt{2}} & \frac{1}{\sqrt{2}} \\ 0 & -\frac{1}{\sqrt{2}} & \frac{1}{\sqrt{2}}\end{array}\right] .$ Which one of the following is matrix $A$ in the diagonal form in the
	basis of $S$ ?
	 \begin{tasks}(2)
		\task[\textbf{a.}] $\left[\begin{array}{lll}2 & 0 & 0 \\ 0 & 2 & 0 \\ 0 & 0 & 0\end{array}\right] \quad$ 
		\task[\textbf{b.}] $\left[\begin{array}{lll}0 & 0 & 0 \\ 0 & 2 & 0 \\ 0 & 0 & 2\end{array}\right] \quad$ 
		\task[\textbf{c.}]$\left[\begin{array}{lll}2 & 0 & 0 \\ 0 & 0 & 0 \\ 0 & 0 & 2\end{array}\right] \quad$
		\task[\textbf{d.}] $\left[\begin{array}{lll}1 & 0 & 2 \\ 0 & 1 & 0 \\ 0 & 0 & 2\end{array}\right]$
	\end{tasks}
	\item Consider the matrix $M=\left(\begin{array}{lll}1 & 1 & 1 \\ 1 & 1 & 1 \\ 1 & 1 & 1\end{array}\right) .$ The eigenvalues of $M$ are
	 \begin{tasks}(2)
		\task[\textbf{a.}] $0,1,2$
		\task[\textbf{b.}]$0,0,3$
		\task[\textbf{c.}]$1,1,1$
		\task[\textbf{d.}] $-1,1,3$
	\end{tasks}
	\item Consider the matrix $M=\left(\begin{array}{lll}1 & 1 & 1 \\ 1 & 1 & 1 \\ 1 & 1 & 1\end{array}\right)$. The exponential of $M$ simplifies to $(I$ is the $3 \times 3$ identity matrix)
	 \begin{tasks}(2)
		\task[\textbf{a.}]$e^{M}=I+\left(\frac{e^{3}-1}{3}\right) M$
		\task[\textbf{b.}] $e^{M}=I+M+\frac{M^{2}}{2 !}$
		\task[\textbf{c.}] $e^{M}=I+3^{3} M$
		\task[\textbf{d.}]  $e^{M}=(e-1) M$
	\end{tasks}
	\item The eigenvalues of the matrix $\left(\begin{array}{lll}2 & 3 & 0 \\ 3 & 2 & 0 \\ 0 & 0 & 1\end{array}\right)$ are
	 \begin{tasks}(2)
		\task[\textbf{a.}]$5,2,-2$
		\task[\textbf{b.}]$-5,-1,-1$
		\task[\textbf{c.}]$5,1,-1$
		\task[\textbf{d.}] $-5,1,1$
	\end{tasks}
	\item The eigen values of the matrix $A=\left(\begin{array}{ccc}1 & 2 & 3 \\ 2 & 4 & 6 \\ 3 & 6 & 9\end{array}\right)$ are
	 \begin{tasks}(2)
		\task[\textbf{a.}]$(1,4,9)$
		\task[\textbf{b.}] $(0,7,7)$
		\task[\textbf{c.}]$(0,1,13)$
		\task[\textbf{d.}] $(0,0,14)$
	\end{tasks}
	\item The eigenvalues of the antisymmetric matrix, $A=\left(\begin{array}{ccc}0 & -n_{3} & n_{2} \\ n_{3} & 0 & -n_{1} \\ -n_{2} & n_{1} & 0\end{array}\right)$ where $n_{1}, n_{2}$ and $n_{3}$ are the components of a unit vector, are
	 \begin{tasks}(2)
		\task[\textbf{a.}]$0, i,-i$
		\task[\textbf{b.}]$0,1,-1$
		\task[\textbf{c.}]$0,1+i,-1,-i$
		\task[\textbf{d.}] $0,0,0$
	\end{tasks}
	\item Consider the matrix $M=\left(\begin{array}{ccc}0 & 2 i & 3 i \\ -2 i & 0 & 6 i \\ -3 i & -6 i & 0\end{array}\right)$. The eigenvalues of $M$ are
	 \begin{tasks}(2)
		\task[\textbf{a.}]$-5,-2,7$
		\task[\textbf{b.}]$-7,0,7$
		\task[\textbf{c.}]$-4 i, 2 i, 2 i$
		\task[\textbf{d.}] $2,3,6$
	\end{tasks}
	\item The column vector $\left(\begin{array}{l}a \\ b \\ a\end{array}\right)$ is a simultaneous eigenvector of
	$A=\left(\begin{array}{lll}0 & 0 & 1 \\ 0 & 1 & 0 \\ 1 & 0 & 0\end{array}\right)$ and $B=\left(\begin{array}{lll}0 & 1 & 1 \\ 1 & 0 & 1 \\ 1 & 1 & 0\end{array}\right)$, if
	 \begin{tasks}(2)
		\task[\textbf{a.}] $b=0$ or $a=0$
		\task[\textbf{b.}] $b=a$ or $b=-2 a$
		\task[\textbf{c.}] $b=2 a$ or $b=-a$
		\task[\textbf{d.}]  $b=a / 2$ or $b=-a / 2$
	\end{tasks}
\item Let $\alpha$ and $\beta$ be complex numbers. Which of the following sets of matrices forms a group under matrix multiplication?
	 \begin{tasks}(2)
		\task[\textbf{a.}] $\left(\begin{array}{ll}\alpha & \beta \\ 0 & 0\end{array}\right)$
		\task[\textbf{b.}] $\left(\begin{array}{lr}1 & \alpha \\ \beta & 1\end{array}\right)$, where $\alpha \beta \neq 1$
		\task[\textbf{c.}] $\left(\begin{array}{cc}\alpha & \alpha^{*} \\ \beta & \beta^{*}\end{array}\right)$, where $\alpha \beta^{*}$ is real
		\task[\textbf{d.}]  $\left(\begin{array}{cc}\alpha & \beta \\ -\beta^{*} & \alpha^{*}\end{array}\right)$, where $|\alpha|^{2}+|\beta|^{2}=1$
	\end{tasks}
	\item A $3 \times 3$ matrix $M$ has $\operatorname{Tr}[M]=6, \operatorname{Tr}\left[M^{2}\right]=26$ and $\operatorname{Tr}\left[M^{3}\right]=90$. Which of the following can be a possible set of eigenvalues of $M$ ?
	 \begin{tasks}(2)
		\task[\textbf{a.}]$\{1,1,4\}$
		\task[\textbf{b.}]$\{-1,0,7\}$
		\task[\textbf{c.}]$\{-1,3,4\}$
		\task[\textbf{d.}] $\{2,2,2\}$
	\end{tasks}
	\item The matrix $A=\frac{1}{\sqrt{3}}\left[\begin{array}{cc}1 & 1+i \\ 1-i & -1\end{array}\right]$ is
	 \begin{tasks}(2)
		\task[\textbf{a.}]Orthogonal
		\task[\textbf{b.}] Symmetric
		\task[\textbf{c.}]anti-symmetric
		\task[\textbf{d.}] Unitary
	\end{tasks}
	\item If $H$ is Hermitian matrix then matrix $A=\exp (i H)$ is
	 \begin{tasks}(2)
		\task[\textbf{a.}] Hermitian
		\task[\textbf{b.}]Unitary
		\task[\textbf{c.}]Skew Hermitian
		\task[\textbf{d.}] Identity
	\end{tasks}
	\item The matrix $A=\left(\begin{array}{ccc}0 & 1 & 0 \\ 1 & 0 & 0 \\ 0 & 0 & 2\end{array}\right)$ is diagonalize in the basis of unitary matrices $U$ and get the diagonalise matrix $\left(\begin{array}{ccc}2 & 0 & 0 \\ 0 & 1 & 0 \\ 0 & 0 & -1\end{array}\right)$ then matrix $U$ is
	 \begin{tasks}(2)
		\task[\textbf{a.}]$\left(\begin{array}{ccc}0 & 0 & 1 \\ \frac{1}{\sqrt{2}} & \frac{1}{\sqrt{2}} & 0 \\ \frac{1}{\sqrt{2}} & -\frac{1}{\sqrt{2}} & 0\end{array}\right)$
		\task[\textbf{b.}]$\left(\begin{array}{ccc}0 & 0 & 1 \\ \frac{1}{\sqrt{2}} & \frac{1}{\sqrt{2}} & 0 \\ -\frac{1}{\sqrt{2}} & \frac{1}{\sqrt{2}} & 0\end{array}\right)$
		\task[\textbf{c.}] $\left(\begin{array}{ccc}0 & 0 & 0 \\ 0 & \frac{1}{\sqrt{2}} & \frac{1}{\sqrt{2}} \\ 1 & \frac{1}{\sqrt{2}} & -\frac{1}{\sqrt{2}}\end{array}\right)$
		\task[\textbf{d.}]  $\left(\begin{array}{ccc}0 & 1 & 0 \\ \frac{1}{\sqrt{2}} & 0 & \frac{1}{\sqrt{2}} \\ \frac{1}{\sqrt{2}} & 0 & -\frac{1}{\sqrt{2}}\end{array}\right)$
	\end{tasks}
	\item Given a $2 \times 2$ unitary matrix $U$ satisfying $U^{\dagger} U=U U^{\dagger}=1$ with $\operatorname{det} U=e^{i \varphi}$, one can construct a unitary matrix $V\left(V^{\dagger} V=V V^{\dagger}=1\right)$ with det $V=1$ from it by
	 \begin{tasks}(1)
		\task[\textbf{a.}] Multiplying $U$ by $e^{-i \varphi / 2}$
		\task[\textbf{b.}] Multiplying any single element of $U$ by $e^{-i \varphi}$
		\task[\textbf{c.}] Multiplying any row or column of $U$ by $e^{-i \varphi / 2}$
		\task[\textbf{d.}]  Multiplying $U$ by $e^{-i \varphi}$
	\end{tasks}
	\item Consider an $n \times n(n>1)$ matrix $A$, in which $A_{i j}$ is the product of the indices $i$ and $j$ (namely $A_{i j}=i j$ ). The matrix $A$
	 \begin{tasks}(1)
		\task[\textbf{a.}]Has one degenerate eigevalue with degeneracy $(n-1)$
		\task[\textbf{b.}]Has two degenerate eigenvalues with degeneracies 2 and $(n-2)$
		\task[\textbf{c.}]Has one degenerate eigenvalue with degeneracy $n$
		\task[\textbf{d.}] Does not have any degenerate eigenvalue
	\end{tasks}
	\item Two matrices $A$ and $B$ are said to be similar if $B=P^{1} A P$ for some invertible matrix $P$. Which of the following statements is NOT TRUE?
	 \begin{tasks}(2)
		\task[\textbf{a.}] $\operatorname{Det} A=\operatorname{Det} B$
		\task[\textbf{b.}] Trace of $A=$ Trace of $B$
		\task[\textbf{c.}]$A$ and $B$ have the same eigenvectors
		\task[\textbf{d.}]  $A$ and $B$ have the same eigenvalues
	\end{tasks}
	\item A $3 \times 3$ matrix has elements such that its trace is 11 and its determinant is 36 . The eigenvalues of the matrix are all known to be positive integers. The largest eigenvalues of the matrix is
	 \begin{tasks}(2)
		\task[\textbf{a.}]18
		\task[\textbf{b.}]12
		\task[\textbf{c.}]9
		\task[\textbf{d.}] 6
	\end{tasks}
	\item The inverse of matrix $M=\left(\begin{array}{lll}0 & 1 & 1 \\ 0 & 0 & 1 \\ 1 & 0 & 0\end{array}\right)$ is
	 \begin{tasks}(2)
		\task[\textbf{a.}]$M-1$
		\task[\textbf{b.}]$M^{2}-1$
		\task[\textbf{c.}] $I-M^{2}$
		\task[\textbf{d.}] $I-M$
	\end{tasks}
       \section{MSQ}  
 \item For matrix $A=\left(\begin{array}{lll}5 & 0 & 2 \\ 0 & 1 & 0 \\ 2 & 0 & 2\end{array}\right)$, which of the following statements are true?
                 \begin{tasks}(1)
                	\task[\textbf{a.}]The degenerate eigen value is 1
                	\task[\textbf{b.}]One of the eigen vector corresponding to degenerate eigen value is $\left[\begin{array}{c}1 \\ 0 \\ -2\end{array}\right]$
                	\task[\textbf{c.}]The eigen vector corresponding to nondegenerate eigen value is $\left[\begin{array}{l}2 \\ 0 \\ 1\end{array}\right]$
                	\task[\textbf{d.}] The eigen vector corresponding to nondegenerate eigen value is $\left[\begin{array}{l}0 \\ 2 \\ 1\end{array}\right]$   
                \end{tasks}
 \item   For matrix $A=\left(\begin{array}{lll}1 & 1 & 0 \\ 1 & 1 & 0 \\ 0 & 0 & 0\end{array}\right)$, which of the following statements are true?    
                 \begin{tasks}(1)
                	\task[\textbf{a.}]The degenerate eigen value is 0
                	\task[\textbf{b.}]One of the eigen vector corresponding to degenerate eigen value is $\left[\begin{array}{c}1 \\ -1 \\ 0\end{array}\right]$
                	\task[\textbf{c.}]One of the eigen vector corresponding to degenerate eigen value is $\left[\begin{array}{l}0 \\ 0 \\ 1\end{array}\right]$
                	\task[\textbf{d.}] The eigen vector corresponding to nondegenerate eigen value is $\left[\begin{array}{l}1 \\ 1 \\ 0\end{array}\right]$
                \end{tasks}
  \item For matrix $A=\left(\begin{array}{ccc}5 & 0 & \sqrt{3} \\ 0 & 3 & 0 \\ \sqrt{3} & 0 & 3\end{array}\right)$, which of the following statements are true?
  \begin{tasks}(1)
 	\task[\textbf{a.}]The eigen value are $2,3,6$
 	\task[\textbf{b.}]The eigen value are 2, 4, 5
 	\task[\textbf{c.}]Eigen vector corresponding eigen value2 is $\left[\begin{array}{c}1 \\ 0 \\ \sqrt{3}\end{array}\right]$
 	\task[\textbf{d.}] Eigen vector corresponding eigen value 2 is $\left[\begin{array}{c}\sqrt{3} \\ 0 \\ 1\end{array}\right]$
 \end{tasks}               
 \item  Which one of following is correct 
   \begin{tasks}(1)
  	\task[\textbf{a.}] If $A^{\dagger}=A$ and $B^{\dagger}=-B$ Then $A B+B A$ is skew Hermitian
  	\task[\textbf{b.}] If $A^{\dagger}=A$ and $B^{\dagger}=-B$ Then $A B+B A$ is Hermitian
  	\task[\textbf{c.}]If $A^{\dagger}=A$ and $B^{\dagger}=-B \quad i(A B+B A)$ is skew Hermitian
  	\task[\textbf{d.}] $A^{\dagger}=A$ and $B^{\dagger}=-B i(A B+B A)$ is Hermitian    
  \end{tasks}              
\item  Which of the following is correct for matrix $A=\left(\begin{array}{lll}1 & 0 & 0 \\ 0 & 0 & 1 \\ 0 & 1 & 0\end{array}\right)$             
   \begin{tasks}(2)
  	\task[\textbf{a.}] It is its own inverse
  	\task[\textbf{b.}]It is its own transpose
  	\task[\textbf{c.}] It has eigen value $\pm 1$
  	\task[\textbf{d.}]  It is orthogonal matrix.
  \end{tasks}              
\section{NAT}              
  \item The degenerate eigenvalue of the matrix $\left[\begin{array}{ccc}4 & -1 & -1 \\ -1 & 4 & -1 \\ -1 & -1 & 4\end{array}\right]$ is (your answer should be an integer)-------------      
\item  The minimum eigenvalues of the matrix $\left(\begin{array}{lll}0 & 1 & 0 \\ 1 & 0 & 1 \\ 0 & 1 & 0\end{array}\right)$ is...........               
  \item The degenerate eigen value of matrix $A=\left[\begin{array}{lll}0 & 0 & 1 \\ 0 & 1 & 0 \\ 1 & 0 & 0\end{array}\right]$ is given by
                
                
                
\end{enumerate}
