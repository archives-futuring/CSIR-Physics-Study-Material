\chapter{Assignment-Matrices-Solutions}
\begin{enumerate}
	\item  $\left. \right. $
	\begin{answer}
		\begin{align*}
		\text{For eigen value }|A-\lambda I|&=0\\
		\left|\begin{array}{ccc}
		1-\lambda & 0 & 1-\lambda \\
		0 & 1-\lambda & 0 \\
		1 & 0 & 1-\lambda
		\end{array}\right|&=(1-\lambda)\left[(1-\lambda)^{2}-0\right]+0+1[-(1-\lambda)]=0 \\
		\Rightarrow-\lambda^{3}+3 \lambda^{2}-2 \lambda&=0 \Rightarrow \lambda=0,1,2
		\end{align*}
		So the correct answer is \textbf{Option (a)}
	\end{answer}
	\item  $\left. \right. $
	\begin{answer}
		\begin{align*}
	\text{ For eigen value }|A-\lambda I|&=0\\
		\left|\begin{array}{ccc}
		1-\lambda & \sqrt{8} & 0 \\
		\sqrt{8} & (1-\lambda) & \sqrt{8} \\
		0 & \sqrt{8} & (1-\lambda)
		\end{array}\right|&=(1-\lambda)\left[(1-\lambda)^{2}-8\right]+\sqrt{8}[\sqrt{8}(1-\lambda)] \\
		-\lambda^{3}+3 \lambda^{2}-3 \lambda+1&=0 \Rightarrow \lambda=-3,1,5
		\end{align*}
			So the correct answer is \textbf{Option (c)}
	\end{answer}
	\item $\left. \right. $
	\begin{answer}
		\begin{align*}
	\text{For eigen value }|A-\lambda I|&=0\\
		\left|\begin{array}{ccc}
		-\lambda & 1 & 0 \\
		1 & -\lambda & 1 \\
		0 & 1 & -\lambda
		\end{array}\right|&=0 \Rightarrow-\lambda\left(\lambda^{2}-1\right)+1[(\lambda)]=0 \Rightarrow-\lambda^{3}+2 \lambda=0 \\
		\text { eigenvalue } \lambda&=0, \sqrt{2},-\sqrt{2}
		\end{align*}
		So the correct answer is \textbf{Option (d)}
	\end{answer}
	\item $\left. \right. $
	\begin{answer}
		\begin{align*}
		 \text{For eigen value }|A-\lambda I|&=0\\
		\left|\begin{array}{ccc}
		-\lambda & 1 & 1 \\
		1 & -\lambda & 1 \\
		1 & 1 & -\lambda
		\end{array}\right|&=0 \Rightarrow-\lambda\left[\left(\lambda^{2}-1\right)\right]+1(1+\lambda)+1(1+\lambda)=0 \\
		\Rightarrow-\lambda^{3}+3 \lambda+2&=0 \Rightarrow \lambda=-1,-1,2
		\end{align*}
		So the correct answer is \textbf{Option (a)}		
	\end{answer}
	\item $\left. \right. $
	\begin{answer}
		\begin{align*}
	\text{ For eigen value }|A-\lambda I|&=0\\
		\left|\begin{array}{ccc}
		1-\lambda & 1 & 1 \\
		1 & 1-\lambda & 1 \\
		1 & 1 & 1-\lambda
		\end{array}\right|&=0 \Rightarrow(1-\lambda)\left[\left(1-\lambda^{2}\right)-1\right]+1[1-(1-\lambda)]+1[1-(1-\lambda)]&=0 \\
		\Rightarrow-\lambda^{3}+3 \lambda^{2}=0 \Rightarrow \lambda=0,0,3
		\end{align*}
		So the correct answer is \textbf{Option (a)}	
	\end{answer}
	\item $\left. \right. $
	\begin{answer}
		\begin{align*}
		 For eigen value |A-\lambda I|&=0\\
		\left|\begin{array}{ccc}
		1-\lambda & 0 & 1-\lambda \\
		0 & 1-\lambda & 0 \\
		1 & 0 & 1-\lambda
		\end{array}\right|&=(1-\lambda)\left[(1-\lambda)^{2}-0\right]+0+1[-(1-\lambda)]=0 \\
		\Rightarrow-\lambda^{3}+3 \lambda^{2}-2 \lambda&=0 \Rightarrow \lambda=0,1,2\\
		\text{Eigen vector can be determine by }A X_{1}&=\lambda X_{1} \Rightarrow\left[\begin{array}{lll}1 & 0 & 1 \\ 0 & 1 & 0 \\ 1 & 0 & 1\end{array}\right]\left[\begin{array}{l}x_{1} \\ x_{2} \\ x_{3}\end{array}\right]=\lambda\left[\begin{array}{l}x_{1} \\ x_{2} \\ x_{3}\end{array}\right]\\
	\text{	For }\lambda=0,
		\left[\begin{array}{lll}
		1 & 0 & 1 \\
		0 & 1 & 0 \\
		1 & 0 & 1
		\end{array}\right]\left[\begin{array}{l}
		x_{1} \\
		x_{2} \\
		x_{3}
		\end{array}\right]=0\left[\begin{array}{l}
		x_{1} \\
		x_{2} \\
		x_{3}
		\end{array}\right]& \Rightarrow x_{1}+x_{3}=0, x_{2}=x_{2}, x_{1}+x_{3}=0 \Rightarrow x_{1}=-x_{3} \text { and } x_{2}=0\\
		X_{1}&=\left[\begin{array}{c}
		x_{1} \\
		0 \\
		-x_{1}
		\end{array}\right]=x_{1}\left[\begin{array}{c}
		1 \\
		0 \\
		-1
		\end{array}\right]\\
		\text{From orthogonal relation }X_{1}^{T} X_{1}&=1 \Rightarrow x_{1}=\frac{1}{\sqrt{2}} \Rightarrow X_{1}=\frac{1}{\sqrt{2}}\left[\begin{array}{c}1 \\ 0 \\ -1\end{array}\right]
		\end{align*}
		So the correct answer is \textbf{Option (b)}
	\end{answer}
	\item $\left. \right. $
	\begin{answer}
		\begin{align*}
	\text{ For eigen value }|A-\lambda I|&=0\\
		\left|\begin{array}{ccc}
		1-\lambda & \sqrt{8} & 0 \\
		\sqrt{8} & (1-\lambda) & \sqrt{8} \\
		0 & \sqrt{8} & (1-\lambda)
		\end{array}\right|&=(1-\lambda)\left[(1-\lambda)^{2}-8\right]+\sqrt{8}[\sqrt{8}(1-\lambda)] \\
		-\lambda^{3}+3 \lambda^{2}-3 \lambda+1&=0 \Rightarrow \lambda=-3,1,5\\
		\text{eigen vector can be determine by }A X_{1}&=\lambda_{1} X_{1}
	\text{	For }\lambda=5,\\
	{\left[\begin{array}{ccc}1 & \sqrt{8} & 0 \\ \sqrt{8} & 1 & \sqrt{8} \\ 0 & \sqrt{8} & 1\end{array}\right]\left[\begin{array}{l}x_{1} \\ x_{2} \\ x_{3}\end{array}\right]=5\left[\begin{array}{l}x_{1} \\ x_{2} \\ x_{3}\end{array}\right] }&\\
	x_{1}+\sqrt{8} x_{2}=5 x_{1}, \sqrt{8} x_{1}+x_{2}+\sqrt{8} x_{3}&=5 x_{2}, \sqrt{8} x_{2}+x_{3}=5 x_{3}\\
	\Rightarrow x_{1}=x_{3}, x_{2}=\sqrt{2} x_{1} \quad X_{1}=\left[\begin{array}{c}x_{1} \\ \sqrt{2} x_{1} \\ x_{1}\end{array}\right] \Rightarrow X_{1}&=x_{1}\left[\begin{array}{c}1 \\ \sqrt{2} \\ 1\end{array}\right]\\
	\text{Form orthogonality condition }x_{1}=\frac{1}{2}, X_{1}&=\frac{1}{2}\left[\begin{array}{c}1 \\ \sqrt{2} \\ 1\end{array}\right]
		\end{align*}
			So the correct answer is \textbf{Option (a)}
	\end{answer}
	\item $\left. \right. $
	\begin{answer}
		\begin{align*}
		 \text{For eigen value }|A-\lambda I|&=0\\
		\left|\begin{array}{ccc}
		-\lambda & 1 & 0 \\
		1 & -\lambda & 1 \\
		0 & 1 & -\lambda
		\end{array}\right|=0 \Rightarrow-&\lambda\left(\lambda^{2}-1\right)+1[(\lambda)]=0 \Rightarrow-\lambda^{3}+2 \lambda=0\\
	\text{	eigenvalue }\lambda=0, \sqrt{2},-\sqrt{2}\\
		\text{eigen vector can be determine by }A X_{1}&=\lambda_{1} X_{1}
	\text{	For }\lambda=-\sqrt{2}\\
		\left[\begin{array}{lll}
		0 & 1 & 0 \\
		1 & 0 & 1 \\
		0 & 1 & 0
		\end{array}\right]\left[\begin{array}{l}
		x_{1} \\
		x_{2} \\
		x_{3}
		\end{array}\right]=-\sqrt{2}\left[\begin{array}{l}
		x_{1} \\
		x_{2} \\
		x_{3}
		\end{array}\right] \Rightarrow x_{2}&=-\sqrt{2} x_{1}, x_{1}+x_{3}=-\sqrt{2} x_{2}, x_{2}=-\sqrt{2} x_{3}\\
		\Rightarrow x_{1}&=x_{3}, x_{2}=-\sqrt{2} x_{1}\\
		X_{1}=\left[\begin{array}{c}x_{1} \\ -\sqrt{2} x_{1} \\ x_{1}\end{array}\right] \text{. From orthogonality condition, }X_{1}^{T} X_{1}&=1 \Rightarrow x_{1}=\frac{1}{2} \Rightarrow X_{1}=\frac{1}{2}\left[\begin{array}{c}1 \\ -\sqrt{2} \\ 1\end{array}\right]\\
	\text{	For }\lambda&=0\\
		\left[\begin{array}{lll}
		0 & 1 & 0 \\
		1 & 0 & 1 \\
		0 & 1 & 0
		\end{array}\right]\left[\begin{array}{l}
		x_{1} \\
		x_{2} \\
		x_{3}
		\end{array}\right]=0\left[\begin{array}{c}
		x_{1} \\
		x_{2} \\
		x_{3}
		\end{array}\right] \Rightarrow x_{2}=0, x_{1}+x_{3}&=0, \Rightarrow x_{1}=-x_{3} \Rightarrow X_{2}=\left[\begin{array}{c}
		x_{1} \\
		0 \\
		-x_{1}
		\end{array}\right]\\
		\text{From orthogonality condition }X_{2}^{T} X_{2}&=1 \Rightarrow x_{1}=\frac{1}{\sqrt{2}} \Rightarrow X_{2}=\frac{1}{\sqrt{2}}\left[\begin{array}{c}1 \\ 0 \\ -1\end{array}\right]\\
		\text{For }\lambda&=\sqrt{2},\\
		\left[\begin{array}{lll}
		0 & 1 & 0 \\
		1 & 0 & 1 \\
		0 & 1 & 0
		\end{array}\right]\left[\begin{array}{l}
		x_{1} \\
		x_{2} \\
		x_{3}
		\end{array}\right]=\sqrt{2}\left[\begin{array}{l}
		x_{1} \\
		x_{2} \\
		x_{3}
		\end{array}\right] \Rightarrow x_{2}=\sqrt{2} x_{1}, x_{1}+x_{3}&=\sqrt{2} x_{2}, x_{2}=\sqrt{2} x_{3} \Rightarrow x_{1}=x_{3}\\
		\Rightarrow X_{3}=\left[\begin{array}{c}x_{1} \\ \sqrt{2} x_{1} \\ x_{1}\end{array}\right] .\text{ From orthogonality condition }X_{3}{ }^{T} X_{3}=1 \Rightarrow x_{1}&=\frac{1}{2} \Rightarrow X_{3}=\frac{1}{2}\left[\begin{array}{c}1 \\ \sqrt{2} \\ 1\end{array}\right]
		\end{align*}
		So the correct answer is \textbf{Option (a)}
	\end{answer}
	\item $\left. \right. $
	\begin{answer}
		\begin{align*}
		\text{For eigen value }|A-\lambda I|&=0
		X_{1}=\frac{1}{2}\left[\begin{array}{c}-2 \\ 1 \\ 1\end{array}\right]\\
		\left|\begin{array}{ccc}-\lambda & 1 & 1 \\ 1 & -\lambda & 1 \\ 1 & 1 & -\lambda\end{array}\right|&=0 \Rightarrow-\lambda\left[\left(\lambda^{2}-1\right)\right]+1(1+\lambda)+1(1+\lambda)=0\\
		\Rightarrow-\lambda^{3}+3 \lambda+2&=0 \Rightarrow \lambda=-1,-1,2\\
		\text{Eigen vector can be determine by }A X_{1}&=\lambda_{1} X_{1}\\
	\text{	For }\lambda&=-1\\
	\left[\begin{array}{lll}0 & 1 & 1 \\ 1 & 0 & 1 \\ 1 & 1 & 0\end{array}\right]\left[\begin{array}{l}x_{1} \\ x_{2} \\ x_{3}\end{array}\right]=-1\left[\begin{array}{c}x_{1} \\ x_{2} \\ x_{3}\end{array}\right] \Rightarrow x_{2}+x_{3}&=-x_{1}, \quad x_{1}+x_{3}=-x_{2}, x_{1}+x_{2}=-x_{3} \Rightarrow x_{1}+x_{2}+x_{3}=0\\
	\text{if }X_{1}&=\left[\begin{array}{c}-\left(x_{2}+x_{3}\right) \\ x_{2} \\ x_{3}\end{array}\right]\text{ if }\Rightarrow x_{2}=k_{1} x_{3}=k_{2}\\
	X_{1}&=\left[\begin{array}{c}
	-\left(k_{1}+k_{2}\right) \\
	k_{1} \\
	k_{2}
	\end{array}\right] \text { if } k_{1}=k_{2}\\
	X_{1}&=k_{1}\left[\begin{array}{c}-2 \\ 1 \\ 1\end{array}\right]\text{ from orthogonality condition }X_{1}^{T} X_{1}=1 \Rightarrow k_{1}=\frac{1}{2}\\
	X_{1}&=\frac{1}{2}\left[\begin{array}{c}-2 \\ 1 \\ 1\end{array}\right]\\
	\text{For }\lambda&=-1\text{ similarly if }k_{1}=1, k_{2}=-1\\
	\text{Then }X_{2}=\left[\begin{array}{c}0 \\ 1 \\ -1\end{array}\right] \quad X_{2}&=\frac{1}{\sqrt{2}}\left[\begin{array}{c}0 \\ 1 \\ -1\end{array}\right]
		\end{align*}
	So the correct answer is \textbf{Option (a)}
	\end{answer}
	\item $\left. \right. $
	\begin{answer}
		\begin{align*}
		 \text{For eigen value }|A-\lambda I|&=0\\
		\left|\begin{array}{ccc}
		1-\lambda & 1 & 1 \\
		1 & 1-\lambda & 1 \\
		1 & 1 & 1-\lambda
		\end{array}\right|&=0 \Rightarrow(1-\lambda)\left[\left(1-\lambda^{2}\right)-1\right]+1[1-(1-\lambda)]+1[1-(1-\lambda)]=0\\
		\Rightarrow-\lambda^{3}+3 \lambda^{2}&=0 \Rightarrow \lambda=0,0,3\\
	\text{	eigen vector can be determine }by A X_{1}&=\lambda_{1} X_{1}\\
		\text{For }\lambda&=0,\\
		\left[\begin{array}{lll}
		1 & 1 & 1 \\
		1 & 1 & 1 \\
		1 & 1 & 1
		\end{array}\right]\left[\begin{array}{l}
		x_{1} \\
		x_{2} \\
		x_{3}
		\end{array}\right]&=0\left[\begin{array}{l}
		x_{1} \\
		x_{2} \\
		x_{3}
		\end{array}\right] \Rightarrow x_{1}+x_{2}+x_{3}=0, x_{1}+x_{2}+x_{3}=0, x_{1}+x_{2}+x_{3}=0\\
		\text{Now }x_{1}&=-x_{2}-x_{3}\text{, if }x_{2}=-1, x_{3}=1 \Rightarrow x_{1}=0\\
		X_{1}&=\left[\begin{array}{c}
		-x_{2}-x_{3} \\
		x_{2} \\
		x_{3}
		\end{array}\right] \Rightarrow X_{1}=\left[\begin{array}{c}
		0 \\
		1 \\
		-1
		\end{array}\right]=X_{1}=\frac{1}{\sqrt{2}}\left[\begin{array}{c}
		0 \\
		1 \\
		-1
		\end{array}\right]\\
	\text{	For }\lambda&=0, X_{2}=\left[\begin{array}{c}-x_{2}-x_{3} \\ x_{2} \\ x_{3}\end{array}\right]\text{ is orthogonal to }X_{1}.\\
	\frac{1}{\sqrt{2}}\left(\begin{array}{lll}
	0 & 1 & -1
	\end{array}\right)\left[\begin{array}{c}
	-x_{2}-x_{3} \\
	x_{2} \\
	x_{3}
	\end{array}\right]&=0 \Rightarrow x_{2}=x_{3} \Rightarrow X_{2}=\left[\begin{array}{c}
	-2 x_{2} \\
	x_{2} \\
	x_{2}
	\end{array}\right]\\
	\text{From orthogonality condition, }X_{2}^{T} X_{2}&=1 \Rightarrow x_{2}=\frac{1}{\sqrt{6}} \Rightarrow X_{2}=\frac{1}{\sqrt{6}}\left[\begin{array}{c}-2 \\ 1 \\ 1\end{array}\right]
		\end{align*}
		So the correct answer is \textbf{Option (a)}
	\end{answer}
		\item $\left. \right. $
	\begin{answer}
		\begin{align}
	\text{$A$ diagonalised }&=S^{-1} A S\label{AMS-01}\\
	S&=\left[\begin{array}{ccc}1 & 0 & 0 \\ 0 & \frac{1}{\sqrt{2}} & \frac{1}{\sqrt{2}} \\ 0 & -\frac{1}{\sqrt{2}} & \frac{1}{\sqrt{2}}\end{array}\right],\text{ and finding $S^{-1}$ and putting in (\ref{AMS-01}), we get}\notag\\
	\text{$A$ diagonalized }&=\left[\begin{array}{lll}2 & 0 & 0 \\ 0 & 0 & 0 \\ 0 & 0 & 2\end{array}\right]\notag
		\end{align}
			So the correct answer is \textbf{Option (c)}
	\end{answer}
	\item $\left. \right. $	
	\begin{answer}
		\begin{align*}
	&\text{ For eigen values }\left[\begin{array}{ccc}1-\lambda & 1 & 1 \\ 1 & 1-\lambda & 1 \\ 1 & 1 & 1-\lambda\end{array}\right]=0\\
	&(1-\lambda)\left((1-\lambda)^{2}-1\right)-(1-\lambda-1)+1(1-(1-\lambda))=0\\
	&(1-\lambda)\left(1+\lambda^{2}-2 \lambda-1\right)+\lambda+\lambda=0 \Rightarrow \lambda^{2}-2 \lambda-\lambda^{3}+2 \lambda^{2}+2 \lambda=0\\
	&\lambda^{3}-3 \lambda^{2}=0 \Rightarrow \lambda^{2}(\lambda-3)=0 \Rightarrow \lambda=0,0,3
	\intertext{For any $n \times n$ matrix having all elements unity eigenvalues are $0,0,0, \ldots, n$.}
		\end{align*}
	So the correct answer is \textbf{Option (b)}		
	\end{answer}
	\item $\left. \right. $	
	\begin{answer}
		\begin{align*}
		\intertext{For $e^{M}$ let us try to diagonalize matrix $M$ using similarity transformation.}
		\text{For }\lambda&=3,\left[\begin{array}{ccc}-2 & 1 & 1 \\ 1 & -2 & 1 \\ 1 & 1 & -2\end{array}\right]\left[\begin{array}{l}x_{1} \\ x_{2} \\ x_{3}\end{array}\right]=\left[\begin{array}{l}0 \\ 0 \\ 0\end{array}\right]\\
		\Rightarrow-2 x_{1}+x_{2}+x_{3}&=0, x_{1}-2 x_{2}+x_{3}=0, x_{1}+x_{2}-2 x_{3}=0 \\
		\Rightarrow-3 x_{2}+3 x_{3}&=0 \text { or } x_{2}=x_{3} \Rightarrow x_{1}=x_{2}=x_{3}=k .\\
		\text{Eigen vector is }&1 \sqrt{3}\left[\begin{array}{l}1 \\ 1 \\ 1\end{array}\right]\text{ where }k=1.\\
	\text{	For }\lambda&=0,\\
		\left[\begin{array}{lll}1 & 1 & 1 \\ 1 & 1 & 1 \\ 1 & 1 & 1\end{array}\right]\left[\begin{array}{l}x_{1} \\ x_{2} \\ x_{3}\end{array}\right]\left[\begin{array}{l}0 \\ 0 \\ 0\end{array}\right] &\Rightarrow x_{1}+x_{2}+x_{3}=0\\
		\intertext{Let $x_{1}=k_{1}, x_{2}=k_{2}$ and $x_{3}=k_{1}+k_{2}$. Eigen vector is $\left[\begin{array}{c}k_{1} \\ k_{2} \\ \left(k_{1}+k_{2}\right)\end{array}\right]=1 / \sqrt{2}\left[\begin{array}{c}1 \\ -1 \\ 1\end{array}\right]$ where $k_{1}=k_{2}=1$.}
		\intertext{Let $x_{1}=k_{1}, x_{2}=k_{2}$ and $x_{3}=-\left(k_{1}+k_{2}\right)$. Other Eigen vector $1 / \sqrt{2}\left[\begin{array}{c}1 \\ 0 \\ -1\end{array}\right]$ where $k_{1}=1, k_{2}=-1$.}
		S=\left[\begin{array}{ccc}0 & 1 & 1 \\ -1 & 0 & 1 \\ 1 & -1 & 1\end{array}\right] \Rightarrow S^{-1}&=\left[\begin{array}{ccc}1 & -2 & 1 \\ 2 & -1 & -1 \\ 1 & -1 & 1\end{array}\right] \Rightarrow D=S^{-1} M S, M=S D S^{-1}\\
		e^{M}=S e^{D} S^{-1} \Rightarrow e^{D}&=\left[\begin{array}{ccc}1 & 0 & 0 \\ 0 & 1 & 0 \\ 0 & 0 & e^{3}\end{array}\right] \Rightarrow e^{M}=1+\frac{\left(e^{3}-1\right) M}{3}
		\end{align*}
			So the correct answer is \textbf{Option (a)}
	\end{answer}
	\item $\left. \right. $	
	\begin{answer}
		\begin{align*}
		 \intertext{The characteristic equation of the matrix $A,|A-\lambda I|=0$}
		\Rightarrow|A-\lambda I|&=\left|\begin{array}{ccc}
		2-\lambda & 3 & 0 \\
		3 & 2-\lambda & 0 \\
		0 & 0 & 1-\lambda
		\end{array}\right|=0 \Rightarrow \lambda^{3}-5 \lambda^{2}-\lambda+5=0 \Rightarrow \lambda=5,1,-1
		\end{align*}
		So the correct answer is \textbf{Option (c)}
	\end{answer}
		\item $\left. \right. $	
	\begin{answer}
		\begin{align*}
		\text{For eigenvalues }|A-\lambda I|&=0 \Rightarrow\left[\begin{array}{ccc}1-\lambda & 2 & 3 \\ 2 & 4-\lambda & 6 \\ 3 & 6 & 9-\lambda\end{array}\right]=0\\
		(1-\lambda)[(4-\lambda)&(9-\lambda)-36]-2[2(9-\lambda)-18]+3[12-3(4-\lambda)]=0\\
		(1-\lambda)(4-\lambda)&(9-\lambda)-36(1-\lambda)-4(9-\lambda)+36+9 \lambda=0\\
		\lambda^{3}-14 \lambda^{2}&=0 \Rightarrow \lambda^{2}(\lambda-14)=0 \Rightarrow \lambda=0,0,14
		\end{align*}
			So the correct answer is \textbf{Option (d)}
	\end{answer}
		\item $\left. \right. $	
	\begin{answer}
		\begin{align*}
		A&=\left[\begin{array}{ccc}0 & -n_{3} & n_{2} \\ n_{3} & 0 & -n_{1} \\ -n_{2} & n_{1} & 0\end{array}\right] \Rightarrow-A^{T}=\left[\begin{array}{ccc}0 & -n_{3} & n_{2} \\ n_{3} & 0 & -n_{1} \\ -n_{2} & n_{1} & 0\end{array}\right]\\
		\Rightarrow \lambda_{1}&=0 \quad \Rightarrow \lambda_{2}=-\sqrt{-n_{1}^{2}-n_{2}^{2}-n_{3}^{2}} \Rightarrow \lambda_{3}=\sqrt{-n_{1}^{2}-n_{2}^{2}-n_{3}^{2}}\\
	\text{ but} \sqrt{n_{1}^{2}+n_{2}^{2}+n_{3}^{2}}&=1 \quad\text{ so, }\quad \lambda_{1}=0, \lambda_{2}=L, \lambda_{3}=-L
	\intertext{$A=-A^{T}$ (Antisymmetric). Eigenvalues are either zero or purely imaginary.}
		\end{align*}
		So the correct answer is \textbf{Option (a)}
	\end{answer}
		\item $\left. \right. $	
	\begin{answer}
		\begin{align*}
		M&=\left(\begin{array}{ccc}0 & 2 i & 3 i \\ -2 i & 0 & 6 i \\ -3 i & -6 i & 0\end{array}\right), M^{+}=\left(\begin{array}{ccc}0 & 2 i & 3 i \\ -2 i & 0 & 6 i \\ -3 i & -6 i & 0\end{array}\right)\\
		&M^{+}=M
	\intertext{	Matrix is Hermitian so roots are real and trace $=0$.}
		&\lambda_{1}+\lambda_{2}+\lambda_{3}=0, \lambda_{1} \cdot \lambda_{2} \cdot \lambda_{3}=0 \Rightarrow \lambda=-7,0,7
		\end{align*}
			So the correct answer is \textbf{Option (b)}
	\end{answer}
	\item $\left. \right. $		
	\begin{answer}
		\begin{align*}
		\text{Let }b&=a\\
		\left(\begin{array}{lll}0 & 0 & 1 \\ 0 & 1 & 0 \\ 1 & 0 & 0\end{array}\right)\left(\begin{array}{l}a \\ a \\ a\end{array}\right)&=\left(\begin{array}{l}a \\ a \\ a\end{array}\right)\text{ and }\left(\begin{array}{lll}0 & 1 & 1 \\ 1 & 0 & 1 \\ 1 & 1 & 0\end{array}\right)\left(\begin{array}{l}a \\ a \\ a\end{array}\right)=\left(\begin{array}{l}a \\ a \\ a\end{array}\right)=\left(\begin{array}{l}a \\ a \\ a\end{array}\right)\\
	\text{	Let }b&=-2 a\\
	\left(\begin{array}{lll}0 & 0 & 1 \\ 0 & 1 & 0 \\ 1 & 0 & 0\end{array}\right)\left(\begin{array}{c}a \\ -2 a \\ a\end{array}\right)&=\left(\begin{array}{c}a \\ -2 a \\ a\end{array}\right)\text{ and }\left(\begin{array}{ccc}0 & 1 & 1 \\ 1 & 0 & 1 \\ 1 & 1 & 0\end{array}\right)\left(\begin{array}{c}a \\ -2 a \\ a\end{array}\right)=\left(\begin{array}{c}-a \\ 2 a \\ -a\end{array}\right)=-1\left(\begin{array}{c}a \\ -2 a \\ a\end{array}\right)
	\intertext{For other combination above relation is not possible.}
		\end{align*}
		So the correct answer is \textbf{Option (b)}
	\end{answer}
	\item $\left. \right. $		
\begin{answer}
	\begin{align*}
	\because\left|\begin{array}{cc}\alpha & \beta \\ -\beta^{+} & \alpha^{*}\end{array}\right|=|\alpha|^{2}+|\beta|^{2}=1
	\end{align*}
	So the correct answer is \textbf{Option (d)}
\end{answer}
	\item $\left. \right. $	
	\begin{answer}
		\begin{align*}
		 \operatorname{Tr}\left[M^{2}\right]=(-1)^{2}+(3)^{2}+(4)^{2}\text{ also } \operatorname{Tr}\left[M^{3}\right]=(-1)^{3}+(3)^{3}+(4)^{3}=90
		\end{align*}
		So the correct answer is \textbf{Option (c)}
	\end{answer}
	\item $\left. \right. $	
	\begin{answer}
		\begin{align*}
	\text{	Unitary }A^{\dagger} A=I
		\end{align*}
		So the correct answer is \textbf{Option (d)}
	\end{answer}
	\item $\left. \right. $		
	\begin{answer}
		\begin{align*}
		A^{\dagger}&=\exp (-i H)=(\exp i H)^{-1}=A^{-1}
		\end{align*}
	So the correct answer is \textbf{Option (b)}
	\end{answer}
		\item $\left. \right. $	
	\begin{answer}
		\begin{align*}
	\text{	Eigen value of matrix }A=\left(\begin{array}{lll}0 & 1 & 0 \\ 1 & 0 & 0 \\ 0 & 0 & 2\end{array}\right)\text{ is }\lambda=2,1,-1.\\
	\text{corresponding eigen vectors are }\left(\begin{array}{l}0 \\ 0 \\ 1\end{array}\right), \quad\left(\begin{array}{c}\frac{1}{\sqrt{2}} \\ \frac{1}{\sqrt{2}} \\ 0\end{array}\right),\left(\begin{array}{c}\frac{1}{\sqrt{2}} \\ -\frac{1}{\sqrt{2}} \\ 0\end{array}\right)\\
	\text{diagonalise matrix }\left(\begin{array}{ccc}2 & 0 & 0 \\ 0 & 1 & 0 \\ 0 & 0 & -1\end{array}\right)\text{ so $U$ is }\left(\begin{array}{ccc}0 & 0 & 0 \\ 0 & \frac{1}{\sqrt{2}} & \frac{1}{\sqrt{2}} \\ 1 & \frac{1}{\sqrt{2}} & -\frac{1}{\sqrt{2}}\end{array}\right)
		\end{align*}
		So the correct answer is \textbf{Option (c)}
	\end{answer}
	\item $\left. \right. $	
	\begin{answer}
		So the correct answer is \textbf{Option (a)}
	\end{answer}
	\item $\left. \right. $	
	\begin{answer}
		\begin{align*}
		\intertext{ If matrix is $2 \times 2$ let $\left(\begin{array}{ll}1 & 2 \\ 2 & 4\end{array}\right)$ then eigen value is given by}\\
		\left(\begin{array}{cc}1-\lambda & 2 \\ 2 & 4-\lambda\end{array}\right)&=0 \Rightarrow(1-\lambda)(4-\lambda)-4=0 \Rightarrow \lambda=0,5
		\intertext{If matrix is $3 \times 3$, let $\left(\begin{array}{ccc}1 & 2 & 3 \\ 2 & 4 & 6 \\ 3 & 6 & 9\end{array}\right)$ then eigen value is given by}
		\left(\begin{array}{ccc}1-\lambda & 2 & 3 \\ 2 & 4-\lambda & 6 \\ 3 & 6 & 9-\lambda\end{array}\right)&=0\\
		(1-\lambda)[(4-\lambda)&(9-\lambda)-36]+2[18-2(9-\lambda)]+3[12-3(4-\lambda)]\\
		(1-\lambda)&\left[\lambda^{2}-13 \lambda+36-36\right]+2[18-18+2 \lambda]+3[12-12+3 \lambda]=0\\
		\lambda^{2}-13 \lambda-\lambda^{3}+13 &\lambda^{2}+13 \lambda=0 \Rightarrow \lambda^{3}-14 \lambda^{2}=0 \Rightarrow \lambda=0,0, \lambda=14
		\intertext{i.e. has one degenerate eigenvalue with degeneracy $2 .$}
		\intertext{Thus one can generalized that for n dimensional matrix has one degenerate eigevalue with degeneracy $(n-1)$.}
		\end{align*}
			So the correct answer is \textbf{Option (a)}
	\end{answer}
	\item $\left. \right. $	
	\begin{answer}
		\begin{align*}
		\intertext{ If $A$ and $P$ be square matrices of the same type and if $P$ be invertible then matrices $A$ and $B=$ $P^{-1} A P$ have the same characteristic roots}
		\text{Then }B-\lambda I&=P^{-1} A P-P^{-1} \lambda I P=P^{-1}(A-\lambda I) P\text{ where $I$ is identity matrix.}\\
		|B-\lambda I|&=\left|P^{-1}(A-\lambda I) P\right|=\left|P^{-1}\right||A-\lambda I||P|=|A-\lambda I|\left|P^{-1}\right||P|=\left|A-\lambda I \| P P^{-1}\right|=|A-\lambda I|
		\intertext{Thus the matrices $A$ and $B\left(=P^{-1} A P\right)$ have the same characteristic equation and hence characteristic roots of eigen values. Since the sum of the eigen values of a matrix and product of eigen values of a matrix is equal to the determinant of matrix hence third alternative is incorrect.}
		\end{align*}
			So the correct answer is \textbf{Option (c)}
	\end{answer}
		\item $\left. \right. $	
	\begin{answer}
		We know that for any matrix\\
		1. The product of eigenvalues is equal to determinant of that matrix.\\
		2. $\lambda_{1}+\lambda_{2}+\lambda_{3}+\ldots \ldots=$ Trace of matrix\\
		$\lambda_{1}+\lambda_{2}+\lambda_{3}=11$ and $\lambda_{1} \lambda_{2} \lambda_{3}=36$. Hence the largest eigen value of the matrix is 9 .\\\\
			So the correct answer is \textbf{Option (c)}
	\end{answer}
		\item $\left. \right. $	
	\begin{answer}
		\begin{align*}
		\text{the characteristic equation is given by}& \left(\begin{array}{ccc}0-\lambda & 1 & 1 \\ 0 & 0-\lambda & 1 \\ 1 & 0 & 0-\lambda\end{array}\right)=0\\
		-\lambda\left((-\lambda)^{2}-0\right)-1(1)+1&\left(0-(-\lambda)=\lambda^{3}-\lambda-1=0\right.
	\intertext{	So from calley hemilton's theorem $M^{3}-M-I=0$ multiply both side by $M^{-1}$}
	M^{2}-I-M^{-1}=0\text{ so }M^{-1}=M^{2}-I.
		\end{align*}
		So the correct answer is \textbf{Option (b)}
	\end{answer}
	\section{MSQ}
		\item $\left. \right. $	
	\begin{answer}
		\begin{align*}
			\text{For eigen value }|A-\lambda I|&=0\\
			\left|\begin{array}{ccc}5-\lambda & 0 & 2 \\ 0 & 1-\lambda & 0 \\ 2 & 0 & 2-\lambda\end{array}\right|&=0 \Rightarrow(5-\lambda)[(1-\lambda)(2-\lambda)]+2(-2)(1-\lambda)=0\\
			\Rightarrow-\lambda^{3}+8 \lambda^{2}-13 \lambda+6&=0,\text{ eigenvalue } \lambda=1,1,6\\
			\text{eigen vector can be determine by }A X_{1}&=\lambda_{1} X_{1}\\
			\text{For }\lambda&=1,\\
			\left[\begin{array}{lll}
			5 & 0 & 2 \\
			0 & 1 & 0 \\
			2 & 0 & 2
			\end{array}\right]\left[\begin{array}{l}
			x_{1} \\
			x_{2} \\
			x_{3}
			\end{array}\right]=\left[\begin{array}{l}
			x_{1} \\
			x_{2} \\
			x_{3}
			\end{array}\right] \Rightarrow 5 x_{1}+2 x_{3}&=x_{1}, x_{2}=x_{2}, 2 x_{1}+2 x_{3}=x_{3} \Rightarrow x_{2}=0, x_{3}=-2 x_{1}\\
			X_{1}=\left[\begin{array}{c}x_{1} \\ 0 \\ -2 x_{1}\end{array}\right]\text{ from orthogonality condition }X_{1}^{T} X_{1}&=1 \Rightarrow x_{1}=\frac{1}{\sqrt{5}} \Rightarrow X_{1}=\frac{1}{\sqrt{5}}\left[\begin{array}{c}1 \\ 0 \\ -2\end{array}\right]
			\intertext{Hence $\lambda=1$ is degenerate its eigen vector will $X_{2}=\left[\begin{array}{c}x_{1} \\ x_{2} \\ -2 x_{1}\end{array}\right]$ will orthogonal to $X_{1}$ so}
			X_{1}^{T} X_{2}&=0 \Rightarrow 4 x_{1}=0\text{ so } X_{2}=\left[\begin{array}{l}0 \\ 1 \\ 0\end{array}\right]\\
			\text{For }\lambda&=6\\
			\left[\begin{array}{lll}
			5 & 0 & 2 \\
			0 & 1 & 0 \\
			2 & 0 & 2
			\end{array}\right]\left[\begin{array}{l}
			x_{1} \\
			x_{2} \\
			x_{3}
			\end{array}\right]=6\left[\begin{array}{l}
			x_{1} \\
			x_{2} \\
			x_{3}
			\end{array}\right] \Rightarrow 5 x_{1}+2 x_{3}&=6 x_{1}, x_{2}=6 x_{2}, 2 x_{1}+2 x_{3}=6 x_{3} \Rightarrow x_{1}=2 x_{3}, x_{2}=0,\\
			X_{3}=\left[\begin{array}{c}2 x_{3} \\ 0 \\ x_{3}\end{array}\right]=x_{3}\left[\begin{array}{l}2 \\ 0 \\ 1\end{array}\right].\text{ From orthogonality }&\text{condition,} x_{3}=\frac{1}{\sqrt{5}} \Rightarrow X_{3}=\frac{1}{\sqrt{5}}\left[\begin{array}{l}2 \\ 0 \\ 1\end{array}\right]
		\end{align*}
		So the correct answers are \textbf{Option (a),(b) and (c)}
	\end{answer}
	\item $\left. \right. $	
	\begin{answer}
		\begin{align*}
		\text{For eigen value }|A-\lambda I|&=0\\
		\left|\begin{array}{ccc}1-\lambda & 1 & 0 \\ 1 & 1-\lambda & 0 \\ 0 & 0 & -\lambda\end{array}\right|&=0 \Rightarrow(1-\lambda)[-\lambda(1-\lambda)]+\lambda=0 \Rightarrow-\lambda^{3}+2 \lambda^{2}=0\\
		\text{eigenvalue }\lambda&=0,0,2\\
		\text{eigen vector can be determine by }A X_{1}&=\lambda_{1} X_{1}\\
		\text{For }\lambda&=0,\\
		\left[\begin{array}{lll}1 & 1 & 0 \\ 1 & 1 & 0 \\ 0 & 0 & 0\end{array}\right]\left[\begin{array}{l}x_{1} \\ x_{2} \\ x_{3}\end{array}\right]&=0\left[\begin{array}{l}x_{1} \\ x_{2} \\ x_{3}\end{array}\right] \Rightarrow x_{1}+x_{2}=0, \quad x_{3}=0 \Rightarrow x_{1}=-x_{2}\\
		X_{1}=\left[\begin{array}{c}x_{1} \\ -x_{1} \\ 0\end{array}\right]. \text{From orthogonality }&\text{ condition,}X_{1}^{T} X_{1}=1 \Rightarrow x_{1}=\frac{1}{\sqrt{2}} \Rightarrow X_{1}=\frac{1}{\sqrt{2}}\left[\begin{array}{c}1 \\ -1 \\ 0\end{array}\right]
		\intertext{One can not find independent vector directly so we can find it from block diagonal method so another eigen vector which is independent to $\Rightarrow X_{1}=\frac{1}{\sqrt{2}}\left[\begin{array}{c}1 \\ -1 \\ 0\end{array}\right]$ and also eigenvector corresponds to $\lambda=0$ is $\Rightarrow X_{2}=\left[\begin{array}{l}0 \\ 0 \\ 1\end{array}\right]$}
		\text{For }\lambda&=2,\\
		\left[\begin{array}{lll}
		1 & 1 & 0 \\
		1 & 1 & 0 \\
		0 & 0 & 0
		\end{array}\right]\left[\begin{array}{l}
		x_{1} \\
		x_{2} \\
		x_{3}
		\end{array}\right]=2\left[\begin{array}{l}
		x_{1} \\
		x_{2} \\
		x_{3}
		\end{array}\right] \Rightarrow x_{1}+x_{2}=2 x_{1}, x_{3}&=0 \Rightarrow x_{1}=x_{2} \Rightarrow X_{3}=\left[\begin{array}{c}
		x_{2} \\
		x_{2} \\
		0
		\end{array}\right]\\
		\text{From orthogonality condition }X_{3}{ }^{T} X_{3}&=1 \Rightarrow x_{2}=\frac{1}{\sqrt{2}} \Rightarrow X_{3}=\frac{1}{\sqrt{2}}\left[\begin{array}{l}1 \\ 1 \\ 0\end{array}\right]
		\end{align*}
			So the correct answers are \textbf{Option (a),(b),(c)and (d)}
	\end{answer}
	\item $\left. \right. $	
	\begin{answer}
		\begin{align*}
	\text{	For eigen value }|A-\lambda I|&=0\\
	\left|\begin{array}{ccc}5-\lambda & 0 & \sqrt{3} \\ 0 & 3-\lambda & 0 \\ \sqrt{3} & 0 & 3-\lambda\end{array}\right|&=0 \Rightarrow(5-\lambda)\left[(3-\lambda)^{2}-0\right]+\sqrt{3}[-\sqrt{3}(3-\lambda)]=0\\
	\Rightarrow-\lambda^{3}+11 \lambda^{2}-36 \lambda+36&=0 \Rightarrow \lambda=2,3,6\\
	\text{eigen vector can be determine by }A X_{1}&=\lambda_{1} X_{1}\\
	\text{	For }\lambda&=2,\\
	\left[\begin{array}{ccc}5 & 0 & \sqrt{3} \\ 0 & 3 & 0 \\ \sqrt{3} & 0 & 3\end{array}\right]\left[\begin{array}{l}x_{1} \\ x_{2} \\ x_{3}\end{array}\right]&=2\left[\begin{array}{l}x_{1} \\ x_{2} \\ x_{3}\end{array}\right]\\
	5 x_{1}+\sqrt{3} x_{3}&=2 x_{1}, \quad 3 x_{2}=2 x_{2}, \sqrt{3} x_{1}+3 x_{3}=2 x_{3} \Rightarrow x_{2}=0, x_{3}=-\sqrt{3} x_{1}\\
	X_{1}=\left[\begin{array}{c}x_{1} \\ 0 \\ \sqrt{3} x_{1}\end{array}\right] .&\text{ From orthogonality condition }X_{1}^{T} X_{1}=1 \Rightarrow x_{1}=\frac{1}{2} \Rightarrow X_{1}=\frac{1}{2}\left[\begin{array}{c}1 \\ 0 \\ -\sqrt{3}\end{array}\right]\\
	\text{For }\lambda&=3,\\
	\left[\begin{array}{ccc}5 & 0 & \sqrt{3} \\ 0 & 3 & 0 \\ \sqrt{3} & 0 & 3\end{array}\right]\left[\begin{array}{l}x_{1} \\ x_{2} \\ x_{3}\end{array}\right]&=3\left[\begin{array}{l}x_{1} \\ x_{2} \\ x_{3}\end{array}\right] \Rightarrow 5 x_{1}+\sqrt{3} x_{3}=3 x_{1}, 3 x_{2}=3 x_{2}, \sqrt{3} x_{1}+3 x_{3}=3 x_{3} \Rightarrow x_{1}=0\\
	X_{2}&=\left[\begin{array}{l}
	0 \\
	1 \\
	0
	\end{array}\right]\\
\text{	For }\lambda&=6,\\
\left[\begin{array}{ccc}5 & 0 & \sqrt{3} \\ 0 & 3 & 0 \\ \sqrt{3} & 0 & 3\end{array}\right]\left[\begin{array}{l}x_{1} \\ x_{2} \\ x_{3}\end{array}\right]&=6\left[\begin{array}{l}x_{1} \\ x_{2} \\ x_{3}\end{array}\right] \Rightarrow 5 x_{1}+\sqrt{3} x_{3}=6 x_{1}, 3 x_{2}=6 x_{2} \Rightarrow x_{1}=\sqrt{3} x_{3}, x_{2}=0\\
X_{3}=\left[\begin{array}{c}\sqrt{3} x_{3} \\ 0 \\ x_{3}\end{array}\right] .&\text{ From orthogonality condition }X_{3}{ }^{T} X_{3}=1 \Rightarrow x_{3}=\frac{1}{2} \Rightarrow X_{3}=\frac{1}{2}\left[\begin{array}{c}\sqrt{3} \\ 0 \\ 1\end{array}\right]
		\end{align*}
			So the correct answers are \textbf{Option (a) and (c)}
	\end{answer}
	\item $\left. \right. $	
	\begin{answer}
		\begin{align*}
		(A B+B A)^{\dagger}&=B^{\dagger} A^{\dagger}+A^{\dagger} B^{\dagger} \Rightarrow-B A-A B=-(A B+B A)\\
		-i(A B+B A)^{\dagger}&=-i\left(B^{\dagger} A^{\dagger}+A^{\dagger} B^{\dagger}\right) \Rightarrow-i(-B A-A B)=i(A B+B A)\\
		\end{align*}
		So the correct answers are \textbf{Option (a) and (d)}	
	\end{answer}
		\item $\left. \right. $	
	\begin{answer}
		\begin{align*}
		\left(\begin{array}{ccc}1-\lambda & 0 & 0 \\ 0 & 0-\lambda & 1 \\ 0 & 1 & 0-\lambda\end{array}\right)=0\\
		(1-\lambda)\left(\lambda^{2}-1\right)&=0 \Rightarrow \lambda^{3}-\lambda^{2}-\lambda+1=0\\
		A^{3}-A^{2}-A+I&=0 \Rightarrow A^{2}-A-I+A^{-1}=0\\
		A^{2}=I \Rightarrow A^{2}-A-I+A^{-1}&=0 \Rightarrow I-A-I-A^{-1}=0 \Rightarrow A^{-1}=A
		\end{align*}
			So the correct answers are \textbf{Option (a),(b),(c) and (d)}
	\end{answer}
	\section{NAT}
		\item $\left. \right. $	
	\begin{answer}
		\begin{align*}
		\left[\begin{array}{ccc}4-\lambda & -1 & -1 \\ -1 & 4-\lambda & -1 \\ -1 & -1 & 4-\lambda\end{array}\right] \Rightarrow(4-\lambda)\left[\begin{array}{ccc}1 & -1 & -1 \\ 0 & 5-\lambda & 0 \\ 0 & 0 & 5-\lambda\end{array}\right]&=(4-\lambda)(5-\lambda)^{2}=0 \Rightarrow \lambda=2,5,5
		\end{align*}
			So the correct answer is \textbf{5}
	\end{answer}
		\item $\left. \right. $	
	\begin{answer}
		\begin{align*}
		|A-\lambda I|=0 \Rightarrow\left|\begin{array}{ccc}-\lambda & 1 & 0 \\ 1 & -\lambda & 1 \\ 0 & 1 & -\lambda\end{array}\right|=0 \Rightarrow-\lambda\left(\lambda^{2}-1\right)+\lambda=0 \Rightarrow \lambda=0,+\sqrt{2},-\sqrt{2}
		\end{align*}
			So the correct answer is \textbf{0}
	\end{answer}
		\item $\left. \right. $	
	\begin{answer}
		\begin{align*}
		A&=\left[\begin{array}{ccc}-\lambda & 0 & 1 \\ 0 & 1-\lambda & 0 \\ 1 & 0 & 0-\lambda\end{array}\right]=0\\
		&\Rightarrow(-\lambda)(1-\lambda)(-\lambda)+(-(1-\lambda)=0\\
		&\Rightarrow(1-\lambda)\left(\lambda^{2}-1\right)=0 \quad \Rightarrow(1-\lambda)\left(\lambda^{2}-1\right)=0 \quad \lambda=1,1,-1
		\end{align*}
		So the correct answer is \textbf{1}
	\end{answer}
	
	
	
	
\end{enumerate}