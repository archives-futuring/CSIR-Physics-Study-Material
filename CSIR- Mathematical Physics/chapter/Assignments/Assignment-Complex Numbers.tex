\chapter{Assignment-Complex Numbers}
\section{MCQ}
\begin{enumerate}
	\item The amplitude of $\frac{1+i \sqrt{3}}{\sqrt{3}+i}$ is
 \begin{tasks}(2)
	\task[\textbf{a.}]$\frac{\pi}{3}$
	\task[\textbf{b.}]$-\frac{\pi}{3}$
	\task[\textbf{c.}]$\frac{\pi}{6}$
	\task[\textbf{d.}]  $-\frac{\pi}{6}$
\end{tasks}	
	\item If $\frac{1-i x}{1+i x}=a+i b$, then $a^{2}+b^{2}$ is
	 \begin{tasks}(2)
		\task[\textbf{a.}]1
		\task[\textbf{b.}]$-1$
		\task[\textbf{c.}] 0
		\task[\textbf{d.}] none of these
	\end{tasks}
	\item If $z=1-\cos \theta+i \sin \theta$, then $|z|$ equals
	 \begin{tasks}(2)
		\task[\textbf{a.}]$2 \sin \frac{\theta}{2}$
		\task[\textbf{b.}]$2 \cos \frac{\theta}{2}$
		\task[\textbf{c.}]$2\left|\sin \frac{\theta}{2}\right|$
		\task[\textbf{d.}] $2\left|\cos \frac{\theta}{2}\right|$
	\end{tasks}
	\item If $z=\frac{1}{(2+3 i)^{2}}$, then $|z|$ equals
	 \begin{tasks}(2)
		\task[\textbf{a.}]$\frac{1}{13}$
		\task[\textbf{b.}]$\frac{1}{15}$
		\task[\textbf{c.}]$\frac{1}{12}$
		\task[\textbf{d.}] none of these
	\end{tasks}
	\item If $x_{n}=\cos \left(\frac{\pi}{2^{n}}\right)+i \sin \left(\frac{\pi}{2^{n}}\right)$
	Then the value of $x_{1} \cdot x_{2} \cdot x_{3} \ldots$, up to infinity is
	 \begin{tasks}(2)
		\task[\textbf{a.}] $-2$
		\task[\textbf{b.}] $+2$
		\task[\textbf{c.}] $-1$
		\task[\textbf{d.}] $+1$
	\end{tasks}
	\item If $\left|z-\frac{4}{z}\right|=2$, then the maximum value of $|z|$ is
	 \begin{tasks}(2)
		\task[\textbf{a.}]$\sqrt{5}$
		\task[\textbf{b.}]$\sqrt{5}+1$
		\task[\textbf{c.}]$\sqrt{5}-1$
		\task[\textbf{d.}] none of these
	\end{tasks}
	\item The value of $1+\sum_{k=0}^{14}\left\{\cos \frac{(2 k+1) \pi}{15}+i \sin \frac{(2 k+1) \pi}{15}\right\}$ is
	 \begin{tasks}(2)
		\task[\textbf{a.}]0
		\task[\textbf{b.}] $-1$
		\task[\textbf{c.}]1
		\task[\textbf{d.}] $i$
	\end{tasks}
	\item The point representing complex number for which $|z+4|^{2}-|z-4|^{2}=8$ lie on
	 \begin{tasks}(2)
		\task[\textbf{a.}]a straight line parallel to $x$-axis
		\task[\textbf{b.}]a straight line parallel to $y$-axis
		\task[\textbf{c.}]a circle with centre as origin
		\task[\textbf{d.}] a circle with centre other than origin
	\end{tasks}
	\item If $\left|\begin{array}{ccc}6 i & -3 i & 1 \\ 4 & 3 i & -1 \\ 20 & 3 & i\end{array}\right|=x+i y$, then
	 \begin{tasks}(2)
		\task[\textbf{a.}] $x=3, y=1$
		\task[\textbf{b.}] $x=1, y=3$
		\task[\textbf{c.}]$x=0, y=3$
		\task[\textbf{d.}]  $x=0, y=0$
	\end{tasks}
	\item If the number $\frac{z-1}{z+1}$ is purely imaginary, then
	 \begin{tasks}(2)
		\task[\textbf{a.}]$|z|=1$
		\task[\textbf{b.}]$|z|>1$
		\task[\textbf{c.}]$|z|<1$
		\task[\textbf{d.}] $|z|>2$
	\end{tasks}
	\item The value of integral $I=\int_{0}^{\pi} \frac{2 d \theta}{R-\cos \theta}$ is given by where $R$ is real constant.
	\begin{tasks}(2)
		\task[\textbf{a.}]$\frac{-1}{2 \sqrt{R^{2}-1}}$
		\task[\textbf{b.}]$\frac{2 \pi}{\sqrt{R^{2}-1}}$
		\task[\textbf{c.}]$\frac{\pi}{\sqrt{1-R^{2}}}$
		\task[\textbf{d.}] $\frac{\pi}{\sqrt{R^{2}-1}}$
	\end{tasks}
	\item The value of integral $\int_{-\infty}^{+\infty} \frac{d x}{\left(1+x^{2}\right)^{2}}$ is given by
	\begin{tasks}(2)
		\task[\textbf{a.}]$\frac{\pi}{2}$
		\task[\textbf{b.}]$\pi$
		\task[\textbf{c.}] $i \frac{\pi}{2}$
		\task[\textbf{d.}]  $\frac{1}{4 i}$
	\end{tasks}
	\item The value of $\oint_{C} \frac{\sin 3 z}{z^{2}} d z$
	$c:|z|=\pi$ is given by
	\begin{tasks}(2)
		\task[\textbf{a.}]$6 \pi i$
		\task[\textbf{b.}]$-6 \pi i$
		\task[\textbf{c.}]0
		\task[\textbf{d.}]  3
	\end{tasks}
	\item The value of integral $\int_{0}^{2 \pi} \frac{\cos ^{2} \theta d \theta}{5-4 \cos \theta}$
	\begin{tasks}(2)
		\task[\textbf{a.}]$\frac{\pi}{12}$
		\task[\textbf{b.}]$-\frac{5 \pi}{12}$
		\task[\textbf{c.}]$\frac{5 \pi}{12}$
		\task[\textbf{d.}] $\frac{-5}{12}$
	\end{tasks}
	\item $\int_{0}^{2 \pi} \frac{\cos \theta}{13-12 \cos 2 \theta} d \theta$
	\begin{tasks}(2)
		\task[\textbf{a.}]0
		\task[\textbf{b.}]$\frac{+1}{10} \sqrt{\frac{2}{3}}$
		\task[\textbf{c.}] $\frac{-1}{10} \sqrt{\frac{2}{3}}$
		\task[\textbf{d.}]  $\frac{i}{10} \sqrt{\frac{2}{3}}$
	\end{tasks}
	\item Consider a complex function $f(z)=\frac{1}{z\left(z+\frac{1}{2}\right) \cos (z \pi)}$. Which one of the following
	statements is correct?
	\begin{tasks}(2)
		\task[\textbf{a.}] $f(z)$ has simple poles at $z=0$ and $z=-\frac{1}{2}$
		\task[\textbf{b.}]$f(z)$ has second order pole at $z=-\frac{1}{2}$
		\task[\textbf{c.}]$f(z)$ has infinite number of second order poles
		\task[\textbf{d.}] $f(z)$ has all simple poles
	\end{tasks}
	\item The value of integral
	$$
	I=\oint_{c} \frac{\sin z}{2 z-\pi} d z
	$$
	with $c$ a circle $|z|=2$, is
	\begin{tasks}(2)
		\task[\textbf{a.}]0
		\task[\textbf{b.}]$2 \pi i$
		\task[\textbf{c.}]$\pi i$
		\task[\textbf{d.}] $-\pi i$
	\end{tasks}
	\item The value of integral $I=\int_{0}^{x} \frac{(\ln x)^{2}}{x^{2}+1} d x$ is given by
	\begin{tasks}(2)
		\task[\textbf{a.}]$\frac{\pi^{3}}{2}$
		\task[\textbf{b.}]$\frac{\pi^{3}}{4}$
		\task[\textbf{c.}]$\frac{\pi^{3}}{8}$
		\task[\textbf{d.}]  $\frac{\pi^{3}}{14}$
	\end{tasks}
	\item $\int_{c} \frac{d z}{z^{2} \sinh z}$ where contour is defined as $|z-1|=2$. The value of integral is
	\begin{tasks}(2)
		\task[\textbf{a.}]$\frac{\pi i}{3}$
		\task[\textbf{b.}]$\frac{-\pi i}{3}$
		\task[\textbf{c.}]$\frac{\pi i}{6}$
		\task[\textbf{d.}] $\frac{-\pi i}{6}$
	\end{tasks}
	\item The value of the integral $\int_{0}^{\infty} \frac{\ln x^{2}}{\left(x^{2}+1\right)^{2}} d x$ is
	\begin{tasks}(2)
		\task[\textbf{a.}]0
		\task[\textbf{b.}] $\frac{-\pi}{4}$
		\task[\textbf{c.}]$\frac{-\pi}{2}$
		\task[\textbf{d.}] $\frac{\pi}{2}$
	\end{tasks}
	\item The value of the integral $\int_{C} \frac{z^{3} d z}{-2 z^{2}+10 z-12}$, where $C$ is a closed contour defined by the cquation $2|=|-5=0$, traversed in the anti-clockwise direction, is
	\begin{tasks}(2)
		\task[\textbf{a.}]$-16 \pi i$
		\task[\textbf{b.}]$16 \pi i$
		\task[\textbf{c.}]$8 \pi i$
		\task[\textbf{d.}] $2 \pi i$
	\end{tasks}
	\item The value of $\oint \frac{\cos \pi z}{z^{2}-1} d z$ around a rectangle with vertices at $2 \pm i,-2 \pm i$ is
	\begin{tasks}(2)
		\task[\textbf{a.}]$\pi$
		\task[\textbf{b.}] $-\pi$
		\task[\textbf{c.}]$2 \pi$
		\task[\textbf{d.}] 0
	\end{tasks}
	\item The principal value of the integral $\int_{-\infty}^{\infty} \frac{\sin (2 x)}{x^{3}} d x$ is
	\begin{tasks}(2)
		\task[\textbf{a.}]$-2 \pi$
		\task[\textbf{b.}]$-\pi$
		\task[\textbf{c.}] $\pi$
		\task[\textbf{d.}] $2 \pi$
	\end{tasks}
	\item The value of $\int_{0}^{2 \pi} \frac{d \theta}{5+4 \cos \theta}$ is:
	\begin{tasks}(2)
		\task[\textbf{a.}]$\frac{2 \pi}{3}$
		\task[\textbf{b.}]$\frac{\pi}{3}$
		\task[\textbf{c.}]$-\frac{\pi}{3}$
		\task[\textbf{d.}] 0
	\end{tasks}
	\item The value of integral $\int_{-\infty}^{+\infty} \frac{x}{\left(x^{2}+1\right)\left(x^{2}+4\right)} d x$ is
	\begin{tasks}(2)
		\task[\textbf{a.}]0
		\task[\textbf{b.}]1
		\task[\textbf{c.}]2
		\task[\textbf{d.}] 3
	\end{tasks}
	\item The value of integral $\int_{0}^{2 \pi} \frac{d \theta}{13-5 \sin \theta}$ is
	\begin{tasks}(2)
		\task[\textbf{a.}]$\frac{\pi}{2}$
		\task[\textbf{b.}]$\frac{\pi}{3}$
		\task[\textbf{c.}]$\frac{\pi}{4}$
		\task[\textbf{d.}] $\frac{\pi}{6}$
	\end{tasks}
	\item The value of $\int_{0}^{\infty} \frac{\ln x}{\left(x^{2}+4\right)^{2}} d x$ is
	\begin{tasks}(2)
		\task[\textbf{a.}]$\frac{\pi}{32}(\ln 2+1)$
		\task[\textbf{b.}] $\frac{\pi}{32}(\ln 2-1)$
		\task[\textbf{c.}]$\frac{\pi}{32} \ln 2$
		\task[\textbf{d.}]  $\frac{\pi}{32}$
	\end{tasks}
	\item If $0<p<1$, then the value of integral $\int_{0}^{\infty} \frac{x^{p-1}}{1+x} d x$ is
	\begin{tasks}(2)
		\task[\textbf{a.}]$\frac{\pi}{2 \sin p \pi}$
		\task[\textbf{b.}] $\frac{1}{\sin \frac{p}{2} \pi}$
		\task[\textbf{c.}]$\frac{\pi}{\sin p \pi}$
		\task[\textbf{d.}]  $\frac{\pi}{\sin \frac{p}{2} \pi}$
	\end{tasks}
	\item If $C$ is the circle given by $|z|=\pi$, and the integral is evaluated counterclockwise then the value of integral $\oint_{c} \frac{z \cosh z \pi}{z^{4}+13 z^{2}+36} d z$ is
	\begin{tasks}(2)
		\task[\textbf{a.}] $\frac{2 \pi i}{5}$
		\task[\textbf{b.}]$\frac{3 \pi i}{5}$
		\task[\textbf{c.}]$\frac{4 \pi i}{5}$
		\task[\textbf{d.}]  $\frac{6 \pi i}{5}$
	\end{tasks}
	\section{NAT}
	\item The least positive integer $n$ for which $\left(\frac{1+i}{1-i}\right)^{n}$ is real is...
	\item The value of $i^{n}+i^{n+1}+i^{n+2}+i^{n+3}$ is
	\item The value of $\frac{(\cos 3 \theta+i \sin 3 \theta)^{4}(\cos 4 \theta-i \sin 4 \theta)^{5}}{(\cos 4 \theta+i \sin 4 \theta)^{3}(\cos 5 \theta+i \sin 5 \theta)^{-4}}$ is
	\item For $n=6 k, k \in z$,\\
	$\left(\frac{1-i \sqrt{3}}{2}\right)^{n}+\left(\frac{-1-i \sqrt{3}}{2}\right)^{n}$ has the value...
	\item If $z=\frac{\sqrt{3}+i}{2}$, then $z^{69}$ equals -----------
	\section{MSQ}
	\item If $z=4+2 i$, then
	 \begin{tasks}(2)
		\task[\textbf{a.}] The magnitude of $z$ is $2 \sqrt{5}$ units
		\task[\textbf{b.}]The magnitude of $z$ is $3 \sqrt{5}$ units
		\task[\textbf{c.}] The argument or amplitude of $z$ is $\tan ^{-1} \frac{1}{3}$
		\task[\textbf{d.}]  The argument or amplitude of $z$ is $\tan ^{-1} \frac{1}{2}$
	\end{tasks}
	\item Consider a complex number $z=1-\sqrt{3} i$
	 \begin{tasks}(2)
		\task[\textbf{a.}]The magnitude of $z$ is 2 units
		\task[\textbf{b.}]The magnitude of $z$ is 4 units
		\task[\textbf{c.}]The complex number z can be represented by a point $(1,-\sqrt{3})$ in the complex or Argand plane
		\task[\textbf{d.}] The complex number z can be represented by a point $(-\sqrt{3},-1)$ in the complex plane.
	\end{tasks}
	\item If $z_{1}=2+3 i$ and $z_{2}=1+2 i$ then
	 \begin{tasks}(2)
		\task[\textbf{a.}] $z_{1}+z_{2}$ is equal to $4+6 i$
		\task[\textbf{b.}]$z_{1}-z_{2}$ is equal to $1+i$
		\task[\textbf{c.}]$z_{1} \cdot z_{2}$ is equal to $-4+7 i$
		\task[\textbf{d.}]  $\frac{z_{1}}{z_{2}}$ is equal to $\frac{8}{5}-\frac{1}{5} i$
	\end{tasks}
	\item Pick out the correct statement (S)
	 \begin{tasks}(2)
		\task[\textbf{a.}]Addition of two complex number is commutative
		\task[\textbf{b.}]For any three complex numbers $z_{1}, z_{2}$ and $z_{3}$ we have $\left(z_{1}+z_{2}\right)+z_{3}=z_{1}+\left(z_{2}+z_{3}\right)$
		\task[\textbf{c.}]For any two complex numbers $z_{1}$ and $z_{2}$ we have $z_{1} \cdot z_{2}=z_{2} \cdot z_{1}$
		\task[\textbf{d.}] Multiplication of two complex number are not commutative
	\end{tasks}
	\item Pick out the correct statement(s)
	 \begin{tasks}(2)
		\task[\textbf{a.}] Let $z$ and $\bar{z}$ denote a complex number and its conjugate respectively then, $z=\bar{z}$ implies $z$ is purely imaginary
		\task[\textbf{b.}]Let $z$ and $\bar{z}$ denote a complex number and its conjugate respectively then, $z+\bar{z}=0$ implies $z$ is purely imaginary
		\task[\textbf{c.}]In polar form the complex number $z=1+i$ can be written as $\sqrt{2}\left(\cos \frac{\pi}{4}+i \sin \frac{\pi}{4}\right)$
		\task[\textbf{d.}]  In polar form the complex number $1+i$ can be written as $\sqrt{2}\left(\cos \frac{\pi}{3}+i \sin \frac{\pi}{3}\right)$
	\end{tasks}
	
	
	
	
	
	
	
	
	
\end{enumerate}