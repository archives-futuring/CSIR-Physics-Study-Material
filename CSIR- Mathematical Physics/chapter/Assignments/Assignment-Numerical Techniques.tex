<<<<<<< Updated upstream
\chapter{Assignment-Numerical Techniques}
\begin{enumerate}
	\item From the Taylor's series for $y(x)$, find $y(0.05)$ correct to four decimal places if $y(x)$ satisfies $y^{\prime \prime}-x y^{\prime}-y=0$ with the conditions $y(0)=1$ and $y^{\prime}(0)=0$.
 \begin{tasks}(4)
	\task[\textbf{a.}]$1.0050$
	\task[\textbf{b.}]$0.0150$
	\task[\textbf{c.}] $0.0155$
	\task[\textbf{d.}]  $0.5025$
\end{tasks}	
	\item The value of $I=\int_{0}^{1} \frac{1}{1+x} d x$ using Simpson $1 / 3$ rule with $h=0.5$ is
	 \begin{tasks}(4)
		\task[\textbf{a.}]$0.69$
		\task[\textbf{b.}]$0.59$
		\task[\textbf{c.}]$0.49$
		\task[\textbf{d.}]  $0.39$
	\end{tasks}
	\item The value of the integral $\int_{0}^{8} \frac{1}{x^{2}+5} d x$, evaluated using Simpson's $\frac{1}{3}$ rule with $h=2$ is
	 \begin{tasks}(4)
		\task[\textbf{a.}]$0.59$
		\task[\textbf{b.}]$0.69$
		\task[\textbf{c.}] $0.79$
		\task[\textbf{d.}] $0.89$
	\end{tasks}
	\item Consider differential equation $y^{\prime}=-y$ with the condition $y(0)=1$. Use $-h=0.01$. The value of $y(0.02)$
	 \begin{tasks}(4)
		\task[\textbf{a.}]$0.9606$
		\task[\textbf{b.}] $0.9703$
		\task[\textbf{c.}]$0.9801$
		\task[\textbf{d.}] $0.99$
	\end{tasks}
	\item Using Euler's method solve the differential equation $y^{\prime}=-y$ with the condition $y(0)=1$ with $h=0.01$. Find $y(0.04)$
	 \begin{tasks}(4)
		\task[\textbf{a.}]$0.99$
		\task[\textbf{b.}] $0.9801$
		\task[\textbf{c.}]$0.9703$
		\task[\textbf{d.}] $0.9606$
	\end{tasks}
	\item Given the initial value problem $y^{(1)}(t)=1-t y(t)$ with $y(0)=1$, approximation of $y(1)$ by Runge-Kutta method with $\mathrm{h}=1$ would be:
	 \begin{tasks}(4)
		\task[\textbf{a.}]$1.35$
		\task[\textbf{b.}]$1.31$
		\task[\textbf{c.}]$1.32$
		\task[\textbf{d.}] $1.34$
	\end{tasks}
	\item Using the Newton's forward interpolating polynomial, $P_{5}(x)$ for the following tabular data find interpolated value of the function at $x=0.0045$.\\\\
	\begin{tabular}{|c|c|c|c|c|c|c|}
		\hline$x$ & 0 & $0.001$ & $0.002$ & $0.003$ & $0.004$ & $0.005$ \\
		\hline$y$ & $1.121$ & $1.123$ & $1.1255$ & $1.127$ & $1.128$ & $1.1285$ \\
		\hline
	\end{tabular}
	 \begin{tasks}(4)
		\task[\textbf{a.}]$-1.12840045$
		\task[\textbf{b.}]$1.12840045$
		\task[\textbf{c.}]$1.128124$
		\task[\textbf{d.}] $0.12840045$
	\end{tasks}
	\item From the Taylor's series for $y(x)$, find $y(0.1)$ correct to four decimal places if $y(x)$ satisfies\\
	$y^{\prime \prime}-x y^{\prime}-y=0$ with the conditions $y(0)=1$ and $y^{\prime}(0)=0$.
	 \begin{tasks}(4)
		\task[\textbf{a.}]$1.0050$
		\task[\textbf{b.}]$1.0150$
		\task[\textbf{c.}]$1.0155$
		\task[\textbf{d.}]$1.0250$
	\end{tasks}
	\item Compute the value of $\frac{2}{\sqrt{\pi}} \int_{0}^{x} e^{-x^{2}} d x$ when $x=0 \cdot 6538$ using Gauss's forward formula also taking origin at $0.65, h=0.01$ and $x=0.6538$, Use the given table\\
	\begin{tabular}{|l|l|}
		\hline \multicolumn{1}{|c|}{$x$} & \multicolumn{1}{c|}{$y$} \\
		\hline $0.62$ & $0.6194114$ \\
		\hline $0.63$ & $0.6270463$ \\
		\hline $0.64$ & $0.6345857$ \\
		\hline $0.65$ & $0.6420292$ \\
		\hline $0.66$ & $0.6493765$ \\
		\hline $0.67$ & $0.6566275$ \\
		\hline $0.68$ & $0.6637820$ \\
		\hline
	\end{tabular}
	 \begin{tasks}(4)
		\task[\textbf{a.}]$0.6448325$
		\task[\textbf{b.}]$0.7448325$
		\task[\textbf{c.}]$0.8448325$
		\task[\textbf{d.}]$0.5448325$
	\end{tasks}
	
	

=======
\chapter{Numerical Techniques}
\begin{enumerate}
	\item From the Taylor's series for $y(x)$, find $y(0.05)$ correct to four decimal places if $y(x)$ satisfies $y^{\prime \prime}-x y^{\prime}-y=0$ with the conditions $y(0)=1$ and $y^{\prime}(0)=0$.
	 \begin{tasks}(2)
		\task[\textbf{a.}]$1.0050$
		\task[\textbf{b.}]$0.0150$
		\task[\textbf{c.}]$0.0155$
		\task[\textbf{d.}]  $0.5025$
	\end{tasks}
	\item The value of $I=\int_{0}^{1} \frac{1}{1+x} d x$ using Simpson $1 / 3$ rule with $h=0.5$ is
	 \begin{tasks}(2)
		\task[\textbf{a.}]$0.69$
		\task[\textbf{b.}]$0.59$
		\task[\textbf{c.}]$0.49$
		\task[\textbf{d.}] $0.39$
	\end{tasks}
	
	
	
	
	
	
	
	
	
	
	
	
	
	
	
	
	
	
	
	
	
>>>>>>> Stashed changes
	
	
	
	
	
	
	
	
	
	
	
	
	
	
\end{enumerate}