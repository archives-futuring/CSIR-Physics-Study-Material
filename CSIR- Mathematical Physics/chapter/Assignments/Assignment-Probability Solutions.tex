\chapter{Assignment-Probability solutions}
\begin{enumerate}
	\item $\left. \right. $
	\begin{answer}
		\begin{align*}
		\text{Total number of ways }&=4 \times 4 \times 4\\
		\text{Number of preferred outcome }&=4 \times 3 \times 3
		\intertext{( $\because$ Any four option in step- 1 and only 3 option in step $2 \& 3$ because he can not go to previous position)}
		\text{Probability }&=\frac{4 \times 3 \times 3}{4 \times 4 \times 4}=\frac{9}{16}
		\end{align*}
		So the correct answer is \textbf{Option (a)}
	\end{answer}
	\item $\left. \right. $
	\begin{answer}
		\begin{align*}
		\intertext{After transferring balls from basket I into II, there can be either (a) 5 red and 3 blue or (b) 4 red and 4 blue balls in the second basket.}
		\text{Probability that the ball transferred is blue }&=\frac{4}{7}\\
		\text{Probability that the ball transferred is red }&=\frac{3}{7}
		\intertext{Hence probability of drawing a blue ball from basket II}
		=\frac{4}{7} \times \frac{4}{8}+\frac{3}{7} \times \frac{3}{8}&=\frac{25}{56}
		\end{align*}
		So the correct answer is \textbf{Option (d)}
	\end{answer}
	\item $\left. \right. $
	\begin{answer}
		\begin{align*}
		\because \text{ Since, }f(x)&=x e^{-x / \lambda},\text{ therefore}\\
		\langle x\rangle&=\frac{\int_{-\infty}^{\infty} x f(x) d x}{\int_{-\infty}^{\infty} f(x) d x}=\frac{\int_{0}^{\infty} x \cdot x e^{-\frac{x}{\lambda}} d x}{\int_{0}^{\infty} x e^{-\frac{x}{\lambda}} d x} \Rightarrow \frac{\int_{0}^{\infty} x^{2} e^{\frac{-x}{\lambda}} d x}{\int_{0}^{\infty} x e^{\frac{-x}{\lambda}} d x}=2 \lambda
		\end{align*}
		So the correct answer is \textbf{Option (b)}
	\end{answer}
		\item $\left. \right. $
	\begin{answer}
		\begin{align*}
		\color{red}{??}\\
		N&=30, m=10
		\end{align*}
			So the correct answer is \textbf{Option (a)}
	\end{answer}
	\item $\left. \right. $
	\begin{answer}
		\begin{align*}
		\intertext{Case I: Four steps in $x$-axis}
		\text{Two steps in }&+x\text{ and two steps in }-x\\
		\therefore\text{ probability }&=^{4} c_{2} \times\left(\frac{1}{4}\right)^{2}\left(\frac{1}{4}\right)^{2}=\frac{6}{4^{4}}
	\intertext{	Case II: Four steps in $y$ - axis}
		\text{Two steps in }&+y\text{ and two steps in }-y\\
		\therefore\text{ probability }&={ }^{4} c_{1} \times{ }^{2} c_{1}\left(\frac{1}{4}\right)^{2}\left(\frac{1}{4}\right)^{2}=\frac{6}{4^{4}}\\
		\text{Case 3: Two steps in }&+x\text{ and two steps in $-y$ axis}\\
		\Rightarrow\text{ One step in each }&+x,-x,+y \&-y\\
		\therefore \text { probability }&={ }^{2} c_{1} \times{ }^{2} c_{2}\left(\frac{1}{4}\right)^{4}=\frac{4}{4^{4}}\\
		\therefore\text{ probability }&={ }^{2} c_{1} \times{ }^{2} c_{2}\\ \therefore\text{ Answer }&=\frac{16}{4^{4}}=\frac{1}{16}
		\end{align*}
		So the correct answer is \textbf{Option (c)}
	\end{answer}
		\item $\left. \right. $
	\begin{answer}
		\begin{align*}
		\intertext{The probability that the random variable takes the value between 2 and 4 is}
		\int_{2}^{4} f(x) d x&=\int_{2}^{4} k x^{2} d x=\frac{k}{3}\left[x^{3}\right]_{2}^{4}=\frac{k}{3} \times 56=\frac{56 k}{3}\\
		\text{since }\int_{-\infty}^{\infty} f(x) d x&=1 \Rightarrow \int_{1}^{5} k x^{2} d x=1\\
		\text{Thus }\frac{k}{3}\left[x^{3}\right]_{1}^{5}&=1 \Rightarrow \frac{124 k}{3}=1 \Rightarrow k=\frac{3}{124}\\
	\text{	Hence required probability }&=\frac{56}{3} \times \frac{3}{124}=\frac{14}{31}
		\end{align*}
		So the correct answer is \textbf{Option (b)}
	\end{answer}
	\item $\left. \right. $
	\begin{answer}
		\begin{align*}
		\int_{-1}^{1} f(x) d x&=1 \Rightarrow c=\frac{3}{2}\\
		E(x)&=\int_{-1}^{1} x f(x) d x=0 \quad E\left(x^{2}\right)=\int_{-1}^{1} x^{2} f(x) d x=\frac{3}{5}\\
		\operatorname{var}(x)&=E\left(x^{2}\right)-[E(x)]^{2}=\frac{3}{5}\\
		\end{align*}
		So the correct answer is \textbf{Option (a)}
	\end{answer}
		\item $\left. \right. $
	\begin{answer}
		\begin{align*}
		\intertext{Experiment I: The probability of getting a doublet in a single throw $=\frac{6}{36}=\frac{1}{6}$}
		\text{probability of not getting }&\text{a doublet in a single throw } =1-\frac{1}{6}=\frac{5}{6}.\\
		\text{Hence, the probability of getting }&\text{a doublet on 5 throws }=6 c_{5}\left(\frac{1}{6}\right)^{5} \times \frac{5}{6}\\
		&=6\left(\frac{1}{6}\right)^{5}\left(\frac{5}{6}\right)=\frac{5}{6^{5}}
		\intertext{Experiment II: The probability of getting a head in a single throw $=\frac{1}{2}$}
		\text{The probability of not getting a head }&=1-\frac{1}{2}=\frac{1}{2}
		\intertext{Hence, probability of getting 4 heads in six tosses of a coin}
		&={ }^{6} c_{4}\left(\frac{1}{2}\right)^{4}\left(\frac{1}{2}\right)^{2}=\frac{15}{2^{6}}\\
		P_{1}&=\frac{5}{6^{5}}, P_{2}=\frac{15}{2^{6}}\\
		\therefore \frac{P_{1}}{P_{2}}&=\frac{5}{2^{5} \times 3^{5}} \times \frac{2^{6}}{5 \times 3}=\frac{2}{3^{6}}
		\end{align*}
		So the correct answer is \textbf{Option (d)}
	\end{answer}
	
	
	
	
	
	
	
	
	
	
	
	
	
	
	
	
	
	
	
	
	
	
	
	
	
	
\end{enumerate}