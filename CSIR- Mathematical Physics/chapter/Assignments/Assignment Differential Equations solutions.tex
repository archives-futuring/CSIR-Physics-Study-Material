\chapter{Assignment- Differential Equations}
\section{MCQ}
\begin{enumerate}
	\item $\left. \right. $
	\begin{answer}
		\begin{align*}
			\int \frac{y}{\sqrt{1+y^{2}}} d y+\int \frac{x}{\sqrt{1+x^{2}}} d x=c \Rightarrow \sqrt{1+y^{2}}+\sqrt{1+x^{2}}=c
		\end{align*}
		So the correct answer is \textbf{Option (c)}
	\end{answer}
	\item $\left. \right. $
	\begin{answer}
		\begin{align*}
		x d y&=-\cot y d x \Rightarrow \tan y d y=-\frac{d x}{x}\\
		\Rightarrow \int \tan y d y&=-\int \frac{d x}{x}+\log A \Rightarrow \log \sec y=-\log x+\log A \Rightarrow \log \sec y=\log \frac{A}{x}\\
		\Rightarrow x&=A \cos y \Rightarrow \sqrt{2}=A \cos \frac{\pi}{4} \Rightarrow A=2 \Rightarrow x=2 \cos y
		\end{align*}
		So the correct answer is \textbf{Option (a)}
	\end{answer}
	\item $\left. \right. $
	\begin{answer}
		\begin{align*}
		\frac{d y}{d x}&=-\frac{x}{y+1} \Rightarrow x d x+y d y+d y=0 \Rightarrow \frac{x^{2}}{2}+\frac{y^{2}}{2}+y=C_{1} \Rightarrow x^{2}+y^{2}+2 y=2 C_{1}\\
		\Rightarrow(x-0)^{2}+(y+1)^{2}&=2 C_{1}+1=C
		\intertext{which is family of circles with different radii.}
		\end{align*}
		So the correct answer is \textbf{Option (a)}
	\end{answer}
	\item $\left. \right. $
\begin{answer}
	\begin{align*}
	x y \frac{d y}{d x}&=3 y^{2}+x^{2} \Rightarrow \frac{d y}{d x}=\frac{3 y^{2}+x^{2}}{x y}\text{ which is a homogeneous equation.}\\
	\text{Put }y&=v x \Rightarrow \frac{d y}{d x}=v+x \frac{d v}{d x} \Rightarrow v+x \frac{d v}{d x}=\frac{3 v^{2} x^{2}+x^{2}}{v x^{2}}=\frac{3 v^{2}+1}{v}\\
	\Rightarrow x \frac{d v}{d x}&=\frac{3 v^{2}+1}{v}-v=\frac{2 v^{2}+1}{v} \Rightarrow \frac{v}{2 v^{2}+1} d v=\frac{d x}{x}\\
	\Rightarrow \frac{1}{4} \ln \left(2 v^{2}+1\right)&=\ln x+\ln c^{\prime} \Rightarrow \ln \left(2 v^{2}+1\right)=\ln x^{4}+\ln c \Rightarrow\left(2 v^{2}+1\right)=c x^{4}\\
	\Rightarrow\left(2 \frac{y^{2}}{x^{2}}+1\right)&=c x^{4}\\
	y=2\text{ when }x&=1 \Rightarrow\left(2 \frac{4}{1}+1\right)=c .1 \Rightarrow c=9 \Rightarrow 2 y^{2}+x^{2}=9 x^{6}
	\end{align*}
		So the correct answer is \textbf{Option (a)}
\end{answer}
	\item $\left. \right. $
\begin{answer}
	\begin{align*}
	\frac{d y}{d x}=\frac{x+2 y-1}{x+2 y+1}\text{. Let }x+2 y&=z \Rightarrow 1+2 \frac{d y}{d x}=\frac{d z}{d x} \Rightarrow \frac{d z}{d x}=\frac{3 z-1}{z+1}\\
	\Rightarrow \int\left(\frac{1}{3}+\frac{1}{4}-\frac{1}{3 z-1}\right) d z&=\int d x+c \Rightarrow \frac{z}{3}+\frac{4}{9} \log (3 z-1)=x+c\\
	\Rightarrow 3 x-3 y+a&=2 \log (3 x+6 y-1)
	\end{align*}
		So the correct answer is \textbf{Option (b)}
\end{answer}
	\item $\left. \right. $
\begin{answer}
	\begin{align*}
	\text{I.F. }=e^{\int 3 d x}&=e^{3 x} \Rightarrow y \times e^{3 x}=\int e^{2 x} \times e^{3 x} d x+c \Rightarrow y \times e^{3 x}=\frac{e^{5 x}}{5}+c \Rightarrow y=c_{1} e^{2 x}+c_{2} e^{-3 x}
	\end{align*}
		So the correct answer is \textbf{Option (c)}
\end{answer}
\item $\left. \right. $
\begin{answer}
	\begin{align*}
	\frac{d y}{d x}+\frac{1}{x} y&=x^{3} \Rightarrow \mathrm{I} . F .=e^{\int \frac{1}{x} d x}=e^{\ln x}=x\\
	\Rightarrow y \times x&=\int x^{3} \times x d x+c \Rightarrow y \times x=\frac{x^{5}}{5}+c \Rightarrow y=\frac{x^{4}}{5}+\frac{c}{x}\\
\text{	Since }y&=1\text{ at }x=1 \Rightarrow 1=\frac{1}{5}+\frac{c}{1} \Rightarrow c=\frac{4}{5} \Rightarrow y=\frac{x^{4}}{5}+\frac{4}{5 x}
	\end{align*}
		So the correct answer is \textbf{Option (d)}
\end{answer}
	\item $\left. \right. $
\begin{answer}
	\begin{align*}
	\frac{d x}{d t}+\frac{K_{2}}{K_{1}} x&=\frac{K_{3}}{K_{1}} \Rightarrow \frac{d x}{d t}+A x=B \Rightarrow I . F .=e^{\int A d t}=e^{A t} \Rightarrow x \times e^{A t}=\int\left(B \times e^{A t}\right) d t+c\\
	\Rightarrow x \times e^{A t}&=B \times \frac{e^{A t}}{A}+c \Rightarrow x=c_{1}+c e^{-A t}\\
	\text{Since }x&=0\text{ at }t=0 \Rightarrow c_{1}+c=0 \Rightarrow c_{1}=-c \Rightarrow x=c\left(1-e^{-A t}\right)
	\end{align*}
		So the correct answer is \textbf{Option (a)}
\end{answer}
	\item $\left. \right. $
\begin{answer}
	\begin{align*}
	\text{Divide by }y t \Rightarrow \frac{1}{y} \frac{d y}{d t}+\frac{1}{t} \log y&=e^{t}\text{, put }\log y=z \Rightarrow \frac{1}{y} \frac{d y}{d t}=\frac{d z}{d t}\\
	\Rightarrow \frac{d z}{d t}+\frac{1}{t} \cdot z&=e^{t} \Rightarrow I \cdot F=e^{\int_{t}^{1} d t}=e^{\log t}=t\\
	\Rightarrow z t&=\int t e^{t} d t+c \Rightarrow t \log y=t e^{t}-e^{t}+c
	\end{align*}
		So the correct answer is \textbf{Option (d)}
\end{answer}
	\item $\left. \right. $
\begin{answer}
	\begin{align*}
	\left(1+y^{2}\right) d x&=\left(\tan ^{-1} y-x\right) d y \Rightarrow \frac{d x}{d y}=\frac{\tan ^{-1} y-x}{1+y^{2}}\\
	\Rightarrow \frac{d x}{d y}+\frac{x}{1+y^{2}}&=\frac{\tan ^{-1} y}{1+y^{2}}.\text{ This is a linear differential equation.}\\
	I . F .&=e^{\int \frac{1}{1+y^{2}} d y}=e^{\tan ^{-1} y}\\
	\text{	Its solution is }x . e^{\tan ^{-1} y}&=\int e^{\tan ^{-1} y} \frac{\tan ^{-1} y}{1+y^{2}} d y+c\\
	\text{Put }\tan ^{-1} y&=t\text{ on R.H.S. so that }\frac{1}{1+y^{2}} d y=d t\\
	x . e^{\tan ^{-1} y}&=\int e^{t} t d t+C=t . e^{t}-e^{t}+C=e^{\tan ^{-1} y}\left(\tan ^{-1} y-1\right)+C\\
	\Rightarrow x&=\left(\tan ^{-1} y-1\right)+c e^{-\tan ^{-1} y}
	\end{align*}
		So the correct answer is \textbf{Option (d)}
\end{answer}
	\item $\left. \right. $
\begin{answer}
	\begin{align*}
	y \log y \frac{d x}{d y}+x-\log y&=0 \Rightarrow \frac{d x}{d y}+\frac{x}{y \log y}=\frac{1}{y}\\
	I . F .&=e^{\int \frac{1}{y \log y} d y}=e^{\log (\log y)}=\log y\\
	\text{Its solution is }x \cdot \log y&=\int \frac{1}{y}(\log y) d y+c \Rightarrow x \cdot \log y=\frac{1}{2}(\log y)^{2}+c
	\end{align*}
		So the correct answer is \textbf{Option (a)}
\end{answer}
\item $\left. \right. $
\begin{answer}
	\begin{align*}
	M&=\left(e^{y}+2\right) \sin x, N=-e^{y} \cos x \Rightarrow \frac{\partial M}{\partial y}=e^{y} \sin x, \frac{\partial N}{\partial x}=e^{y} \sin x\\
	\Rightarrow \frac{\partial M}{\partial y}&=\frac{\partial N}{\partial x} \Rightarrow \int\left(e^{y}+2\right) \sin x d x+0=c^{\prime} \Rightarrow\left(e^{y}+2\right) \cos x=-c^{\prime}=c
	\end{align*}
\end{answer}
\item $\left. \right. $
\begin{answer}
	\begin{align*}
	M&=\left(y^{4}+2 y\right), N=\left(x y^{3}+2 y^{4}-4 x\right)\\
	\Rightarrow \frac{\partial M}{\partial y}&=4 y^{3}+2, \frac{\partial N}{\partial x}=y^{3}-4 \Rightarrow \frac{\frac{\partial N}{\partial x}-\frac{\partial M}{\partial y}}{M}=-\frac{3}{y}=f(y)\\
\text{	then I.F. }&=e^{\int \frac{3}{y} d y}=e^{-3 \log y}=\frac{1}{y^{3}}\\
\text{Multiplying by }&\frac{1}{y^{3}}\text{ we get }\frac{1}{y^{3}}\left(y^{4}+2 y\right) d x+\frac{1}{y^{3}}\left(x y^{3}+2 y^{4}-4 x\right) d y=0\text{ which is an exact}
	\end{align*}
	So the correct answer is \textbf{Option (c)}
\end{answer}
\item $\left. \right. $
\begin{answer}
	\begin{align*}
	\left(D^{2}-3 D+2\right)&=0 \Rightarrow D=1,2 \Rightarrow C . F .=c_{1} e^{t}+c_{2} e^{2 t}\\
	P . I .&=\frac{1}{\left(D^{2}-3 D+2\right)} e^{3 t}=\frac{1}{\left(3^{2}-3 \times 3+2\right)} e^{3 t}=\frac{1}{2} e^{3 t}
	\end{align*}
		So the correct answer is \textbf{Option (a)}
\end{answer}
\item $\left. \right. $
\begin{answer}
	\begin{align*}
	\left(D^{2}+2 D+101\right)&=0 \Rightarrow D=\frac{-2 \pm \sqrt{4-4 \times 101}}{2}=\frac{-2 \pm 20 i}{2}\\&=-1 \pm 10 iC.F. =e^{-x}(A \cos 10 x+B \sin 10 x)\\
	P.I. &=\frac{1}{\left(D^{2}+2 D+101\right)} 10.4 e^{x}=\frac{1}{(1+2 \times 1+101)} 10.4 e^{x}=\frac{10.4}{104} e^{x}=0.1 e^{x}\\
	\Rightarrow y&=e^{-x}(A \cos 10 x+B \sin 10 x)+0.1 e^{x}\\
	\because y(0)&=1.1 \Rightarrow 0=1(A+0)+0.1=1.1 \Rightarrow A=1 .\\
	\Rightarrow y&=e^{-x}(\cos 10 x+B \sin 10 x)+0.1 e^{x}\\
	\Rightarrow \frac{d y}{d x}&=e^{-x}(-10 \sin 10 x+10 B \cos 10 x)-e^{-x}(\cos 10 x+B \sin 10 x)+0.1 e^{x}\\
	\left.\because \frac{d y}{d x}\right|_{x=0}&=-0.9 \Rightarrow-0.9=1(-0+10 B)-1(1+0)+0.1 \Rightarrow 10 B=1-0.1-0.9 \Rightarrow B=0 .\\
	\Rightarrow y&=e^{-x} \cos 10 x+0.1 e^{x}
	\end{align*}
		So the correct answer is \textbf{Option (a)}
\end{answer}
\item $\left. \right. $
\begin{answer}
	\begin{align*}
	\left(D^{2}+4\right)=0 \Rightarrow D=+2 i,-2 i \Rightarrow C . F .=A \cos 2 t+B \sin 2 t
	\end{align*}
	So the correct answer is \textbf{Option (c)}
\end{answer}
\item $\left. \right. $
\begin{answer}
	\begin{align*}
	\left(D^{2}-1\right)&=0 \Rightarrow D=+1,-1 \Rightarrow C . F .=c_{1} e^{t}+c_{2} e^{-t}\\
	P.I. &=\frac{1}{\left(D^{2}-1\right)} 2\left(\frac{e^{t}+e^{-t}}{2}\right)=\frac{1}{\left(D^{2}-1\right)}\left(e^{t}+e^{-t}\right)=t \frac{1}{2 D}\left(e^{t}+e^{-t}\right)=\frac{t}{2}\left(e^{t}-e^{-t}\right)\\
	\Rightarrow y(t)&=c_{1} e^{t}+c_{2} e^{-t}+\frac{t}{2}\left(e^{t}-e^{-t}\right)\\
	y(0)&=0 \Rightarrow c_{1}+c_{2}=0 \\
	\frac{d y}{d t}&=c_{1} e^{t}-c_{2} e^{-t}+\frac{1}{2}\left(e^{t}-e^{-t}\right)+\frac{t}{2}\left(e^{t}+e^{-t}\right),\left.\frac{d y}{d t}\right|_{t=0}=0 \Rightarrow c_{1}-c_{2}=0\\
\text{	Thus }c_{1}&=c_{2}=0 \Rightarrow y(t)=\frac{t}{2}\left(e^{t}-e^{-t}\right)=t \sinh (t)
	\end{align*}
	So the correct answer is \textbf{Option (d)}
\end{answer}
\item $\left. \right. $
\begin{answer}
	\begin{align*}
	\left(D^{2}-1\right)&=0 \Rightarrow D=+1,-1 \Rightarrow C . F .=C_{1} e^{t}+C_{2} e^{-t}\\
	P.I. &=\frac{1}{\left(D^{2}-1\right)} 2\left(\frac{e^{t}-e^{-t}}{2}\right)=\frac{1}{\left(D^{2}-1\right)}\left(e^{t}-e^{-t}\right)=t \frac{1}{2 D}\left(e^{t}-e^{-t}\right)=\frac{t}{2}\left(e^{t}+e^{-t}\right)\\
	\Rightarrow y(t)&=c_{1} e^{t}+c_{2} e^{-t}+\frac{t}{2}\left(e^{t}+e^{-t}\right)\\
	y(0)&=0 \Rightarrow c_{1}+c_{2}=0\\
	\frac{d y}{d t}&=c_{1} e^{t}-c_{2} e^{-t}+\frac{1}{2}\left(e^{t}+e^{-t}\right)+\left.\frac{t}{2}\left(e^{t}-e^{-t}\right) \Rightarrow \frac{d y}{d t}\right|_{t=0}=0 \Rightarrow c_{1}-c_{2}=0\\
\text{	Thus }c_{1}&=c_{2}=0 \Rightarrow y(t)=\frac{t}{2}\left(e^{t}+e^{-t}\right)=t \cosh (t)
	\end{align*}
		So the correct answer is \textbf{Option (c)}
\end{answer}
\end{enumerate}
\section{NAT}
\begin{enumerate}
	\item $\left. \right. $
	\begin{answer}
		\begin{align*}
		\frac{d x}{d t}&=x^{2} \Rightarrow \int \frac{d x}{x^{2}}=\int d t \Rightarrow \frac{x^{-2+1}}{-2+1}=t+C \Rightarrow \frac{-1}{x}=t+C\\
		\Rightarrow x(0)&=1 \Rightarrow \frac{-1}{1}=0+C \Rightarrow C=-1 \Rightarrow \frac{-1}{x}=t-1 \Rightarrow x=\frac{1}{1-t}\text{ as }t \rightarrow 1, x\text{ blows up.}\\
		\end{align*}
			So the correct answer is 1
	\end{answer}
		\item $\left. \right. $
	\begin{answer}
		\begin{align*}
		M&=\left(x y^{2}+\lambda x^{2} y\right), N=(x+y) x^{2} d y \Rightarrow \frac{\partial M}{\partial y}=2 x y+\lambda x^{2}, \frac{\partial N}{\partial x}=3 x^{2}+2 x y\\
		\Rightarrow \frac{\partial M}{\partial y}&=\frac{\partial N}{\partial x} \Rightarrow \lambda=3
		\end{align*}
			So the correct answer is 3
	\end{answer}
		\item $\left. \right. $
	\begin{answer}
		\begin{align*}
			P.I. &=\frac{1}{D^{2}+D+1} \cdot \cos 2 x=\frac{1}{-2^{2}+D+1} \cdot \cos 2 x=\frac{1}{D-3} \cdot \cos 2 x\\
			\Rightarrow P . I .&=\frac{D+3}{D^{2}-9} \cdot \cos 2 x=\frac{D+3}{-2^{2}-9} \cdot \cos 2 x=\frac{1}{13}(2 \sin 2 x-3 \cos 2 x)
		\end{align*}
		So the correct answer is 13
	\end{answer}
		\item $\left. \right. $
\begin{answer}
	\begin{align*}
	P \cdot I .&=\frac{1}{D^{2}+2 D+2} \sin x=\frac{1}{-1+2 D+2} \sin x=\frac{2 D-1}{4 D^{2}-1} \sin x=-\frac{1}{5}(2 D-1) \sin x\\
	\Rightarrow P \cdot I .&=-\frac{1}{5}(2 \cos x-\sin x)=\frac{1}{5}(\sin x-2 \cos x)
	\end{align*}
		So the correct answer is 2
\end{answer}
		\item $\left. \right. $
		\begin{answer}
			\begin{align*}
			P \cdot I \cdot&=\frac{1}{D^{2}-5 D+6} e^{t} \cos 2 t=e^{t} \frac{1}{(D+1)^{2}-5(D+1)+6} \cos 2 t\\
			P \cdot I \cdot&=e^{t} \frac{1}{D^{2}-3 D+2} \cos 2 t=e^{t} \frac{1}{-4-3 D+2} \cos 2 t\\
			P \cdot I \cdot&=-e^{t} \frac{1}{3 D+2} \cos 2 t=-e^{t} \frac{3 D-2}{9 D^{2}-4} \cos 2 t\\
			P \cdot I \cdot&=-e^{t} \frac{3 D-2}{9 \times-4-4} \cos 2 t=\frac{e^{t}}{40}(3 D-2) \cos 2 t=-\frac{e^{t}}{20}(3 \sin 2 t+\cos 2 t)
			\end{align*}
				So the correct answer is 3
		\end{answer}
		\item $\left. \right. $
\begin{answer}
	\begin{align*}
	P . I .&=\frac{1}{D^{2}-4 D+4} \cdot x^{3} e^{2 x}=e^{2 x} \frac{1}{(D+2)^{2}-4(D+2)+4} \cdot x^{3}\\
	\Rightarrow P . I .&=e^{2 x} \frac{1}{D^{2}} \cdot x^{3}=e^{2 x} \frac{1}{D}\left(\frac{x^{4}}{4}\right)=e^{2 x} \frac{x^{5}}{20}
	\end{align*}
		So the correct answer is 5
\end{answer}
		\item $\left. \right. $
	\begin{answer}
		\begin{align*}
		 P.I. &=\left[D^{2}+5 D+4\right]^{-1}(3-2 x)=\frac{1}{4}\left[1+\frac{5}{4} D+\frac{5}{4} D^{2}\right]^{-1}(3-2 x)\\
		\Rightarrow P.I. &=\frac{1}{4}\left[1-\frac{5}{4} D-\frac{5}{4} D^{2}\right](3-2 x)=\frac{1}{4}\left[3-2 x-\frac{5}{4} \times-2\right]=\frac{1}{8}[11-4 x]
		\end{align*}
		So the correct answer is 11
	\end{answer}
		\item $\left. \right. $
	\begin{answer}
		\begin{align*}
		\left(D^{2}+1\right)&=0 \Rightarrow D=\pm i \Rightarrow y=c_{1} e^{i x}+c_{2} e^{-i x}=A \cos x+B \sin x\\
		\because y(0)&=1 \Rightarrow 1=A \times 1+B \times 0 \Rightarrow A=1 \Rightarrow \dot{y}=-A \sin x+B \cos x \\
		\Rightarrow \dot{y}(0)&=1 \Rightarrow 1=-A \times 0+B \times 1 \Rightarrow B=1 \Rightarrow y=\cos x+\sin x\\
		\text{For maxima, }y^{\prime}&=-\sin x+\cos x=0 \Rightarrow \sin x=\cos x \Rightarrow x=45^{\circ}\\
		y^{\prime \prime}&=-\cos x-\sin x, \quad y^{\prime \prime}<0 \text { for } x=45^{\circ} \\
		\Rightarrow y(\max )&=\cos 45^{\circ}+\sin 45^{\circ}=\frac{1}{\sqrt{2}}+\frac{1}{\sqrt{2}}=\frac{2}{\sqrt{2}}=\sqrt{2}
		\end{align*}
			So the correct answer is 1.41
	\end{answer}
		\item $\left. \right. $
		\begin{answer}
			\begin{align*}
				\left(D^{2}+2 \alpha D+1\right)=0 \Rightarrow m_{1}, m_{1}=\frac{-2 \alpha \pm \sqrt{4 \alpha^{2}-4}}{2} \quad \because m_{1}=m_{1} \Rightarrow \alpha=1
			\end{align*}
			So the correct answer is 1
		\end{answer}
\item $\left. \right. $
	\begin{answer}
		\begin{align*}
		\left(D^{2}+3 D+2\right)&=0 \Rightarrow D=-1,-2 \Rightarrow x(t)=C_{1} e^{-t}+C_{2} e^{-2 t}\\
		 \Rightarrow x(1)&=\frac{10}{e}=C_{1} e^{-1}+C_{2} e^{-2} \Rightarrow C_{1}+C_{2} e^{-1}=10\text{ and }C_{1}+C_{2}=20\\
		\Rightarrow C_{1}&=\frac{10 e-20}{e-1} ; C_{2}=\frac{10 e}{e-1} \\
		x(2)&=\left(\frac{10 e-20}{e-1}\right) e^{-2}+\left(\frac{10 e}{e-1}\right) e^{-4}=0.8566
		\end{align*}
			So the correct answer is 0.8566
	\end{answer}
\end{enumerate}