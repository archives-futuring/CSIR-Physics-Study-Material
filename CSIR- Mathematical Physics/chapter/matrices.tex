% !TeX spellcheck = en_GB
\chapter{Matrices}
We  come  across  matrices often while  we  deal  with various physical systems. When we need to solve  a  Set  of  linear  equations  ,  when  we  need  to  rotate  vectors, in quantum mechanics etc. Let us find what a matrix is.
\begin{definition}
A matrix can be defned as a collection of numbers arranged in a rectangular way, in rows and columns and bounded by brackets , $[\hspace{0.2cm} ]$  or  $(\hspace{0.2cm})$. Individual numbers or functions inside a matrix  are known as elements of the matrix.\\ A matrix can also be viewed as an operator (a linear transformation) from $\mathbb{R}^{n}$ to $\mathbb{R}^{n}$.	
\end{definition}
\begin{example}
$\left[\begin{array}{ll}2 & 3 \\ 4 & 7\end{array}\right]$,$\left[\begin{array}{lll}-7 & 5 &3i\\ 4 & -7i&8\\4 & 5-i &3i \end{array}\right]$,$\left[\begin{array}{llll}x & y &z&w\\ p & q&r&s\\a & b &c&d \end{array}\right]$		
\end{example}
\textbf{Order of a matrix}
\newline A matrix of order $m\times n$ has $m$ number of rows and  $n$  number of columns.Every elment of the matrix is charecterized by a row index $i$ and a column index $j$.Then an element of $i^{th}$ row and $j^{th}$ column of a matrix can be represented as, $a_{ij}$.
\\A matrix $A$ of $m$ rows and $n$ columns is written as,
$A=[a_{ij}]_{m\times n}$

\section{Types of matrices}
\begin{itemize}
	\item \textbf{Row Matrix}\\
	If a matrix has only one row and any number of columns, it is called a Row matrix, \begin{example}
		$\left[\begin{array}{llll}
		2 & 7 & 3 & 9
	\end{array}\right]$
	\end{example}

	\item \textbf{Column Matrix}\\
	 A matrix, having one column and any number of rows, is called a Column  matrix .
	\begin{example}
		$
	\text {}\left[\begin{array}{l}
		1 \\
		2 \\
		3
	\end{array}\right]
	$
	\end{example}
	\item \textbf{Null Matrix or Zero Matrix}\\
	 Any matrix, in which all the elements are zeros, is called a Zero matrix or Null matrix .
	\begin{example}
		$
	\left[\begin{array}{llll}
		0 & 0 & 0 & 0 \\
		0 & 0 & 0 & 0
	\end{array}\right]
	$
	\end{example}
	\item \textbf{Square Matrix }\\
	A matrix, in which the number of rows is equal to the number of columns, is called a square matrix 
\begin{example}
		$\left[\begin{array}{ll}
		2 & 5 \\
		1 & 4
	\end{array}\right]$
\end{example}

	 \item \textbf{Diagonal matrix}\\
	 Matrix having elements other than the principal diagonal elements are zero
	 i.e $a_{i j}=0$ for $i \neq j$.
	\begin{example}
		 $\left[\begin{array}{ll}a & 0 \\ 0 & b\end{array}\right]$
	\end{example}
	 \item \textbf{Scalar matrix}
	 \\ A diagonal matrix in which all the diagonal elements are equal to a scalar, say $(k)$ is called a scalar matrix. i.e., $\mathrm{A}=\left[a_{i j}\right]_{n \times n}$ is a scalar matrix if $a_{i j}=\left\{\begin{array}{ll}0, & \text { when } i \neq j \\ k, & \text { when } i=j\end{array}\right.$ 
	 \begin{example}
	 	$
	 \left[\begin{array}{lll}
	 	2 & 0 & 0 \\
	 	0 & 2 & 0 \\
	 	0 & 0 & 2
	 \end{array}\right],\left[\begin{array}{rrrr}
	 	-6 & 0 & 0 & 0 \\
	 	0 & -6 & 0 & 0 \\
	 	0 & 0 & -6 & 0 \\
	 	0 & 0 & 0 & -6
	 \end{array}\right]
	 $
	
	 \end{example}
	 \item  \textbf{ldentity/unit matrix}\\
	 Diagonal matrix having all principal diagonal elements equal to one.
	 i.e. $a_{i j}=0$ for $i \neq j$ and $a_{i i}$ is equal to 1 for all $i$.
	\begin{example}
		  $\left[\begin{array}{ll}1 & 0 \\ 0 & 1\end{array}\right]$
	\end{example}
	 \item\textbf{ Triangular Matrix}\\  A square matrix, all of whose elements below the leading diagonal are zero, is called an upper triangular matrix. A square matrix, all of whose elements above the leading diagonal are zero, is called a lower triangular matrix
	
	\begin{example}
		 $\text{Upper triangular matrix}\rightarrow
	 \left[\begin{array}{lll}
	 	1 & 3 & 2 \\
	 	0 & 4 & 1 \\
	 	0 & 0 & 6
	 \end{array}\right]\\ \rightarrow\text{Lower triangular matrix}\left[\begin{array}{lll}
	 	2 & 0 & 0 \\
	 	4 & 1 & 0 \\
	 	5 & 6 & 7
	 \end{array}\right]
	 $
	\end{example}
	 \item \textbf{ Periodic matrix}\\
	  A square matrix ' $A$ ' for which $A^{k+1}=A,$ is called periodic matrix of period $k$.
	 \begin{example}
	 	 $A=\left[\begin{array}{cc}0 & -1 \\ 1 & 0\end{array}\right] \Rightarrow A^{5}=A$ i.e $A$ is periodic matrix of period 4
	 \end{example}
	 \item \textbf{ Idempotent matrix}\\
	 A square matrix ' $A$ ' for which $A^{2}=A,$ is called idempotent matrix. It is a periodic matrix of period 1 . Any idempotent matrix will be either singular matrix or non-singular unit matrix.
	\begin{example}
		 $\left[\begin{array}{ll}1 & 1 \\ 0 & 0\end{array}\right],\left[\begin{array}{ccc}2 & -2 & -4 \\ -1 & 3 & 4 \\ 1 & -2 & -3\end{array}\right]$
	\end{example}
	 \item \textbf{Nilpotent matrix}\\
	  A square matrix for which $A^{p}=0,$ is called nilpotent matrix of index ' $p$ '. The trace and determinant of the nilpotent matrix is always zero.
	\begin{example}
		 $A=\left[\begin{array}{ll}0 & 0 \\ 1 & 0\end{array}\right]$ is a nilpotent matrix of index '2'.
	\end{example}
	 \item  \textbf{Involutory Matrix}\\ A square matrix for which $A^{2}=I$, is called involutory matrix. It is a self-inverse matrix.
	 \begin{example}
	 	$\left[\begin{array}{lll}1 & 0 & 0 \\ 0 & 0 & 1 \\ 0 & 1 & 0\end{array}\right]$
	 \end{example}
	 \item \textbf{Symmetric Matrix}\\
	 A square matrix will be called symmetric, if for all values of $i$ and $j,$ $a_{i j}=a_{j i}$ i.e., $A^{\prime}=A$
	\begin{example}
		 $
	 \left[\begin{array}{lll}
	 	a & h & g \\
	 	h & b & f \\
	 	g & f & c
	 \end{array}\right]
	 $
	\end{example}
	 \item \textbf{Skew Symmetric Matrix}\\A square matrix is called skew symmetric matrix, if
	 (1) $a_{i j}=-a_{j i}$ for all values of $i$ and $j,$ or $A^{\prime}=-A$
	 (2) All diagonal elements are zero.
	\begin{example}
		 $
	 \left[\begin{array}{ccc}
	 	0 & -h & -g \\
	 	h & 0 & -f \\
	 	g & f & 0
	 \end{array}\right]
	 $
	\end{example}
	 \end{itemize}
	 \section{Elementary matrix arithmetic}
	 \subsection{Matrix addition}
	 The operation of addition of two matrices is only defined when both matrices have the same dimensions. If ${A}$ and ${B}$ are both $(m \times n)$, then the sum,
	 \\$
	 {C}={A}+{B}
	 $\\
	 the order of the sum is also $(m \times n)$ and is defined to have each element the sum of the corresponding elements of ${A}$ and ${B},$ thus
	 \\$
	 c_{i j}=a_{i j}+b_{i j}
	 $
	 \subsubsection{Properties of vector addition}
	 Only matrices of the same order can be added or subtracted.
	 \begin{itemize}
	 	\item  Commutative Law \quad  $A+B=B+A$.
	 	\item Associative law \quad $A+(B+C)=(A+B)+C$.
	 \end{itemize}
	
	 \begin{exercise}
	 	If $\begin{aligned} A &=\left[\begin{array}{rrr}4 & 2 & 5 \\ 1 & 3 & -6\end{array}\right] \text{and} & B=\left[\begin{array}{lll}1 & 0 & 2 \\ 3 & 1 & 4\end{array}\right] \end{aligned}$ find (A+B).\end{exercise}
 	\begin{answer}
 			$$\begin{aligned}
 			A+B &=\left[\begin{array}{lrl}4+1 & 2+0 & 5+2 \\ 1+3 & 3+1 & -6+4\end{array}\right]=\left[\begin{array}{lll}5 & 2 & 7 \\ 4 & 4 & -2\end{array}\right] \end{aligned}$$
 	\end{answer}
	 
	 
	 %........................................................
	 \subsection{Matrix multiplication}
	 \subsubsection{$\bullet$Scalar multiple of a matrix}
	 If a matrix is multiplied by a scalar quantity $k$, then each element o fthe matrix is multiplied by $k$.
	 \subsubsection{$\bullet$Multiplication between matrices}
	The product of two matrices $A$ and $\mathrm{B}$ is only possible if the number of columns in $A$ is equal to the number of rows in $B$.
	 
	 Let $A=\left[a_{i j}\right]$ be an $m \times n$ matrix and $B=\left[b_{i j}\right]$ be an $n \times p$ matrix. Then the product $A B$ of these matrices is an $m \times p$ matrix $C=\left[c_{i j}\right]$ where
	 $$
	 c_{i j}=a_{i 1} b_{1 j}+a_{i 2} b_{2 j}+a_{i 3} b_{3 j}+\ldots .+a_{i n} b_{n j}
	 $$
	 $$\left[\begin{array}{ll}a_{1 1} & a_{1 2} \\ a_{2 1} & a_{2 2 }\end{array}\right] \cdot\left[\begin{array}{ll} b_{1 1} &  b_{1 2} \\b_{2 1} & b_{22 }\end{array}\right]=\left[\begin{array}{ll}{a_{1}} \cdot {b_{1}} & {a_{1}} \cdot b_{2} \\ {a_{2}} \cdot {b_{1}} & {a_{2}} \cdot {b_{2}}\end{array}\right]$$\\\\
	 \textbf{Properties of matrix multiplication}
	 \begin{itemize}
	 	\item Multiplication of matrices is not commutative.
	 	$$
	 	A B \neq B A
	 	$$
	 	\item Matrix multiplication is associative, if conformability is assured.
	 	$$
	 	A(B C)=(A B) C
	 	$$
	 	\item  Matrix multiplication is distributive with respect to addition.
	 	$$
	 	A(B+C)=A B+A C
	 	$$
	 	\item Multiplication of matrix $A$ by unit matrix.
	 	$$
	 	A I=I A=A
	 	$$
	 	\item Multiplicative inverse of a matrix exists if $|\mathrm{A}| \neq 0$.
	 	$$
	 	A \cdot A^{-1}=A^{-1} \cdot A=I
	 	$$
	 	\item  If $\mathrm{A}$ is a square then $A \times A=A^{2}, A \times A \times A=\mathrm{A}^{3}$.
	 	\item  $\quad A^{0}=I$
	 	\item  $\quad I^{n}=I,$ where $n$ is positive integer.
	 \end{itemize}
 \begin{exercise}
 	$$
 	\text {  If } A=\left[\begin{array}{rrr}
 		1 & -2 & 3 \\
 		2 & 3 & -1 \\
 		-3 & 1 & 2
 	\end{array}\right] \text { and } B=\left[\begin{array}{lll}
 		1 & 0 & 2 \\
 		0 & 1 & 2 \\
 		1 & 2 & 0
 	\end{array}\right]
 	$$ \end{exercise}
 \begin{answer}
 	from the products $A B$ and $B A,$ and show that $A B \neq B A$.\\
 \\Here,
 $$
 A B=\left[\begin{array}{rrr}
 	1 & -2 & 3 \\
 	2 & 3 & -1 \\
 	-3 & 1 & 2
 \end{array}\right]\left[\begin{array}{lll}
 	1 & 0 & 2 \\
 	0 & 1 & 2 \\
 	1 & 2 & 0
 \end{array}\right]
 $$
 $$
 =\left[\begin{array}{ccc}
 	1-0+3 & 0-2+6 & 2-4+0 \\
 	2+0-1 & 0+3-2 & 4+6-0 \\
 	-3+0+2 & 0+1+4 & -6+2+0
 \end{array}\right]=\left[\begin{array}{rrr}
 	4 & 4 & -2 \\
 	1 & 1 & 10 \\
 	-1 & 5 & -4
 \end{array}\right]
 $$$$
 \begin{array}{l}
 	B A=\left[\begin{array}{lll}
 		1 & 0 & 2 \\
 		0 & 1 & 2 \\
 		1 & 2 & 0
 	\end{array}\right]\left[\begin{array}{rrr}
 		1 & -2 & 3 \\
 		2 & 3 & -1 \\
 		-3 & 1 & 2
 	\end{array}\right]\\=\left[\begin{array}{ccc}
 		1+0-6 & -2+0+2 & 3-0+4 \\
 		0+2-6 & 0+3+2 & 0-1+4 \\
 		1+4+0 & -2+6+0 & 3-2+0
 	\end{array}\right]=\left[\begin{array}{rrr}
 		-5 & 0 & 7 \\
 		-4 & 5 & 3 \\
 		5 & 4 & 1
 	\end{array}\right] 
 	
 \end{array}
 $$
 
 \end{answer}
 


\subsection{Trace of a matrix}
In linear algebra, tthe trace of a matrix is defined as the sum of the principal diagonal elements of the matrix i.e.
$$
\operatorname{Tr}({A})=a_{11}+a_{22}+a_{33}+\ldots \ldots \ldots \ldots+a_{n n}=\sum_{i=1}^{n} a_{i i}
$$
\textbf{Properties of trace}
\begin{itemize}
	\item  $\operatorname{Tr}(A+B)=\operatorname{Tr}(A)+\operatorname{Tr}(B)$
	\item $\operatorname{Tr}(c A)=c \operatorname{Tr}(A)[$ where $c$ is constant $]$
	\item $\operatorname{Tr}(A B)=\operatorname{Tr}(B A)$
	\item $\operatorname{Tr}(A B C D)=\operatorname{Tr}(B C D A)=\operatorname{Tr}(C D A B)=\operatorname{Tr}(D A B C)$
\end{itemize}
\begin{example}
	\leavevmode
	\newline
	$
	\begin{array}{l}
		A=\left[\begin{array}{rrr}
			3 & 5 & 6 \\
			7 & 8 & 9 \\
			12 & 15 & 12
		\end{array}\right] \\\\
		\operatorname{Tr}(\mathrm{A})=\mathrm{a}_{11}+\mathrm{a}_{22}+\mathrm{a}_{33}=8+3+12=23
	\end{array}
	$
\end{example}
\subsection{Transpose of a matrix}
The transpose of an arbitrary matrix $A$ is written as $A^{T}$ is obtained by interchanging corresponding rows into column of $A$ i.e. if element of $A$ is $a_{i j}$ then element of $A^{T}$ is $a_{j i}$.\\
\textbf{Properties:}
\begin{itemize}
	\item $\left(A^{T}\right)^{T}=A$
	\item $(A B)^{T}=B^{T} A^{T}$
\end{itemize}
 
\begin{example}
$
A=\left[\begin{array}{lll}
	2 & 3 & 4 \\
	1 & 0 & 5 \\
	6 & 7 & 8
\end{array}\right], A^{T}=\left[\begin{array}{lll}
	2 & 1 & 6 \\
	3 & 0 & 7 \\
	4 & 5 & 8
\end{array}\right]
$	
\end{example}
\subsection{Conjugate of a Matrix}
Conjugate of a matrix obtained by taking the complex conjugate of each element  i.e
\begin{example}
	$
	A=\left[\begin{array}{ccc}
		2+i & 2-3 i & 0 \\
		1-2 i & -i & 3-2 i
	\end{array}\right]
	 \Rightarrow{A^{\ast}}=\left[\begin{array}{ccc}
		2-i & 2+3 i & 0 \\
		1+2 i & i & 3+2 i
	\end{array}\right]
	$
\end{example}

\subsection{Transpose Conjugate of a Matrix:}
Transpose conjugate of a matrix obtained by taking the transpose of the matrix and then taking the complex conjugate of each element or vice versa i.e
$$
A^{\dagger}=\left(A^{T}\right)^{*}=\left(A^{*}\right)^{T}
$$
\begin{example}
	$
	 \quad A=\left[\begin{array}{ccc}
		2 & 4+i & 1 \\
		-i & 5 & 4-i \\
		3i & 0 & 3i
	\end{array}\right] \Rightarrow A^{\dagger}=\left[\begin{array}{ccc}
		2 & i & -3i \\
		4-i & 5 & 0 \\
		1 & 4+ i & -3i
	\end{array}\right]
	$
\end{example}
\subsection{Unitary Matrix}
 A square matrix $A$ is said to be unitary if,
 \begin{equation*}
 A^{\dagger}A=I
 \end{equation*}
Where $A^{\dagger}$ is the conjugate transpose of the matrix, $ A$ and $I$ is a unit matrix. And 
 \begin{equation*}
 A^{\dagger}=A^{-1}
 \end{equation*} 
\subsection{Symmetric matrix}
A square matrix is called symmetric if for all values of i and j, $a_{i j}=a_{j i}$, then $A^{T}=A$ 
\begin{example} $\left[\begin{array}{lll}a & b & f \\ b & c & d \\ f & d & e\end{array}\right]$	 
	\end{example}
\subsubsection{Skew-Symmetric matrix}
A square matrix is called skew-symmetric if for all values of i and j, $a_{i j}=-a_{j i}$, then $A^{T}=-A$ 
\begin{example} $\left[\begin{array}{ccc}a & -b & -f\\ b & c & -d \\ f & d & e\end{array}\right]$	 
\end{example}
\begin{note}
Every square matrix can be uniquely expressed as the sum of symmetric and a skew symmetric matrix.$$  A=\frac{1}{2}(A+A^{T})+\frac{1}{2}(A-A^{T})$$	
\end{note}
\subsection{Orthogonal matrix}
 A square matrix $A$ is called an orthogonal matrix if the product of the matrix $A$ and the transpose matrix $A^{T}$ , is an identity matrix 
\\$
\text { A. } A^{T}=I
$\hspace{1cm}if $|A|=1,$ matrix $A$ is proper.
\begin{example}
	$$\begin{aligned}
		\text{If  }A&=\left[\begin{array}{ccc}1 / 3 & 2 / 3 & -2 / 3 \\ -2 / 3 & 2 / 3 & 1 / 3 \\ 2 / 3 & 1 / 3 & 2 / 3\end{array}\right]\\
	A^{T} A&=\left[\begin{array}{ccc}1 / 3 & 2 / 3 & -2 / 3 \\ -2 / 3 & 2 / 3 & 1 / 3 \\ 2 / 3 & 1 / 3 & 2 / 3\end{array}\right]\left[\begin{array}{ccc}1 / 3 & -2 / 3 & 2 / 3 \\ 2 / 3 & 2 / 3 & 1 / 3 \\ -2 / 3 & 1 / 3 & 2 / 3\end{array}\right]&=\left[\begin{array}{ccc}1 & 0 & 0 \\ 0 & 1 & 0 \\ 0 & 0 & 1\end{array}\right]
	\end{aligned}$$
	

	\end{example}
\subsection{Hermitian Matrix}
A square matrix $A=\left(a_{i j}\right)$ is called Hermitian matrix, if every $i$ -jth element of $A$ is equal to conjugate complex $j-i$ th element of $A$. Then, $A^{\dagger}=A$
\begin{example}
	$\text{A=}\left[\begin{array}{lll}
			1 & 2+3 i & 3+i \\
			2-3 i & 2 & 1-2 i \\
			3-i & 1+2 i & 5
	\end{array}\right]$
\end{example}
\subsubsection{Skew Hermitian Matrix}
 A square matrix $A=\left(a_{i j}\right)$ will be called a Skew Hermitian matrix if every $i$ -$j^{th}$ element of $A$ is equal to negative conjugate complex of $j$ -$i^{th}$ element of $A$.
\\In other words, $\quad a_{i j}=-{a}_{j i}^{\ast}$.\\All the diagonal elements of a Skew Hermitian matrix are either zeros or pure imaginary.
\begin{example}
	$\left[\begin{array}{ccc}i & 1-3 i & 4+7 i \\ -(1+3 i) & 0 &  i \\ -(4-7i) &  i & -5 i\end{array}\right]$
\end{example}
\subsection{Singular Matrix} If the determinant of the matrix is zero, then the matrix is known as singular matrix \begin{example}
	 $A=\left[\begin{array}{ll}1 & 2 \\ 3 & 6\end{array}\right]$ is singular matrix, because $|A|=6-6=0$..
\end{example}
%............................................................................................
\subsection{Adjoint of a square matrix}
\textbf{Minor}:\\In a square matrix, each element possesses its own minor. The minor is defined as a value obtained from the determinant of a square matrix by deleting out a row and a column corresponding to the element of a matrix.
\\\\\textbf{Cofactor}:\\ The cofactor is defined the signed minor. An (i,j) cofactor is computed by multiplying
(i, j) minor by $(-1)^{i+j}$ and is denoted by $C_{i j}$. The formula to find cofactor $=C_{i j}=(-1)^{i+j} . M_{i j}$ where $M_{i j}$ denotes the minor of $i^{t h}$ row and $j^{\text {th }}$ column of a matrix.
\\\\\textbf{Adjoint:}\\
Let $A=\left[a_{i j}\right]$ be a square matrix of order $n$ and let $C_{i j}$ be the cofactor of $a_{i j}$ in $A .$ Then the transpose of the matrix of cofactors of ellements of $A$ is called the adjoint of $A$ and is denoted by adj $A$.
\\Thus, adj $A=\left[C_{i j}\right]^{T} \Rightarrow(\operatorname{adj} A)_{i j}=C_{j i}=$ cofactor of
$a_{j i}$ in
$A .$
$$
\begin{array}{l}
	\text { If } A=\left[\begin{array}{lll}
		a_{11} & a_{12} & a_{13} \\
		a_{21} & a_{22} & a_{23} \\
		a_{31} & a_{32} & a_{33}
	\end{array}\right] \\
	\text { Then } \operatorname{adj}(A)=\left[\begin{array}{lll}
		c_{11} & c_{12} & c_{13} \\
		c_{21} & c_{22} & c_{23} \\
		c_{31} & c_{32} & c_{33}
	\end{array}\right]=\left[\begin{array}{lll}
		c_{11} & c_{21} & c_{31} \\
		c_{12} & c_{22} & c_{32} \\
		c_{13} & c_{23} & c_{33}
	\end{array}\right]
\end{array}
$$
\subsection{Inverse of a matrix}
A square matrix of order $n$ is invertible if there exists a square matrix $B$ of the same order such that
$$
A B=I_{n}=B A
$$
In the above case, $B$ is called the inverse of $A$ and is denoted by $A^{-1}$ where,
$$
A^{-1}=\frac{(\operatorname{adj} A)}{|A|}
$$
\subsubsection{Properties}
\begin{itemize}
	\item  $A^{-1}$ exists only when $A$ is non-singular, i.e. $|A| \neq 0$.
	\item The inverse of a matrix is unique.
	\item Reversal laws: If $A$ and $B$ are invertible matrices of the same order, then,
	$$
	(A B)^{-1}=B^{-1} A^{-1}
	$$
	\item If $A$ is an invertible square matrix, then, $$\left(A^{T}\right)^{-1}= \left(A^{-1}\right)^{T}$$
   \item The inverse of an invertible symmetric matrix is a symmetric matrix.
	\item Let $A$ be a non-singular square matrix of order $n$. Then,
	$$
	|\operatorname{adj} A|=|A|^{n-1}
	$$
	\item If $A$ is an invertible square matrix, then, $$\operatorname{adj} A^{T}=(\operatorname{adj} A)^{T}$$
	\item If $A$ and $B$ are non-singular square matrices of the same order. then, $$\operatorname{adj}(A B)=(\operatorname{adj} B)(\operatorname{adj} A)$$
	\item If $A$ is a non-singular matrix, then,
	$$
	\left|A^{-1}\right|=|A|^{-1}, \text {i.e. }\left|A^{-1}\right|=\frac{1}{|A|}
	$$
\end{itemize}
\begin{exercise}
	If $A=\left[\begin{array}{rrr}3 & -3 & 4 \\ 2 & -3 & 4 \\ 0 & -1 & 1\end{array}\right],$ find $A^{-1}$\end{exercise}
\begin{answer}
$$
\begin{aligned}
	A&=\left[\begin{array}{rrr}3 & -3 & 4 \\ 2 & -3 & 4 \\ 0 & -1 & 1\end{array}\right]\\
	|A|&=3(-3+4)+3(2-0)+4(-2-0)=3+6-8=1\\
	\text{cofactor of A}&=\left[\begin{array}{lll}
		(-3+4) & (-2-0) & (-2-0) \\
		(3-4) & (3-0) & (3-0) \\
		(-12+12) & (-12+8) & (-9+6)
	\end{array}\right]\\&= \left[\begin{array}{rrr}1 & -2 & -2 \\ -1 & 3 & 3 \\ 0 & -4 & -3\end{array}\right]\\\text { Adj. } A&=\left[\begin{array}{rrr}1 & -1 & 0 \\ -2 & 3 & -4 \\ -2 & 3 & -3\end{array}\right]\\ A^{-1}&=\frac{1}{|A|} \text { Adj.A }\\ &=\frac{1}{1}\left[\begin{array}{rrr}1 & -1 & 0 \\ -2 & 3 & -4 \\ -2 & 3 & -3\end{array}\right]=\left[\begin{array}{rrr}1 & -1 & 0 \\ -2 & 3 & -4 \\ -2 & 3 & -3\end{array}\right]
\end{aligned}$$
\end{answer}
	

\subsubsection{Rank of a matrix}

The rank of a matrix is said to be $r$ if,
\begin{itemize}
	\item It has at least one non-zero minor of order $r$.
	\item Every minor of $A$ of order higher than $r$ is zero. 
\end{itemize}
 
 \begin{note}
 	\begin{itemize}
 		\item Non-zero row is that row in which all the elements are not zero.\\
 		\item  $\text{Rank(AB)}< \text{Rank(A) or} \text{ Rank(B)}$
 		The rank of the product matrix $A B$ of two matrices $A$ and $B$ is less than the rank of either of the matrices $A$ and $B$.\\
 		\item  Corresponding to every matrix $A$ of rank $r,$ there exist non-singular matrices $P$ and $Q$ such
 		that $P A Q=\left[\begin{array}{cc}I_{r} & 0 \\ 0 & 0\end{array}\right]$
 	\end{itemize}
 \end{note}
\section{Determinants}
The concept of determinant and the notation were introduced by the renowned German mathematician and philosopher Gottfried Wilhelm von Leibniz. A determinant can be defined as  a number associated with any \textbf{square} matrix. We'll write it as,  $\operatorname{det}A$ or $|A|$. The determinant encodes a lot of information about the matrix. 
\subsection{Properties}
\begin{enumerate}
	\item The matrix is invertible exactly when the determinant is non-zero.
	\item Determinant of identity Matrix is unity $\operatorname{det} I=1$
	\item If you exchange two rows of a matrix, you reverse the sign of its determinant from positive to negative or from negative to positive.
	i.e., If $\left|\begin{array}{ll}
	1 & 0 \\
	0 & 1
	\end{array}\right|=1$ ,\ then, $\left|\begin{array}{ll}0 & 1 \\ 1 & 0\end{array}\right|=-1$
	\item 
	\begin{itemize}
		\item If we multiply one row of a matrix by $p$, the determinant is multiplied by $p:$\\\\ $\left|\begin{array}{rr}p a & p b \\ c & d\end{array}\right|=p\left|\begin{array}{cc}a & b \\ c & d\end{array}\right|$
		\item The determinant behaves like a linear function on the rows of the
		matrix:\\\\
		$
		\left|\begin{array}{cc}
		a+a^{\prime} & b+b^{\prime} \\
		c & d
		\end{array}\right|=\left|\begin{array}{ll}
		a & b \\
		c & d
		\end{array}\right|+\left|\begin{array}{cc}
		a^{\prime} & b^{\prime} \\
		c & d
		\end{array}\right|
		$
	\end{itemize}
	\item If any two rows or columns of a determinant are identical, then the value of the
	determinant is zero.
	\item The determinant of a triangular matrix is the product of the diagonal elemets. This propery holds true for diagonal matrix also.
	\item $\operatorname{det} A=0$\\  Exactly when $A$ is singular.
	\item $\operatorname{det} A B=(\operatorname{det} A)(\operatorname{det} B)$\\
	Although the determinant of a sum does not equal the sum of the determinants, it is true that the determinant of a product equals the product of the determinants. For example:
	$$
	\operatorname{det} A^{-1}=\frac{1}{\operatorname{det} A}\quad 	(\text{Because,}\ A^{-1} A=1)
	$$
	(Note that if $A$ is singular then,\  $A^{-1}$ does not exist and $\operatorname{det} A^{-1}$ is undefined.) Also, $\operatorname{det} A^{2}=(\operatorname{det} A)^{2}$ and $\operatorname{det} 2 A=2^{n} \operatorname{det} A$
	(applying property 3 to each row of the matrix). 
	\item Determinant of Matrix and its Transpose are equal. $$\operatorname{det} A^{T}=\operatorname{det} A$$
\end{enumerate}
\section{Eigen values and eigen vectors}

 If $A=\left[a_{i j}\right]_{n \times n}$ is a square matrix of order $n,$ then the vector equation $$A X=\lambda X$$ Where $X$ is an unknown column vector and $\lambda$ is an unknown scalar value, is called an eigenvalue problem. To solve this, we need to determine the value of $X$ 's and $\lambda$ 's to satisfy the above mentioned vector.
 Take all unknowns to one side:$$(A-\lambda I) X=0$$
 Where $I$ is a unit matrix with the same dimensions as $A$.
 (Note that $A X-\lambda X=0$ does not simplify to $(A-\lambda) X=0$ as you cannot subtract a scalar $\lambda$ from a matrix $A$ ).
 For this system non-trivial solutions will only exist if the determinant of the coefficient matrix is zero.
 $$
 \operatorname{det}(A-\lambda I)=0
 $$
 \begin{itemize}
 	\item \textbf{Characteristic Polynomial}:\\ The determinant $|A-\lambda I|$ when expanded will give a polynomial, which we call as characteristic polynomial of matrix $A$.With degree being the same order of $A$.
 	\item \textbf{Characteristic Equation:}\\ The equation $|A-\lambda I|=0$ is called the characteristic equation of the matrix $A$.
 	\item \textbf{Characteristic Roots or Eigen Values:}\\ The roots of characteristic equation $|A-\lambda I|=0$ are called characteristic roots of matrix .
 \end{itemize}
For every $\lambda$ there corresponds $X\neq 0$ which satisfies the equation $A X=\lambda X$
Then $X$ is said to be Eigen vector corresponding to Eigen value $\lambda$ of matrix $A$.


\section{Eigenvalues}
The eigenvalues of a square matrix A are the roots of the characteristic equation
 of A.\\
Hence an n $\times$ n matrix has at least one eigenvalue and at most 'n' numerically
different eigenvalues.
\begin{note}
	\begin{itemize}
		\item 	 The trace of a matrix is equal to the sum of the
		eigenvalues of a matrix.
		\[
		\sum_{\substack{i }} a_{ii}=\sum_{\substack{i }} \lambda_{i}
		\]
		\item The product of the eigenvalues of a matrix is equal
		to the determinant of that matrix.
		\[
		Det(A)=\prod_{\substack{i }} \lambda_{i}
		\]
		
	\end{itemize}
	
\end{note}
\begin{exercise}
	Obtain eigen values of the following matrices.\\$1.\left[\begin{array}{ccc}2 & 1 & 1\\ 1 & 2 & 1\\ 0 & 0& 1\end{array}\right]$\\$2.\left[\begin{array}{cc}4 & 1 \\ 2 & 3\end{array}\right]$
\end{exercise}
\begin{answer}
	$$
	\begin{aligned}
		1.\text{Trace}&=5, \text{Determinant}=3\Longrightarrow \lambda=1,1,3(\because 1+1+3=5,1\times1\times3=3)\\
		2.\text{Trace}&=7, \text{Determinant}=10\Longrightarrow \lambda=5,2(\because 5+2=7,5\times2=10)	
	\end{aligned}$$
 
\end{answer}



\subsection{Properties of eigen values}
\begin{itemize}
	\item  Any square matrix $A$ and its transpose $A^{T}$ have the same eigen values.
    \item If $\lambda_{1}, \lambda_{2}, \ldots \lambda_{n}$ are the eigen values of $A,$ then the eigen values of,
	\begin{itemize}
		\item  $k A$ are $k \lambda_{1}, \quad k \lambda_{2}, \ldots \ldots, k \lambda_{n}$\\
		\item  $A^{m}$ are $\lambda_{1}^{m}, \lambda_{2}^{m}, \ldots \ldots ., \lambda_{n}^{m}$\\
		\item $A^{-1}$ are $\frac{1}{\lambda_{1}}, \frac{1}{\lambda_{2}}, \ldots, \frac{1}{\lambda_{n}}$.
	\end{itemize}
	\item  If $A$ and $B$ are similar matrices, i.e. $A=I B,$ then $A$ and $B$ have the same eigenvalues.
\item  If $A$ and $B$ are two matrices of same order, then the matrices $A B$ and $B A$ have the same eigenvalues.
\item  The eigenvalues of a triangular matrix are equal to the diagonal elements of the matrix.
\begin{example}
		$\left[\begin{array}{ccc}a & d & e \\ 0 & b & f \\ 0 & 0 & c\end{array}\right]$, $ \lambda=a,b,c$
\end{example}
\item The eigen values of diagonal matrix are equal to the diagonal elements of the matrix.
\begin{example}
	$\left[\begin{array}{ccc}a & 0 & 0 \\ 0 & b & 0 \\ 0 & 0 & c\end{array}\right]$, $ \lambda=a,b,c$
\end{example}
\item For matrix whose rows or columns are identical, then $ \lambda=\text{Trace},0,0,0,..(n-1)$
\begin{example}
		$\left[\begin{array}{ccc}a & a & a \\ a & a & a \\ a & a & a\end{array}\right]$, $ \lambda=3a,0,0$,$\left[\begin{array}{ccc}2 & 1 & 3 \\ 2 & 1 & 3 \\ 2 & 1 & 3\end{array}\right]$, $ \lambda=6,0,0$
\end{example}
\item For a skew symmetric matrix eigen value is , $\lambda= 0,\pm i\sqrt {\text{sum of squares of non diagonal elements}}$
\begin{example}
$\left[\begin{array}{ccc}0 & -a & -b \\ a & 0 & -c \\ b & c & 0\end{array}\right]$ $ \lambda=0,\pm i\sqrt{a^{2}+b^{2}+c^{2}}$	
\end{example}

\end{itemize}
\begin{exercise}
	 Find the characteristic roots of the matrix $\left[\begin{array}{rrr}6 & -2 & 2 \\ -2 & 3 & -1 \\ 2 & -1 & 3\end{array}\right]$ \end{exercise}
 \begin{answer}
 	 The characteristic equation of the given matrix is $\left|\begin{array}{rrr}6-\lambda & -2 & 2 \\ -2 & 3-\lambda & -1 \\ 2 & -1 & 3-\lambda\end{array}\right|=0$
 	\\\\$\Rightarrow \quad(6-\lambda)\left(9-6 \lambda+\lambda^{2}-1\right)+2(-6+2 \lambda+2)+2(2-6+2 \lambda)=0$
 	\\\\	$\Rightarrow$
 	$-\lambda^{3}+12 \lambda^{2}-36 \lambda+32=0$
 	\\\\By trial, $\lambda=2$ is a root of this equation. \\\\$\Rightarrow \quad(\lambda-2)\left(\lambda^{2}-10 \lambda+16\right)=0 \\\\\Rightarrow(\lambda-2)(\lambda-2)(\lambda-8)=0$
 	\\\\$\Rightarrow \quad \lambda=2,2,8$ are the characteristic roots or Eigen values.
 	
 \end{answer}
 
\begin{exercise}
	 The matrix $A$ is defined as $A=\left[\begin{array}{rrr}1 & 2 & -3 \\ 0 & 3 & 2 \\ 0 & 0 & -2\end{array}\right]$Find the eigen values of $3 A^{3}+5 A^{2}-6 A+2 I$
	\end{exercise}
\begin{answer}
	 
	\begin{align*}
		|A-\lambda I|&=0\\
		\Rightarrow \quad(1-\lambda)(3-\lambda)(-2-\lambda)&=0\\ \text { or } \lambda&=1,3,-2\\
		\text{ie,Eigen values of } A&=1,3,-2\\
		\text{Eigen values of } A^{3}&=1,27,-8\\
		\text{Eigen values of }A^{2}&=1,9,4\\
		\text{Eigen values of }I&=1,1,1\\
		\text{Eigen values of }3 A^{3}+5 A^{2}-6 A+2 I,\\
		\text{First eigen value}&=3(1)^{3}+5(1)^{3}-6(1)+2(1)=4\\	
		\text{Second eigen value}&=3(27)+5(9)-6(3)+2(1)=110\\
		\text{Third eigen value}&=3(-8)+5(4)-6(-2)+2(1)=10\\\text{Then Required eigen values are } 4,110,10
	\end{align*}
	
\end{answer}
	
\subsection{Eigen vectors}
\subsubsection{Properties of eigen vectors}
\begin{itemize}
	\item The eigen vector $X$ of a matrix $A$ is not unique.
	\item  If $\lambda_{1}, \lambda_{2}, \ldots, \lambda_{n}$ be distinct eigen values of an $n \times n$ matrix then corresponding eigen vectors $X_{1}, X_{2}, \ldots \ldots ., X_{n}$ form a linearly independent Set.
	\item If two or more eigen values are equal it may or may not be possible to get linearly independent eigen vectors corresponding to the equal roots.
	\item  Two eigen vectors $X_{1}$ and $X_{2}$ are called orthogonal vectors if $X_{1}^{T} X_{2}=0$.
	\item Eigen vectors of a symmetric matrix corresponding to different eigen values are orthogonal.
	
\end{itemize}
\begin{note}
	  To find normalised form of $\left[\begin{array}{l}a \\ b \\ c\end{array}\right],$ we divide each element by $\sqrt{a^{2}+b^{2}+c^{2}}$
	  \begin{exampleT}
	  	normalised form of $\left[\begin{array}{l}1 \\ 2 \\ 3\end{array}\right]$ is $\left[\begin{array}{l}1 / \sqrt{14} \\ 2 / \sqrt{14} \\ 2 / \sqrt{14}\end{array}\right]$\\$ \left[ \because \sqrt {1^{2}+2^{2}+3^{2}}=\sqrt{14}\right]$
	  \end{exampleT}
\end{note} 

\begin{exercise}
	 Find the eigen values and eigen vectors of matrix $A=\left[\begin{array}{lll}3 & 1 & 4 \\ 0 & 2 & 6 \\ 0 & 0 & 5\end{array}\right]$\end{exercise}
 \begin{answer}
 	\begin{align*}
 		|A-\lambda I|&=\left|\begin{array}{ccc}
 			3-\lambda & 1 & 4 \\
 			0 & 2-\lambda & 6 \\
 			0 & 0 & 5-\lambda
 		\end{array}\right|\\&=(3-\lambda)(2-\lambda)(5-\lambda)\\
 	\end{align*}
 	\begin{align*}
 		\text{Hence the characteristic equation of matrix $A$ is given by},\\
 		|A-\lambda I| &=0 \\\Rightarrow (3-\lambda)(2-\lambda)(5-\lambda)&=0 \\
 		\therefore  \lambda &=2,3,5\\
 		\text{Thus the eigen values of matrix A are 2,3,5 .}
 		\\\text{The eigen vectors of the matrix A corresponding to the eigen value}\\\text{$\lambda$ is given by the nonzero solution of the equation (A-$\lambda I)$ X=0}\\
 		\left[\begin{array}{lll}
 			3-\lambda & 1 & 4 \\
 			0 & 2-\lambda & 6 \\
 			0 & 0 & 5-\lambda
 		\end{array}\right]\left[\begin{array}{l}
 			x_{1} \\
 			x_{2} \\
 			x_{3}
 		\end{array}\right]=\left[\begin{array}{l}
 			0 \\
 			0 \\
 			0
 		\end{array}\right]\\
 		\text{When $\lambda=2,$ the corresponding eigen vector is given by}\\
 		\left[\begin{array}{lll}
 			3-2 & 1 & 4 \\
 			0 & 2-2 & 6 \\
 			0 & 0 & 5-2
 		\end{array}\right]\left[\begin{array}{l}
 			x_{1} \\
 			x_{2} \\
 			x_{3}
 		\end{array}\right]=\left[\begin{array}{l}
 			0 \\
 			0 \\
 			0
 		\end{array}\right]\\
 		\Rightarrow\left[\begin{array}{lll}
 			1 & 1 & 4 \\
 			0 & 0 & 6 \\
 			0 & 0 & 3
 		\end{array}\right]\left[\begin{array}{l}
 			x_{1} \\
 			x_{2} \\
 			x_{3}
 		\end{array}\right]=\left[\begin{array}{l}
 			0 \\
 			0 \\
 			0
 		\end{array}\right]\\
 		\begin{array}{r}
 			x_{1}+x_{2}+4 x_{3}=0 \\
 			0 x_{1}+0 x_{2}+6 x_{3}=0 \\
 			\frac{x_{1}}{6-0}=\frac{x_{2}}{0-6}=\frac{x_{3}}{0-0}=k
 		\end{array}\\
 		\Rightarrow \quad \frac{x_{1}}{1}=\frac{x_{2}}{-1}=\frac{x_{3}}{0}=k \quad \Rightarrow \quad x_{1}=k, x_{2}=-k, x_{3}=0\\
 		\text { Hence } X_{1}=\left[\begin{array}{r}
 			k \\
 			-k \\
 			0
 		\end{array}\right]=k\left[\begin{array}{r}
 			1 \\
 			-1 \\
 			0
 		\end{array}\right] \\
 		\text { can be taken as an eigen vector of A corresponding to the eigen } \\
 		\text { value } \lambda=2
 	\end{align*}
 \end{answer}
	


\subsection{Cayley-Hamilton theorem}
According to the Cayley-Hamilton theorem, every square matrix satisfies its own characteristic equations. 
\begin{equation}
|A-\lambda I|=(-1)^{n}\left(\lambda^{n}+a_{1} \lambda^{n-1}+a_{2} \lambda^{n-2}+\cdots+a_{n}\right)
\label{cayley hamilton}
\end{equation}

Hence, if  equation \ref{cayley hamilton}  is the characteristic polynomial of a matrix $A$ of order $n$, then the matrix equation,
\begin{equation}
X^{n}+a_{1} X^{n-1}+a_{2} X^{n-2}+\cdots+a_{n} I=0
\end{equation}

Is satisfied by $X=A$.ie,
\begin{equation}
A^{n}+a_{1} A^{n-1}+a_{2} A^{n-2}+\ldots+a_{n} I=0
\end{equation} 
\begin{exercise}
	 Verify Cayley-Hamilton theorem for the matrix $A=\left(\begin{array}{cc}1 & 2 \\ 2 & -1\end{array}\right)$ and hence find $A^{-1}$.\end{exercise}


\begin{answer}
 The characteristic equation of the matrix is 

\begin{align*}
	| A-\lambda I|&=0\\
	\begin{array}{cc}|1-\lambda & 2 \\ 2 & -1-\lambda\end{array}|=0\\
	(1-\lambda)(-1-\lambda)-4&=0\\ \Rightarrow-1+\lambda^{2}-4&=0 \\\Rightarrow \lambda^{2}-5&=0\\
	\newline\text{By Cayley-Hamilton Theorem,} A^{2}-5 I&=0\\
	A^{2}=A . A &=\left[\begin{array}{rr}1 & 2 \\ 2 & -1\end{array}\right]\left[\begin{array}{rr}1 & 2 \\ 2 & -1\end{array}\right]=\left[\begin{array}{cc}5 & 0 \\ 0 & 5\end{array}\right] \\A^{2}-5 I&=\left[\begin{array}{ll}5 & 0 \\ 0 & 5\end{array}\right]-5\left[\begin{array}{ll}1 & 0 \\ 0 & 1\end{array}\right]\\&=\left[\begin{array}{ll}5 & 0 \\ 0 & 5\end{array}\right]+\left[\begin{array}{rr}-5 & 0 \\ 0 & -5\end{array}\right]\\&=\left[\begin{array}{ll}0 & 0 \\ 0 & 0\end{array}\right]=0 \\
	\text{ Thus Cayley hamilton theorem is verified.}\\
	A^{2}-5 I&=0\\\text{Multiplying by}\ A^{-1} \ \text{we get},\\
	A-5 A^{-1}&=0 \\\Rightarrow A^{-1}=\frac{1}{5} A \\\Rightarrow A^{-1}=\frac{1}{5}\left[\begin{array}{cc}
		1 & 2 \\
		2 & -1
	\end{array}\right]=\left[\begin{array}{cc}
		\frac{1}{5} & \frac{2}{5} \\
		\frac{2}{5} & -\frac{1}{5}
	\end{array}\right]
\end{align*}

\end{answer}

\subsection{Similiarity transformation}
Let $A$ and $B$ be two square matrices of order $n$. Then $B$ is said to be similar to $A$ if there exists a non-singular matrix $P$ such that
\begin{equation}
B=P^{-1} A P
\end{equation}

This transformation that gives B from A is called a similarity transformation.
\begin{note}
	If A and B are similiar matrices then both of them have same eigen values.Furthermore, if $\mathbf{x}$ is an eigenvector of $\mathbf{A},$ then $\mathbf{y}=\mathbf{P}^{-1} \mathbf{x}$ is an eigenvector of $\mathbf{B}$ corresponding to the same eigenvalue.\end{note}
\subsection{Diagonalisation}
Diagonalisation of a matrix A is the process of reduction of A to a diagonal form D.

If an $n \times n$ matrix $\mathbf{A}$ has a basis of eigenvectors, then
\begin{equation}
\mathbf{D}=\mathbf{P}^{-1} \mathbf{A} \mathbf{P}
\end{equation}
is diagonal, with the eigenvalues of $\mathbf{A}$ as the entries on the main diagonal. Here $\mathbf{X}$ is the matrix with these eigenvectors as column vectors. Also,
\begin{equation}
\mathbf{D}^{m}=\mathbf{P}^{-1} \mathbf{A}^{m} \mathbf{P} \quad(m=2,3, \cdots)
\end{equation}
\subsubsection{Theorem on diagonalisation of matrix}
\begin{theorem}
	If a square matrix $A$ of order $n$ has $n$ linearly independent eigen vectors, then a matrix $P$ can be found such that $P^{-1} A P$ is a diagonal matrix.
\end{theorem}

	$\bullet$ The square matrix $P$, which diagonalises $A$, is found by grouping the eigen vectors of $A$ into square-matrix and the resulting diagonal matrix has the eigen values of $A$ as its diagonal elements.
	 
\begin{exercise}
	Let $A=\left[\begin{array}{rrr}6 & -2 & 2 \\ -2 & 3 & -1 \\ 2 & -1 & 3\end{array}\right]$ Find matrix $P$ such that $P^{-1} A P$ is diagonal matrix.\end{exercise}
\begin{answer}
The characteristic equation of the  matrix $A$\\
$$\left|\begin{array}{rrr}6-\lambda & -2 & 2 \\ -2 & 3-\lambda & -1 \\ 2 & -1 & 3-\lambda\end{array}\right|=0$$
\begin{align*}
	(6-\lambda)\left[9+\lambda^{2}-6 \lambda-1\right]+2[-6+2 \lambda+2]+2[2-6+2 \lambda]&=0 \\(6-\lambda)\left(\lambda^{2}-6 \lambda+8\right)-8+4 \lambda-8+4 \lambda&=0 \\
	6 \lambda^{2}-36 \lambda+48-\lambda^{3}+6 \lambda^{2}-8 \lambda-16+8 \lambda&=0 \\-\lambda^{3}+12 \lambda^{2}-36 \lambda+32=0 \quad \Rightarrow \quad \lambda^{3}-12 \lambda^{2}+36 \lambda-32&=0 \\
	(\lambda-2)^{2}(\lambda-8)=0 \quad \Rightarrow \quad \lambda=2,2,8
\end{align*}
Eigen vector for $\lambda=2$\\
\begin{align*}
	\left[\begin{array}{ccc}
		4 & -2 & 2 \\
		-2 & 1 & -1 \\
		2 & -1 & 1
	\end{array}\right]\left[\begin{array}{l}
		x_{1} \\
		x_{2} \\
		x_{3}
	\end{array}\right]&=\left[\begin{array}{l}
		0 \\
		0 \\
		0
	\end{array}\right]\\\left[\begin{array}{rrr}
		2 & -1 & 1 \\
		-2 & 1 & -1 \\
		2 & -1 & 1
	\end{array}\right]\left[\begin{array}{l}
		x_{1} \\
		x_{2} \\
		x_{3}
	\end{array}\right]&=\left[\begin{array}{l}
		0 \\
		0 \\
		0
	\end{array}\right] \begin{array}{l}
		R_{2} \rightarrow R_{1}+R_{2} \\
		R_{3} \rightarrow R_{2}+R_{3}
	\end{array}\\
	\left[\begin{array}{ccc}
		2 & -1 & 1 \\
		0 & 0 & 0 \\
		0 & 0 & 0
	\end{array}\right]\left[\begin{array}{l}
		x_{1} \\
		x_{2} \\
		x_{3}
	\end{array}\right]&=\left[\begin{array}{l}
		0 \\
		0 \\
		0
	\end{array}\right] \text { or } 2 x_{1}-x_{2}+x_{3}=0\\
	\intertext{This equation is satisfied by $ x_{1}=0, x_{2}=1, x_{3}=1$}
X_{1}&=\left[\begin{array}{l}
		0 \\
		1 \\
		1
	\end{array}\right]
	\text{and again }
	x_{1}=1, x_{2}=3, x_{3}=1
	\\X_{2}&=\left[\begin{array}{l}
		1 \\
		3 \\
		1
	\end{array}\right]
	\intertext{Eigen vector for $ \lambda=8$}\begin{array}{r}
		{\left[\begin{array}{rrr}
				-2 & -2 & 2 \\
				-2 & -5 & -1 \\
				2 & -1 & -5
			\end{array}\right]\left[\begin{array}{l}
				x_{1} \\
				x_{2} \\
				x_{3}
			\end{array}\right]=\left[\begin{array}{l}
				0 \\
				0 \\
				0
			\end{array}\right]} \\
		-2 x_{1}-2 x_{2}+2 x_{3}=0 \\
		-2 x_{1}-5 x_{2}-x_{3}=0
	\end{array}
\end{align*}
\begin{align*}
	\frac{x_{1}}{2+10}&=\frac{x_{2}}{-4-2}=\frac{x_{3}}{10-4}\\ \Rightarrow \frac{x_{1}}{12}&=\frac{x_{2}}{-6}=\frac{x_{3}}{6}\\ \Rightarrow \frac{x_{1}}{2}&=\frac{x_{2}}{-1}=\frac{x_{3}}{1}\\X_{3}&=\left[\begin{array}{r}
		2 \\
		-1 \\
		1
	\end{array}\right]
\end{align*}
\begin{align*}
	\therefore \quad P&=\left[\begin{array}{rrr}
		0 & 1 & 2 \\
		1 & 3 & -1 \\
		1 & 1 & 1
	\end{array}\right], \quad P^{-1}=-\frac{1}{6}\left[\begin{array}{rrr}
		4 & 1 & -7 \\
		-2 & -2 & 2 \\
		-2 & 1 & -1
	\end{array}\right]\intertext { Now }\\ \quad P^{-1} A P&=-\frac{1}{6}\left[\begin{array}{rrr}
		4 & 1 & -7 \\
		-2 & -2 & 2 \\
		-2 & 1 & -1
	\end{array}\right]\left[\begin{array}{rrr}
		6 & -2 & 2 \\
		-2 & 3 & -1 \\
		2 & -1 & 3
	\end{array}\right]\left[\begin{array}{rrr}
		0 & 1 & 2 \\
		1 & 3 & -1 \\
		1 & 1 & 1
	\end{array}\right]|\\&=\left[\begin{array}{lll}
		2 & 0 & 0 \\
		0 & 2 & 0 \\
		0 & 0 & 8
	\end{array}\right]
\end{align*}

\end{answer}



\subsubsection{Power of a matrix}
We can obtain powers of a matrix by using diagonalisation.
We know that
$$
D=P^{-1} A P
$$
\\We have ,\ $${D}^{m}={P}^{-1} {A}^{m} {P} \quad(m=2,3, \cdots)$$
\\Pre-multiply by $P$ and post-multiply by $P^{-1}$ we get\\
$$
\begin{aligned}
	P D^{m} P^{-1} &=P\left(P^{-1} A^{m} P\right) P^{-1} \\
	&=\left(P P^{-1}\right) A^{m}\left(P P^{-1}\right) \\
	&=A^{m}
\end{aligned}
$$
\begin{itemize}
	\item  Find eigen values for a square matrix $A$.
	\item  Find eigen vectors to get the matrix $P$.
	\item  Find the diagonal matrix $D,$ by the formula $D=P^{-1} \mathrm{AP}$
	\item  Obtain $A^{m}$ by the formula $A^{m}=P D^{m} P^{-1}$.
\end{itemize}
\begin{exercise}
	Evaluate $A^{50}$ for the matrix $A=\left[ \begin{array}{cc}
	\frac{4}{3}& \frac{\sqrt{2}}{3}\\
	\frac{\sqrt{2}}{3}& \frac{5}{3} 
	\end{array}\right]$\\Their eigen values and eigen vectors are given as,$1\Longrightarrow \left\lbrace \sqrt{2},1\right\rbrace $ and for $2\Longrightarrow \left\lbrace  1,\sqrt{2}\right\rbrace $\end{exercise}
\begin{answer}
	$P=\left[ \begin{array}{cc}
		\sqrt{2}& 1\\
		1& \sqrt{2} 
	\end{array}\right]\Longrightarrow p^{-1}AP=\left[ \begin{array}{cc}
		1& 0\\
		0& 2 
	\end{array}\right]=D$\\
	$|A|\neq 0\Longrightarrow\text{matrix A is non singular}$\\
	hence, $$
	\begin{aligned}
		A^{50}=PD^{50}P^{-1}&=\left[ \begin{array}{cc}
			\sqrt{2}& 1\\
			1& \sqrt{2} 
		\end{array}\right] \left[ \begin{array}{cc}
			1& 0\\
			0& 2^{50} 
		\end{array}\right] \left[ \begin{array}{cc}
			\frac{\sqrt{2}}{3}& \frac{-1}{3}\\
			\frac{{1}}{3}& \frac{\sqrt{2}}{3} 
		\end{array}\right]
		\\&=\frac{1}{3} \left[ \begin{array}{cc}
			2^{50}+2& 2^{50}-{\sqrt{2}}\\
			(2^{50}-1)\sqrt{2}& 2^{51}+1 
		\end{array}\right]
	\end{aligned}$$
	
	
\end{answer}

\subsubsection{Exponential of matrix}
According to similarity tranformation, $$D=P^{-I} A P \quad \Rightarrow A=P D P^{-1}$$Then,
$$(\exp A)=P(e x p D) P^{-1}$$
\begin{exercise}
	Evaluate $ e^{A}$ where matrix A is given by, $\left[ \begin{array}{cc}
		1& 0\\
		0& 2 
	\end{array}\right] $
The eigen values of the given diagonal matrix are,$ \lambda_{1}=1  \text{and}  \lambda_{2}=2$
\end{exercise}
\begin{answer}
It is given that,
$$ \begin{aligned}
	(\exp A)&=P(e x p D) P^{-1} \\
	&\text{then,}\\
	e^{A}=\left[ \begin{array}{cc}
		e^{\lambda_{1}}& 0\\
		0& e^{\lambda_{2}}
	\end{array}\right]&=\left[ \begin{array}{cc}
	e^{1}& 0\\
	0& e^{2}
\end{array}\right]
\end{aligned}$$
\end{answer}
\subsubsection{Logarithm of a matrix}
According to similarity tranformation, $$D=P^{-I} A P \quad \Rightarrow A=P D P^{-1}$$
$$
\Rightarrow(\ln A)=P(\ln D) P^{-1}
$$
\section{Applications}
\subsection{Matrix representation of vector}
We can use matrix formalism to represent vectors.a vector can be representsed as a one-column matrix.\\
$\vec{A}=a\hat{i}+b\hat{j}+c\hat{k}$
$\Longrightarrow \left[ \begin{array}{c}
	a\\b\\c\end{array}\right] $
\\We can also use the matrix formalism to generate scalar products, but in
order to do so we must convert one of the column vectors into a row vector. The operation
of transposition provides a way to do this. Thus, letting $a$ and $b$ stand for vectors in $R^{3}$,\\\\
if $\vec{A}=a_{1}\hat{i}+a_{2}\hat{j}+a_{3}\hat{k}$ and $\vec{B}=b_{1}\hat{i}+b_{2}\hat{j}+b_{3}\hat{k}$ then their scalar product,
$${A} \cdot {B} \quad \longrightarrow \quad\left(\begin{array}{lll}a_{1} & a_{2} & a_{3}\end{array}\right)\left(\begin{array}{l}b_{1} \\ b_{2} \\ b_{3}\end{array}\right)=a_{1} b_{1}+a_{2} b_{2}+a_{3} b_{3}$$
If in a matrix context we regard a and ${b}$ as column vectors, the above equation assumes the form
$$
{A} \cdot {B} \quad \longrightarrow  {A}^{T} {B}
$$
\subsection{Cooordinate transformation}
Matrices are linear operators or maps such as rotations. In many problems we will need to use different coordinate systems inorder to describe different vector quantities .Components of a vector are transformed when we change the reference frame.
\newline Let $(x, y, z)$ be a cartesian coordinate system in a three dimension space. \\\\Let $\vec{r}=x \hat{i}+y \hat{j}$ be a vector in $x-y$ plane. \\\\If we consider a rotation of coordinate system about $z$ -axis through an angle $\theta$ in anticlockwise sense
and indicate the new axis by then the components of the same vector \ $\vec{r}=x^{\prime} \hat{i}+y^{\prime} \hat{j}^{\prime}$\  relative to the new system may be expressed as,
\\\\$x^{\prime}=x \cos \theta+y \sin \theta$\\
$y^{\prime}=-x \sin \theta+y \cos \theta$\\\\
Then equations in matrix form can be represented as ,\\\\
$\left[\begin{array}{l}x \\ y\end{array}\right]=\left[\begin{array}{cc}\cos \theta & \sin \theta \\ -\sin \theta & \cos \theta\end{array}\right]\left[\begin{array}{l}x \\ y\end{array}\right]=R_{z}(\theta)\left[\begin{array}{l}x \\ y\end{array}\right]$
\\\\
Where $R_{z}(\theta)$ is the transformation matrix corresponding to the rotation in $x-y$ plane about $z$ -axis.\\\\\textbf{Properties of Transformation matrix matrix}
\begin{itemize}
	\item $R_{z}(\theta)$ is orthogonal matrix.
	\item Determinant of $R_{=}(\theta)$ is 1.
	\item  Eigenvalues of $R_{z}(\theta)$ are $e^{i \theta}$ and $e^{-i \theta}$.
\end{itemize}
Let us consider a vector in space i.e. $\vec{r}=x \hat{i}+y \hat{j}+z \hat{k} .$ Here, three types of rotation is possible:
\\\\\textbf{Type I: Rotation about z-axis:}
Rotation matrix. $$R_{z}(\theta)=\left[\begin{array}{ccc}\cos \theta & \sin \theta & 0 \\ -\sin \theta & \cos \theta & 0 \\ 0 & 0 & 1\end{array}\right]$$
\\\textbf{Type  II: Rotation about x-axis:}
Rotation matrix. $$R_{x}(\theta)=\left[\begin{array}{ccc}1 & 0 & 0 \\ 0 & \cos \theta & \sin \theta \\ 0 & -\sin \theta & \cos \theta\end{array}\right]$$
\textbf{Type III: Rotation about y-axis:}
Rotation matrix. $$R_{y}(\theta)=\left[\begin{array}{ccc}\cos \theta & 0 & -\sin \theta \\ 0 & 1 & 0 \\ \sin \theta & 0 & \cos \theta\end{array}\right]$$
%.................................net..............................................

\section{Linear system of equations and  their solutions}
One of the major applications of determinants is in the establishment of a condition for the existence of a nontrivial solution for a Set of linear homogeneous algebraic equations.\\
Let $\mathrm{a}_{1}, \ldots \mathrm{a}_{\mathrm{n}}$ be elements of a vector  field , and let $\mathrm{x}_{1}, \ldots \mathrm{x}_{\mathrm{n}}$ be unknowns
(also called variables or indeterminates). Then an equation of the form
$$
\mathrm{a}_{1} \mathrm{x}_{1}+\cdots+\mathrm{a}_{\mathrm{n}} \mathrm{x}_{\mathrm{n}}=\mathrm{y}
$$
is called a linear equation in $n$ unknowns (over the vector field.)
in which case we say that $\left(c_{1}, \ldots, c_{n}\right)$ satisfies the equation. The Set of all such solutions is called the solution Set (or the general solution).
Now consider the following system of ${m}$ linear equations in '${n}$' unknowns:
$$
\begin{array}{c}
	a_{11} x_{1}+\cdots+a_{1 n} x_{n}=y_{1} \\
	a_{21} x_{1}+\cdots+a_{2 n} x_{n}=y_{2} \\
	\vdots \\
	a_{m 1} x_{1}+\cdots+a_{m n} x_{n}=y_{m}
\end{array}
$$
We abbreviate this system as,
$$
\sum_{j=1}^{n} a_{i j} x_{j}=y_{i}, \quad i=1, \ldots m
$$
Suppose we have a system of '$ i $' linear equation in '$ j $' unknowns, such as,
$$
\begin{array}{c}
	a_{11} x_{1}+a_{12} x_{2}+\cdots+a_{i j} x_{j}=b_{1} \\
	a_{221} x_{1}+a_{22} x_{2}+\cdots+a_{2 j} x_{j}=b_{2} \\
	: \quad: \quad: \quad: \quad:
\end{array}
$$
The above system of equations can be written in the matrix form as follows:
$$
\left[\begin{array}{cccc}
	a_{11} & a_{12} & \cdots & a_{i j} \\
	a_{21} & a_{22} & \cdots & a_{2 j} \\
	\vdots & \vdots & \vdots & \vdots \\
	a_{i 1} & a_{i 2} & \cdots & a_{i j}
\end{array}\right]\left[\begin{array}{c}
	x_{1} \\
	x_{2} \\
	\vdots \\
	x_{j}
\end{array}\right]=\left[\begin{array}{c}
	b_{1} \\
	b_{2} \\
	\vdots \\
	b_{j}
\end{array}\right]
$$
The equation can be represented by the form $A X=B$.\\
$x=\left[\begin{array}{c}x_{1} \\ x_{2} \\ \vdots \\ x_{j}\end{array}\right]$ is of the order of $j \times 1$ and\\
$B=\left[\begin{array}{c}b_{1} \\ b_{2} \\ \vdots \\ b_{i}\end{array}\right]$ is of the order of $i \times 1 .$\\ $[A]_{i \times j}$ is called the coefficient matrix of system of linear equations.
\subsection{Solution of Homogeneous System of Linear Equations}
As already discussed, for a homogeneous system of linear equation with ' $j$ unknowns,
$$
\begin{array}{c}
	A X=B \text { becomes } \\
	A X=0(\because B=0)
\end{array}
$$
There are two cases that arise for homogeneous systems:
\begin{enumerate}
	\item  \textbf{Matrix $A$ is non-singular or $|A| \neq 0$.}\\
	The solution of the homogeneous system in the above equation has a unique solution, $X=0,$ i.e. $x_{1}=x_{2}=\cdots=x_{j}=0$
		\item \textbf{Matrix $A$ is singular or $|A|=0.$}\\ Then it has infinite many solutions. To find the solution when $|A|=0$, put $z=k$ (where $k$ is any real number) and solve any two equations for $x$ and $y$ using the matrix method. The values obtained for $x$ and $y$ with $z=$ $k$ is the solution of the system.
\end{enumerate}

\subsection{Solution of Non-Homogeneous System of Simultaneous Linear Equations}
To solve a non-homogeneous system of simultaneous linear equations. We have to find the number of unknowns and the number of equations.
\begin{enumerate}
	\item  Given that $A$ is a non-singular matrix, then a system of equations represented by $A X=B$ has the unique solution which can be calculated by $X=A^{-1} B$
	\item  If $A X=B$ is a system with linear equations equal to the number of unknowns, then three cases arise:
\end{enumerate}

\begin{itemize}
	\item  If $|A| \neq 0$, system is consistent and has a unique solution given by $X=A^{-1} B$.
	\item If $|A|=0$ and $(\operatorname{adj} A) B=0,$ system is consistent and has infinite solutions.
	\item If $|A|=0$ and $(\operatorname{adj} A) B \neq 0,$ system is inconsistent.
	
\end{itemize}
\subsection{Cramer's rule}
Our real problem in solving a linear equation is to determine under what conditions there is any solution, apart from the trivial one $x_{1}=0, x_{2}=0, x_{3}=0$. 
Suppose we have the following system of linear equations:
\begin{equation}
\begin{array}{l}
a_{1} x+b_{1} y+c_{1} z=k_{1} \\
a_{2} x+b_{2} y+c_{2} z=k_{2} \\
a_{3} x+b_{3} y+c_{3} z=k_{3}
\end{array}\label{Matrices linear 001}
\end{equation}
If we use vector notation $\mathbf{x}=\left(x_{1}, x_{2}, x_{3}\right)$ for the solution and three rows $\mathbf{a}=\left(a_{1}, a_{2}, a_{3}\right), \mathbf{b}=\left(b_{1}, b_{2}, b_{3}\right), \mathbf{c}=\left(c_{1}, c_{2}, c_{3}\right)$ of coefficients, then the three equations, Equation.\label{Matrices linear 001}, become
Now, if
$$
\begin{array}{l}
	\Delta=\left[\begin{array}{lll}
		a_{1} & b_{1} & c_{1} \\
		a_{2} & b_{2} & c_{2} \\
		a_{3} & b_{3} & c_{3}
	\end{array}\right] \neq 0 \\
	\Delta_{1}=\left[\begin{array}{lll}
		k_{1} & b_{1} & c_{1} \\
		k_{2} & b_{2} & c_{2} \\
		k_{3} & b_{3} & c_{3}
	\end{array}\right] \neq 0 \\
	\Delta_{2}=\left[\begin{array}{lll}
		a_{1} & k_{1} & c_{1} \\
		a_{2} & k_{2} & c_{2} \\
		a_{3} & k_{3} & c_{3}
	\end{array}\right] \neq 0 \\
	\Delta_{3}=\left[\begin{array}{lll}
		a_{1} & b_{1} & k_{1} \\
		a_{2} & b_{2} & k_{2} \\
		a_{3} & b_{3} & k_{3}
	\end{array}\right] \neq 0
\end{array}
$$
Thus, the solution of the system of equations is given by
$$
\begin{array}{l}
	x=\frac{\Delta_{1}}{\Delta} \\
	y=\frac{\Delta_{2}}{\Delta} \\
	z=\frac{\Delta_{3}}{\Delta}
\end{array}
$$\subsection{Augmented matrix}
Consider the following system of equations:
$$
\begin{array}{c}
	a_{11} x_{1}+a_{12} x_{2}+\cdots+a_{1 n} x_{n}=b_{1} \\
	a_{21} x_{1}+a_{22} x_{2}+\cdots+a_{2 n} x_{n}=b_{2} \\
	\vdots \quad \vdots \\
	a_{m 1} x_{1}+a_{m 2} x_{2}+\cdots+a_{m n} x_{n}=b_{m}
\end{array}
$$
This system can be represented as $A X=B$.
$$
\begin{aligned}
	\text { where } A &=\left[\begin{array}{cccc}
		a_{11} & a_{12} & \cdots & a_{1 n} \\
		a_{21} & a_{22} & \cdots & a_{2 n} \\
		\vdots & \vdots & \vdots & \vdots \\
		a_{m 1} & a_{m 2} & \cdots & a_{m n}
	\end{array}\right], X=\left[\begin{array}{l}
		x_{1} \\
		x_{2} \\
		\vdots \\
		x_{n}
	\end{array}\right] \text { and } \\
	& B=\left[\begin{array}{c}
		b_{1} \\
		b_{2} \\
		\vdots \\
		b_{n}
	\end{array}\right] .
\end{aligned}
$$
The matrix $[A \mid B]=\left[\begin{array}{cccc|c}a_{11} & a_{12} & \cdots & a_{1 n} & b_{1} \\ a_{21} & a_{22} & \cdots & a_{2 n} & b_{2} \\ \vdots & \vdots & \vdots & \vdots & \vdots \\ a_{m 1} & a_{m 2} & \cdots & a_{m n} & b_{m}\end{array}\right]$ is called
augmented matrix
\begin{exercise}
	Show that the homogeneous system of equations has a non-trivial solution and find the solution.
	$$
	\begin{array}{r}
		x-2 y+z=0 \\
		x+y-z=0 \\
		3 x+6 y-5 z=0
	\end{array}
	$$
\end{exercise}
\begin{answer}[H]
	The given system of equations can be written in the matrix form as follows:
	\begin{align*}
		\left[\begin{array}{ccc}
			1 & -2 & 1 \\
			1 & 1 & -1 \\
			3 & 6 & -5
		\end{array}\right]\left[\begin{array}{l}
			x \\
			y \\
			z
		\end{array}\right]&=\left[\begin{array}{l}
			0 \\
			0 \\
			0
		\end{array}\right]\\
\text{	Which is similar to }\ A X=O, \\\text{Where}\ A&=\left[\begin{array}{ccc}3 & 6 & -5\end{array}\right]\left[\begin{array}{ccc}1 & -2 & 1 \\ 1 & 1 & -1 \\ 3 & 6 & -5\end{array}\right],\\
	X&=\left[\begin{array}{l}x \\ y \\ z\end{array}\right]\text{ and } O=\left[\begin{array}{l}0 \\ 0 \\ 0\end{array}\right]\\\text{Now,}\
|A|&=\left|\begin{array}{ccc}
			1 & -2 & 1 \\
			1 & 1 & -1 \\
			3 & 6 & -5
		\end{array}\right|\\&=1(-5+6)-1(10-6)+3(2-1)=0 
	\end{align*}
Thus, $|A|=0$ and hence the given system of equations has a non-trivial solution.
Now, to find the solution, we put $z=k$ in the first two equations.
\begin{align*}
	\begin{array}{c}
		x-2 y=-k \\
		x+y=k \\
		{\left[\begin{array}{cc}
				1 & -2 \\
				1 & 1
			\end{array}\right]\left[\begin{array}{l}
				x \\
				y
			\end{array}\right]=\left[\begin{array}{l}
				-k \\
				k
			\end{array}\right]}
	\end{array}\\
\text{which is similar to}\ A X=B, \text{where}\ A=\left[\begin{array}{cc}1 & -2 \\ 1 & 1\end{array}\right],\\ X=\left[\begin{array}{l}x \\ y\end{array}\right] \text{and}\ B=\left[\begin{array}{l}-k \\ k\end{array}\right]\\
\text{Now,}
|A|=\left|\begin{array}{cc}
	1 & -2 \\
	1 & 1
\end{array}\right|=3 \neq 0\\\text{Hence,}\  A^{-1} exists.
\\
\begin{array}{r}
\text { Now, adj } A=\left[\begin{array}{cc}
	1 & 2 \\
	-1 & 1
\end{array}\right] \\
A^{-1}=\frac{1}{3}\left[\begin{array}{cc}
	1 & 2 \\
	-1 & 1
\end{array}\right]
\end{array}
\\
\text { Now, } X=A^{-1} B \\ \Rightarrow\left[\begin{array}{l}
		x \\
		y
	\end{array}\right]=\frac{1}{3}\left[\begin{array}{cc}
		1 & 2 \\
		-1 & 1
	\end{array}\right]\left[\begin{array}{l}
		-k \\
		k
	\end{array}\right]=\left[\begin{array}{l}
		k / 3 \\
		2 k / 3
	\end{array}\right]
	\\ \Rightarrow x=\frac{k }{3}, y=\frac{2 k}{3}\\
\text{Hence,}\ x=\frac{k }{3}, y=\frac{2 k}{3}\ \text{and}\ z=k
\end{align*}
Where k is any real number that satisfies the given set of equations.
\end{answer}

\newpage
\pagestyle{plain}


\begin{abox}
	Practice Set-1
	\end{abox}
\begin{enumerate}[label=\color{ocre}\textbf{\arabic*.}]
	\item Consider the matrix $M=\left(\begin{array}{lll}1 & 1 & 1 \\ 1 & 1 & 1 \\ 1 & 1 & 1\end{array}\right)$\\
	\textbf{A.} The eigenvalues of $M$ are
	{\exyear{NET/JRF(JUNE-2011)}}
	\begin{tasks}(4)
		\task[\textbf{A.}] $0,1,2$
		\task[\textbf{B.}] $0,0,3$
		\task[\textbf{C.}] $1,1,1$
		\task[\textbf{D.}] $-1,1,3$
	\end{tasks}
	\textbf{B.} The exponential of $M$ simplifies to ( $I$ is the $3 \times 3$ identity matrix)
	\begin{tasks}(2)
		\task[\textbf{A.}] $e^{M}=I+\left(\frac{e^{3}-1}{3}\right) M$
		\task[\textbf{B.}] $e^{M}=I+M+\frac{M^{2}}{2 !}$
		\task[\textbf{C.}] $e^{M}=I+3^{3} M$
		\task[\textbf{D.}] $e^{M}=(e-1) M$
	\end{tasks}
	\item A $3 \times 3$ matrix $M$ has $\operatorname{Tr}[M]=6, \operatorname{Tr}\left[M^{2}\right]=26$ and $\operatorname{Tr}\left[M^{3}\right]=90$. Which of the following can be a possible Set of eigenvalues of $M ?$
	{\exyear{NET/JRF(DEC-2011)}}
	\begin{tasks}(4)
		\task[\textbf{A.}] $\{1,1,4\}$
		\task[\textbf{B.}] $\{-1,0,7\}$
		\task[\textbf{C.}] $\{-1,3,4\}$
		\task[\textbf{D.}] $\{2,2,2\}$
	\end{tasks}
	\item The eigen values of the matrix $A=\left(\begin{array}{lll}1 & 2 & 3 \\ 2 & 4 & 6 \\ 3 & 6 & 9\end{array}\right)$ are
	{\exyear{NET/JRF(JUNE-2012)}}
	\begin{tasks}(4)
		\task[\textbf{A.}] $(1,4,9)$
		\task[\textbf{B.}] $(0,7,7)$
		\task[\textbf{C.}] $(0,1,13)$
		\task[\textbf{D.}] $(0,0,14)$
	\end{tasks}
	\item The eigenvalues of the antisymmetric matrix,
	$$
	A=\left(\begin{array}{ccc}
	0 & -n_{3} & n_{2} \\
	n_{3} & 0 & -n_{1} \\
	-n_{2} & n_{1} & 0
	\end{array}\right)
	$$
	where $n_{1}, n_{2}$ and $n_{3}$ are the components of a unit vector, are
	{\exyear{NET/JRF(JUNE-2012)}}
	\begin{tasks}(4)
		\task[\textbf{A.}] $0, i,-i$
		\task[\textbf{B.}] $0,1,-1$
		\task[\textbf{C.}] $0,1+i,-1,-i$
		\task[\textbf{D.}]  $0,0,0$
	\end{tasks}
	\item Consider an $n \times n(n>1)$ matrix $A$, in which $A_{i j}$ is the product of the indices $i$ and $j$ $\left(\right.$ namely $\left.A_{i j}=i j\right)$. The matrix $A$
	{\exyear{NET/JRF(DEC-2013)}}
	\begin{tasks}(1)
		\task[\textbf{A.}]  Has one degenerate eigevalue with degeneracy $(n-1)$
		\task[\textbf{B.}] Has two degenerate eigenvalues with degeneracies 2 and $(n-2)$
		\task[\textbf{C.}] Has one degenerate eigenvalue with degeneracy $n$
		\task[\textbf{D.}] Does not have any degenerate eigenvalue
	\end{tasks}
	\item Consider the matrix
	$$
	M=\left(\begin{array}{ccc}
	0 & 2 i & 3 i \\
	-2 i & 0 & 6 i \\
	-3 i & -6 i & 0
	\end{array}\right)
	$$
	The eigenvalues of $M$ are
	{\exyear{NET/JRF(JUNE-2014)}}
	\begin{tasks}(4)
		\task[\textbf{A.}] $-5,-2,7$
		\task[\textbf{B.}] $-7,0,7$
		\task[\textbf{C.}] $-4 i, 2 i, 2 i$
		\task[\textbf{D.}] $2,3,6$
	\end{tasks}
	\item The column vector $\left(\begin{array}{l}a \\ b \\ a\end{array}\right)$ is a simultaneous eigenvector of $A=\left(\begin{array}{ccc}0 & 0 & 1 \\ 0 & 1 & 0 \\ 1 & 0 & 0\end{array}\right)$ and $B=\left(\begin{array}{lll}0 & 1 & 1 \\ 1 & 0 & 1 \\ 1 & 1 & 0\end{array}\right)$ if
	{\exyear{NET/JRF(DEC-2014)}}
	\begin{tasks}(2)
		\task[\textbf{A.}] $b=0$ or $a=0$
		\task[\textbf{B.}] $b=a$ or $b=-2 a$
		\task[\textbf{C.}] $b=2 a$ or $b=-a$
		\task[\textbf{D.}] $b=a / 2$ or $b=-a / 2$
	\end{tasks}
	\item The matrix $M=\left(\begin{array}{ccc}1 & 3 & 2 \\ 3 & -1 & 0 \\ 0 & 0 & 1\end{array}\right)$ satisfies the equation
	{\exyear{NET/JRF(DEC-2016)}}
	\begin{tasks}(2)
		\task[\textbf{A.}] $M^{3}-M^{2}-10 M+12 I=0$
		\task[\textbf{B.}] $M^{3}+M^{2}-12 M+10 I=0$
		\task[\textbf{C.}] $M^{3}-M^{2}-10 M+10 I=0$
		\task[\textbf{D.}] $M^{3}+M^{2}-10 M+10 I=0$
	\end{tasks}
	\item   Which of the following can not be the eigen values of a real $3 \times 3$ matrix
	{\exyear{NET/JRF(JUNE-2017)}}
	\begin{tasks}(4)
		\task[\textbf{A.}]  $2 i, 0,-2 i$
		\task[\textbf{B.}] $1,1,1$
		\task[\textbf{C.}] $e^{i \theta}, e^{-i \theta}, 1$
		\task[\textbf{D.}] $i, 1,0$
	\end{tasks}
	\item  Let $\sigma_{x}, \sigma_{y}, \sigma_{z}$ be the Pauli matrices and $x^{\prime} \sigma_{x}+y^{\prime} \sigma_{y}+z^{\prime} \sigma_{z}=\exp \left(\frac{i \theta \sigma_{z}}{2}\right) \times$
	$$
	\left[x \sigma_{x}+y \sigma_{y}+z \sigma_{z}\right] \exp \left(-\frac{i \theta \sigma_{z}}{2}\right)
	$$
	Then the coordinates are related as follows
	{\exyear{NET/JRF(JUNE-2017)}}
	\begin{tasks}(1)
		\task[\textbf{A.}] $\left(\begin{array}{l}x^{\prime} \\ y^{\prime} \\ z^{\prime}\end{array}\right)=\left(\begin{array}{ccc}\cos \theta & -\sin \theta & 0 \\ \sin \theta & \cos \theta & 0 \\ 0 & 0 & 1\end{array}\right)\left(\begin{array}{l}x \\ y \\ z\end{array}\right)$
		\task[\textbf{B.}] $\left(\begin{array}{l}x^{\prime} \\ y^{\prime} \\ z^{\prime}\end{array}\right)=\left(\begin{array}{ccc}\cos \theta & \sin \theta & 0 \\ -\sin \theta & \cos \theta & 0 \\ 0 & 0 & 1\end{array}\right)\left(\begin{array}{l}x \\ y \\ z\end{array}\right)$
		\task[\textbf{C.}] $\left(\begin{array}{l}x^{\prime} \\ y^{\prime} \\ z^{\prime}\end{array}\right)=\left(\begin{array}{ccc}\cos \frac{\theta}{2} & \sin \frac{\theta}{2} & 0 \\ -\sin \frac{\theta}{2} & \cos \frac{\theta}{2} & 0 \\ 0 & 0 & 1\end{array}\right)\left(\begin{array}{l}x \\ y \\ z\end{array}\right)$
		\task[\textbf{D.}] $\left(\begin{array}{l}x^{\prime} \\ y^{\prime} \\ z^{\prime}\end{array}\right)=\left(\begin{array}{ccc}\cos \frac{\theta}{2} & -\sin \frac{\theta}{2} & 0 \\ \sin \frac{\theta}{2} & \cos \frac{\theta}{2} & 0 \\ 0 & 0 & 1\end{array}\right)\left(\begin{array}{l}x \\ y \\ z\end{array}\right)$
	\end{tasks}
	\item  Let $A$ be a non-singular $3 \times 3$ matrix, the columns of which are denoted by the vectors $\vec{a}, \vec{b}$ and $\vec{c}$, respectively. Similarly, $\vec{u}, \vec{v}$ and $\vec{w}$ denote the vectors that form the corresponding columns of $\left(A^{T}\right)^{-1}$. Which of the following is true?
	{\exyear{NET/JRF(DEC-2017)}}
	\begin{tasks}(2)
		\task[\textbf{A.}] $\vec{u} \cdot \vec{a}=0, \vec{u} \cdot \vec{b}=0, \vec{u} \cdot \vec{c}=1$
		\task[\textbf{B.}]  $\vec{u} \cdot \vec{a}=0, \vec{u} \cdot \vec{b}=1, \vec{u} \cdot \vec{c}=0$
		\task[\textbf{C.}] $\vec{u} \cdot \vec{a}=1, \vec{u} \cdot \vec{b}=0, \vec{u} \cdot \vec{c}=0$
		\task[\textbf{D.}]  $\vec{u} \cdot \vec{a}=0, \vec{u} \cdot \vec{b}=0, \vec{u} \cdot \vec{c}=0$
	\end{tasks}
	\item Consider the matrix equation
	$$
	\left(\begin{array}{llc}
	1 & 1 & 1 \\
	1 & 2 & 3 \\
	2 & b & 2 c
	\end{array}\right)\left(\begin{array}{l}
	x \\
	y \\
	z
	\end{array}\right)=\left(\begin{array}{l}
	0 \\
	0 \\
	0
	\end{array}\right)
	$$
	The condition for existence of a non-trivial solution and the corresponding normalised solution (upto a sign) is
	{\exyear{NET/JRF(DEC-2017)}}
	\begin{tasks}(2)
		\task[\textbf{A.}] $b=2 c$ and $(x, y, z)=\frac{1}{\sqrt{6}}(1,-2,1)$
		\task[\textbf{B.}] $c=2 b$ and $(x, y, z)=\frac{1}{\sqrt{6}}(1,1,-2)$
		\task[\textbf{C.}] $c=b+1$ and $(x, y, z)=\frac{1}{\sqrt{6}}(2,-1,-1)$
		\task[\textbf{D.}] $b=c+1$ and $(x, y, z)=\frac{1}{\sqrt{6}}(1,-2,1)$
	\end{tasks}
	\item  Which of the following statements is true for a $3 \times 3$ real orthogonal matrix with determinant $+1 ?$
	{\exyear{NET/JRF(JUNE-2018)}}
	\begin{tasks}(1)
		\task[\textbf{A.}] The modulus of each of its eigenvalues need not be 1, but their product must be 1
		\task[\textbf{B.}] At least one of its eigenvalues is $+1$
		\task[\textbf{C.}] All of its eigenvalues must be real
		\task[\textbf{D.}]  None of its eigenvalues must be real
	\end{tasks}
	\item One of the eigenvalues of the matrix $e^{A}$ is $e^{a}$, where $A=\left(\begin{array}{ccc}a & 0 & 0 \\ 0 & 0 & a \\ 0 & a & 0\end{array}\right)$. The product of the other two eigenvalues of $e^{A}$ is
	{\exyear{NET/JRF(DEC-2018)}}
	\begin{tasks}(4)
		\task[\textbf{A.}] $e^{2 a}$
		\task[\textbf{B.}] $e^{-a}$
		\task[\textbf{C.}]  $e^{-2 a}$
		\task[\textbf{D.}] 1
	\end{tasks}
	\item A $4 \times 4$ complex matrix $A$ satisfies the relation $A^{\dagger} A=4 I$, where $I$ is the $4 \times 4$ identity matrix. The number of independent real parameters of $A$ is
	{\exyear{NET/JRF(DEC-2018)}}
	\begin{tasks}(4)
		\task[\textbf{A.}] 32
		\task[\textbf{B.}] 10
		\task[\textbf{C.}] 12
		\task[\textbf{D.}] 16
	\end{tasks}
	\item  The element of a $3 \times 3$ matrix $A$ are the products if its row and column indices $A_{i j}=i j$ (where $i, j=1,2,3)$. The eigenvalues of $A$ are
	{\exyear{NET/JRF(JUNE-2019)}}
	\begin{tasks}(4)
		\task[\textbf{A.}] $(7,7,0)$
		\task[\textbf{B.}]  $(7,4,3)$
		\task[\textbf{C.}] $(14,0,0)$
		\task[\textbf{D.}] $\left(\frac{14}{3}, \frac{14}{3}, \frac{14}{3}\right)$
	\end{tasks}
	\item  The operator $A$ has a matrix representation $\left(\begin{array}{ll}2 & 1 \\ 1 & 2\end{array}\right)$ in the basis spanned by $\left(\begin{array}{l}1 \\ 0\end{array}\right)$ and $\left(\begin{array}{l}0 \\ 1\end{array}\right) .$ In another basis spanned by $\frac{1}{\sqrt{2}}\left(\begin{array}{l}1 \\ 1\end{array}\right)$ and $\frac{1}{\sqrt{2}}\left(\begin{array}{c}1 \\ -1\end{array}\right)$, the matrix representation of $A$ is
	{\exyear{NET/JRF(JUNE-2019)}}
	\begin{tasks}(4)
		\task[\textbf{A.}] $\left(\begin{array}{ll}2 & 0 \\ 0 & 2\end{array}\right)$
		\task[\textbf{B.}] $\left(\begin{array}{ll}3 & 0 \\ 0 & 1\end{array}\right)$
		\task[\textbf{C.}] $\left(\begin{array}{ll}3 & 1 \\ 0 & 1\end{array}\right)$
		\task[\textbf{D.}] $\left(\begin{array}{ll}3 & 0 \\ 1 & 1\end{array}\right)$
	\end{tasks}
	\item  If the rank of an $n \times n$ matrix $A$ is $m$, where $m$ and $n$ are positive integers with $1 \leq m \leq n$, then the rank of the matrix $A^{2}$ is
	{\exyear{NET/JRF(DEC-2019)}}
	\begin{tasks}(4)
		\task[\textbf{A.}]  $m$
		\task[\textbf{B.}] $m-1$
		\task[\textbf{C.}] $2 \mathrm{~m}$
		\task[\textbf{D.}] $m-2$
	\end{tasks}
	\item   The eigenvalues of the $3 \times 3$ matrix $M=\left(\begin{array}{lll}a^{2} & a b & a c \\ a b & b^{2} & b c \\ a c & b c & c^{2}\end{array}\right)$ are
	{\exyear{NET/JRF(JUNE-2020)}}
	\begin{tasks}(2)
		\task[\textbf{A.}] $a^{2}+b^{2}+c^{2}, 0,0$
		\task[\textbf{B.}] $b^{2}+c^{2}, a^{2}, 0$
		\task[\textbf{C.}] $a^{2}+b^{2}, c^{2}, 0$
		\task[\textbf{D.}] $a^{2}+c^{2}, b^{2}, 0$
	\end{tasks}
\end{enumerate}
 \colorlet{ocre1}{ocre!70!}
\colorlet{ocrel}{ocre!30!}
\setlength\arrayrulewidth{1pt}
\begin{table}[H]
	\centering
	\arrayrulecolor{ocre}
	\begin{tabular}{|p{1.5cm}|p{1.5cm}||p{1.5cm}|p{1.5cm}|}
		\hline
		\multicolumn{4}{|c|}{\textbf{Answer key}}\\\hline\hline
		\rowcolor{ocrel}Q.No.&Answer&Q.No.&Answer\\\hline
		1&\textbf{B} &2&\textbf{C}\\\hline 
		3&\textbf{D} &4&\textbf{A} \\\hline
		5&\textbf{A} &6&\textbf{B} \\\hline
		7&\textbf{B}&8&\textbf{C}\\\hline
		9&\textbf{D}&10&\textbf{B}\\\hline
		11&\textbf{C} &12&\textbf{D}\\\hline
		13&\textbf{B}&14&\textbf{D}\\\hline
		15&\textbf{D}&16&\textbf{C}\\\hline
		17&\textbf{B} &18&\textbf{A}\\\hline
		19&\textbf{A}& & \\\hline
	\end{tabular}
\end{table}
\newpage
\begin{abox}
	Practice Set-2
\end{abox}
\begin{enumerate}[label=\color{ocre}\textbf{\arabic*.}]
	\item The eigenvalues of the matrix $\left(\begin{array}{lll}2 & 3 & 0 \\ 3 & 2 & 0 \\ 0 & 0 & 1\end{array}\right)$ are
	{\exyear{GATE 2010}}
	\begin{tasks}(4)
		\task[\textbf{A.}] $5,2,-2$
		\task[\textbf{B.}] $-5,-1,-1$
		\task[\textbf{C.}]  $5,1,-1$
		\task[\textbf{D.}] $-5,1,1$
	\end{tasks}
	\item Two matrices $A$ and $B$ are said to be similar if $B=P^{-1} A P$ for some invertible matrix $P$.Which of the following statements is NOT TRUE?
	{\exyear{GATE 2011}}
	\begin{tasks}(2)
		\task[\textbf{A.}] Det $A=\operatorname{Det} B$
		\task[\textbf{B.}]  Trace of $A=$ Trace of $B$
		\task[\textbf{C.}] $A$ and $B$ have the same eigenvectors
		\task[\textbf{D.}] $A$ and $B$ have the same eigenvalues
	\end{tasks}
	\item A $3 \times 3$ matrix has elements such that its trace is 11 and its determinant is 36 . The eigenvalues of the matrix are all known to be positive integers. The largest eigenvalues of the matrix is
	{\exyear{GATE 2011}}
	\begin{tasks}(4)
		\task[\textbf{A.}] 18
		\task[\textbf{B.}]  12
		\task[\textbf{C.}] 9
		\task[\textbf{D.}] 6
	\end{tasks}
	\item  The number of independent components of the symmetric tensor $A_{i j}$ with indices $i, j=1,2,3$ is
	{\exyear{GATE 2012}}
	\begin{tasks}(4)
		\task[\textbf{A.}] 1
		\task[\textbf{B.}] 3
		\task[\textbf{C.}] 6
		\task[\textbf{D.}] 9
	\end{tasks}
	\item  The eigenvalues of the matrix $\left(\begin{array}{lll}0 & 1 & 0 \\ 1 & 0 & 1 \\ 0 & 1 & 0\end{array}\right)$ are
	{\exyear{GATE 2012}}
	\begin{tasks}(4)
		\task[\textbf{A.}] $0,1,1$
		\task[\textbf{B.}] $0,-\sqrt{2}, \sqrt{2}$
		\task[\textbf{C.}]  $\frac{1}{\sqrt{2}}, \frac{1}{\sqrt{2}}, 0$
		\task[\textbf{D.}] $\sqrt{2}, \sqrt{2}, 0$
	\end{tasks}
	\item    The degenerate eigenvalue of the matrix $\left[\begin{array}{ccc}4 & -1 & -1 \\ -1 & 4 & -1 \\ -1 & -1 & 4\end{array}\right]$ is (your answer should be an
	integer)---
	{\exyear{GATE 2013}}
	\item  The matrix
	$$
	A=\frac{1}{\sqrt{3}}\left[\begin{array}{cc}
	1 & 1+i \\
	1-i & -1
	\end{array}\right] \text { is }
	$$
	{\exyear{GATE 2014}}
	\begin{tasks}(4)
		\task[\textbf{A.}] Orthogonal
		\task[\textbf{B.}] Symmetric
		\task[\textbf{C.}]  Anti-symmetric
		\task[\textbf{D.}]  Unitary
	\end{tasks}
	\item  Let $X$ be a column vector of dimension $n>1$ with at least one non-zero entry. The number of non-zero eigenvalues of the matrix $M=X X^{T}$ is
	{\exyear{GATE 2017}}
	 \begin{tasks}(4)
	 	\task[\textbf{A.}] 0
	 	\task[\textbf{B.}] $n$
	 	\task[\textbf{C.}] 1
	 	\task[\textbf{D.}] $n-1$
	 \end{tasks}
	 \item The eigenvalues of a Hermitian matrix are all
	 {\exyear{GATE 2018}}
	 \begin{tasks}(4)
	 	\task[\textbf{A.}]  Real
	 	\task[\textbf{B.}] Imaginary
	 	\task[\textbf{C.}] Of modulus one
	 	\task[\textbf{D.}] Real and positive
	 \end{tasks}
	 \item During a rotation, vectors along the axis of rotation remain unchanged. For the rotation matrix $\left(\begin{array}{ccc}0 & 1 & 0 \\ 0 & 0 & -1 \\ -1 & 0 & 0\end{array}\right)$, the vector along the axis of rotation is
	 {\exyear{GATE 2019}}
	 \begin{tasks}(2)
	 	\task[\textbf{A.}] $\frac{1}{3}(2 \hat{i}-\hat{j}+2 \hat{k})$
	 	\task[\textbf{B.}]  $\frac{1}{\sqrt{3}}(\hat{i}+\hat{j}-\hat{k})$
	 	\task[\textbf{C.}] $\frac{1}{\sqrt{3}}(\hat{i}-\hat{j}-\hat{k})$
	 	\task[\textbf{D.}] $\frac{1}{3}(2 \hat{i}+2 \hat{j}-\hat{k})$
	 \end{tasks}
\end{enumerate}
 \colorlet{ocre1}{ocre!70!}
\colorlet{ocrel}{ocre!30!}
\setlength\arrayrulewidth{1pt}
\begin{table}[H]
	\centering
	\arrayrulecolor{ocre}
	\begin{tabular}{|p{1.5cm}|p{1.5cm}||p{1.5cm}|p{1.5cm}|}
		\hline
		\multicolumn{4}{|c|}{\textbf{Answer key}}\\\hline\hline
		\rowcolor{ocrel}Q.No.&Answer&Q.No.&Answer\\\hline
		1&\textbf{C} &2&\textbf{C}\\\hline 
		3&\textbf{D} &4&\textbf{C} \\\hline
		5&\textbf{B} &6&\textbf{-} \\\hline
		7&\textbf{D}&8&\textbf{C}\\\hline
		9&\textbf{A}&10&\textbf{B}\\\hline
		
	\end{tabular}
\end{table}
\newpage
\begin{abox}
	Practice Set-3
	\end{abox}
\begin{enumerate}[label=\color{ocre}\textbf{\arabic*.}]
	\item Consider the matrices $X_{(4 \times 3)}, Y_{(4 \times 3)}$ and $P_{(2 \times 3)}$ The order of $\left[P\left(X^{T} Y\right)^{-1} P^{T}\right]^{T}$ will be
	
	\begin{tasks}(2)
		\task[\textbf{a.}] $(2 \times 2)$  
		\task[\textbf{b.}]$(3 \times 3)$
		\task[\textbf{c.}]$(4 \times 3)$ 
		\task[\textbf{d.}]$(3 \times 4)$ 
	\end{tasks}
	
	\begin{answer}
	We know that the order of
	\begin{align*}
	X \rightarrow 4 \times 3, Y &\rightarrow 4 \times 3 \text { and } P \rightarrow 2 \times 3\\
	\text{Hence, we can calculate the following orders:}\\
	X^{T} \rightarrow 3 \times 4, X^{T} Y &\rightarrow 3 \times 3 \quad\text{and}
	\\
	\left(X^{T} Y\right)^{-1} &\rightarrow 3 \times 3 \quad\text{also}\\
	P^{T} &\rightarrow 3 \times 2\\\therefore \left(P\left(X^{T} Y\right)^{-1} P^{T}\right)^{T} &\rightarrow 2 \times 2\\
	\text{and}\\P\left(X^{T} Y\right)^{-1}\\
	P^{T} \rightarrow(2 \times 3)(3 \times 3)(3 \times 2) &\rightarrow 2 \times 2\\
	\text{ Thus the order of the matrix is $2 \times 2$}
	\end{align*}
\end{answer}
\item What are the eigenvalues of the following $2 \times 2$ matrix?
$
\left[\begin{array}{rr}
2 & -1 \\
-4 & 5
\end{array}\right]
$
\begin{tasks}(2)
	\task[\textbf{a.}]-1 and 1  
	\task[\textbf{b.}]1 and 6
	\task[\textbf{c.}]2 and 5 
	\task[\textbf{d.}]4 and -1 
\end{tasks}	
\begin{answer}
	We have
	$$
	A=\left[\begin{array}{rr}
	2 & -1 \\
	-4 & 5
	\end{array}\right]
	$$
	The characteristic equation of this matrix is given by
	$$
	\begin{aligned}
	|A-\lambda I| &=0 \\
	\left|\begin{array}{rr}
	2-\lambda & -1 \\
	-4 & 5-\lambda
	\end{array}\right| &=0 \\
	(2-\lambda)(5-\lambda)-4 &=0 \\
	\lambda &=1,6
	\end{aligned}
	$$
	Therefore, the eigenvalues of $A$ are 1 and 6 .\\Hence, option(b)is correct.
\end{answer}
\item Given a $2 \times 2$ unitary matrix $U$ satisfying $U^{\dagger} U=U U^{\dagger}=1$ with $\operatorname{det} U=e^{i \varphi}$, one can construct a
unitary matrix $V\left(V^{\dagger} V=V V^{\dagger}=1\right)$ with $\operatorname{det} V=1$ from it by
\begin{tasks}(1)
	\task[\textbf{a.}]multiplying $U$ by $e^{-i \varphi / 2}$  
	\task[\textbf{b.}] multiplying any single element of $U$ by $e^{-i \varphi}$
	\task[\textbf{c.}]multiplying any row or column of $U$ by $e^{-i \varphi / 2}$ 
	\task[\textbf{d.}]multiplying $U$ by $e^{-i \varphi}$ 
\end{tasks}
\begin{answer}
	\begin{align*}
	\text{Let}\quad U=\left(\begin{array}{ll}a & b \\ c & d\end{array}\right)&\\\text{and}\quad \operatorname{det} U=a d-b c=e^{i \phi}( \text{Given })\\\text{Let,}\quad V=e^{-i \phi / 2} U \quad \Rightarrow \quad V=\left(\begin{array}{ll}e^{-i \phi / 2} a & e^{-i \phi / 2} b \\ e^{-i \phi / 2} c & e^{-i \phi / 2} d\end{array}\right)\\
	\operatorname{det} V=e^{-i \phi} a d-e^{-i \phi} b c=e^{-i \phi}(a d-b c)=e^{-i \phi} e^{i \phi}=1\\
	\text{We have to multiply U with}\quad {e}^{-\mathrm{i} \phi / 2}
	\text{to get,}\mathrm{V}\text{with determinant =1}
	\end{align*}
	Correct answer is (a)
	
\end{answer} 
\item Determine the values of $\alpha, \beta, \gamma$ when
$$
\left[\begin{array}{rrr}
0 & 2 \beta & \gamma \\
\alpha & \beta & -\gamma \\
\alpha & -\beta & \gamma
\end{array}\right] \text { is orthogonal. }
$$
\begin{answer}
	$$
	\text { Let } A=\left[\begin{array}{rrr}
	0 & 2 \beta & \gamma \\
	\alpha & \beta & -\gamma \\
	\alpha & -\beta & \gamma
	\end{array}\right]
	$$
	On transposing $A,$ we have
	$$
	A^{T}=\left[\begin{array}{rrr}
	0 & \alpha & \alpha \\
	2 \beta & \beta & -\beta \\
	\gamma & -\gamma & \gamma
	\end{array}\right]
	$$
	If $A$ is orthogonal, then $A A^{T}=I$
	Equating the corresponding elements, we have
	But
	$$
	\left.\begin{array}{r}
	4 \beta^{2}+\gamma^{2}=1 \\
	2 \beta^{2}-\gamma^{2}=0
	\end{array}\right\} \Rightarrow \beta=\pm \frac{1}{\sqrt{6}}, \gamma=\pm \frac{1}{\sqrt{3}}
	$$
	But $\quad \alpha^{2}+\beta^{2}+\gamma^{2}=1$ as $\beta=\pm \frac{1}{\sqrt{6}}, \gamma=\pm \frac{1}{\sqrt{3}}, \alpha=\pm \frac{1}{\sqrt{2}}$
\end{answer}
\item Find for what values of $\lambda$ and $\mu$ the system of linear equations.
$$
\begin{aligned}
x+y+z &=6 \\
x+2 y+5 z &=10 \\
2 x+3 y+\lambda z &=\mu
\end{aligned}
$$
\begin{tasks}(2)
	\task[\textbf{a.}] a unique solution  
	\task[\textbf{b.}]no solution
	\task[\textbf{c.}]infinite solutions 
\\Also find the solution for $\lambda=2$ and $\mu=8$.
\end{tasks}
\begin{answer}
$\begin{aligned}\left[\begin{array}{lll}1 & 1 & 1 \\ 1 & 2 & 5 \\ 2 & 3 & \lambda\end{array}\right]\left[\begin{array}{l}x \\ y \\ z\end{array}\right]=\left[\begin{array}{c}6 \\ 10 \\ \mu\end{array}\right] \\ A X &=B \end{aligned}$.\\\\
$\begin{aligned} C=(A, B) &=\left[\begin{array}{lllll}1 & 1 & 1 & : & 6 \\ 1 & 2 & 5 & : & 10 \\ 2 & 3 & \lambda & : & \mu\end{array}\right] \sim\left[\begin{array}{ccccc}1 & 1 & 1 & : & 6 \\ 0 & 1 & 4 & : & 4 \\ 0 & 1 & \lambda-2 & : & \mu-12\end{array}\right] \begin{array}{l}R_{2} \rightarrow R_{2}-R_{1} \\ R_{3} \rightarrow R_{3}-2 R_{1}\end{array} \\ & \sim\left[\begin{array}{ccccc}1 & 1 & 1 & : & 6 \\ 0 & 1 & 4 & : & 4 \\ 0 & 0 & \lambda-6 & : & \mu-16\end{array}\right] R_{3} \rightarrow R_{3}-R_{2} \end{aligned}$......(1)\\
\\\textbf{(i)} $\quad$ A unique solution If $R(A)=R(C)=3$
\\then $\lambda-6 \neq 0 \Rightarrow \lambda \neq 6$ and $\mu-16 \neq 0 \Rightarrow \mu \neq 16$
\\\\\textbf{(ii)} No solutions If $R(A) \neq R(C),$ \\then $R(A)=2$ and $R(C)=3$ $\lambda-6=0 \Rightarrow \lambda=6$ and $\mu-16 \neq 0 \Rightarrow \mu \neq 16$
\\\\\textbf{(iii)} Infinite solutions If $R(A)=R(C)=2$
\\then $\lambda-6=0$ and $\mu-16=0$
$\Rightarrow \quad \lambda=6$ and $\mu=16$\\
\\\\\textbf{(iv)} Putting $\lambda=2$ and $\mu=8$ in (1), we get
$$
\begin{aligned}
\left[\begin{array}{rrrrr}
1 & 1 & 1 & : & 6 \\
0 & 1 & 4 & : & 4 \\
0 & 0 & -4 & : & -8
\end{array}\right] & \Rightarrow\left[\begin{array}{rrr}
1 & 1 & 1 \\
0 & 1 & 4 \\
0 & 0 & -4
\end{array}\right]\left[\begin{array}{l}
x \\
y \\
z
\end{array}\right]=\left[\begin{array}{r}
6 \\
4 \\
-8
\end{array}\right] \\
& x+y+z=6 \\
& y+4 z=4 \\
&-4 z=-8 & \Rightarrow \quad z=2
\end{aligned}
$$
Putting $z=2$ in (3), we get
$$
y+8=4 \quad \Rightarrow \quad y=-4
$$
Putting $y=-4, z=2$ in (1), we get\\

$$
x-4+2=6 \quad \Rightarrow \quad x=8
$$
Hence, $\quad x=8, \quad y=-4, \quad z=2$
\end{answer}
\item Consider the following system of linear equations:
$$
\left[\begin{array}{rrr}
2 & 1 & -4 \\
4 & 3 & -12 \\
1 & 2 & -8
\end{array}\right]\left[\begin{array}{l}
x \\
y \\
z
\end{array}\right]=\left[\begin{array}{l}
\alpha \\
5 \\
7
\end{array}\right]
$$
Notice that the second and third columns of the
coefficient matrix are linearly dependent.For how many values of $\alpha,$ does this system of equations have infinitely many solutions?
\begin{tasks}(2)
	\task[\textbf{a.}]0  
	\task[\textbf{b.}]1
	\task[\textbf{c.}]2 
	\task[\textbf{d.}]Infinitely many 
\end{tasks}
\begin{answer}
 The given system of equations is
$$
\left[\begin{array}{rrr}
2 & 1 & -4 \\
4 & 3 & -12 \\
1 & 2 & -8
\end{array}\right]\left[\begin{array}{l}
x \\
y \\
z
\end{array}\right]=\left[\begin{array}{l}
\alpha \\
5 \\
7
\end{array}\right]
$$
The augmented matrix for the given system is
$$
\left[\begin{array}{rrr|r}
2 & 1 & -4 & \alpha \\
4 & 3 & -12 & 5 \\
1 & 2 & -8 & 7
\end{array}\right]
$$
For infinite solutions to exist, the rank of the augmented matrix should be less than the total unknown variables, i.e. 3 . Therefore,
$$
\begin{array}{l}
\left|\begin{array}{lll}
2 & 1 & \alpha \\
4 & 3 & 5 \\
1 & 2 & 7
\end{array}\right|=0 \\
\Rightarrow \alpha(8-3)-5(4-1)+7(6-4)=0 \\
\Rightarrow \alpha=\frac{1}{5}
\end{array}
$$
Therefore, there is only one value of $\alpha$ for which infinite solutions exist.	
\end{answer}
\item The eigen value of matrix $A=\left(\begin{array}{lll}1 & 0 & 1 \\ 0 & 1 & 0 \\ 1 & 0 & 1\end{array}\right)$ is
\begin{tasks}(2)
	\task[\textbf{a.}]$\lambda=1,0,2$  
	\task[\textbf{b.}]$\lambda=-1, \quad 2,2$
	\task[\textbf{c.}]$\lambda=0,0,3$ 
	\task[\textbf{d.}]$\lambda=1,1,1$ 
\end{tasks}
\begin{answer}
 For eigen value $|A-\lambda I|=0$
$$
\begin{array}{l}
\left|\begin{array}{ccc}
1-\lambda & 0 & 1-\lambda \\
0 & 1-\lambda & 0 \\
1 & 0 & 1-\lambda
\end{array}\right|=(1-\lambda)\left[(1-\lambda)^{2}-0\right]+0+1[-(1-\lambda)]=0 \\
\Rightarrow-\lambda^{3}+3 \lambda^{2}-2 \lambda=0 \Rightarrow \lambda=0,1,2
\end{array}
$$	
\end{answer}
\item  Which one of the following is the inverse of the matrix $\left(\begin{array}{cc}1 & -1 \\ 0 & 1\end{array}\right)?$
\begin{tasks}(2)
	\task[\textbf{a.}]$\left(\begin{array}{cc}1 & 1 \\ -1 & 1\end{array}\right)$  
	\task[\textbf{b.}]$\left(\begin{array}{ll}1 & 0 \\ 1 & 1\end{array}\right)$
	\task[\textbf{c.}] $\left(\begin{array}{ll}1 & 1 \\ 0 & 1\end{array}\right)$
	\task[\textbf{d.}]$\left(\begin{array}{cc}-1 & 1 \\ 0 & -1\end{array}\right)$ 
\end{tasks}
\begin{answer}
$A^{-1}=\frac{a d j A}{|A|}=\frac{1}{1}\left[\begin{array}{ll}1 & 1 \\ 0 & 1\end{array}\right]=\left[\begin{array}{ll}1 & 1 \\ 0 & 1\end{array}\right]$
\\Correct option is (c)	
\end{answer}
\item The eigenvalues and eigenvectors of the matrix $\left[\begin{array}{ll}5 & 4 \\ 1 & 2\end{array}\right]$ are
\begin{tasks}(2)
	\task[\textbf{a.}]6,1 and $\left[\begin{array}{c}4 \\ 1\end{array}\right],\left[\begin{array}{c}1 \\ -1\end{array}\right]$  
	\task[\textbf{b.}]2,5 and $\left[\begin{array}{c}4 \\ 1\end{array}\right],\left[\begin{array}{c}1 \\ -1\end{array}\right]$
	\task[\textbf{c.}] 6,1 and $\left[\begin{array}{c}4 \\ 1\end{array}\right],\left[\begin{array}{c}1 \\ -1\end{array}\right]$
	\task[\textbf{d.}]$(\mathrm{d}) 2,5$ $\left[\begin{array}{c}4 \\ 1\end{array}\right],\left[\begin{array}{c}1 \\ -1\end{array}\right]$ 
\end{tasks}
\begin{answer}
\begin{align*}
A&=\left[\begin{array}{ll}
5 & 4 \\
1 & 2
\end{array}\right] ;\\ \text { eigenvalue equation: }|A-\lambda I|&=0\\
\Rightarrow\left|\begin{array}{cc}
5-\lambda & 4 \\
1 & 2-\lambda
\end{array}\right|&=0 \quad\\ \Rightarrow(\lambda-2)(\lambda-5)-4&=0 \Rightarrow \lambda^{2}-7 \lambda+6=0 \Rightarrow \lambda=6,1\\
\begin{array}{l}
\text { Now, }(A-\lambda I) X=0 \\
\Rightarrow \quad\left(\begin{array}{cc}
5-\lambda & 4 \\
1 & 2-\lambda
\end{array}\right)\left(\begin{array}{c}4 \\ 1\end{array}\right),\left(\begin{array}{c}1 \\ -1\end{array}\right)=0
\end{array}
\end{align*}	
\end{answer}
\item $ \mathrm{A}\  3 \times 3$ matrix $M$ has $\operatorname{Tr}[M]=6, \operatorname{Tr}\left[M^{2}\right]=26, \operatorname{Tr}\left[M^{3}\right]=90 .$ Which of the following can be possible set of eigenvalues of M?
\begin{tasks}(2)
	\task[\textbf{a.}]\{1,1,4\}  
	\task[\textbf{b.}]\{-1,0,7\}
	\task[\textbf{c.}] \{-1,3,4\} 
	\task[\textbf{d.}]\{2,2,2\} 
\end{tasks}
\begin{answer}
$$\lambda_{1}+\lambda_{2}+\lambda_{3}=6 ; \quad \lambda_{1}^{2}+\lambda_{2}^{2}+\lambda_{3}^{2}=26 ; \lambda_{1}^{3}+\lambda_{2}^{3}+\lambda_{3}=90$$
\\It is trivial to check that only \{-1,3,4\} satisfies there three equation.	
\end{answer}

\end{enumerate}

