\chapter{Power series Solution and Special functions}
\section{Series Solution Method}
Series expansion is a  method of obtaining one solution of the linear, second-order, homogeneous ODE. The method, will always work, provided the point of expansion is no worse than a regular singular point.In physics this very gentle condition is almost always satisfied. 
A linear, second-order, homogeneous ODE can be written in the form
\begin{equation}
\frac{d^{2} y}{d x^{2}}+P(x) \frac{d y}{d x}+Q(x) y=0 \label{DE002}
\end{equation}
The most general solution of the equation \ref{DE002} may be written as,
\begin{equation}
y(x)=c_{1} y_{1}(x)+c_{2} y_{2}(x)
\end{equation}
But a physical problem may lead to a nonhomogeneous, linear, second-order ODE
\begin{equation}
\frac{d^{2} y}{d x^{2}}+P(x) \frac{d y}{d x}+Q(x) y=F(x)\label{DE003}
\end{equation}
Hence the most general solution to the equation \label{DE003} will be of the form,
\begin{equation}
y(x)=c_{1} y_{1}(x)+c_{2} y_{2}(x)+y_{p}(x)
\end{equation}
The constants $c_{1}$ and $c_{2}$ will eventually be fixed by boundary conditions.\\\\
There are two series solution method  for differential equation,
\begin{enumerate}
	\item \textbf{Simple series expansion method}
	\item \textbf{Frobenious Method}
\end{enumerate}
\subsection{Simple Power Series Expansion Method}
The simple series expansion method works for differential equations whose solutions are well-behaved at the expansion point $x = 0$.
This method can be illustrated by Linear classical oscillator problem
\subsection{Classical Linear Oscillator}
\begin{align}
\frac{d^{2} y}{d x^{2}}+\omega^{2} y&=0 \label{DE003}\\
\text{with known solutions} \ y&=\sin \omega x, \cos \omega x\\
\text{We try}\ y(x) &=x^{k}\left(a_{0}+a_{1} x+a_{2} x^{2}+a_{3} x^{3}+\cdots\right) \\
&=\sum_{\lambda=0}^{\infty} a_{\lambda} x^{k+\lambda}, \quad a_{0} \neq 0 \label{DE004}\\
\intertext{with the exponent $k$ and all the coefficients $a_{\lambda}$ still undetermined. Note that $k$ need not be an integer. By differentiating twice, we obtain}
\frac{d y}{d x} &=\sum_{\lambda=0}^{\infty} a_{\lambda}(k+\lambda) x^{k+\lambda-1} \\
\frac{d^{2} y}{d x^{2}} &=\sum_{\lambda=0}^{\infty} a_{\lambda}(k+\lambda)(k+\lambda-1) x^{k+\lambda-2}
\intertext{By substituting into equation.\ref{DE003}, we have}
\sum_{\lambda=0}^{\infty} a_{\lambda}(k+\lambda)(k+\lambda-1) x^{k+\lambda-2}+\omega^{2} \sum_{\lambda=0}^{\infty} a_{\lambda} x^{k+\lambda}&=0 \label{DE005}
\intertext{The coefficients of each power of $x$ on the left-hand side of equation.\ref{DE005} must vanish individually.The lowest power of $x$ appearing in equation.\ref{DE005} is $x^{k-2}$, for $\lambda=0$ in the first summation. The requirement that the coefficient vanish  yields,}
a_{0} k(k-1)&=0
\intertext{We had chosen $a_{0}$ as the coefficient of the lowest nonvanishing terms of the series \ref{DE004}, hence, by definition, $a_{0} \neq 0$. Therefore we have,}
k(k-1)&=0 \label{DE006}
\end{align}
\textbf{This equation, coming from the coefficient of the lowest power of $x$, we call the {indicial equation}.} The indicial equation and its roots are of critical importance to our analysis.
\\From equation.\ref{DE006}, \qquad $k=0 $ or $k=1$\\
The only way a power series can be zero is, it's coefficients must be equal to zero. But here the power of $x$ in the equation do not match up. The Coefficent of $x$ in the first term is,${k+\lambda-2} $ and for the second term it is,$k+\lambda$, to make them equal, we can replace $\lambda$ by $\lambda+2$ in the first term. Then we get,
\begin{align}
\sum_{\lambda=2}^{\infty} a_{\lambda+2}(k+\lambda+2)(k+\lambda+1) x^{k+\lambda}+\omega^{2} \sum_{\lambda=0}^{\infty} a_{\lambda} x^{k+\lambda}&=0\\
\sum_{\lambda=2}^{\infty} a_{\lambda+2}(k+\lambda+2)(k+\lambda+1) +\omega^{2} \sum_{\lambda=0}^{\infty} a_{\lambda} &=0
\intertext{Here the coefficients  are independent summations and $\lambda $ is a dummy index. Then we get,}
a_{\lambda+2}(k+\lambda+2)(k+\lambda+1) +\omega^{2} a_{\lambda} &=0\\
a_{\lambda+2}&=-a_{\lambda} \frac{\omega^{2}}{(k+\lambda+2)(k+\lambda+1)}\label{DE007}
\end{align}
For this example, if we start with $a_{0}$, Equation.\ref{DE007} leads to the even coefficients $a_{2}, a_{4}$, and so on, and ignores $a_{1}, a_{3}, a_{5}$, and so on. Since $a_{1}$ is arbitrary if $k=0$ and necessarily zero if $k=1$, 
$$
a_{3}=a_{5}=a_{7}=\cdots=0
$$
and all the odd-numbered coefficients vanish. The odd powers of $x$ will actually reappear when the second root of the indicial equation is used.
\begin{align}
a_{\lambda+2}&=-a_{\lambda} \frac{\omega^{2}}{(\lambda+2)(\lambda+1)}
\intertext{which leads to}
a_{2}&=-a_{0} \frac{\omega^{2}}{1 \cdot 2}=-\frac{\omega^{2}}{2 !} a_{0} \\
a_{4}&=-a_{2} \frac{\omega^{2}}{3 \cdot 4}=+\frac{\omega^{4}}{4 !} a_{0} \\
a_{6}&=-a_{4} \frac{\omega^{2}}{5 \cdot 6}=-\frac{\omega^{6}}{6 !} a_{0}, \quad \text { and so on. }
\intertext{By inspection (and mathematical induction),}
a_{2 n}&=(-1)^{n} \frac{\omega^{2 n}}{(2 n) !} a_{0}
\intertext{and our solution is}
y(x)_{k=0}&=a_{0}\left[1-\frac{(\omega x)^{2}}{2 !}+\frac{(\omega x)^{4}}{4 !}-\frac{(\omega x)^{6}}{6 !}+\cdots\right]\\&=a_{0} \cos \omega x\\
\intertext{If we choose the indicial equation root $k=1$ Equation.\ref{DE007}, the recurrence relation becomes}
a_{j+2}&=-a_{j} \frac{\omega^{2}}{(j+3)(j+2)}\\
\intertext{Substituting in $j=0,2,4$, successively, we obtain}
a_{2}=-a_{0} \frac{\omega^{2}}{2 \cdot 3}&=-\frac{\omega^{2}}{3 !} a_{0} \\
a_{4}=-a_{2} \frac{\omega^{2}}{4 \cdot 5}&=+\frac{\omega^{4}}{5 !} a_{0} \\
a_{6}=-a_{4} \frac{\omega^{2}}{6 \cdot 7}&=-\frac{\omega^{6}}{7 !} a_{0}, \quad \text { and so on. }
\intertext{Again, by inspection and mathematical induction,}
a_{2 n}&=(-1)^{n} \frac{\omega^{2 n}}{(2 n+1) !} a_{0}\\
\intertext{For this choice, $k=1$, we obtain}
y(x)_{k=1} &=a_{0} x\left[1-\frac{(\omega x)^{2}}{3 !}+\frac{(\omega x)^{4}}{5 !}-\frac{(\omega x)^{6}}{7 !}+\cdots\right] \\
&=\frac{a_{0}}{\omega}\left[(\omega x)-\frac{(\omega x)^{3}}{3 !}+\frac{(\omega x)^{5}}{5 !}-\frac{(\omega x)^{7}}{7 !}+\cdots\right] \\
&=\frac{a_{0}}{\omega} \sin \omega x
\end{align}
\subsubsection{Power Series Solution (About an Ordinary Point)}
Find the power series solution of $\left(1-x^{2}\right) y^{\prime \prime}-2 x y^{\prime}+2 y=0$ about $x=0$\\\\
Since $x=0$ is an ordinary point of the given differential equation, the solution can be written as
\begin{align*}
y&=\sum_{k=0}^{\infty} a_{k} x^{k} \\ \frac{d y}{d x}&=\sum_{k=0}^{\infty} k a_{k} x^{k-1} \\ \frac{d^{2} y}{d x^{2}}&=\sum_{k=0}^{\infty} a_{k} k(k-1) x^{k-2}
\intertext{Substituting these values in the given equation we get,}
\left(1-x^{2}\right) \sum_{k} a_{k} k(k-1) x^{k-2}&-2 x \sum_{k} a_{k}(k) x^{k-1}+2 \sum_{k} a_{k} x^{k}=0 \\
\sum_{k=2} a_{k} k(k-1) x^{k-2}&-\sum\left(k^{2}+k-2\right) a_{k} x^{k}=0
\intertext{now equating the coefficient of $x^{k}$ then}
(k+2)(k+1) a_{k+2}-\left(k^{2}+k-2\right) a_{k}&=0 \\a_{k+2}&=\frac{k-1}{(k+1)} a_{k}\\
\text{For} \ k&=0 \Rightarrow a_{2}=-a_{0} \\ k&=1 \Rightarrow a_{3}=0 \\
k&=2 \Rightarrow a_{4}=\frac{a_{2}}{3}=\frac{-a_{0}}{3}  \\ k&=3 \Rightarrow a_{5}=\frac{2}{4} a_{3}=0\\
\text{Therefore, solution}\ y&=a_{0}+a_{1} x+a_{2} x^{2}+\ldots \ldots\\&=a_{0}\left[1-x^{2}-\frac{x^{4}}{3} \ldots . .\right]+a_{1} x
\end{align*}
\subsection{Frobenious Method}
Even though the simple power series expansion method works for many functions there are some whose behaviour  precludes the simple series method like the Bessel's function. The need of Frobenious method  lies under the fact that, \textbf{any functions involving negative or fractional powers would not be amenable to power series solution method}. The Frobenious method extends the simple power series solution method to include negative and fractional powers, and it also allows a natural extension involving logarithm terms.\\
The basic idea of the Frobenius method is to look for solutions of the form
\begin{align*}
y(x) &=a_{0} x^{\lambda}+a_{1} x^{\lambda+1}+a_{2} x^{\lambda+2}+a_{3} x^{\lambda+3}+\ldots \\
&=x^{\lambda}\left(a_{0}+a_{1} x+a_{2} x^{2}+a_{3} x^{3}+\ldots\right) \\
&=x^{\lambda} \sum_{k=0}^{\infty} a_{k} x^{k} \\
&= \sum_{k=0}^{\infty} a_{k} x^{k+\lambda}
\end{align*}
The extension of the simple power series method is all in the factor $x^{\lambda}$. The power $c$ must now be determined, as well as the coefficients $a_{k}$. Since $\lambda$ may be negative, positive, and possibly non-integral, this extends considerably the range of functions which may be treated. Note that $a_{0}$ is the lowest non-zero coefficient, so by definition it cannot be zero.
\subsection{Bessel Function}
\newpage
\begin{abox}
	Problem Set -1
\end{abox}
\begin{enumerate}[label=\color{ocre}\textbf{\arabic*.}]
	\item  Let $p_{n}(x)$ (where $n=0,1,2, \ldots \ldots$ ) be a polynomial of degree $n$ with real coefficients, defined in the interval $2 \leq n \leq 4$. If $\int_{2}^{4} p_{n}(x) p_{m}(x) d x=\delta_{n m}$, then
	{\exyear{NET/JRF(JUNE-2011)}}
	\begin{tasks}(2)
		\task[\textbf{A.}] $p_{0}(x)=\frac{1}{\sqrt{2}}$ and $p_{1}(x)=\sqrt{\frac{3}{2}}(-3-x)$
		\task[\textbf{B.}]  $p_{0}(x)=\frac{1}{\sqrt{2}}$ and $p_{1}(x)=\sqrt{3}(3+x)$
		\task[\textbf{C.}] $p_{0}(x)=\frac{1}{2}$ and $p_{1}(x)=\sqrt{\frac{3}{2}}(3-x)$
		\task[\textbf{D.}] $p_{0}(x)=\frac{1}{\sqrt{2}}$ and $p_{1}(x)=\sqrt{\frac{3}{2}}(3-x)$
	\end{tasks}
	\item  The generating function $F(x, t)=\sum_{n=0}^{\infty} P_{n}(x) t^{n}$ for the Legendre polynomials $P_{n}(x)$ is $F(x, t)=\left(1-2 x t+t^{2}\right)^{-1 / 2}$. The value of $P_{3}(-1)$ is
	{\exyear{NET/JRF(DEC-2011)}}
	\begin{tasks}(4)
		\task[\textbf{A.}] $5 / 2$
		\task[\textbf{B.}] $3 / 2$
		\task[\textbf{C.}] $+1$
		\task[\textbf{D.}] $-1$
	\end{tasks}
	\item  The graph of the function $f(x)$ shown below is best described by
	{\exyear{NET/JRF(DEC-2012)}}
	\begin{figure}[H]
		\centering
		\includegraphics[height=6cm,width=8cm]{diagram-20211005(12)-crop}
	\end{figure}
	\begin{tasks}(2)
		\task[\textbf{A.}]  The Bessel function $J_{0}(x)$
		\task[\textbf{B.}] $\cos x$
		\task[\textbf{C.}] $e^{-x} \cos x$
		\task[\textbf{D.}] $\frac{1}{x} \cos x$
	\end{tasks}
	\item Given that $\sum_{n=0}^{\infty} H_{n}(x) \frac{t^{n}}{n !}=e^{-t^{2}+2 t x}$ the value of $H_{4}(0)$ is
	{\exyear{NET/JRF(JUNE-2013)}}
	\begin{tasks}(4)
		\task[\textbf{A.}] 12
		\task[\textbf{B.}] 6
		\task[\textbf{C.}] 24
		\task[\textbf{D.}] $-6$
	\end{tasks}
	\item   Given $\sum_{n=0}^{\infty} P_{n}(x) t^{n}=\left(1-2 x t+t^{2}\right)^{-1 / 2}$, for $|t|<1$, the value of $P_{5}(-1)$ is
	{\exyear{NET/JRF(JUNE-2014)}}
	\begin{tasks}(4)
		\task[\textbf{A.}] $0.26$
		\task[\textbf{B.}] 1
		\task[\textbf{C.}] $0.5$
		\task[\textbf{D.}] $-1$
	\end{tasks}
	\item The function $f(x)=\sum_{n=0}^{\infty} \frac{(-1)^{n}}{n !(n+1) !}\left(\frac{x}{2}\right)^{2 n+1}$, satisfies the differential equation
	{\exyear{NET/JRF(DEC-2014)}}
	\begin{tasks}(2)
		\task[\textbf{A.}]  $x^{2} \frac{d^{2} f}{d x^{2}}+x \frac{d f}{d x}+\left(x^{2}+1\right) f=0$
		\task[\textbf{B.}]  $x^{2} \frac{d^{2} f}{d x^{2}}+2 x \frac{d f}{d x}+\left(x^{2}-1\right) f=0$
		\task[\textbf{C.}] $x^{2} \frac{d^{2} f}{d x^{2}}+x \frac{d f}{d x}+\left(x^{2}-1\right) f=0$
		\task[\textbf{D.}] $x^{2} \frac{d^{2} f}{d x^{2}}-x \frac{d f}{d x}+\left(x^{2}-1\right) f=0$
	\end{tasks}
	\item
	 The Hermite polynomial $H_{n}(x)$, satisfies the differential equation
	$$
	\frac{d^{2} H_{n}}{d x^{2}}-2 x \frac{d H_{n}}{d x}+2 n H_{n}(x)=0
	$$
	The corresponding generating function $G(t, x)=\sum_{n=0}^{\infty} \frac{1}{n !} H_{n}(x) t^{n}$, satisfies the equation
	{\exyear{NET/JRF(DEC-2015)}}
	\begin{tasks}(2)
		\task[\textbf{A.}] $\frac{\partial^{2} G}{\partial x^{2}}-2 x \frac{\partial G}{\partial x}+2 t \frac{\partial G}{\partial t}=0$
		\task[\textbf{B.}] $\frac{\partial^{2} G}{\partial x^{2}}-2 x \frac{\partial G}{\partial x}-2 t^{2} \frac{\partial G}{\partial t}=0$
		\task[\textbf{C.}] $\frac{\partial^{2} G}{\partial x^{2}}-2 x \frac{\partial G}{\partial x}+2 \frac{\partial G}{\partial t}=0$
		\task[\textbf{D.}]  $\frac{\partial^{2} G}{\partial x^{2}}-2 x \frac{\partial G}{\partial x}+2 \frac{\partial^{2} G}{\partial x \partial t}=0$
	\end{tasks}
	\item A stable asymptotic solution of the equation $x_{n+1}=1+\frac{3}{1+x_{n}}$ is $x=2$. If we take $x_{n}=2+\epsilon_{n}$ and $x_{n+1}=2+\epsilon_{n+1}$, where $\epsilon_{n}$ and $\epsilon_{n+1}$ are both small, the ratio $\frac{\epsilon_{n+1}}{\epsilon_{n}}$ is approximately
	{\exyear{NET/JRF(DEC-2016)}}
	\begin{tasks}(4)
		\task[\textbf{A.}] $-\frac{1}{2}$
		\task[\textbf{B.}] $-\frac{1}{4}$
		\task[\textbf{C.}]  $-\frac{1}{3}$
		\task[\textbf{D.}] $-\frac{2}{3}$
	\end{tasks}
	\item  The Green's function satisfying
	$$
	\frac{d^{2}}{d x^{2}} g\left(x, x_{0}\right)=\delta\left(x-x_{0}\right)
	$$
	with the boundary conditions $g\left(-L, x_{0}\right)=0=g\left(L, x_{0}\right)$, is
	{\exyear{NET/JRF(JUNE-2017)}}
	\begin{tasks}(1)
		\task[\textbf{A.}] $\left\{\begin{array}{ll}\frac{1}{2 L}\left(x_{0}-L\right)(x+L), & -L \leq x<x_{0} \\ \frac{1}{2 L}\left(x_{0}+L\right)(x-L), & x_{0} \leq x \leq L\end{array}\right.$
		\task[\textbf{B.}]  $\left\{\begin{array}{ll}\frac{1}{2 L}\left(x_{0}+L\right)(x+L), & -L \leq x<x_{0} \\ \frac{1}{2 L}\left(x_{0}-L\right)(x-L), & x_{0} \leq x \leq L\end{array}\right.$
		\task[\textbf{C.}] $\left\{\begin{array}{ll}\frac{1}{2 L}\left(L-x_{0}\right)(x+L), & -L \leq x<x_{0} \\ \frac{1}{2 L}\left(x_{0}+L\right)(L-x), & x_{0} \leq x \leq L\end{array}\right.$
		\task[\textbf{D.}] $\frac{1}{2 L}(x-L)(x+L), \quad-L \leq x \leq L$
	\end{tasks}
	\item  The generating function $G(t, x)$ for the Legendre polynomials $P_{n}(t)$ is
	$$
	G(t, x)=\frac{1}{\sqrt{1-2 x t+x^{2}}}=\sum_{n=0}^{\infty} x^{n} P_{n}(t), \text { for }|x|<1
	$$
	If the function $f(x)$ is defined by the integral equation $\int_{0}^{x} f\left(x^{\prime}\right) d x^{\prime}=x G(1, x)$, it can be expressed as
	{\exyear{NET/JRF(DEC-2017)}}
	\begin{tasks}(2)
		\task[\textbf{A.}] $\sum_{n, m=0}^{\infty} x^{n+m} P_{n}(1) P_{m}\left(\frac{1}{2}\right)$
		\task[\textbf{B.}] $\sum_{n, m=0}^{\infty} x^{n+m} P_{n}(1) P_{m}(1)$
		\task[\textbf{C.}] $\sum_{n, m=0}^{\infty} x^{n-m} P_{n}(1) P_{m}(1)$
		\task[\textbf{D.}] $\sum_{n, m=0}^{\infty} x^{n-m} P_{n}(0) P_{m}(1)$
	\end{tasks}
	\item In the function $P_{n}(x) e^{-x^{2}}$ of a real variable $x, P_{n}(x)$ is polynomial of degree $n$. The maximum number of extrema that this function can have is
	{\exyear{NET/JRF(JUNE-2018)}}
	\begin{tasks}(4)
		\task[\textbf{A.}] $n+2$
		\task[\textbf{B.}]  $n-1$
		\task[\textbf{C.}] $n+1$
		\task[\textbf{D.}] $n$
	\end{tasks}
	\item  The Green's function $G\left(x, x^{\prime}\right)$ for the equation $\frac{d^{2} y(x)}{d x^{2}}+y(x)=f(x)$, with the boundary values $y(0)=y\left(\frac{\pi}{2}\right)=0$, is
	{\exyear{NET/JRF(JUNE-2018)}}
	\begin{tasks}(1)
		\task[\textbf{A.}] $G\left(x, x^{\prime}\right)=\left\{\begin{array}{ll}x\left(x^{\prime}-\frac{\pi}{2}\right), & 0<x<x^{\prime}<\frac{\pi}{2} \\ \left(x-\frac{\pi}{2}\right) x^{\prime}, & 0<x^{\prime}<x<\frac{\pi}{2}\end{array}\right.$
		\task[\textbf{B.}] $G\left(x, x^{\prime}\right)=\left\{\begin{array}{ll}-\cos x^{\prime} \sin x, & 0<x<x^{\prime}<\frac{\pi}{2} \\ -\sin x^{\prime} \cos x, & 0<x^{\prime}<x<\frac{\pi}{2}\end{array}\right.$
		\task[\textbf{C.}] $G\left(x, x^{\prime}\right)=\left\{\begin{array}{ll}\cos x^{\prime} \sin x, & 0<x<x^{\prime}<\frac{\pi}{2} \\ \sin x^{\prime} \cos x, & 0<x^{\prime}<x<\frac{\pi}{2}\end{array}\right.$
		\task[\textbf{D.}] $G\left(x, x^{\prime}\right)=\left\{\begin{array}{ll}x\left(\frac{\pi}{2}-x^{\prime}\right), & 0<x<x^{\prime}<\frac{\pi}{2} \\ x^{\prime}\left(\frac{\pi}{2}-x\right), & 0<x^{\prime}<x<\frac{\pi}{2}\end{array}\right.$
	\end{tasks}
	\item The polynomial $f(x)=1+5 x+3 x^{2}$ is written as linear combination of the Legendre polynomials
	$\left(P_{0}(x)=1, P_{1}(x), P_{2}(x)=\frac{1}{2}\left(3 x^{2}-1\right)\right)$ as $f(x)=\sum_{n} c_{n} P_{n}(x)$. The value of $c_{0}$ is
	{\exyear{NET/JRF(DEC-2018)}}
	\begin{tasks}(4)
		\task[\textbf{A.}] $\frac{1}{4}$
		\task[\textbf{B.}] $\frac{1}{2}$
		\task[\textbf{C.}]  2
		\task[\textbf{D.}]  4
	\end{tasks}
	\item The Green's function $G\left(x, x^{\prime}\right)$ for the equation $\frac{d^{2} y(x)}{d x^{2}}=f(x)$, with the boundary values $y(0)=0$ and $y(1)=0$, is
	{\exyear{NET/JRF(DEC-2018)}}
	\begin{tasks}(1)
		\task[\textbf{A.}] $G\left(x, x^{\prime}\right)=\left\{\begin{array}{ll}\frac{1}{2} x\left(1-x^{\prime}\right), & 0<x<x^{\prime}<1 \\ \frac{1}{2} x^{\prime}(1-x) & 0<x^{\prime}<x<1\end{array}\right.$
		\task[\textbf{B.}] $G\left(x, x^{\prime}\right)=\left\{\begin{array}{ll}x\left(x^{\prime}-1\right), & 0<x<x^{\prime}<1 \\ x^{\prime}(1-x) & 0<x^{\prime}<x<1\end{array}\right.$
		\task[\textbf{C.}] $G\left(x, x^{\prime}\right)=\left\{\begin{array}{ll}-\frac{1}{2} x\left(1-x^{\prime}\right), & 0<x<x^{\prime}<1 \\ \frac{1}{2} x^{\prime}(1-x) & 0<x^{\prime}<x<1\end{array}\right.$
		\task[\textbf{D.}]  $G\left(x, x^{\prime}\right)=\left\{\begin{array}{ll}x\left(x^{\prime}-1\right), & 0<x<x^{\prime}<1 \\ x^{\prime}(x-1) & 0<x^{\prime}<x<1\end{array}\right.$
	\end{tasks}
	\item  The Green's function for the differential equation $\frac{d^{2} x}{d t^{2}}+x=f(t)$, satisfying the initial conditions $x(0)=\frac{d x}{d t}(0)=0$ is\\
	$$G(t, \tau)=\left\{\begin{array}{ll}0 & \text { for } \quad 0<t<\tau \\ \sin (t-\tau) & \text { for } \quad t>\tau\end{array}\right.$$\\
	The solution of the differential equation when the source $f(t)=\theta(t)$ (the Heaviside step function) is
	{\exyear{NET/JRF(JUNE-2020)}}
	\begin{tasks}(4)
		\task[\textbf{A.}] $\sin t$
		\task[\textbf{B.}] $1-\sin t$
		\task[\textbf{C.}] $1-\cos t$
		\task[\textbf{D.}] $\cos ^{2} t-1$
	\end{tasks}
\end{enumerate}
 \colorlet{ocre1}{ocre!70!}
\colorlet{ocrel}{ocre!30!}
\setlength\arrayrulewidth{1pt}
\begin{table}[H]
	\centering
	\arrayrulecolor{ocre}
	\begin{tabular}{|p{1.5cm}|p{1.5cm}||p{1.5cm}|p{1.5cm}|}
		\hline
		\multicolumn{4}{|c|}{\textbf{Answer key}}\\\hline\hline
		\rowcolor{ocrel}Q.No.&Answer&Q.No.&Answer\\\hline
		1&\textbf{D} &2&\textbf{D}\\\hline 
		3&\textbf{A} &4&\textbf{A} \\\hline
		5&\textbf{D} &6&\textbf{C} \\\hline
		7&\textbf{A}&8&\textbf{C}\\\hline
		9&\textbf{A}&10&\textbf{B}\\\hline
		11&\textbf{C} &12&\textbf{B}\\\hline
		13&\textbf{C}&14&\textbf{D}\\\hline
		15&\textbf{C}& &\\\hline
		
	\end{tabular}
\end{table}
\begin{abox}
	Problem Set -3
\end{abox}
\begin{enumerate}[label=\color{ocre}\textbf{\arabic*.}]
	\item 48Green function for time dependent Schrödinger wave equation is defined as $G\left(\vec{r}, t: r^{\prime}, t^{\prime}\right)$. If $H$ is Hamiltonion of system then $G\left(\vec{r}, t: r^{\prime}, t^{\prime}\right)$ will satisfied the equation
	 \begin{tasks}(1)
		\task[\textbf{a.}]$\left(i \hbar \frac{\partial}{\partial t}-H\right) G\left(\vec{r}, t ; \vec{r}^{\prime}, t^{\prime}\right)=0$
		\task[\textbf{b.}]$\left(i \hbar \frac{\partial}{\partial t}-H\right) G\left(\vec{r}, t ; \vec{r}^{\prime}, t^{\prime}\right)=\delta\left(\vec{r}-\vec{r}^{\prime}\right)$
		\task[\textbf{c.}] $\left(i \hbar \frac{\partial}{\partial t}-H\right) G\left(\vec{r}, t ; \vec{r}^{\prime}, t^{\prime}\right)=\delta\left(t-t^{\prime}\right)$
		\task[\textbf{d.}]  $\left(i \hbar \frac{\partial}{\partial t}-H\right) G\left(\vec{r}, t ; \vec{r}^{\prime}, t^{\prime}\right)=\delta\left(\vec{r}-\vec{r}^{\prime}\right) \delta\left(t-t^{\prime}\right)$
	\end{tasks}
\begin{answer}
So the correct answer is \textbf{Option (d)}
\end{answer}
	\item 49$G\left(x, x_{0}\right)$ is the Green's function associated with the boundary value problem consisting of ordinary differential equation.
	$$
	\frac{d}{d x}\left(p(x) \frac{d u}{d x}\right)=f(x) \text { with } u(0)=0, u(L)=0
	$$
	The discontinuity condition on the derivative $\frac{d G\left(x, x_{0}\right)}{d x}$ at $x=x_{0}$ is
	 \begin{tasks}(4)
		\task[\textbf{a.}]0
		\task[\textbf{b.}]$p\left(x_{0}\right)$
		\task[\textbf{c.}]1
		\task[\textbf{d.}] $\frac{1}{p\left(x_{0}\right)}$
	\end{tasks}
\begin{answer}
	\begin{align*}
	\left.\frac{d G}{d x}\right|_{x=x_{0}^{+}}-\left.\frac{d G}{d x}\right|_{x=x_{i 1}^{-}}=\frac{1}{p\left(x_{0}\right)}
	\end{align*}
	So the correct answer is \textbf{Option (d)}
\end{answer}
\item 50Consider the steady state heat equation $\frac{d^{2} u}{d x^{2}}=f(x)$ with boundary condition,
$$
u(0)=0, u(L)=0
$$
The Green's function associated with the above equation
 \begin{tasks}(2)
	\task[\textbf{a.}]Constant
	\task[\textbf{b.}] Linear function
	\task[\textbf{c.}] Parabolic function
	\task[\textbf{d.}] Hyperbolic function
\end{tasks}
\begin{answer}
	\begin{align*}
	\intertext{The Green's function satisfies}
	\frac{d^{2} G\left(x, x_{0}\right)}{d x^{2}}&=\delta\left(x-x_{0}\right)\\
\text{	with }G\left(0, x_{0}\right)&=0\text{ and }G\left(L, x_{0}\right)=0
\intertext{Corresponding homogeneous equation is:}
\frac{d^{2} G}{d x^{2}}&=0\\
\text{Solution for }x \neq x_{0}&\text{ are, }G\left(x, x_{0}\right)= \begin{cases}a+b x_{2} & x<x_{1+} \\ c+d x, & x>x_{0}\end{cases}
	\end{align*}
		So the correct answer is \textbf{Option (b)}
\end{answer}
\item 51Consider the steady state heat equation $\frac{d^{2} u}{d x^{2}}=f(x)$ with boundary condition. $u(0)=0, u(L)=0$
The Green's function associated with the above equation is
 \begin{tasks}(1)
	\task[\textbf{a.}] $G\left(x, x_{0}\right)= \begin{cases}\frac{x}{L}\left(x_{0}-L\right), & 0 \leq x \leq x_{0} \\ \frac{x_{0}}{L}(x-L), & x_{0} \leq x \leq L\end{cases}$
	\task[\textbf{b.}] $G\left(x, x_{0}\right)= \begin{cases}\frac{x}{L}\left(L-x_{0}\right), & 0 \leq x \leq x_{0} \\ \frac{x_{0}}{L}(L-x), & x_{0} \leq x \leq L\end{cases}$
	\task[\textbf{c.}] $G\left(x, x_{0}\right)= \begin{cases}\sqrt{\frac{x}{L},} &\quad 0 \leq x \leq x_{0} \\ \sqrt{\frac{(x-L)}{L}}, & \quad x_{0} \leq x \leq L\end{cases}$
	\task[\textbf{d.}] $G\left(x, x_{0}\right)= \begin{cases}\sqrt{\frac{L-x}{L},}, & 0 \leq x \leq x_{0} \\ \sqrt{\frac{(x)}{L}}, & x_{0} \leq x \leq L\end{cases}$
\end{tasks}
\begin{answer}
	\begin{align*}
	\intertext{The Green's function satisfies}
	\frac{d^{2} G\left(x, x_{0}\right)}{d x^{2}}&=\delta\left(x-x_{0}\right)\\
	\text{with }G\left(0, x_{0}\right)&=0\text{ and }G\left(L, x_{0}\right)=0
	\intertext{Corresponding homogeneous equation is:}
	\frac{d^{2} G}{d x^{2}}&=0\\
	\text{Solution for }&x \neq x_{0}\text{ are}\\
	G\left(x, x_{0}\right)&= \begin{cases}a+b x, & x<x_{0} \\ c+d x, & x>x_{0}\end{cases}
	\intertext{From boundary conditions:}
	G\left(0, x_{0}\right)&=0 \Rightarrow a=0\\
	G\left(L, x_{0}\right)&=0 \Rightarrow c=-d L\\
	\therefore G\left(x, x_{0}\right)&= \begin{cases}b x, & x<x_{0} \\ d(x-L), & x>x_{0}\end{cases}\\
	\text{From continuity of }&\text{Green's function at }x=x_{0},\text{ we have}\\
	b x_{0}&=d\left(x_{0}-L\right)\\
	b&=\frac{d\left(x_{0}-L\right)}{x_{0}}\\
	\text{From discontinuity of }&\frac{\partial G}{\partial x}\text{ at }x=x_{0}\text{, we have}\\
	\left.\frac{\partial G}{\partial x}\right|&_{x=x_{0}^{+}}-\left.\frac{\partial G}{\partial x}\right|_{x=x_{0}^{-}}=1\\
	d-b&=1\\
	\Rightarrow d&=b+1 \Rightarrow d=\frac{d\left(x_{0}-L\right)}{x_{0}}+1 \Rightarrow d x_{0}=d x_{0}-d L+x_{0}\\
	\Rightarrow d&=\frac{x_{0}}{L}, b=d-1=\left(\frac{x_{0}}{L}-1\right)\\
	\therefore G\left(x, x_{0}\right)&= \begin{cases}\frac{x}{L}\left(x_{0}-L\right), & 0 \leq x \leq x_{0} \\ \frac{x_{0}}{L}(x-L), & x_{0} \leq x \leq L\end{cases}
	\end{align*}
		So the correct answer is \textbf{Option (a)}
\end{answer}
\item 52The differential equation defined as $\frac{d^{2} y}{d x^{2}}=f(x)$ With boundary conditions $\quad y(0)=0$ and $y^{\prime}(1)=0$
The green function $G\left(x, x_{0}\right)$ satisfy the
 \begin{tasks}(2)
	\task[\textbf{a.}]$G\left(x, x_{0}\right)= \begin{cases}x & \text { if } x<x_{0} \\ x_{0} & \text { if } x>x_{0}\end{cases}$
	\task[\textbf{b.}]$G\left(x, x_{0}\right)= \begin{cases}-x & \text { if } x<x_{0} \\ -x_{0} & \text { if } x>x_{0}\end{cases}$
	\task[\textbf{c.}]$G\left(x, x_{0}\right)= \begin{cases}x^{2} & \text { if } x<x_{0} \\ -x_{0} & \text { if } x>x_{0}\end{cases}$
	\task[\textbf{d.}] $G\left(x, x_{0}\right)= \begin{cases}-x^{2} & \text { if } x<x_{0} \\ -x_{0} & \text { if } x>x_{0}\end{cases}$
\end{tasks}
\begin{answer}
	\begin{align}
	\intertext{The corresponding non-homogenous differential equation for Green's function is}\notag\\
	\frac{\partial^{2}}{\partial x^{2}} G\left(x, x_{0}\right)&=\delta\left(x-x_{0}\right)\\
	\text{With }G\left(0, x_{0}\right)&=0\text{ and }G^{\prime}\left(1, x_{0}\right)=0\notag\notag\\
\text{	Let }&\frac{\partial^{2}}{\partial x^{2}} G\left(x, x_{0}\right)=0\notag\\
\Rightarrow G\left(x, x_{0}\right)&= \begin{cases}A x+B, & x<x_{0} \\ C x+D, & x>x_{0}\end{cases}\label{SF-01}
\intertext{Using booundary condition, we have}\notag\\
B&=0\text{ and }C=0\notag\\
\therefore&\text{ equation (\ref{SF-01}) becomes}\notag\\
G\left(x, x_{b}\right)&= \begin{cases}A x, & x<x_{0} \\ D, & x>x_{i 1}\end{cases}\notag\\
\text{From continuity of }&\left(f\left(x, x_{0}\right)\right.\text{ at }x=x_{0}\text{, we have}\notag\\
A x_{0}&=D
\intertext{From discontinuity of first derivative of Green's function i.c. $\frac{\partial G}{\partial x}$ at $x=x_{0}$ we have}
\left.\frac{\partial G}{\partial x}\right|_{x=x_{0}^{+}}-\left.\frac{\partial G}{\partial x}\right|&_{x=x_{0}^{-}}=1\notag\\
\Rightarrow 0-A&=1 \Rightarrow A=-1\notag\\
\text{and }D&=-x_{0}\notag\\
\therefore G\left(x, x_{0}\right)&= \begin{cases}-x & \text { if } x<x_{0} \notag\\ -x_{0} & \text { if } x>x_{0}\end{cases}
	\end{align}
	So the correct answer is \textbf{Option (b)}
\end{answer}
\item 53For real $n$ the cylindrical Bessel function of order $n$ is $J_{n}(x)$ then $J_{1 / 2}$ will converge to
 \begin{tasks}(4)
	\task[\textbf{a.}]0
	\task[\textbf{b.}]1
	\task[\textbf{c.}] $-1$
	\task[\textbf{d.}] $\frac{1}{2}$
\end{tasks}
\begin{answer}
	\begin{align*}
	{{\color{red}{Not completed}}}\\
	\end{align*}
	So the correct answer is \textbf{Option (a)}
\end{answer}
\item 54For real $n$ the cylindrical Bessel function is $J_{n}(x)$ of order $n$ then behavior $J_{1 / 2}$ will behave $x \approx 0$ as
 \begin{tasks}(4)
	\task[\textbf{a.}] 0
	\task[\textbf{b.}]$\sqrt{\frac{2 x}{\pi}}$
	\task[\textbf{c.}]$\sqrt{\frac{x}{\pi}}$
	\task[\textbf{d.}]  $\sqrt{\frac{x}{2 \pi}}$
\end{tasks}
\begin{answer}
	\begin{align*}
	{{\color{red}{Not completed}}}\\
	J_{n}(x)&=\sum_{0}^{\infty} \frac{(-1)^{r}}{[r \mid n+r}\left(\frac{x}{2}\right)^{n+2 r} \Rightarrow J_{1 / 2}(x)=\sum_{0}^{\infty} \frac{(-1)^{r}}{\left\lfloor\frac{1}{2}+r\right.}\left(\frac{x}{2}\right)^{\frac{1}{2}+2 r}\\
	\text{Put }r&=0 \frac{\sqrt{x / 2}}{\frac{1}{2}}=\sqrt{\frac{2 x}{\pi}} \text{where }\frac{1}{2}=\frac{\sqrt{\pi}}{2}
	\end{align*}
		So the correct answer is \textbf{Option (b)}
\end{answer}
\item 55For real $n$ the cylindrical Bessel function is $J_{n}(x)$ of order $n$ then behavior $J_{1 / 2}$ will equivalent to (it is given that $\underline{r} \cdot\left\lfloor r-\frac{1}{2}=\left[(2 r) 2^{-r} \sqrt{\pi}\right)\right.$
 \begin{tasks}(4)
	\task[\textbf{a.}] $\sqrt{\frac{2}{\pi}} \frac{\sin x}{\sqrt{x}}$
	\task[\textbf{b.}]$\sqrt{\frac{2}{\pi}} \frac{\sin x}{x}$
	\task[\textbf{c.}]$\sqrt{\frac{2}{\pi}} \frac{\cos x}{\sqrt{x}}$
	\task[\textbf{d.}] $\sqrt{\frac{2}{\pi}} \frac{\cos }{x}$
\end{tasks} 
\begin{answer}
	\begin{align*}
	{{\color{red}{Not completed}}}\\
	\end{align*}
\end{answer}
\item 56For real $n$ the cylindrical Bessel function is $J_{n}(x)$ of order $n$ then $J_{n}(x)$ will satisfied differential equation
 \begin{tasks}(1)
	\task[\textbf{a.}]$\frac{d^{2} J_{n}}{d x^{2}}+\frac{1}{x}\left(\frac{d J_{n}}{d x}\right)+\left(1+\frac{n^{2}}{x^{2}}\right) J_{n}=0$
	\task[\textbf{b.}] $\frac{d^{2} J_{n}}{d x^{2}}+\frac{1}{x}\left(\frac{d J_{n}}{d x}\right)+\left(1-\frac{n^{2}}{x^{2}}\right) J_{n}=0$
	\task[\textbf{c.}] $\frac{d^{2} J_{n}}{d x^{2}}+x\left(\frac{d J_{n}}{d x}\right)+\left(1+\frac{n^{2}}{x^{2}}\right) J_{n}=0$
	\task[\textbf{d.}] $\frac{d^{2} J_{n}}{d x^{2}}+x\left(\frac{d J_{n}}{d x}\right)+\left(1-\frac{n^{2}}{x^{2}}\right) J_{n}=0$
\end{tasks}
\begin{answer}
	\begin{align*}
\text{The Bessel function is given by }\frac{d^{2} J_{n}}{d x^{2}}+\frac{1}{x}\left(\frac{d J_{n}}{d x}\right)+\left(1-\frac{n^{2}}{x^{2}}\right) J_{n}=0
	\end{align*}
		So the correct answer is \textbf{Option (b)}
\end{answer}
\item 57For real $n$ the cylindrical Bessel function is $J_{n}(x)$ of order $n$ then value of $\frac{d J_{0}}{d x}$ is equivalent to 
 \begin{tasks}(4)
	\task[\textbf{a.}] $J_{1}$
	\task[\textbf{b.}]$-J_{1}$
	\task[\textbf{c.}]$2 J_{1}$
	\task[\textbf{d.}]$-2 J_{1}$
\end{tasks}
\begin{answer}
	\begin{align*}
J_{n+1}(x)=-J_{n}^{\prime}(x)+\frac{n}{x} J_{n}\text{. for }n=0, J_{1}=-J_{0}^{\prime}
	\end{align*}
		So the correct answer is \textbf{Option (b)}
\end{answer}
\item 58 The differential equation $x^{2} \frac{d^{2} y}{d x^{2}}+2 x \frac{d y}{d x}+\left[x^{2}-\lambda\right] y(x)=0$ is spherical Bessel's differential equation of order $n$ then value of $\lambda$ is given by
 \begin{tasks}(4)
	\task[\textbf{a.}]$n$
	\task[\textbf{b.}]$n(n+1)$
	\task[\textbf{c.}] $n(n-1)$
	\task[\textbf{d.}]  $n^{2}$
\end{tasks}
\begin{answer}
	\begin{align*}
\text{Spherical Bessel's differential equation }x^{2} \frac{d^{2} y}{d x^{2}}+2 x \frac{d y}{d x}+\left[x^{2}-n(n+1)\right] y(x)=0
	\end{align*}
		So the correct answer is \textbf{Option (b)}
\end{answer}
\item 59If $J_{n}(x)$ is spherical Bessel function of order $n$ if $N_{n}(x)$ is spherical Neumann function of order $n$ and $h_{n}^{\prime}$ is spherical Hankel function of type one of order $n$. Then $h_{0}^{1}$ is given by
 \begin{tasks}(4)
	\task[\textbf{a.}]$i \frac{e^{-i x}}{x}$
	\task[\textbf{b.}]$-i \frac{e^{-i x}}{x}$
	\task[\textbf{c.}] $i \frac{e^{i x}}{x}$
	\task[\textbf{d.}] $-i \frac{e^{i x}}{x}$
\end{tasks}
\begin{answer}
	\begin{align*}
		h_{n}^{1}&=J_{n}+i N_{n}\\
	J_{0}(x)&=\frac{\sin x}{x}, N_{0}(x)=-\frac{\cos x}{x} \Rightarrow h_{0}^{\prime^{\prime}}=J_{0}+i N_{0}=\frac{\sin x-i \cos x}{\because x}=-i \frac{e^{i x}}{x}
	\end{align*}
	So the correct answer is \textbf{Option (d)}
\end{answer}
\item 60If $J_{n}(x)$ is spherical Bessel function of order $n$ if $N_{n}(x)$ is spherical Neumann function of order $n$ and $h_{n}^{2}$ is spherical Hankel function of type two of order $n$. Then $h_{0}^{2}$ is given by
 \begin{tasks}(4)
	\task[\textbf{a.}] $i \frac{e^{-i x}}{x}$
	\task[\textbf{b.}] $-i \frac{e^{-i x}}{x}$
	\task[\textbf{c.}]$i \frac{e^{i x}}{x}$
	\task[\textbf{d.}] $-i \frac{e^{i x}}{x}$
\end{tasks}
\begin{answer}
	\begin{align*}
	h_{n}^{2}&=J_{n}-i N_{n}\\
	J_{0}(x)&=\frac{\sin x}{x},N_{0}(x)=-\frac{\cos x}{x} \Rightarrow h_{0}^{1}=J_{0}+i N_{0} \Rightarrow \frac{\sin x+i \cos x}{x}=i \frac{e^{-i x}}{x}
	\end{align*}
		So the correct answer is \textbf{Option (a)}
\end{answer}
\item 61If $J_{n}(x)$ is spherical Bessel function of order $n$ then $j_{0}^{\prime}(x)$ is equivalent to
 \begin{tasks}(4)
	\task[\textbf{a.}]$j_{1}(x)$
	\task[\textbf{b.}]$-j_{1}(x)$
	\task[\textbf{c.}]$\frac{j_{1}(x)}{2}$
	\task[\textbf{d.}]$-\frac{j_{1}(x)}{2}$
\end{tasks}
\begin{answer}
	\begin{align*}
	\frac{d}{d x}\left(j_{0}(x)\right)&=\frac{d}{d x}\left(\frac{\sin x}{x}\right)=\frac{\cos x}{x}-\frac{\sin x}{x^{2}}=-J_{1}(x)\\
	\text{Where }j_{1}(x)&=-\frac{\cos x}{x}+\frac{\sin x}{x^{2}}
	\end{align*}
	So the correct answer is \textbf{Option (b)}
\end{answer}
\item62 The solution of the differential equation $x^{2} \frac{d^{2} y}{d x^{2}}+2 x \frac{d y}{d x}+x^{2} y(x)=0$ subjected to the condition is given by $y(0)=1$.
 \begin{tasks}(4)
	\task[\textbf{a.}] $\frac{\sin x}{x}$
	\task[\textbf{b.}] $\frac{\cos x}{x}$
	\task[\textbf{c.}]$\frac{\exp (-i x)}{x}$
	\task[\textbf{d.}] $\frac{\exp i x}{x}$
\end{tasks}
\begin{answer}
	\begin{align*}
 \text{Spherical Bessel's differential equation }&x^{2} \frac{d^{2} y}{d x^{2}}+2 x \frac{d y}{d x}+\left[x^{2}-n(n+1)\right] y(x)=0\\
 \text{ then }x^{2} \frac{d^{2} y}{d x^{2}}+2 x \frac{d y}{d x}+x^{2} y(x)=0 &\text{ is spherical Bessel's differential equation for order}\\
 n&=0\\
	\text{then solution is }J_{0}(x)&=\frac{\sin x}{x}\text{ with boundary condition }y(0)=1.
	\end{align*}
	So the correct answer is \textbf{Option (a)}
\end{answer}
\item63 $H_{n}(x)$ is Hermite polynomials of order $n$ then $H_{n}(x)=(-1)^{n} f(x) \frac{d^{n}(W(x))}{d x^{n}}$, then $f(x)$ and $W(x)$ are respectively
 \begin{tasks}(1)
	\task[\textbf{a.}]$f(x)=\exp \left(x^{2}\right), W(x)=\exp \left(-x^{2}\right)$
	\task[\textbf{b.}]$f(x)=\exp \left(-x^{2}\right), W=\exp \left(x^{2}\right)$
	\task[\textbf{c.}] $f(x)=W(x)=\exp \left(x^{2}\right)$
	\task[\textbf{d.}] $f(x)=W(x)=\exp \left(-x^{2}\right)$
\end{tasks}
\begin{answer}
	\begin{align*}
	H_{n}(x)&=(-1)^{n} \exp \left(x^{2}\right) \frac{d^{n}\left(\exp \left(-x^{2}\right)\right)}{d x^{n}}\\
	\text{So after comparing }H_{n}(x)&=(-1)^{n} f(x) \frac{d^{n}(W(x))}{d x^{n}}\\
	f(x)&=\exp \left(x^{2}\right), W(x)=\exp \left(-x^{2}\right)
	\end{align*}
		So the correct answer is \textbf{Option (a)}
\end{answer}
\item 64The solution of differential equation $\frac{d^{2} y}{d x^{2}}-2 x \frac{d y}{d x}+\lambda y(x)=0$ is Hermilte polynomial of order $n$ then value of $\lambda$ is
 \begin{tasks}(4)
	\task[\textbf{a.}]$n$
	\task[\textbf{b.}] $-n$
	\task[\textbf{c.}]$2 n$
	\task[\textbf{d.}] $-2 n$
\end{tasks}
\begin{answer}
	\begin{align*}
	\frac{d^{2} y}{d x^{2}}-2 x \frac{d y}{d x}+2 n y(x)=0\text{ is Hermite differential equation}
	\end{align*}
		So the correct answer is \textbf{Option (c)}
\end{answer}
\item 65The Rodrigues formula for Laguerre polunomial is given by
 \begin{tasks}(2)
	\task[\textbf{a.}]$L_n(x)=\frac{e^{-x}}{n !}\left(\frac{d}{d x}\right)^{n}\left(x^{n} e^{-x}\right)$
	\task[\textbf{b.}]$L_{n}(x)=\frac{e^{x}}{n !}\left(\frac{d}{d x}\right)^{n}\left(x^{n} e^{x}\right)$
	\task[\textbf{c.}]$L_n(x)=\frac{e^{-x}}{n !}\left(\frac{d}{d x}\right)^{n}\left(x^{n} e^{x}\right)$
	\task[\textbf{d.}] $L_{n}(x)=\frac{e^{x}}{n !}\left(\frac{d}{d x}\right)^{n}\left(x^{n} e^{-x}\right)$
\end{tasks}
\begin{answer}
	\begin{align*}
	L_{n}(x)=\frac{e^{x}}{n !}\left(\frac{d}{d x}\right)^{n}\left(x^{n} e^{-x}\right)
	\end{align*}
		So the correct answer is \textbf{Option (d)}
\end{answer}
\item 66It is given that operator $x-\frac{d}{d x}=-\exp \left(\frac{x^{2}}{2}\right) \frac{d}{d x} \exp \left(-\frac{x^{2}}{2}\right)$
If then the normalized wave function for harmonic oscillation is $\psi(x)=\left(\pi^{1 / 2} 2^{n}\lfloor n)^{-1 / 2} \exp \left(-\frac{x^{2}}{2}\right) H_{n}(x)\right.$, then $\psi_n(x)$ is equivalent to 
 \begin{tasks}(1)
	\task[\textbf{a.}]$\psi_{n}(x)=\left(\pi^{1 / 2} 2^{n}\lfloor n)^{-1 / 2}\left(x-\frac{d}{d x}\right)^{n} \exp \left(-\frac{x^{2}}{2}\right)\right.$
	\task[\textbf{b.}] $\psi_{n}(x)=\left(\pi^{1 / 2} 2^{n}\lfloor n)^{-1 / 2}\left(x-\frac{d}{d x}\right)^{2 n} \exp \left(\frac{x^{2}}{2}\right)\right.$
	\task[\textbf{c.}] $\psi_{n}(x)=\left(\pi^{1 / 2} 2^{n}\lfloor n)^{-1 / 2}\left(x-\frac{d}{d x}\right)^{n} \exp \left(-x^{2}\right)\right.$
	\task[\textbf{d.}] $\psi_{n}(x)=\left(\pi^{k / 2} 2^{n}\lfloor n)^{-1 / 2}\left(x-\frac{d}{d x}\right)^{2 n} \operatorname{cxp}\left(-x^{2}\right)\right.$
\end{tasks}
\begin{answer}
	\begin{align*}
	H_{n}(x)&=(-1)^{n} \exp \left(x^{2}\right) \frac{d^{n}\left(\exp \left(-x^{2}\right)\right)}{d x^{n}}\\
	x-\frac{d}{d x} &=-\exp \left(\frac{x^{2}}{2}\right) \frac{d}{d x} \exp \left(-\frac{x^{2}}{2}\right) \Rightarrow\left(x-\frac{d}{d x}\right) \exp \left(-\frac{x^{2}}{2}\right) \\ &\left.=-\exp \left(\frac{x^{2}}{2}\right) \frac{d}{d x} \exp \left(-\frac{x^{2}}{2}\right)\right) \exp \left(-\frac{x^{2}}{2}\right)\\
	x \exp \left(-\frac{x^{2}}{2}\right)-\frac{d \exp \left(-\frac{x^{2}}{2}\right)}{d x}&=-\exp \left(\frac{x^{2}}{2}\right) \frac{d}{d x} \exp \left(-x^{2}\right)\\
	\Rightarrow\left(x-\frac{d}{d x}\right) \exp \left(-\frac{x^{2}}{2}\right)&=\exp \left(\frac{x^{2}}{2}\right)\left(-2 x \exp \left(-x^{2}\right)\right)=2 x \exp -\frac{x^{2}}{2}=H_{1}\left(\exp -\frac{x^{2}}{2}\right)\\
	\text{where }2 x&=H_{1}(x)\\
	\text{Similarly }\left(x-\frac{d}{d x}\right)^{n} \exp \left(-\frac{x^{2}}{2}\right)&=H_{n} \exp \left(-\frac{x^{2}}{2}\right)\\
	\psi_{n}(x)&=\left(\pi^{1 / 2} 2^{n}\lfloor n)^{-1 / 2}\left(x-\frac{d}{d x}\right)^{n} \exp \left(-\frac{x^{2}}{-2}\right)\right.
	\end{align*}
	So the correct answer is \textbf{Option (a)}
\end{answer}
\item 67The solution of differential equation $x \frac{d^{2} y}{d x^{2}}+(1-x) \frac{d y}{d x}+\lambda y(x)=0$ is Laguerre polynomials of order $n$ then value of $\lambda$ is
 \begin{tasks}(4)
	\task[\textbf{a.}]$n$
	\task[\textbf{b.}]$-n$
	\task[\textbf{c.}] $2 n$
	\task[\textbf{d.}] $-2 n$
\end{tasks}
\item 68The generating function $F(x, t)=\sum_{n=0}^{\infty} P_{n}(x) t^{n}$ for the Legendre polynomials $P_{n}(x)$ is $F(x, t)=\left(1-2 x t+t^{2}\right)^{-1 / 2}$. The value of $P_{2}(-1)$ is
 \begin{tasks}(4)
	\task[\textbf{a.}]$5 / 2$
	\task[\textbf{b.}]$3 / 2$
	\task[\textbf{c.}] $+1$
	\task[\textbf{d.}] $-1$
\end{tasks}
\item 69If we observe plot of Bessel functions $J_{0}(x), J_{1}(x)$, and $J_{2}(x)$ we find their maxima at $x_{0}, x_{1}$ and $x_{2}$ respectively. Then which of the following is true
 \begin{tasks}(2)
	\task[\textbf{a.}]$x_{0}<x_{1}<x_{2}$
	\task[\textbf{b.}]$x_{0}>x_{1}>x_{2}$
	\task[\textbf{c.}]$x_{0}<x_{1}=x_{2}$
	\task[\textbf{d.}] $x_{0}=x_{1}<x_{2}$
\end{tasks}
\item 70Which one of the following is correctly matched?\\
\begin{figure}[H]
	\centering
	\includegraphics[height=3.5cm,width=6.5cm]{SF-01}
\end{figure}
 \begin{tasks}(2)
	\task[\textbf{a.}](1) $J_{0}$,
	(2) $J_{2}$, (3) $J_{1}$
	\task[\textbf{b.}]$(1) J_{0}$,
	(2) $J_{1}, \quad(3) J_{2}$
	\task[\textbf{c.}](1) $J_{2}$,
	(2) $J_{1}$,
	(3) $J_{0}$
	\task[\textbf{d.}] None of the above
\end{tasks}
\item 71If the generating function of Legendre polynomial is $\frac{1}{\sqrt{1-6 t+t^{2}}}$, then coefficient of $t^{2}$ is
 \begin{tasks}(4)
	\task[\textbf{a.}] 11
	\task[\textbf{b.}]$-11$
	\task[\textbf{c.}]13
	\task[\textbf{d.}] $-13$
\end{tasks}
\item72 Which of the following relation is true for Bessel's differential equation?
 \begin{tasks}(2)
	\task[\textbf{a.}]$J_{0}^{\prime}(x)=J_{1}(x)$
	\task[\textbf{b.}]$J_{0}^{\prime}(x)=-J_{2}(x)$
	\task[\textbf{c.}]$J_{0}^{\prime}(x)=J_{2}(x)$
	\task[\textbf{d.}] $J_{0}^{\prime}(x)=-J_{1}(x)$
\end{tasks}
\item 73Given that $\sum_{n=0}^{\infty} H_{n}(x) \frac{t^{n}}{n !}=e^{-t^{2}+2 x x}$ the value of $H_{6}(0)$ is
 \begin{tasks}(4)
	\task[\textbf{a.}]$-120$
	\task[\textbf{b.}]$+120$
	\task[\textbf{c.}]12
	\task[\textbf{d.}]  $-12$
\end{tasks}
\item74Given that $\sum_{n=0}^{\infty} H_{n}(x) \frac{t^{n}}{n !}=e^{-t^{2}+2 e x}$ the value of $H_{4}(0)$ is
 \begin{tasks}(4)
	\task[\textbf{a.}]12
	\task[\textbf{b.}] 6
	\task[\textbf{c.}]24
	\task[\textbf{d.}] $-6$
\end{tasks}
\item 75If Hermite polynomial of order 2 is given by $H_{2}(x)=a x^{2}-2 ; a>0$, then the value of $a$ is
 \begin{tasks}(4)
	\task[\textbf{a.}]3
	\task[\textbf{b.}]4
	\task[\textbf{c.}]5
	\task[\textbf{d.}] 6
\end{tasks}
\item 76The value of Legendre polynomial $p_{n}(x)$ for odd $n$ and $x=0$. i.e., $p_{n}(0)$ is
 \begin{tasks}(4)
	\task[\textbf{a.}]1
	\task[\textbf{b.}]0
	\task[\textbf{c.}]$-1$
	\task[\textbf{d.}]  $0.5$
\end{tasks}
\item77 For the Legendre's polynomial $P_{n}(x)$, given below are two statements. Study these carefully and pick out the correct option.\\
Statement I: $\quad \int_{-1}^{1} x\left[P_{n}(x)\right]^{2} d x=0$\\
Statement I: $\lim _{n \rightarrow \infty}\left[\int_{-1}^{1} x P_{n}(x) P_{n+1}(x) d x\right]=0$
 \begin{tasks}(1)
	\task[\textbf{a.}]Only statement (I) is correct
	\task[\textbf{b.}]Only statement (II) is correct
	\task[\textbf{c.}]Both (I) and (II) are correct
	\task[\textbf{d.}]Neither (I) nor (II) is correet
\end{tasks}
\item 78Which of the following statements is Incorrect about the Hermite polynomials $H_{n}(x)$ ?
 \begin{tasks}(1)
	\task[\textbf{a.}] The value of integral $\frac{1}{\sqrt{\pi}} \int_{-\infty}^{\infty} e^{-x^{2}}\left[H_{4}(x)\right]^{2} d x$ is 384
	\task[\textbf{b.}] Hermite polynomial of order $3, H_{3}(x)$, satisfies the differential equation $y^{\prime \prime}-2 x y^{\prime}+6 y=0$
	\task[\textbf{c.}] The value of $\mathrm{H}_{4}(\mathrm{l})$ is $-20$
	\task[\textbf{d.}] $H_{n}(x)=\frac{H_{n+1}(x)+2 n H_{n-1}(x)}{x}$
\end{tasks}
\item 79If $P_{n}(x)$ denotes the Legendre polynomials of order $n$, then which of the following statements is incorrect?
 \begin{tasks}(1)
	\task[\textbf{a.}]$P_{n}(x)=\frac{1}{2^{n} n !} \frac{d^{n}}{d x^{n}}\left[\left(x^{2}-1\right)^{n}\right]$ where $n=0,1,2 \ldots$
	\task[\textbf{b.}]The Legendre polynomials satisfy the differential equation\\$
	\left(1-x^{2}\right) \frac{d^{2} y}{d x^{2}}-2 x \frac{d y}{d x}+n(n+1) y=0
	$
	\task[\textbf{c.}] For each value of $n$ the Legendre polynomials satisfy the relation $P_{n}(1)=1$.
	\task[\textbf{d.}] The value of integral $\int_{-1}^{1}\left[P_{4}(x)\right]^{2} d x$ is $\frac{2}{7}$.
\end{tasks}
\end{enumerate}