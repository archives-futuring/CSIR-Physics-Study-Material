\chapter{Power series Solution and Special functions}
\section{Series Solution Method}
Series expansion is a  method of obtaining one solution of the linear, second-order, homogeneous ODE. The method, will always work, provided the point of expansion is no worse than a regular singular point.In physics this very gentle condition is almost always satisfied. 
A linear, second-order, homogeneous ODE can be written in the form
\begin{equation}
\frac{d^{2} y}{d x^{2}}+P(x) \frac{d y}{d x}+Q(x) y=0 \label{DE002}
\end{equation}
The most general solution of the equation \ref{DE002} may be written as,
\begin{equation}
y(x)=c_{1} y_{1}(x)+c_{2} y_{2}(x)
\end{equation}
But a physical problem may lead to a nonhomogeneous, linear, second-order ODE
\begin{equation}
\frac{d^{2} y}{d x^{2}}+P(x) \frac{d y}{d x}+Q(x) y=F(x)\label{DE003}
\end{equation}
Hence the most general solution to the equation \label{DE003} will be of the form,
\begin{equation}
y(x)=c_{1} y_{1}(x)+c_{2} y_{2}(x)+y_{p}(x)
\end{equation}
The constants $c_{1}$ and $c_{2}$ will eventually be fixed by boundary conditions.\\\\
There are two series solution method  for differential equation,
\begin{enumerate}
	\item \textbf{Simple series expansion method}
	\item \textbf{Frobenious Method}
\end{enumerate}
\subsection{Simple Power Series Expansion Method}
The simple series expansion method works for differential equations whose solutions are well-behaved at the expansion point $x = 0$.
This method can be illustrated by Linear classical oscillator problem
\subsubsection{title}
 We illustrate the method of series solution by solving the following simple equation (which you can easily solve by elementary methods also!):
\begin{align}
y^{\prime}&=2 x y\label{SP-05}
\intertext{We assume a solution of this differential equation in the form of a power series, namely}
y&=a_{0}+a_{1} x+a_{2} x^{2}+a_{3} x^{3}+\cdots+a_{n} x^{n}+\cdots\notag\\
&=\sum_{n=0}^{\infty} a_{n} x^{n}\label{SP-06}
\intertext{where the $a$ 's are to be found. Differentiating $(1.2)$ term by term, we get}
 y^{\prime} &=a_{1}+2 a_{2} x+3 a_{3} x^{2}+\cdots+n a_{n} x^{n-1}+\cdots \notag\\ &=\sum_{n=1}^{\infty} n a_{n} x^{n-1} \label{SP-07}
 \intertext{We substitute (\ref{SP-06}) and (\ref{DE007}) into the differential equation (\ref{SP-05}); we then have two power series equal to each other. Now the original differential equation is to be satisfied for all values of $x$, that is, $y^{\prime}$ and $2 x y$ are to be the same function of $x$. Since a given function has only one series expansion in powers of $x$ (see Chapter 1 , Section 11), the two series must be identical, that is, the coefficients of corresponding powers of $x$ must be equal. We get the following set of equations for the $a$ 's:}
 a_{1}&=0, \quad a_{2}=a_{0}, \quad a_{3}=\frac{2}{3} a_{1}=0, \quad a_{4}=\frac{1}{2} a_{0}
 \intertext{or in general:}
 n a_{n}&=2 a_{n-2}, \quad a_{n}= \begin{cases}0, & \text { odd } n, \\ \frac{2}{n} a_{n-2}, & \text { even } n .\end{cases}\\
\text{ Putting }n&=2 m\text{ (since only even terms appear in this series), we get}\notag\\
a_{2 m}&=\frac{2}{2 m} a_{2 m-2}=\frac{1}{m} a_{2 m-2}=\frac{1}{m} \frac{1}{m-1} a_{2 m-4}=\cdots=\frac{1}{m !} a_{0}
\intertext{Substituting these values of the coefficients into the assumed solution (\ref{SP-06}) gives the solution}
y&=a_{0}+a_{0} x^{2}+\frac{1}{2 !} a_{0} x^{4}+\cdots+\frac{1}{m !} a_{0} x^{2 m}+\cdots=a_{0} \sum_{m=0}^{\infty} \frac{x^{2 m}}{m !}
\end{align}
\subsection{Classical Linear Oscillator}
\begin{align}
\frac{d^{2} y}{d x^{2}}+\omega^{2} y&=0 \label{DE003}\\
\text{with known solutions} \ y&=\sin \omega x, \cos \omega x\\
\text{We try}\ y(x) &=x^{k}\left(a_{0}+a_{1} x+a_{2} x^{2}+a_{3} x^{3}+\cdots\right) \\
&=\sum_{\lambda=0}^{\infty} a_{\lambda} x^{k+\lambda}, \quad a_{0} \neq 0 \label{DE004}\\
\intertext{with the exponent $k$ and all the coefficients $a_{\lambda}$ still undetermined. Note that $k$ need not be an integer. By differentiating twice, we obtain}
\frac{d y}{d x} &=\sum_{\lambda=0}^{\infty} a_{\lambda}(k+\lambda) x^{k+\lambda-1} \\
\frac{d^{2} y}{d x^{2}} &=\sum_{\lambda=0}^{\infty} a_{\lambda}(k+\lambda)(k+\lambda-1) x^{k+\lambda-2}
\intertext{By substituting into equation.\ref{DE003}, we have}
\sum_{\lambda=0}^{\infty} a_{\lambda}(k+\lambda)(k+\lambda-1) x^{k+\lambda-2}+\omega^{2} \sum_{\lambda=0}^{\infty} a_{\lambda} x^{k+\lambda}&=0 \label{DE005}
\intertext{The coefficients of each power of $x$ on the left-hand side of equation.\ref{DE005} must vanish individually.The lowest power of $x$ appearing in equation.\ref{DE005} is $x^{k-2}$, for $\lambda=0$ in the first summation. The requirement that the coefficient vanish  yields,}
a_{0} k(k-1)&=0
\intertext{We had chosen $a_{0}$ as the coefficient of the lowest nonvanishing terms of the series \ref{DE004}, hence, by definition, $a_{0} \neq 0$. Therefore we have,}
k(k-1)&=0 \label{DE006}
\end{align}
\textbf{This equation, coming from the coefficient of the lowest power of $x$, we call the {indicial equation}.} The indicial equation and its roots are of critical importance to our analysis.
\\From equation.\ref{DE006}, \qquad $k=0 $ or $k=1$\\
The only way a power series can be zero is, it's coefficients must be equal to zero. But here the power of $x$ in the equation do not match up. The Coefficent of $x$ in the first term is,${k+\lambda-2} $ and for the second term it is,$k+\lambda$, to make them equal, we can replace $\lambda$ by $\lambda+2$ in the first term. Then we get,
\begin{align}
\sum_{\lambda=2}^{\infty} a_{\lambda+2}(k+\lambda+2)(k+\lambda+1) x^{k+\lambda}+\omega^{2} \sum_{\lambda=0}^{\infty} a_{\lambda} x^{k+\lambda}&=0\\
\sum_{\lambda=2}^{\infty} a_{\lambda+2}(k+\lambda+2)(k+\lambda+1) +\omega^{2} \sum_{\lambda=0}^{\infty} a_{\lambda} &=0
\intertext{Here the coefficients  are independent summations and $\lambda $ is a dummy index. Then we get,}
a_{\lambda+2}(k+\lambda+2)(k+\lambda+1) +\omega^{2} a_{\lambda} &=0\\
a_{\lambda+2}&=-a_{\lambda} \frac{\omega^{2}}{(k+\lambda+2)(k+\lambda+1)}\label{DE007}
\end{align}
For this example, if we start with $a_{0}$, Equation.\ref{DE007} leads to the even coefficients $a_{2}, a_{4}$, and so on, and ignores $a_{1}, a_{3}, a_{5}$, and so on. Since $a_{1}$ is arbitrary if $k=0$ and necessarily zero if $k=1$, 
$$
a_{3}=a_{5}=a_{7}=\cdots=0
$$
and all the odd-numbered coefficients vanish. The odd powers of $x$ will actually reappear when the second root of the indicial equation is used.
\begin{align}
a_{\lambda+2}&=-a_{\lambda} \frac{\omega^{2}}{(\lambda+2)(\lambda+1)}
\intertext{which leads to}
a_{2}&=-a_{0} \frac{\omega^{2}}{1 \cdot 2}=-\frac{\omega^{2}}{2 !} a_{0} \\
a_{4}&=-a_{2} \frac{\omega^{2}}{3 \cdot 4}=+\frac{\omega^{4}}{4 !} a_{0} \\
a_{6}&=-a_{4} \frac{\omega^{2}}{5 \cdot 6}=-\frac{\omega^{6}}{6 !} a_{0}, \quad \text { and so on. }
\intertext{By inspection (and mathematical induction),}
a_{2 n}&=(-1)^{n} \frac{\omega^{2 n}}{(2 n) !} a_{0}
\intertext{and our solution is}
y(x)_{k=0}&=a_{0}\left[1-\frac{(\omega x)^{2}}{2 !}+\frac{(\omega x)^{4}}{4 !}-\frac{(\omega x)^{6}}{6 !}+\cdots\right]\\&=a_{0} \cos \omega x\\
\intertext{If we choose the indicial equation root $k=1$ Equation.\ref{DE007}, the recurrence relation becomes}
a_{j+2}&=-a_{j} \frac{\omega^{2}}{(j+3)(j+2)}\\
\intertext{Substituting in $j=0,2,4$, successively, we obtain}
a_{2}=-a_{0} \frac{\omega^{2}}{2 \cdot 3}&=-\frac{\omega^{2}}{3 !} a_{0} \\
a_{4}=-a_{2} \frac{\omega^{2}}{4 \cdot 5}&=+\frac{\omega^{4}}{5 !} a_{0} \\
a_{6}=-a_{4} \frac{\omega^{2}}{6 \cdot 7}&=-\frac{\omega^{6}}{7 !} a_{0}, \quad \text { and so on. }
\intertext{Again, by inspection and mathematical induction,}
a_{2 n}&=(-1)^{n} \frac{\omega^{2 n}}{(2 n+1) !} a_{0}\\
\intertext{For this choice, $k=1$, we obtain}
y(x)_{k=1} &=a_{0} x\left[1-\frac{(\omega x)^{2}}{3 !}+\frac{(\omega x)^{4}}{5 !}-\frac{(\omega x)^{6}}{7 !}+\cdots\right] \\
&=\frac{a_{0}}{\omega}\left[(\omega x)-\frac{(\omega x)^{3}}{3 !}+\frac{(\omega x)^{5}}{5 !}-\frac{(\omega x)^{7}}{7 !}+\cdots\right] \\
&=\frac{a_{0}}{\omega} \sin \omega x
\end{align}
\subsubsection{Power Series Solution (About an Ordinary Point)}
Find the power series solution of $\left(1-x^{2}\right) y^{\prime \prime}-2 x y^{\prime}+2 y=0$ about $x=0$\\\\
Since $x=0$ is an ordinary point of the given differential equation, the solution can be written as
\begin{align*}
y&=\sum_{k=0}^{\infty} a_{k} x^{k} \\ \frac{d y}{d x}&=\sum_{k=0}^{\infty} k a_{k} x^{k-1} \\ \frac{d^{2} y}{d x^{2}}&=\sum_{k=0}^{\infty} a_{k} k(k-1) x^{k-2}
\intertext{Substituting these values in the given equation we get,}
\left(1-x^{2}\right) \sum_{k} a_{k} k(k-1) x^{k-2}&-2 x \sum_{k} a_{k}(k) x^{k-1}+2 \sum_{k} a_{k} x^{k}=0 \\
\sum_{k=2} a_{k} k(k-1) x^{k-2}&-\sum\left(k^{2}+k-2\right) a_{k} x^{k}=0
\intertext{now equating the coefficient of $x^{k}$ then}
(k+2)(k+1) a_{k+2}-\left(k^{2}+k-2\right) a_{k}&=0 \\a_{k+2}&=\frac{k-1}{(k+1)} a_{k}\\
\text{For} \ k&=0 \Rightarrow a_{2}=-a_{0} \\ k&=1 \Rightarrow a_{3}=0 \\
k&=2 \Rightarrow a_{4}=\frac{a_{2}}{3}=\frac{-a_{0}}{3}  \\ k&=3 \Rightarrow a_{5}=\frac{2}{4} a_{3}=0\\
\text{Therefore, solution}\ y&=a_{0}+a_{1} x+a_{2} x^{2}+\ldots \ldots\\&=a_{0}\left[1-x^{2}-\frac{x^{4}}{3} \ldots . .\right]+a_{1} x
\end{align*}
%\subsection{Frobenious Method}
%Even though the simple power series expansion method works for many functions there are some whose behaviour  precludes the simple series method like the Bessel's function. The need of Frobenious method  lies under the fact that, \textbf{any functions involving negative or fractional powers would not be amenable to power series solution method}. The Frobenious method extends the simple power series solution method to include negative and fractional powers, and it also allows a natural extension involving logarithm terms.\\
%The basic idea of the Frobenius method is to look for solutions of the form
%\begin{align*}
%y(x) &=a_{0} x^{\lambda}+a_{1} x^{\lambda+1}+a_{2} x^{\lambda+2}+a_{3} x^{\lambda+3}+\ldots \\
%&=x^{\lambda}\left(a_{0}+a_{1} x+a_{2} x^{2}+a_{3} x^{3}+\ldots\right) \\
%&=x^{\lambda} \sum_{k=0}^{\infty} a_{k} x^{k} \\
%&= \sum_{k=0}^{\infty} a_{k} x^{k+\lambda}
%\end{align*}
%The extension of the simple power series method is all in the factor $x^{\lambda}$. The power $c$ must now be determined, as well as the coefficients $a_{k}$. Since $\lambda$ may be negative, positive, and possibly non-integral, this extends considerably the range of functions which may be treated. Note that $a_{0}$ is the lowest non-zero coefficient, so by definition it cannot be zero.
\section{Singularity}
Consider differential equation,
\begin{equation}
\frac{d^2y}{dx^2} +P(x)\frac{dy}{dx}+Q(x)y=0
\end{equation}
\begin{enumerate}
	\item If $P(x)$ and $Q(x)$ remain finite at $x=x_0$ then it is O.P
	\item If either $P(x)/Q(x)$ or both tends to $\infty$ at $x=x_0$
	then it is singular point.
\end{enumerate}
\textbf{Kinds of singular points}
\begin{enumerate}
	\item Regular singularity\\
	IF $P(x), Q(x)\text{ diverges but } (x-x_0) P(x) \text{ and }(x-x_0)^2Q(x)$ remain finite at $x\rightarrow x_0$ then $x=x_0$ is called regular or non essencial singular points
	\item Irregular singularity\\
	$P(x), Q(x)$ diverge either $(x-x_0)P(x)$ or $(x-x_0)^2Q(x)$ or both  remain infinite at $x\rightarrow x_0$ then $x=x_0$ is called irregular or essential singular points
\end{enumerate}
Little bit review of ODES
\begin{equation*}
\frac{d^2y }{dx^2}+P(x)\frac{dy}{dx}+Q(x)y=r(x)
\end{equation*}
remember, Tailor series of $y$ about $x_0$. If a $f_n$ was continues and differentable you could express it as an infinite series of polynomial terms being added
\begin{align*}
y(x)&=\sum\limits_{n=0}^{\infty}a_n(x-x_0)^n\\
y^\prime(x)&=\sum\limits_{n=0}^{\infty}na_n(x-x_0)^{n-1}\\
y^{\prime\prime}(x)&=\sum\limits_{n=0}^{\infty}\\
\sum\limits_{n=0}^{\infty}n(a-1)a_n(x-x_0)^{n-2}&+\sum\limits_{n=0}^{\infty}na_n(x-x_0)^{n-1}P(x)+Q(x)\sum\limits_{n=0}^{\infty}a_n(x-x_0)^n=r(x)
\intertext{$P(x),Q(x)$ and $r(x)$ are sitting there a far of $x$. It's also possible to express there as power series} 
p(x)&=\sum\limits_{n=0}^{\infty}b_n(x-x_0)^n\quad Q(x)=\sum\limits_{n=0}^{\infty}c_n(x-x_0)^n,r(x)=\sum\limits_{n=0}^{\infty}d_n(x-x_0)^n\\
\intertext{$\left(\sum\limits_{n=0}^{\infty}n(n-1)a_n(x-x_0)^{n-2}+\left[\sum\limits_{n=0}^{\infty}b_n(x-x_0)^n \right]\sum\limits_{n=0}^{\infty}na_n(x-x_0)^{n-1}+\left[\sum\limits_{n=0}^{\infty}c_n(x-x_0)^n \right] \sum\limits_{n=0}^{\infty}a_n(x-x_0)^n   \right) $}
&=\sum\limits_{n=0}^{\infty}d_n(x-x_0)^n
\end{align*}
We can find these since $P,Q$ and $r$ given what we want to find is $a_n$\\
We can only do this if we properly expand $P(x),Q(x)$ and $R(x)$ abt $x=x_0$ if one of function is undefined at $x_0$ then we would be able to do it. Properly expand and simplify the left or series solution methode will not work to combat this we have to go back and express $y(x)$ as power series abt some other value of $x$ where $p,q$ and $r$ we defined but there are other methodes we will discuss in later\\
Let go with this simple examble
\begin{equation*}
\frac{d^2y}{dx^2}-4y=0, \ y(x)=\sum\limits_{n=0}^{\infty}a_n(x-x_0)^n
\end{equation*}
Using series solution method lets make $y$ a power series. The only $Q_n$
what point power series expand about earliest point $x_0=0$\\
It will always continuos and differentiable about $x_0=0$.Therefore $-4$ is a constant its always continues and differentiable for then any value of $x$
\begin{align*}
y(x)&=\sum\limits_{n=0}^{\infty}a_nx^n,\ y^\prime(x)=\sum\limits_{n=0}^{\infty}na_n x^{n-1}\\
y^{\prime\prime}(x)&=\sum\limits_{n=0}^{\infty}n(n-1)a_nx^(n-2)
\end{align*}
substitute this into ODE only way to combine them is make power of $x$ same
\begin{align*}
\sum\limits_{n=0}^{\infty}n(n-1)a_nx^(n-2)-4\sum\limits_{n=0}^{\infty}a_n x^n&=0\\
\text{Let }n=m-2\text{ when }n=0 \ n=2&\text{ so in terms of $m$ change into $n-2$}\\
\sum\limits_{n=0}^{\infty}n(n-1)a_n x^(n-2)-4\sum\limits_{n=2}^{\infty}a_(n-2)x^{n-2}=0&\text{ still there is problem expand $n=0,n=1$}
\end{align*}
\begin{align*}
&0(0-1)a-0x^{-2}+1(1-1)a_1x^{-1}+\sum\limits_{n=2}^{\infty}n(n-1)a_nx^{n-2}-4\sum\limits_{n=0}^{\infty}a_{n-2}x^{n-2}=0\\
&\sum\limits_{n=2}^{\infty}n(n-1)a_nx^{n-2}-4\sum\limits_{n=2}^{\infty}a_{n-2}x^{n-2}=0\\
&\text{	combining }\sum\limits_{n=2}^{\infty}\left[n(n-1)a_n-4a_{n-2} \right] x^{n-2}=0\\
&\text{	since r.h.s }=0 \text{ all coefficient of l.h.s must be zero}\\
&\implies n(n-1)a_n-4a_{n-2}=0
\intertext{If relates the $n^{th}$ term in the sequence of coefficient to terms that occured  previously in that sequence}
&a_n=\frac{4a_n-2}{n(n-1)}
\end{align*} 
$n^{th}$ coefficient is related to two terms back in the sequence\\
we have solutions for D.E even coefficients related to even previous ans similarly. 'n' can be odd and even.
\begin{align*}
\intertext{n=Even}
a_2&=\frac{4a_0}{2.1}\\
a_4&=\frac{4a_2}{4.3}\\
&=\frac{4.4 \ a_0}{4.3\  2.1}\\
&\text{A pattern is developing}\\
a_4&=\frac{2^4\ a_0}{4!}\quad \text{it follows that}\\
a_n&=\frac{2^2 a_0}{n!}\\
&\text{Since a is even}\\
a_{2k}&=\frac{2^{2k\ a_0}}{(2k)!}\\
&\text{any even number can represent as multiple of $2$}\\
\text{here}&n=0,1,2,3 \\\\
&\intertext{n=Odd}\\
a_3&=\frac{4\ a_1}{3.2},\ a_5=\frac{4\ a_3}{5.4}=\frac{4.4.a_1}{5.4.3.2}\\
a_5&=\frac{2^4\ a_1}{5!}\quad \therefore a_n=\frac{2^{n-1}\ a_1}{n!}\\
&\text{ any odd number can}\\
a_{2k+1}&=\frac{2^{2k}\ a_1}{(2k+1)!}\\
\intertext{can substitute this in the very beginning $y(x)=\sum\limits_{n=0}^{\infty}a_nx^n$ we know all the formulas for $a_n$}
&=a_0+a_1x+a_2x^2+a_3x^3+\dots a_{2k}x^{2k}+a_{2k+1}x^{2k+1}+\dots\\
&=a_0+a_1x+\frac{2^2\ a_0}{2!}x^2+\frac{2^2\ a_1}{3!}x^3+\dots+\frac{2^{2k}\ a_0}{(2k)!}+\frac{2^{2k}\ a_1}{(2k+1)!}+\dots\\
&\text{ can seperate even and odd}\\
y(x)&=\sum\limits_{k=0}^{\infty}\frac{2^{2k}\ a_0}{(2k)!}x^{2k}+\sum\limits_{k=0}^{\infty}\frac{2^{2k}\ a_1}{(2k+1)!}x^{2k+1}
\intertext{this tylor expanssion can leave this here or go one step further }
y(x)&=a_0\cosh(2x)+a_1\sinh(2x)
\end{align*}
Here we discuss series solution and how to use power series to solve differential equations. How to use this to solve one of the important problems in maths and physics.\\
Legendre ODR problem\\
It is a l.d.e
\begin{align*}
(1-x^2)\frac{d^2y}{dx^2}&-2x\frac{dy}{dx}+k(k+1)y=0\\
&\text{to expand it in}\\
x_0&=0\\
\frac{d^2y}{dx^2}&+\frac{P(x)dy}{dx}+Q(x)y=r(x)\\
\frac{d^2y}{dx^2}&-\frac{2x}{(1-x^2)}\frac{dy}{dx}+\frac{k(k+1)}{1-x^2}y=0\text{$r$ is constant}\\
&\text{let check $p(x),q(x)$ and $r(x)$ are all defined at $x=0$, they are}\\
&\text{$r(x)=0$ so it can be defined for all $x$ $p$ and $q$ are only  undefined}\\
&\text{at $x=-1$ and $x=1$ at $x=0$ then both ok. So it is possible for regular series solutionabout $x_0=0$}
\end{align*}
\begin{align*}
y(x)&=\sum\limits_{n=0}^{\infty}a_nx^n\\
y^{\prime\prime}(x)&=\sum\limits_{n=0}^{\infty}n(n-1)a_nx^{n-2}\\
y^{\prime}(x)&=\sum\limits_{n=0}^{\infty}na_nx^{n-1}\\
(1-x^2)\sum\limits_{n=0}^{\infty}n(n-1)&a_nx^{n-2}-2x\sum\limits_{n=0}^{\infty}na_nx^{n-1}+k(k+1)\sum\limits_{n=0}^{\infty}a_nx^{n}=0\\
\sum\limits_{n=0}^{\infty}n(n-1)&a_nx^{n-2}-\sum\limits_{n=0}^{\infty}n(n-1)a_nx^{n}-\sum\limits_{n=0}^{\infty}2na_nx^n+\sum\limits_{n=0}^{\infty}k(k+11)a_nx^{n}=0\\
m&=n-2 \text{ at }n=0\\
m&=-2, n=m+2\\
\sum\limits_{m=-2}^{\infty}&a_{m+2}(m+2)(m+1)x^m
\intertext{expanding this $1^{st}$ two term becomes to $0$}
\sum\limits_{n=0}^{\infty}&a_{n+2}(n+2)(n+1)x^n-\sum\limits_{n=0}^{\infty}n(n-1)a_nx^n-\sum\limits_{n=0}^{\infty}2na_nx^n+\sum\limits_{n=0}^{\infty}k(k+1)a_nx^n=0\\
\sum\limits_{n=0}^{\infty}&\left[ a_{n+2}(n+2)(n+1)-a_nn(n-1)-2na_n+k(k+1)a_n\right] x^n=0\\
a_{n+2}&(n+2)(n+1)-a_nn(n-1)-2na_n+k(k+1)a_n=0\\
a_{n+2}&=a\\
a_{n+2}&=\frac{a_{n}[n(n-1)+2 n-k(k+1)]}{(n+2)(n+1)}=\frac{ a_{n}\left(n^{2}+n-k^{2}-k\right)}{(n+2)(n+1)}\\\\
a_{n+2}&=\frac{a_{n}[(n-k)(n+k)+(n-k)] }{(n+2)(n+1)}=\frac{a_{n}(n-k)(n+k+1)}{(n+2)(n+1)}\\
a_{0},a_2&=\frac{a_{0} \cdot(-k)(x+1)}{2 \cdot 1}\\
a_4&=\frac{a_{2}(2-k)(3+k)}{4\cdot 3}=\frac{(k-2) k(k+1)(k+3) a_{0}}{4!}
\intertext{patterm developing}
a_3&=\frac{a_{1}(1-k)(2+k)}{3\cdot2}\\
a_5&=\frac{9_{3}(3-k)(4+k)}{5.4}=\frac{a_{1}(k-3)(k-1)(k+2)(k+4)}{5!}
\end{align*}
1.\quad What if $k=1$ \quad $a_3=0$
\begin{align*}
\text{all coefficients come later on }a_3&=0\\
\therefore \text{ they dipend on }&a_3\\
\text{so solution}\\
y_{odd}=a_px&=a_1P_1(x)\\
P_1(x)\text{ is just a function of $x$ will explain later}\\
\text{even series }&\text{doesnot terminate}
\end{align*}
2.\quad What if $k=2$
\begin{align*}
a_2&=a_{0} \frac{(-2)(3)}{2}=-3 a_{0}, a_{4}=0\\
y_{\text {even }}&=a_{0}-3 a_{0} x^{2}=a_{0}\left(1-3 x^{2}\right)\\
&=\cdot 5\left(3 x^{2}-1\right)=P_{2}(x) \text{ for }a_0=\frac{-1}{2}\\
\text{for }k&=2,\text{ odd series doesnot terminate}\\
k&=3\\
y_ {odd} &=-\frac{5}{3} a_1 x^{3}+a_{1} x\\
&=\frac{1}{2}\left(5 x^{3}-3 x\right)=P_{3}(x)\text{ for }a_{1}=-3 / 2\\
\text{even series }&\text{doesnot terminate for odd value of $k$}
\end{align*}
we can repeat this for different value of $k$. \\for even $k$ tends to even series terminates.\\
for odd $k$ tends to odd series terminates\\
when even series terminate we end up with a polynomial of $k$ is the highest power in the series \\
When $k=3$  highest power will be an $x^3$ term $P_1, P_2, P_3$ they result from series terminating are not like just other polynomial they are special polynomial.\\
$P_1, P_2, P_3$ tends to these polynomials are called legendre polynomials, they are solutions to legendre ODES they come up in electromagnetism and quantum mechanics. Bunch of techniques already developed to calculate legendre polynomials more easily one of these techniques is rodrigues formula
$P_{k}(x)=\frac{1}{2^{k} k !} \frac{d^{k}}{d x^{k}}\left[\left(x^{2}-1\right)^{k}\right]$\\
Here one series terminates giving a polynomial solution. Other series divergent\\
\textbf{Theree Possibilities}\\
1.\quad $r_1\text{ and }R_2$ distinct not differ by integer\\
$y(x)=k_1y_1+k_2y_2=k_1(x-x_0)^{r_1}\sum\limits_{n=0}^{\infty}a_n(x-x_0)^n+k_2(x-x_0)^{r_2}$\\
2.\quad $r_1=r_2=r,y_1=(x-x_0)^r\sum_{n=0}^{\infty} a_{n}\left(x-x_{0}\right)^{n}$\\
3.\quad $r_1\text{ and }R_2$ distinct and differ by integer\\
$y_1=(x-x_0)^{r_1}\sum_{n=0}^{\infty} a_{n}\left(x-x_{0}\right)^{n}$ is similar using recursion relation 
\section{Bessel Function}
Special D.E when we solving P.D.E especially when solving problems in cylindrical coordinates.
\begin{align*}
x^{2} y^{\prime }+x y^\prime+\left(x^{2}-p^{2}\right) y&=0
\intertext{We want a solution that can be expanded about $x_0=0$}
\intertext{Before starting we can check $x_0=0$ is singular point or not}
y^{\prime \prime}+\frac{1}{x} y^{\prime}+\left(1-\frac{p^{2}}{x^{2}}\right) y&=0\hspace{3cm}P(x)=\frac{1}{x},\quad Q(x)=1-\frac{p^{2}}{x^{2}}\\
x p(x)&=1\\
X^{2} Q(x)&=x^{2}-P^{2}\\
X_{0}=0 &\text{is regular singular point}\\
y=\sum_{n=0}^{\infty} a_{n} x^{n+r}\quad y^{\prime}&=\sum_{n=0}^{\infty}(n+r) a_{n} x^{n+r-1}\quad y^{\prime \prime}=\sum_{n=0}^{\infty}(n+r)(n+r-1) a_{n}x^{n+r-2}\\
\sum_{n=0}^{\infty}(n+r)(n+r-1) a_{n} x^{n+r-2} &+\frac{1}{x} \sum_{n=0}^{\infty}(n+r) a_{n}x^{n+r-1}+\left(1-\frac{p^{2}}{x^{2}}\right) \sum_{n=0}^{\infty} a_{n} x^{n+r}=0\\
\sum_{n=0}^{\infty}(n+r)(n+r-1) a_{n} x^{n+r-2)} &+\sum_{n=0}^{\infty}(n+r) a_{n} x^{n+r-2}+\sum_{n=0}^{\infty} a_{n} x^{n+r}-p^{2}\sum_{n=0}^{\infty} a_{n}x^{n+r-2} =0\\
\sum_{n=0}^{\infty}(n+r)(n+r-1) a_{n} x^{n+r-2)} &+\sum_{n=0}^{\infty}(n+r) a_{n} x^{n+r-2}+ \sum_{n=2}^{\infty} a_{n-2} x^{n+r-2}\sum_{n=0}^{\infty} a_{n}x^{n+r-2} =0\\
+r(r-1) a_{0} x^{r-2}+r(r+1) a_{1} x^{r-1}&+r a_{0} x^{r-2}+(r+1) a_{1} x^{r-1}-P^2{a_{0}} x^{r-2}-p^{2} a_1 x^{r-1}=0
\intertext{$\therefore$ every value of r.h.s=0, coefficient of every power of l.h.s coefficients must be zero}
\text{for} x^{r-2}:r(r-1) a_{0}+r a_{0}-p^{2} a_{0}&=0, \text{we want }a_{0} \neq 0 \text{so we can write as,}\\
\gamma(r-1)+r-p^{2}=0\quad \gamma^{2}-p^{2}&=0 \quad r=\pm p\\
\intertext{this is our indicial equations}
\text{For} x^{8-1}:[r(\partial+1)+(r+1)-P^2] a_{1}&=0\quad \text { So } a_{1}=0\\
\sum_{n=2}^{\infty}(n+r)(n+r-1) a_{n} x^{n+r-2}&+\sum_{n=2}^{\infty}(n+r) a_{n} x^{n+r-2}+\sum_{n=2}^{\infty}a_{n-2}x^{n+r-2}-p^{2} \sum_{n=2}^{\infty} a_{n}  x^{n+r-2}=0
\intertext{gather everything}
\sum_{n=2}^{\infty}[(n+r)(n+\gamma-1)&+(n+r) P^2] a_{n}+\left.a_{n-2}\right\} x^{n+r-2}=0\\
\left[(n+r)(n+r-1)+(n+r)-p^{2}\right] a_{n}+a_{n-2}&=0\\
a_{n}&=\frac{-a_{n-2}}{\left[(n+r)^{2}-p^{2}\right]}\\
\text{substitute }r_1&=P\\
a_{n}=\frac{-a_{n-2}}{n(n+2 P)}\implies \text{recursion}\\
\therefore a_1=0 &\text{and indices are added with two terms behind}\\
a_{3}, a_{5}, .&=0
\intertext{For this we only carried about even indice coefficients $n=2k$ where $k$ is another integer}
a_{2 k}&=-\frac{a_{2 k-2}}{4 k(k+P)}\\
a_{2}&=\frac{-a_{0}}{4 \cdot 1 (1+P)}a_{4}=-\frac{-a_{2}}{4 \cdot 2(2+P)}=\frac{(-1)^{2} a_{0}}{4^{2}(2 \cdot 1)(2+P)(1+P)}\\
\therefore a_{2 k}&=\frac{(-1)^{k}}{4{k}}\frac{a_{0}}{k !(P+k) \cdots(P+1)}\\
y_{1}&=x^{p}\sum_{k=0}^{\infty} \frac{(-1)^{k}}{4 k}\frac{a_{0}}{k !(p+k) . .(P+1)} x^{2 k}
\intertext{For perticular value of $a_0$ this solution is given a special name $~\quad J_P(x)$ Bessel function of $I$ kind $P$ is order of Bessel function.}  
\end{align*}
\section{Legendre Differential Equation}
\subsection{Rodrigue's Formula for Legendre's Polynomial $P_{n}(x)$ :}
\begin{align*}
\text { Legendre polynomial of order ' } n \text { ' : } P_{n}(x)&=\frac{1}{2^{n} n !} \frac{d^{n}}{d x^{n}}\left(x^{2}-1\right)^{n}\\
P_{0}(x)=1 ; \quad P_{1}(x)=x ; \quad P_{2}(x)&=\frac{1}{2}\left(3 x^{2}-1\right) ; \quad P_{3}(x)=\frac{1}{2}\left(5 x^{3}-3 x\right)
\end{align*}
\begin{figure}[H]
	\centering
	\includegraphics[height=3.8cm,width=6cm]{SF-02}
\end{figure}
\subsection{Properties of Legendre's Polynomial}
\begin{align*}
\text{(1) }P_{n}(1)&=1\qquad
\text{(2) }P_{n}(-1)=(-1)^{n}\\
\text { (3) } P_{n}(x&)=\text { even function, if } \mathrm{n}=\text { even }\\
&=\text { odd function if } \mathrm{n}=\text { odd }\\
\text { (4) } P_{n}(-x)&=(-1)^{n} P_{n}(x)
\end{align*}
\subsection{Generating function of Legendre's polynomial:}
$\left(1-2 x z+z^{2}\right)^{-1 / 2}$ is the generating function for legendre's polynomials i.e. $\left(1-2 x z+z^{2}\right)^{-1 / 2}=\sum_{n=0}^{\infty} z^{n} P_{n}(x)$\\
The co-efficient of $z^{n}$ in the exapansion of the generating function is the Legendre's polynomial of order $n$.
\subsection{Orthogonal properties of Legendre's polynomial:}
\begin{align*}
\int_{-1}^{+1} P_{m}(x) P_{n}(x) d x=\frac{2}{2 n+1} \delta_{m n} \text { where } \delta_{m n}=\text { Kronecker delta }&=1 \quad \text { if } \mathrm{m}=\mathrm{n}\\
&=0 \quad \text { if } m \neq n
\end{align*}
\subsection{Recurrence relations for $P_{n}(x)$ :}
\begin{align*}
\text{(i) }&n P_{n}=(2 n-1) x P_{n-1}-(n-1) P_{n-2}\hspace{1cm}
\text{(ii) }n P_{n}=x P_{n}^{\prime}-P_{n-1}^{\prime}\\
\text{(iii) }&(2 n+1) P_{n}=P_{n+1}^{\prime}-P_{n-1}^{\prime}\hspace{2.5cm}
\text{(iv) }(n+1) P_{n}=P_{n+1}^{\prime}-x P_{n}^{\prime}\\
\text{(v) }&\left(1-x^{2}\right) P_{n}^{\prime}=n\left(P_{n-1}-x P_{n}\right)\hspace{2.1cm}
\text{(vi) }\left(1-x^{2}\right) P_{n}^{\prime}=(n+1)\left(x P_{n}-P_{n+1}\right).
\end{align*}
\begin{exercise}
	Express $4 x^{3}+6 x^{2}+7 x+2$ in terms ofLegendre Polynpmials.
\end{exercise}
\begin{answer}
	\begin{align*}
	\text { Let, } 4 x^{3}+6 x^{2}+7 x+2&=a_{3} P_{3}(x)+a_{2} P_{2}(x)+a_{1} P_{1}(x)+a_{0} P_{0}(x)\\
	&=\frac{a_{3}}{2}\left(5 x^{3}-3 x\right)+\frac{a_{2}}{2}\left(3 x^{2}-1\right)+a_{1} x+a_{0}\\
	\text { Comparing the co-efficient of } &\mathrm{x}^{3}: \frac{5 a_{3}}{2}=4 \Rightarrow a_{3}=\frac{8}{5}\\
	\text { Comparing the co-efficient of }& x^{2}: \frac{3 a_{2}}{2}=6 \Rightarrow a_{2}=4\\
	\text { Comparing the co-efficient of }& x^{1}:-\frac{3 a_{3}}{2}+a_{1}=7 \Rightarrow a_{1}=\frac{47}{5}\\
	\text { Comparing the co-efficient of }& \mathrm{x}^{0}:-\frac{a_{2}}{2}+a_{0}=2 \Rightarrow a_{0}=4\\
	\text { Therefore, } 4 x^{3}+6 x^{2}+7 x+2&=\frac{8}{5} P_{3}(x)+4 P_{2}(x)+\frac{47}{5} P_{1}(x)+4 P_{0}(x)
	\end{align*}
\end{answer}
\begin{exercise}
	Show that, $\int_{-1}^{1} x^{2} P_{n+1} P_{n-1} d x=\frac{2 n(n+1)}{(2 n-1)(2 n+1)(2 n+3)}$
\end{exercise}
\begin{answer}
	\begin{align*}
	\text{Using the relation: }&(2 n+1) x P_{n}=(n+1) P_{n+1}+n P_{n-1}\\
	\text{Putting $(\mathrm{n}+1)$ in place of }&\mathrm{n} \Rightarrow(2 n+3) x P_{n+1}=(n+2) P_{n+2}+(n+1) P_{\mathrm{n}}\\
	\text{Putting $(\mathrm{n}-1)$ in place of }&\mathrm{n} \Rightarrow(2 n-1) x P_{n-1}=n P_{n}+(n-1) P_{n-2}\\
	\int_{-1}^{1}(2 n+3)(2 n-1) x^{2} P_{n+1} P_{n-1} d x&=n(n+2) \int_{-1}^{1} P_{n+2} P_{n} d x+n(n+1) \int_{-1}^{1} P_{n} P_{n} d x+\\
	&(n-1)(n+2) \int_{-1}^{1} P_{n-2} P_{n+2} d x+\int_{-1}^{1} P_{n} P_{n-2} d x\\
	\Rightarrow \int_{-1}^{1} x^{2} P_{n-1} P_{n+1} d x&=\frac{2 n(n+1)}{(2 n-1)(2 n+1)(2 n+3)}
	\end{align*}
\end{answer}
\begin{exercise}
	Show that $\sum_{n} P_{n}(x)=\frac{1}{\sqrt{2-2 x}}$
\end{exercise}
\begin{answer}
	\begin{align*}
	\left(1-2 x z+z^{2}\right)^{-1 / 2}&=\sum_{n} z^{n} P_{n}(x)\\
	\text { Putting } z&=1 \text { in equation, }(1-2 x+1)^{-1 / 2}=\sum_{n} P_{n}(x) \quad \Rightarrow \sum_{n} P_{n}(x)=\frac{1}{\sqrt{2-2 x}}
	\end{align*}
\end{answer}
\begin{exercise}
	Prove that $P_{n+1}^{\prime}(x)+P_{n}^{\prime}(x)=P_{0}+3 P_{1}+5 P_{2}+\ldots . .+(2 n+1) P_{n}$
\end{exercise}
\begin{answer}
	\begin{align*}
	\text { Using the relation }&(2 n+1) P_{n}(x)=P_{n+1}^{\prime}(x)-P_{n-1}^{\prime}(\dot{x})\\
	\text { putting } \mathrm{n}&=1 \quad \Rightarrow 3 P_{1}(x)=P_{2}^{\prime}(x)-P_{0}^{\prime}(x)\\
	\text { putting } \mathrm{n}&=2 \quad \Rightarrow 5 P_{2}(x)=P_{3}^{\prime}(x)-P_{1}^{\prime}(x)\\
	\text { putting } \mathrm{n}&=3 \quad \Rightarrow 7 P_{3}(x)=P_{4}^{\prime}(x)-P_{2}^{\prime}(x) \text { and so on. }\\
	\text{Adding all}&\text{ the equations we get,}\\
	3 P_{1}(x)+5 P_{2}(x)+7 P_{3}(x)&+\ldots \ldots(2 n+1) P_{n}(x)\\
	=-P_{0}^{\prime}(x)-P_{1}^{\prime}(x)+P_{n}^{\prime}(x)&+P_{n+1}^{\prime}(x)=0-P_{0}(x)+P_{n}^{\prime}(x)+P_{n+1}^{\prime}{ }^{\cdot}(x)\\
	\Rightarrow P_{n+1}^{\prime}(x)+P_{n}^{\prime}(x)&=P_{0}+3 P_{1}+5 P_{2}+\ldots . .+(2 n+1) P_{n}
	\end{align*}
\end{answer}
\section{Bessel Differential Equation}
\begin{equation*}
x^{2} \frac{d^{2} y}{d x^{2}}+x \frac{d y}{d x}+\left(x^{2}-n^{2}\right) y=0
\end{equation*}
$x=0$ is a regular singular point of the Bessel differential equation.\\\\
\begin{align*}
\text{Bessel Functions }&\mathrm{J}_{\mathrm{n}}(\mathrm{x})\\
\mathrm{J}_{\mathrm{n}}(\mathrm{x})&=y=a_{0} \sum_{r=0}^{\infty} \frac{(-1)^{r}}{r !(n+r) !}\left(\frac{x}{2}\right)^{n+2 r} \quad \text { where } \mathrm{a}_{0}=\frac{1}{2^{n} \mid n+1}\\
\mathrm{n}=0 \quad \Rightarrow \quad \mathrm{J}_{0}(\mathrm{x})&=\sum_{r=0}^{\infty} \frac{(-1)^{r}}{(r !)^{2}}\left(\frac{x}{2}\right)^{2 r}=1-\frac{x^{2}}{2^{2}}+\frac{x^{4}}{2^{2} \cdot 4^{2}}-\frac{x^{6}}{2^{2} \cdot 4^{2} \cdot 6^{2}}+\ldots . .\\
\mathrm{n}=1 \quad \Rightarrow \mathrm{J}_{1}(x)&=\frac{x}{2}-\frac{x^{3}}{2^{2} 4}+\frac{x^{5}}{2^{2} \cdot 4^{2} \cdot 6}-\ldots .
\end{align*}
\begin{figure}[H]
	\centering
	\includegraphics[height=3.5cm,width=5cm]{SF-03}
\end{figure}
(i) Bessel functions are oscillatory function with varying period and decreasing amplitude.\\
(ii) $J_{n}(x)$ is an even function when $n$ is even and $J_{n}(x)$ is an odd function when $n$ is odd.\\
(iii) $\mathrm{J}_{0}(\mathrm{x})$ has nodes at $x=2.40 \% 8,5.5201$......and $J_{1}(x)$ has nodes at $x=3.8317,7.015$.......
\begin{align*}
J_{1 / 2}(x)=\sqrt{\frac{2}{\pi x}} \sin x \quad \& \quad J_{-1 / 2}(x)=\sqrt{\frac{2}{\pi x}} \cos x
\end{align*}
\subsection{Generating function of Bessel's function:}
\begin{align*}
e^{\frac{x}{2}\left(z-\frac{1}{2}\right)}=\sum_{n=-\infty}^{\infty} z^{n} J_{n}(x)
\intertext{The coefficient of $z^{n}$ in the expansion of the generating tünction is the Bessel function of order $1 .$}
\end{align*}
\subsection{Orthogonal condition of Bessel Function:}
\begin{align*}
\int_{0}^{1} x_{n} \cdot(\alpha x) J_{n}(\beta x) d x=\frac{\delta_{\alpha \beta}}{2} \cdot\left[J_{n+1}(\alpha)\right]^{2}
\end{align*}
\subsection{Recurrencé relations for $J_{n}(x)$ :}
\begin{align*}
(i)\ & x J_{n}^{\prime}(x)=n J_{n}-x J_{n+1}\hspace{1cm}(ii)\  x J_{n}^{\prime}=-n J_{n} \times J_{n-1}.\\
(iii)\ & 2 J_{n}^{\prime}=J_{n-1}-J_{n+1}\hspace{1.5cm}(iv)\  2 n J_{n}=x\left[J_{n-1}+J_{n+1}\right]\\
(v)\ & \frac{d}{d x}\left[x^{-n} J_{n}\right]=-x^{-n} J_{n+1} \hspace{.5cm} (vi)\  \frac{d}{d x}\left[x^{n} J_{n}\right]=x^{n} J_{n-1}
\end{align*}
\begin{exercise}
	Show that $\sqrt{\frac{\pi x}{2}} J_{3 / 2}(x)=\frac{\sin x}{x}-\cos x$
\end{exercise}
\begin{answer}
	\begin{align*}
	\text{Using recurrence relation, }2 n J_{n}&=x\left[J_{n+1}+J_{n-1}\right]\\
	\text{putting}\mathrm{n}&=1 / 2,\text{ we get, }J_{1 / 2}=x\left[J_{3 / 2}+J_{-1 / 2}\right]\\
	\Rightarrow x J_{3 / 2}=J_{1 / 2}-x J_{-1 / 2} \quad \Rightarrow \sqrt{\frac{\pi x}{2}} J_{3 / 2}&=\frac{\sin x}{x}-\cos x \text { (using the expressions of } J_{1 / 2} \& J_{-1 / 2} \text { ) }
	\end{align*}
\end{answer}
\section{Hermite Differentiation Equation}
\begin{align*}
\frac{d^{2} y}{d x^{2}}-2 x \frac{d y}{d x}+2 n y&=0,(\mathrm{n}=0,1 \ldots . .)\\
\text{$x=0$ is an ordinary point of the }&\text{Hermite differential equation}\\
\intertext{The solution of the Hermite differential equation is a terminating series, so it is called Hermite polynomial of}
\intertext{order ni.e. $\mathrm{H}_{\mathrm{n}}(\mathrm{x})$}
H_{n}(x)=\sum_{r=0}^{P} \frac{(-1)^{r} n !(2 x)^{n-2 r}}{r !(n-2 r) !} \quad \Rightarrow\left[\begin{array}{ll}
P=\frac{n}{2} & \text { if } n=\text { even } \\
=\frac{n-1}{2} & \text { if } \mathrm{n} \text { is odd }
\end{array}\right]
\end{align*}
\subsection{Rodrigue formula for $H_{n}(x)$ :}
\begin{align*}
\begin{array}{ll}
H_{n}(x)=(-1)^{n} e^{x^{2}} \frac{d^{n}}{d x^{n}}\left(e^{-x^{2}}\right) \\
H_{0}(x)=1 & H_{3}(x)=8 x^{3}-12 x \\
H_{1}(x)=2 x & H_{4}(x)=16 x^{4}-48 x^{2}+12 \\
H_{2}(x)=4 x^{2}-2 &
\end{array}
\end{align*}
These are Hermitite polynomial of order zero, one, two, three, four .... respectively. \\
$H_{\mathrm{n}}(x)$ will be even function if $n$ is even and will be odd function if $n$ is odd.\\
The Hermite polynomial of different order differ from the legendre polynomials with respect to the ceefficient.\\
 So, the nature of the graphs of $\mathrm{H}_{\mathrm{n}}(\mathrm{x})$ will be same as.Legendre polynomial $P_{\mathrm{n}}(x)$.
\subsection{Generating function for hermite polynomial:}
\begin{align*}
\intertext{$\mathrm{e}^{2 \mathrm{zx}-\mathrm{z}^{2}}$ is the generating function of hemite polynomial i.e.}
e^{2 z x-z^{2}}&=\sum_{n=0}^{\infty} \frac{H_{n}(x)}{n !} z^{n}\\
\text{The coefficient of $z^{n}$ in the expansion of }e^{2 z x-z^{2}}&\text{ is } \frac{H_{n}(x)}{n !}
\end{align*}
\subsection{Orthogonal property of $H_{n}(x)$ :}
\begin{align*}
\int_{-\infty}^{\infty} e^{-x^{2}} H_{n}(x) H_{m}(x) d x=2^{n} n ! \sqrt{\pi} \delta_{m n}
\end{align*}
\subsection{Recurrence relation for $\mathrm{H}_{\mathrm{n}}(\mathbf{x})$ :}
\begin{align*}
(i)&\  H_{n}^{\prime}(x)=2 n H_{n-1}(x)\\
(ii)& \ 2 x H_{n}(x)=2 n H_{n-1}(x)+H_{n+1}(x)\\
(iii)& \ H_{n}^{\prime}(x)=2 x H_{n}(x)-H_{n+1}(x)\\
(iv)&\  H_{n}^{\prime \prime}(x)-2 x H_{n}^{\prime}(x)+2 n H_{n}(x)=0\\
(v)&\  H_{n}(-x)=(-1)^{n} H_{n}(x)\\
\end{align*}
\section{Lagauerre Differential Equation}
\begin{align*}
x \frac{d^{2} y}{d x^{2}}+(1-x) \frac{d y}{d x}+n y&=0\\
\text{$x=0$ is a regular singular point of the Laguerre differential }& \text{equation.}\\\\
\intertext{The solution ofLaguerre differentialequation is known as Laguerre polynomial of order ni.e. $L_{n}(x)$}
L_{n}(x)=\sum \frac{(-1)^{r} n ! x^{r}}{(r !)^{2}(n-r) !}&
\end{align*}
\subsection{Rodrigue formula for $L_{n}(x)$ :}
\begin{align*}
\begin{aligned}
&L_{n}(x)=\frac{e^{x}}{n !} \frac{d^{n}}{d x^{n}}\left(x^{n} e^{-x}\right) \\
&L_{0}(x)=1 \quad L_{1}(x)=1-x \quad L_{2}(x)=\frac{1}{2}\left(x^{2}-4 x+2\right) \\
&L_{3}(x)=\frac{1}{6}\left(-x^{3}+9 x^{2}-18 x+6\right)
\intertext{Laguerre polynomials are neither even nor odd.}
\end{aligned}
\end{align*}
\subsection{Generating function for Laguerre polynomial:}
\begin{align*}
\frac{e^{-x z /(1-z)}}{(1-z)}=\sum_{n=0}^{\infty} z^{n} L_{n}(x)
\end{align*}
\subsection{Orthogonal property of Laguerre's polynomial:}
\begin{align*}
\int_{0}^{\infty} e^{-x} L_{n}(x) L_{m}(x) d x=\delta_{m n}
\end{align*}



\newpage
\begin{abox}
	Problem Set -1
\end{abox}
\begin{enumerate}[label=\color{ocre}\textbf{\arabic*.}]
	\item  Let $p_{n}(x)$ (where $n=0,1,2, \ldots \ldots$ ) be a polynomial of degree $n$ with real coefficients, defined in the interval $2 \leq n \leq 4$. If $\int_{2}^{4} p_{n}(x) p_{m}(x) d x=\delta_{n m}$, then
	{\exyear{NET/JRF(JUNE-2011)}}
	\begin{tasks}(2)
		\task[\textbf{A.}] $p_{0}(x)=\frac{1}{\sqrt{2}}$ and $p_{1}(x)=\sqrt{\frac{3}{2}}(-3-x)$
		\task[\textbf{B.}]  $p_{0}(x)=\frac{1}{\sqrt{2}}$ and $p_{1}(x)=\sqrt{3}(3+x)$
		\task[\textbf{C.}] $p_{0}(x)=\frac{1}{2}$ and $p_{1}(x)=\sqrt{\frac{3}{2}}(3-x)$
		\task[\textbf{D.}] $p_{0}(x)=\frac{1}{\sqrt{2}}$ and $p_{1}(x)=\sqrt{\frac{3}{2}}(3-x)$
	\end{tasks}
	\item  The generating function $F(x, t)=\sum_{n=0}^{\infty} P_{n}(x) t^{n}$ for the Legendre polynomials $P_{n}(x)$ is $F(x, t)=\left(1-2 x t+t^{2}\right)^{-1 / 2}$. The value of $P_{3}(-1)$ is
	{\exyear{NET/JRF(DEC-2011)}}
	\begin{tasks}(4)
		\task[\textbf{A.}] $5 / 2$
		\task[\textbf{B.}] $3 / 2$
		\task[\textbf{C.}] $+1$
		\task[\textbf{D.}] $-1$
	\end{tasks}
	\item  The graph of the function $f(x)$ shown below is best described by
	{\exyear{NET/JRF(DEC-2012)}}
	\begin{figure}[H]
		\centering
		\includegraphics[height=6cm,width=8cm]{diagram-20211005(12)-crop}
	\end{figure}
	\begin{tasks}(2)
		\task[\textbf{A.}]  The Bessel function $J_{0}(x)$
		\task[\textbf{B.}] $\cos x$
		\task[\textbf{C.}] $e^{-x} \cos x$
		\task[\textbf{D.}] $\frac{1}{x} \cos x$
	\end{tasks}
	\item Given that $\sum_{n=0}^{\infty} H_{n}(x) \frac{t^{n}}{n !}=e^{-t^{2}+2 t x}$ the value of $H_{4}(0)$ is
	{\exyear{NET/JRF(JUNE-2013)}}
	\begin{tasks}(4)
		\task[\textbf{A.}] 12
		\task[\textbf{B.}] 6
		\task[\textbf{C.}] 24
		\task[\textbf{D.}] $-6$
	\end{tasks}
	\item   Given $\sum_{n=0}^{\infty} P_{n}(x) t^{n}=\left(1-2 x t+t^{2}\right)^{-1 / 2}$, for $|t|<1$, the value of $P_{5}(-1)$ is
	{\exyear{NET/JRF(JUNE-2014)}}
	\begin{tasks}(4)
		\task[\textbf{A.}] $0.26$
		\task[\textbf{B.}] 1
		\task[\textbf{C.}] $0.5$
		\task[\textbf{D.}] $-1$
	\end{tasks}
	\item The function $f(x)=\sum_{n=0}^{\infty} \frac{(-1)^{n}}{n !(n+1) !}\left(\frac{x}{2}\right)^{2 n+1}$, satisfies the differential equation
	{\exyear{NET/JRF(DEC-2014)}}
	\begin{tasks}(2)
		\task[\textbf{A.}]  $x^{2} \frac{d^{2} f}{d x^{2}}+x \frac{d f}{d x}+\left(x^{2}+1\right) f=0$
		\task[\textbf{B.}]  $x^{2} \frac{d^{2} f}{d x^{2}}+2 x \frac{d f}{d x}+\left(x^{2}-1\right) f=0$
		\task[\textbf{C.}] $x^{2} \frac{d^{2} f}{d x^{2}}+x \frac{d f}{d x}+\left(x^{2}-1\right) f=0$
		\task[\textbf{D.}] $x^{2} \frac{d^{2} f}{d x^{2}}-x \frac{d f}{d x}+\left(x^{2}-1\right) f=0$
	\end{tasks}
	\item
	 The Hermite polynomial $H_{n}(x)$, satisfies the differential equation
	$$
	\frac{d^{2} H_{n}}{d x^{2}}-2 x \frac{d H_{n}}{d x}+2 n H_{n}(x)=0
	$$
	The corresponding generating function $G(t, x)=\sum_{n=0}^{\infty} \frac{1}{n !} H_{n}(x) t^{n}$, satisfies the equation
	{\exyear{NET/JRF(DEC-2015)}}
	\begin{tasks}(2)
		\task[\textbf{A.}] $\frac{\partial^{2} G}{\partial x^{2}}-2 x \frac{\partial G}{\partial x}+2 t \frac{\partial G}{\partial t}=0$
		\task[\textbf{B.}] $\frac{\partial^{2} G}{\partial x^{2}}-2 x \frac{\partial G}{\partial x}-2 t^{2} \frac{\partial G}{\partial t}=0$
		\task[\textbf{C.}] $\frac{\partial^{2} G}{\partial x^{2}}-2 x \frac{\partial G}{\partial x}+2 \frac{\partial G}{\partial t}=0$
		\task[\textbf{D.}]  $\frac{\partial^{2} G}{\partial x^{2}}-2 x \frac{\partial G}{\partial x}+2 \frac{\partial^{2} G}{\partial x \partial t}=0$
	\end{tasks}
	\item A stable asymptotic solution of the equation $x_{n+1}=1+\frac{3}{1+x_{n}}$ is $x=2$. If we take $x_{n}=2+\epsilon_{n}$ and $x_{n+1}=2+\epsilon_{n+1}$, where $\epsilon_{n}$ and $\epsilon_{n+1}$ are both small, the ratio $\frac{\epsilon_{n+1}}{\epsilon_{n}}$ is approximately
	{\exyear{NET/JRF(DEC-2016)}}
	\begin{tasks}(4)
		\task[\textbf{A.}] $-\frac{1}{2}$
		\task[\textbf{B.}] $-\frac{1}{4}$
		\task[\textbf{C.}]  $-\frac{1}{3}$
		\task[\textbf{D.}] $-\frac{2}{3}$
	\end{tasks}
	\item  The Green's function satisfying
	$$
	\frac{d^{2}}{d x^{2}} g\left(x, x_{0}\right)=\delta\left(x-x_{0}\right)
	$$
	with the boundary conditions $g\left(-L, x_{0}\right)=0=g\left(L, x_{0}\right)$, is
	{\exyear{NET/JRF(JUNE-2017)}}
	\begin{tasks}(1)
		\task[\textbf{A.}] $\left\{\begin{array}{ll}\frac{1}{2 L}\left(x_{0}-L\right)(x+L), & -L \leq x<x_{0} \\ \frac{1}{2 L}\left(x_{0}+L\right)(x-L), & x_{0} \leq x \leq L\end{array}\right.$
		\task[\textbf{B.}]  $\left\{\begin{array}{ll}\frac{1}{2 L}\left(x_{0}+L\right)(x+L), & -L \leq x<x_{0} \\ \frac{1}{2 L}\left(x_{0}-L\right)(x-L), & x_{0} \leq x \leq L\end{array}\right.$
		\task[\textbf{C.}] $\left\{\begin{array}{ll}\frac{1}{2 L}\left(L-x_{0}\right)(x+L), & -L \leq x<x_{0} \\ \frac{1}{2 L}\left(x_{0}+L\right)(L-x), & x_{0} \leq x \leq L\end{array}\right.$
		\task[\textbf{D.}] $\frac{1}{2 L}(x-L)(x+L), \quad-L \leq x \leq L$
	\end{tasks}
	\item  The generating function $G(t, x)$ for the Legendre polynomials $P_{n}(t)$ is
	$$
	G(t, x)=\frac{1}{\sqrt{1-2 x t+x^{2}}}=\sum_{n=0}^{\infty} x^{n} P_{n}(t), \text { for }|x|<1
	$$
	If the function $f(x)$ is defined by the integral equation $\int_{0}^{x} f\left(x^{\prime}\right) d x^{\prime}=x G(1, x)$, it can be expressed as
	{\exyear{NET/JRF(DEC-2017)}}
	\begin{tasks}(2)
		\task[\textbf{A.}] $\sum_{n, m=0}^{\infty} x^{n+m} P_{n}(1) P_{m}\left(\frac{1}{2}\right)$
		\task[\textbf{B.}] $\sum_{n, m=0}^{\infty} x^{n+m} P_{n}(1) P_{m}(1)$
		\task[\textbf{C.}] $\sum_{n, m=0}^{\infty} x^{n-m} P_{n}(1) P_{m}(1)$
		\task[\textbf{D.}] $\sum_{n, m=0}^{\infty} x^{n-m} P_{n}(0) P_{m}(1)$
	\end{tasks}
	\item In the function $P_{n}(x) e^{-x^{2}}$ of a real variable $x, P_{n}(x)$ is polynomial of degree $n$. The maximum number of extrema that this function can have is
	{\exyear{NET/JRF(JUNE-2018)}}
	\begin{tasks}(4)
		\task[\textbf{A.}] $n+2$
		\task[\textbf{B.}]  $n-1$
		\task[\textbf{C.}] $n+1$
		\task[\textbf{D.}] $n$
	\end{tasks}
	\item  The Green's function $G\left(x, x^{\prime}\right)$ for the equation $\frac{d^{2} y(x)}{d x^{2}}+y(x)=f(x)$, with the boundary values $y(0)=y\left(\frac{\pi}{2}\right)=0$, is
	{\exyear{NET/JRF(JUNE-2018)}}
	\begin{tasks}(1)
		\task[\textbf{A.}] $G\left(x, x^{\prime}\right)=\left\{\begin{array}{ll}x\left(x^{\prime}-\frac{\pi}{2}\right), & 0<x<x^{\prime}<\frac{\pi}{2} \\ \left(x-\frac{\pi}{2}\right) x^{\prime}, & 0<x^{\prime}<x<\frac{\pi}{2}\end{array}\right.$
		\task[\textbf{B.}] $G\left(x, x^{\prime}\right)=\left\{\begin{array}{ll}-\cos x^{\prime} \sin x, & 0<x<x^{\prime}<\frac{\pi}{2} \\ -\sin x^{\prime} \cos x, & 0<x^{\prime}<x<\frac{\pi}{2}\end{array}\right.$
		\task[\textbf{C.}] $G\left(x, x^{\prime}\right)=\left\{\begin{array}{ll}\cos x^{\prime} \sin x, & 0<x<x^{\prime}<\frac{\pi}{2} \\ \sin x^{\prime} \cos x, & 0<x^{\prime}<x<\frac{\pi}{2}\end{array}\right.$
		\task[\textbf{D.}] $G\left(x, x^{\prime}\right)=\left\{\begin{array}{ll}x\left(\frac{\pi}{2}-x^{\prime}\right), & 0<x<x^{\prime}<\frac{\pi}{2} \\ x^{\prime}\left(\frac{\pi}{2}-x\right), & 0<x^{\prime}<x<\frac{\pi}{2}\end{array}\right.$
	\end{tasks}
	\item The polynomial $f(x)=1+5 x+3 x^{2}$ is written as linear combination of the Legendre polynomials
	$\left(P_{0}(x)=1, P_{1}(x), P_{2}(x)=\frac{1}{2}\left(3 x^{2}-1\right)\right)$ as $f(x)=\sum_{n} c_{n} P_{n}(x)$. The value of $c_{0}$ is
	{\exyear{NET/JRF(DEC-2018)}}
	\begin{tasks}(4)
		\task[\textbf{A.}] $\frac{1}{4}$
		\task[\textbf{B.}] $\frac{1}{2}$
		\task[\textbf{C.}]  2
		\task[\textbf{D.}]  4
	\end{tasks}
	\item The Green's function $G\left(x, x^{\prime}\right)$ for the equation $\frac{d^{2} y(x)}{d x^{2}}=f(x)$, with the boundary values $y(0)=0$ and $y(1)=0$, is
	{\exyear{NET/JRF(DEC-2018)}}
	\begin{tasks}(1)
		\task[\textbf{A.}] $G\left(x, x^{\prime}\right)=\left\{\begin{array}{ll}\frac{1}{2} x\left(1-x^{\prime}\right), & 0<x<x^{\prime}<1 \\ \frac{1}{2} x^{\prime}(1-x) & 0<x^{\prime}<x<1\end{array}\right.$
		\task[\textbf{B.}] $G\left(x, x^{\prime}\right)=\left\{\begin{array}{ll}x\left(x^{\prime}-1\right), & 0<x<x^{\prime}<1 \\ x^{\prime}(1-x) & 0<x^{\prime}<x<1\end{array}\right.$
		\task[\textbf{C.}] $G\left(x, x^{\prime}\right)=\left\{\begin{array}{ll}-\frac{1}{2} x\left(1-x^{\prime}\right), & 0<x<x^{\prime}<1 \\ \frac{1}{2} x^{\prime}(1-x) & 0<x^{\prime}<x<1\end{array}\right.$
		\task[\textbf{D.}]  $G\left(x, x^{\prime}\right)=\left\{\begin{array}{ll}x\left(x^{\prime}-1\right), & 0<x<x^{\prime}<1 \\ x^{\prime}(x-1) & 0<x^{\prime}<x<1\end{array}\right.$
	\end{tasks}
	\item  The Green's function for the differential equation $\frac{d^{2} x}{d t^{2}}+x=f(t)$, satisfying the initial conditions $x(0)=\frac{d x}{d t}(0)=0$ is\\
	$$G(t, \tau)=\left\{\begin{array}{ll}0 & \text { for } \quad 0<t<\tau \\ \sin (t-\tau) & \text { for } \quad t>\tau\end{array}\right.$$\\
	The solution of the differential equation when the source $f(t)=\theta(t)$ (the Heaviside step function) is
	{\exyear{NET/JRF(JUNE-2020)}}
	\begin{tasks}(4)
		\task[\textbf{A.}] $\sin t$
		\task[\textbf{B.}] $1-\sin t$
		\task[\textbf{C.}] $1-\cos t$
		\task[\textbf{D.}] $\cos ^{2} t-1$
	\end{tasks}
\end{enumerate}
 \colorlet{ocre1}{ocre!70!}
\colorlet{ocrel}{ocre!30!}
\setlength\arrayrulewidth{1pt}
\begin{table}[H]
	\centering
	\arrayrulecolor{ocre}
	\begin{tabular}{|p{1.5cm}|p{1.5cm}||p{1.5cm}|p{1.5cm}|}
		\hline
		\multicolumn{4}{|c|}{\textbf{Answer key}}\\\hline\hline
		\rowcolor{ocrel}Q.No.&Answer&Q.No.&Answer\\\hline
		1&\textbf{D} &2&\textbf{D}\\\hline 
		3&\textbf{A} &4&\textbf{A} \\\hline
		5&\textbf{D} &6&\textbf{C} \\\hline
		7&\textbf{A}&8&\textbf{C}\\\hline
		9&\textbf{A}&10&\textbf{B}\\\hline
		11&\textbf{C} &12&\textbf{B}\\\hline
		13&\textbf{C}&14&\textbf{D}\\\hline
		15&\textbf{C}& &\\\hline
		
	\end{tabular}
\end{table}
\begin{abox}
	Problem Set -3
\end{abox}
\begin{enumerate}[label=\color{ocre}\textbf{\arabic*.}]
	\item Green function for time dependent Schrödinger wave equation is defined as $G\left(\vec{r}, t: r^{\prime}, t^{\prime}\right)$. If $H$ is Hamiltonion of system then $G\left(\vec{r}, t: r^{\prime}, t^{\prime}\right)$ will satisfied the equation
	 \begin{tasks}(1)
		\task[\textbf{a.}]$\left(i \hbar \frac{\partial}{\partial t}-H\right) G\left(\vec{r}, t ; \vec{r}^{\prime}, t^{\prime}\right)=0$
		\task[\textbf{b.}]$\left(i \hbar \frac{\partial}{\partial t}-H\right) G\left(\vec{r}, t ; \vec{r}^{\prime}, t^{\prime}\right)=\delta\left(\vec{r}-\vec{r}^{\prime}\right)$
		\task[\textbf{c.}] $\left(i \hbar \frac{\partial}{\partial t}-H\right) G\left(\vec{r}, t ; \vec{r}^{\prime}, t^{\prime}\right)=\delta\left(t-t^{\prime}\right)$
		\task[\textbf{d.}]  $\left(i \hbar \frac{\partial}{\partial t}-H\right) G\left(\vec{r}, t ; \vec{r}^{\prime}, t^{\prime}\right)=\delta\left(\vec{r}-\vec{r}^{\prime}\right) \delta\left(t-t^{\prime}\right)$
	\end{tasks}
\begin{answer}
So the correct answer is \textbf{Option (d)}
\end{answer}
	\item $G\left(x, x_{0}\right)$ is the Green's function associated with the boundary value problem consisting of ordinary differential equation.
	$$
	\frac{d}{d x}\left(p(x) \frac{d u}{d x}\right)=f(x) \text { with } u(0)=0, u(L)=0
	$$
	The discontinuity condition on the derivative $\frac{d G\left(x, x_{0}\right)}{d x}$ at $x=x_{0}$ is
	 \begin{tasks}(4)
		\task[\textbf{a.}]0
		\task[\textbf{b.}]$p\left(x_{0}\right)$
		\task[\textbf{c.}]1
		\task[\textbf{d.}] $\frac{1}{p\left(x_{0}\right)}$
	\end{tasks}
\begin{answer}
	\begin{align*}
	\left.\frac{d G}{d x}\right|_{x=x_{0}^{+}}-\left.\frac{d G}{d x}\right|_{x=x_{i 1}^{-}}=\frac{1}{p\left(x_{0}\right)}
	\end{align*}
	So the correct answer is \textbf{Option (d)}
\end{answer}
\item Consider the steady state heat equation $\frac{d^{2} u}{d x^{2}}=f(x)$ with boundary condition,
$$
u(0)=0, u(L)=0
$$
The Green's function associated with the above equation
 \begin{tasks}(2)
	\task[\textbf{a.}]Constant
	\task[\textbf{b.}] Linear function
	\task[\textbf{c.}] Parabolic function
	\task[\textbf{d.}] Hyperbolic function
\end{tasks}
\begin{answer}
	\begin{align*}
	\intertext{The Green's function satisfies}
	\frac{d^{2} G\left(x, x_{0}\right)}{d x^{2}}&=\delta\left(x-x_{0}\right)\\
\text{	with }G\left(0, x_{0}\right)&=0\text{ and }G\left(L, x_{0}\right)=0
\intertext{Corresponding homogeneous equation is:}
\frac{d^{2} G}{d x^{2}}&=0\\
\text{Solution for }x \neq x_{0}&\text{ are, }G\left(x, x_{0}\right)= \begin{cases}a+b x_{2} & x<x_{1+} \\ c+d x, & x>x_{0}\end{cases}
	\end{align*}
		So the correct answer is \textbf{Option (b)}
\end{answer}
\item Consider the steady state heat equation $\frac{d^{2} u}{d x^{2}}=f(x)$ with boundary condition. $u(0)=0, u(L)=0$
The Green's function associated with the above equation is
 \begin{tasks}(1)
	\task[\textbf{a.}] $G\left(x, x_{0}\right)= \begin{cases}\frac{x}{L}\left(x_{0}-L\right), & 0 \leq x \leq x_{0} \\ \frac{x_{0}}{L}(x-L), & x_{0} \leq x \leq L\end{cases}$
	\task[\textbf{b.}] $G\left(x, x_{0}\right)= \begin{cases}\frac{x}{L}\left(L-x_{0}\right), & 0 \leq x \leq x_{0} \\ \frac{x_{0}}{L}(L-x), & x_{0} \leq x \leq L\end{cases}$
	\task[\textbf{c.}] $G\left(x, x_{0}\right)= \begin{cases}\sqrt{\frac{x}{L},} &\quad 0 \leq x \leq x_{0} \\ \sqrt{\frac{(x-L)}{L}}, & \quad x_{0} \leq x \leq L\end{cases}$
	\task[\textbf{d.}] $G\left(x, x_{0}\right)= \begin{cases}\sqrt{\frac{L-x}{L},}, & 0 \leq x \leq x_{0} \\ \sqrt{\frac{(x)}{L}}, & x_{0} \leq x \leq L\end{cases}$
\end{tasks}
\begin{answer}
	\begin{align*}
	\intertext{The Green's function satisfies}
	\frac{d^{2} G\left(x, x_{0}\right)}{d x^{2}}&=\delta\left(x-x_{0}\right)\\
	\text{with }G\left(0, x_{0}\right)&=0\text{ and }G\left(L, x_{0}\right)=0
	\intertext{Corresponding homogeneous equation is:}
	\frac{d^{2} G}{d x^{2}}&=0\\
	\text{Solution for }&x \neq x_{0}\text{ are}\\
	G\left(x, x_{0}\right)&= \begin{cases}a+b x, & x<x_{0} \\ c+d x, & x>x_{0}\end{cases}
	\intertext{From boundary conditions:}
	G\left(0, x_{0}\right)&=0 \Rightarrow a=0\\
	G\left(L, x_{0}\right)&=0 \Rightarrow c=-d L\\
	\therefore G\left(x, x_{0}\right)&= \begin{cases}b x, & x<x_{0} \\ d(x-L), & x>x_{0}\end{cases}\\
	\text{From continuity of }&\text{Green's function at }x=x_{0},\text{ we have}\\
	b x_{0}&=d\left(x_{0}-L\right)\\
	b&=\frac{d\left(x_{0}-L\right)}{x_{0}}\\
	\text{From discontinuity of }&\frac{\partial G}{\partial x}\text{ at }x=x_{0}\text{, we have}\\
	\left.\frac{\partial G}{\partial x}\right|&_{x=x_{0}^{+}}-\left.\frac{\partial G}{\partial x}\right|_{x=x_{0}^{-}}=1\\
	d-b&=1\\
	\Rightarrow d&=b+1 \Rightarrow d=\frac{d\left(x_{0}-L\right)}{x_{0}}+1 \Rightarrow d x_{0}=d x_{0}-d L+x_{0}\\
	\Rightarrow d&=\frac{x_{0}}{L}, b=d-1=\left(\frac{x_{0}}{L}-1\right)\\
	\therefore G\left(x, x_{0}\right)&= \begin{cases}\frac{x}{L}\left(x_{0}-L\right), & 0 \leq x \leq x_{0} \\ \frac{x_{0}}{L}(x-L), & x_{0} \leq x \leq L\end{cases}
	\end{align*}
		So the correct answer is \textbf{Option (a)}
\end{answer}
\item The differential equation defined as $\frac{d^{2} y}{d x^{2}}=f(x)$ With boundary conditions $\quad y(0)=0$ and $y^{\prime}(1)=0$
The green function $G\left(x, x_{0}\right)$ satisfy the
 \begin{tasks}(2)
	\task[\textbf{a.}]$G\left(x, x_{0}\right)= \begin{cases}x & \text { if } x<x_{0} \\ x_{0} & \text { if } x>x_{0}\end{cases}$
	\task[\textbf{b.}]$G\left(x, x_{0}\right)= \begin{cases}-x & \text { if } x<x_{0} \\ -x_{0} & \text { if } x>x_{0}\end{cases}$
	\task[\textbf{c.}]$G\left(x, x_{0}\right)= \begin{cases}x^{2} & \text { if } x<x_{0} \\ -x_{0} & \text { if } x>x_{0}\end{cases}$
	\task[\textbf{d.}] $G\left(x, x_{0}\right)= \begin{cases}-x^{2} & \text { if } x<x_{0} \\ -x_{0} & \text { if } x>x_{0}\end{cases}$
\end{tasks}
\begin{answer}
	\begin{align}
	\intertext{The corresponding non-homogenous differential equation for Green's function is}\notag\\
	\frac{\partial^{2}}{\partial x^{2}} G\left(x, x_{0}\right)&=\delta\left(x-x_{0}\right)\\
	\text{With }G\left(0, x_{0}\right)&=0\text{ and }G^{\prime}\left(1, x_{0}\right)=0\notag\notag\\
\text{	Let }&\frac{\partial^{2}}{\partial x^{2}} G\left(x, x_{0}\right)=0\notag\\
\Rightarrow G\left(x, x_{0}\right)&= \begin{cases}A x+B, & x<x_{0} \\ C x+D, & x>x_{0}\end{cases}\label{SF-01}
\intertext{Using booundary condition, we have}\notag\\
B&=0\text{ and }C=0\notag\\
\therefore&\text{ equation (\ref{SF-01}) becomes}\notag\\
G\left(x, x_{b}\right)&= \begin{cases}A x, & x<x_{0} \\ D, & x>x_{i 1}\end{cases}\notag\\
\text{From continuity of }&\left(f\left(x, x_{0}\right)\right.\text{ at }x=x_{0}\text{, we have}\notag\\
A x_{0}&=D
\intertext{From discontinuity of first derivative of Green's function i.c. $\frac{\partial G}{\partial x}$ at $x=x_{0}$ we have}
\left.\frac{\partial G}{\partial x}\right|_{x=x_{0}^{+}}-\left.\frac{\partial G}{\partial x}\right|&_{x=x_{0}^{-}}=1\notag\\
\Rightarrow 0-A&=1 \Rightarrow A=-1\notag\\
\text{and }D&=-x_{0}\notag\\
\therefore G\left(x, x_{0}\right)&= \begin{cases}-x & \text { if } x<x_{0} \notag\\ -x_{0} & \text { if } x>x_{0}\end{cases}
	\end{align}
	So the correct answer is \textbf{Option (b)}
\end{answer}
\item For real $n$ the cylindrical Bessel function of order $n$ is $J_{n}(x)$ then $J_{1 / 2}$ will converge to
 \begin{tasks}(4)
	\task[\textbf{a.}]0
	\task[\textbf{b.}]1
	\task[\textbf{c.}] $-1$
	\task[\textbf{d.}] $\frac{1}{2}$
\end{tasks}
\begin{answer}
	\begin{align*}
	{{\color{red}{Not completed}}}\\
	\end{align*}
	So the correct answer is \textbf{Option (a)}
\end{answer}
\item For real $n$ the cylindrical Bessel function is $J_{n}(x)$ of order $n$ then behavior $J_{1 / 2}$ will behave $x \approx 0$ as
 \begin{tasks}(4)
	\task[\textbf{a.}] 0
	\task[\textbf{b.}]$\sqrt{\frac{2 x}{\pi}}$
	\task[\textbf{c.}]$\sqrt{\frac{x}{\pi}}$
	\task[\textbf{d.}]  $\sqrt{\frac{x}{2 \pi}}$
\end{tasks}
\begin{answer}
	\begin{align*}
	{{\color{red}{Not completed}}}\\
	J_{n}(x)&=\sum_{0}^{\infty} \frac{(-1)^{r}}{[r \mid n+r}\left(\frac{x}{2}\right)^{n+2 r} \Rightarrow J_{1 / 2}(x)=\sum_{0}^{\infty} \frac{(-1)^{r}}{\left\lfloor\frac{1}{2}+r\right.}\left(\frac{x}{2}\right)^{\frac{1}{2}+2 r}\\
	\text{Put }r&=0 \frac{\sqrt{x / 2}}{\frac{1}{2}}=\sqrt{\frac{2 x}{\pi}} \text{where }\frac{1}{2}=\frac{\sqrt{\pi}}{2}
	\end{align*}
		So the correct answer is \textbf{Option (b)}
\end{answer}
\item For real $n$ the cylindrical Bessel function is $J_{n}(x)$ of order $n$ then behavior $J_{1 / 2}$ will equivalent to (it is given that $\underline{r} \cdot\left\lfloor r-\frac{1}{2}=\left[(2 r) 2^{-r} \sqrt{\pi}\right)\right.$
 \begin{tasks}(4)
	\task[\textbf{a.}] $\sqrt{\frac{2}{\pi}} \frac{\sin x}{\sqrt{x}}$
	\task[\textbf{b.}]$\sqrt{\frac{2}{\pi}} \frac{\sin x}{x}$
	\task[\textbf{c.}]$\sqrt{\frac{2}{\pi}} \frac{\cos x}{\sqrt{x}}$
	\task[\textbf{d.}] $\sqrt{\frac{2}{\pi}} \frac{\cos }{x}$
\end{tasks} 
\begin{answer}
	\begin{align*}
	{{\color{red}{Not completed}}}\\
	\end{align*}
\end{answer}
\item For real $n$ the cylindrical Bessel function is $J_{n}(x)$ of order $n$ then $J_{n}(x)$ will satisfied differential equation
 \begin{tasks}(1)
	\task[\textbf{a.}]$\frac{d^{2} J_{n}}{d x^{2}}+\frac{1}{x}\left(\frac{d J_{n}}{d x}\right)+\left(1+\frac{n^{2}}{x^{2}}\right) J_{n}=0$
	\task[\textbf{b.}] $\frac{d^{2} J_{n}}{d x^{2}}+\frac{1}{x}\left(\frac{d J_{n}}{d x}\right)+\left(1-\frac{n^{2}}{x^{2}}\right) J_{n}=0$
	\task[\textbf{c.}] $\frac{d^{2} J_{n}}{d x^{2}}+x\left(\frac{d J_{n}}{d x}\right)+\left(1+\frac{n^{2}}{x^{2}}\right) J_{n}=0$
	\task[\textbf{d.}] $\frac{d^{2} J_{n}}{d x^{2}}+x\left(\frac{d J_{n}}{d x}\right)+\left(1-\frac{n^{2}}{x^{2}}\right) J_{n}=0$
\end{tasks}
\begin{answer}
	\begin{align*}
\text{The Bessel function is given by }\frac{d^{2} J_{n}}{d x^{2}}+\frac{1}{x}\left(\frac{d J_{n}}{d x}\right)+\left(1-\frac{n^{2}}{x^{2}}\right) J_{n}=0
	\end{align*}
		So the correct answer is \textbf{Option (b)}
\end{answer}
\item For real $n$ the cylindrical Bessel function is $J_{n}(x)$ of order $n$ then value of $\frac{d J_{0}}{d x}$ is equivalent to 
 \begin{tasks}(4)
	\task[\textbf{a.}] $J_{1}$
	\task[\textbf{b.}]$-J_{1}$
	\task[\textbf{c.}]$2 J_{1}$
	\task[\textbf{d.}]$-2 J_{1}$
\end{tasks}
\begin{answer}
	\begin{align*}
J_{n+1}(x)=-J_{n}^{\prime}(x)+\frac{n}{x} J_{n}\text{. for }n=0, J_{1}=-J_{0}^{\prime}
	\end{align*}
		So the correct answer is \textbf{Option (b)}
\end{answer}
\item  The differential equation $x^{2} \frac{d^{2} y}{d x^{2}}+2 x \frac{d y}{d x}+\left[x^{2}-\lambda\right] y(x)=0$ is spherical Bessel's differential equation of order $n$ then value of $\lambda$ is given by
 \begin{tasks}(4)
	\task[\textbf{a.}]$n$
	\task[\textbf{b.}]$n(n+1)$
	\task[\textbf{c.}] $n(n-1)$
	\task[\textbf{d.}]  $n^{2}$
\end{tasks}
\begin{answer}
	\begin{align*}
\text{Spherical Bessel's differential equation }x^{2} \frac{d^{2} y}{d x^{2}}+2 x \frac{d y}{d x}+\left[x^{2}-n(n+1)\right] y(x)=0
	\end{align*}
		So the correct answer is \textbf{Option (b)}
\end{answer}
\item If $J_{n}(x)$ is spherical Bessel function of order $n$ if $N_{n}(x)$ is spherical Neumann function of order $n$ and $h_{n}^{\prime}$ is spherical Hankel function of type one of order $n$. Then $h_{0}^{1}$ is given by
 \begin{tasks}(4)
	\task[\textbf{a.}]$i \frac{e^{-i x}}{x}$
	\task[\textbf{b.}]$-i \frac{e^{-i x}}{x}$
	\task[\textbf{c.}] $i \frac{e^{i x}}{x}$
	\task[\textbf{d.}] $-i \frac{e^{i x}}{x}$
\end{tasks}
\begin{answer}
	\begin{align*}
		h_{n}^{1}&=J_{n}+i N_{n}\\
	J_{0}(x)&=\frac{\sin x}{x}, N_{0}(x)=-\frac{\cos x}{x} \Rightarrow h_{0}^{\prime^{\prime}}=J_{0}+i N_{0}=\frac{\sin x-i \cos x}{\because x}=-i \frac{e^{i x}}{x}
	\end{align*}
	So the correct answer is \textbf{Option (d)}
\end{answer}
\item If $J_{n}(x)$ is spherical Bessel function of order $n$ if $N_{n}(x)$ is spherical Neumann function of order $n$ and $h_{n}^{2}$ is spherical Hankel function of type two of order $n$. Then $h_{0}^{2}$ is given by
 \begin{tasks}(4)
	\task[\textbf{a.}] $i \frac{e^{-i x}}{x}$
	\task[\textbf{b.}] $-i \frac{e^{-i x}}{x}$
	\task[\textbf{c.}]$i \frac{e^{i x}}{x}$
	\task[\textbf{d.}] $-i \frac{e^{i x}}{x}$
\end{tasks}
\begin{answer}
	\begin{align*}
	h_{n}^{2}&=J_{n}-i N_{n}\\
	J_{0}(x)&=\frac{\sin x}{x},N_{0}(x)=-\frac{\cos x}{x} \Rightarrow h_{0}^{1}=J_{0}+i N_{0} \Rightarrow \frac{\sin x+i \cos x}{x}=i \frac{e^{-i x}}{x}
	\end{align*}
		So the correct answer is \textbf{Option (a)}
\end{answer}
\item If $J_{n}(x)$ is spherical Bessel function of order $n$ then $j_{0}^{\prime}(x)$ is equivalent to
 \begin{tasks}(4)
	\task[\textbf{a.}]$j_{1}(x)$
	\task[\textbf{b.}]$-j_{1}(x)$
	\task[\textbf{c.}]$\frac{j_{1}(x)}{2}$
	\task[\textbf{d.}]$-\frac{j_{1}(x)}{2}$
\end{tasks}
\begin{answer}
	\begin{align*}
	\frac{d}{d x}\left(j_{0}(x)\right)&=\frac{d}{d x}\left(\frac{\sin x}{x}\right)=\frac{\cos x}{x}-\frac{\sin x}{x^{2}}=-J_{1}(x)\\
	\text{Where }j_{1}(x)&=-\frac{\cos x}{x}+\frac{\sin x}{x^{2}}
	\end{align*}
	So the correct answer is \textbf{Option (b)}
\end{answer}
\item The solution of the differential equation $x^{2} \frac{d^{2} y}{d x^{2}}+2 x \frac{d y}{d x}+x^{2} y(x)=0$ subjected to the condition is given by $y(0)=1$.
 \begin{tasks}(4)
	\task[\textbf{a.}] $\frac{\sin x}{x}$
	\task[\textbf{b.}] $\frac{\cos x}{x}$
	\task[\textbf{c.}]$\frac{\exp (-i x)}{x}$
	\task[\textbf{d.}] $\frac{\exp i x}{x}$
\end{tasks}
\begin{answer}
	\begin{align*}
 \text{Spherical Bessel's differential equation }&x^{2} \frac{d^{2} y}{d x^{2}}+2 x \frac{d y}{d x}+\left[x^{2}-n(n+1)\right] y(x)=0\\
 \text{ then }x^{2} \frac{d^{2} y}{d x^{2}}+2 x \frac{d y}{d x}+x^{2} y(x)=0 &\text{ is spherical Bessel's differential equation for order}\\
 n&=0\\
	\text{then solution is }J_{0}(x)&=\frac{\sin x}{x}\text{ with boundary condition }y(0)=1.
	\end{align*}
	So the correct answer is \textbf{Option (a)}
\end{answer}
\item $H_{n}(x)$ is Hermite polynomials of order $n$ then $H_{n}(x)=(-1)^{n} f(x) \frac{d^{n}(W(x))}{d x^{n}}$, then $f(x)$ and $W(x)$ are respectively
 \begin{tasks}(1)
	\task[\textbf{a.}]$f(x)=\exp \left(x^{2}\right), W(x)=\exp \left(-x^{2}\right)$
	\task[\textbf{b.}]$f(x)=\exp \left(-x^{2}\right), W=\exp \left(x^{2}\right)$
	\task[\textbf{c.}] $f(x)=W(x)=\exp \left(x^{2}\right)$
	\task[\textbf{d.}] $f(x)=W(x)=\exp \left(-x^{2}\right)$
\end{tasks}
\begin{answer}
	\begin{align*}
	H_{n}(x)&=(-1)^{n} \exp \left(x^{2}\right) \frac{d^{n}\left(\exp \left(-x^{2}\right)\right)}{d x^{n}}\\
	\text{So after comparing }H_{n}(x)&=(-1)^{n} f(x) \frac{d^{n}(W(x))}{d x^{n}}\\
	f(x)&=\exp \left(x^{2}\right), W(x)=\exp \left(-x^{2}\right)
	\end{align*}
		So the correct answer is \textbf{Option (a)}
\end{answer}
\item The solution of differential equation $\frac{d^{2} y}{d x^{2}}-2 x \frac{d y}{d x}+\lambda y(x)=0$ is Hermilte polynomial of order $n$ then value of $\lambda$ is
 \begin{tasks}(4)
	\task[\textbf{a.}]$n$
	\task[\textbf{b.}] $-n$
	\task[\textbf{c.}]$2 n$
	\task[\textbf{d.}] $-2 n$
\end{tasks}
\begin{answer}
	\begin{align*}
	\frac{d^{2} y}{d x^{2}}-2 x \frac{d y}{d x}+2 n y(x)=0\text{ is Hermite differential equation}
	\end{align*}
		So the correct answer is \textbf{Option (c)}
\end{answer}
\item The Rodrigues formula for Laguerre polunomial is given by
 \begin{tasks}(2)
	\task[\textbf{a.}]$L_n(x)=\frac{e^{-x}}{n !}\left(\frac{d}{d x}\right)^{n}\left(x^{n} e^{-x}\right)$
	\task[\textbf{b.}]$L_{n}(x)=\frac{e^{x}}{n !}\left(\frac{d}{d x}\right)^{n}\left(x^{n} e^{x}\right)$
	\task[\textbf{c.}]$L_n(x)=\frac{e^{-x}}{n !}\left(\frac{d}{d x}\right)^{n}\left(x^{n} e^{x}\right)$
	\task[\textbf{d.}] $L_{n}(x)=\frac{e^{x}}{n !}\left(\frac{d}{d x}\right)^{n}\left(x^{n} e^{-x}\right)$
\end{tasks}
\begin{answer}
	\begin{align*}
	L_{n}(x)=\frac{e^{x}}{n !}\left(\frac{d}{d x}\right)^{n}\left(x^{n} e^{-x}\right)
	\end{align*}
		So the correct answer is \textbf{Option (d)}
\end{answer}
\item It is given that operator $x-\frac{d}{d x}=-\exp \left(\frac{x^{2}}{2}\right) \frac{d}{d x} \exp \left(-\frac{x^{2}}{2}\right)$
If then the normalized wave function for harmonic oscillation is $\psi(x)=\left(\pi^{1 / 2} 2^{n}\lfloor n)^{-1 / 2} \exp \left(-\frac{x^{2}}{2}\right) H_{n}(x)\right.$, then $\psi_n(x)$ is equivalent to 
 \begin{tasks}(1)
	\task[\textbf{a.}]$\psi_{n}(x)=\left(\pi^{1 / 2} 2^{n}\lfloor n)^{-1 / 2}\left(x-\frac{d}{d x}\right)^{n} \exp \left(-\frac{x^{2}}{2}\right)\right.$
	\task[\textbf{b.}] $\psi_{n}(x)=\left(\pi^{1 / 2} 2^{n}\lfloor n)^{-1 / 2}\left(x-\frac{d}{d x}\right)^{2 n} \exp \left(\frac{x^{2}}{2}\right)\right.$
	\task[\textbf{c.}] $\psi_{n}(x)=\left(\pi^{1 / 2} 2^{n}\lfloor n)^{-1 / 2}\left(x-\frac{d}{d x}\right)^{n} \exp \left(-x^{2}\right)\right.$
	\task[\textbf{d.}] $\psi_{n}(x)=\left(\pi^{k / 2} 2^{n}\lfloor n)^{-1 / 2}\left(x-\frac{d}{d x}\right)^{2 n} \operatorname{cxp}\left(-x^{2}\right)\right.$
\end{tasks}
\begin{answer}
	\begin{align*}
	H_{n}(x)&=(-1)^{n} \exp \left(x^{2}\right) \frac{d^{n}\left(\exp \left(-x^{2}\right)\right)}{d x^{n}}\\
	x-\frac{d}{d x} &=-\exp \left(\frac{x^{2}}{2}\right) \frac{d}{d x} \exp \left(-\frac{x^{2}}{2}\right) \Rightarrow\left(x-\frac{d}{d x}\right) \exp \left(-\frac{x^{2}}{2}\right) \\ &\left.=-\exp \left(\frac{x^{2}}{2}\right) \frac{d}{d x} \exp \left(-\frac{x^{2}}{2}\right)\right) \exp \left(-\frac{x^{2}}{2}\right)\\
	x \exp \left(-\frac{x^{2}}{2}\right)-\frac{d \exp \left(-\frac{x^{2}}{2}\right)}{d x}&=-\exp \left(\frac{x^{2}}{2}\right) \frac{d}{d x} \exp \left(-x^{2}\right)\\
	\Rightarrow\left(x-\frac{d}{d x}\right) \exp \left(-\frac{x^{2}}{2}\right)&=\exp \left(\frac{x^{2}}{2}\right)\left(-2 x \exp \left(-x^{2}\right)\right)=2 x \exp -\frac{x^{2}}{2}=H_{1}\left(\exp -\frac{x^{2}}{2}\right)\\
	\text{where }2 x&=H_{1}(x)\\
	\text{Similarly }\left(x-\frac{d}{d x}\right)^{n} \exp \left(-\frac{x^{2}}{2}\right)&=H_{n} \exp \left(-\frac{x^{2}}{2}\right)\\
	\psi_{n}(x)&=\left(\pi^{1 / 2} 2^{n}\lfloor n)^{-1 / 2}\left(x-\frac{d}{d x}\right)^{n} \exp \left(-\frac{x^{2}}{-2}\right)\right.
	\end{align*}
	So the correct answer is \textbf{Option (a)}
\end{answer}
\item The solution of differential equation $x \frac{d^{2} y}{d x^{2}}+(1-x) \frac{d y}{d x}+\lambda y(x)=0$ is Laguerre polynomials of order $n$ then value of $\lambda$ is
 \begin{tasks}(4)
	\task[\textbf{a.}]$n$
	\task[\textbf{b.}]$-n$
	\task[\textbf{c.}] $2 n$
	\task[\textbf{d.}] $-2 n$
\end{tasks}
\begin{answer}
	\begin{align*}
	x \frac{d^{2} y}{d x^{2}}+(1-x) \frac{d y}{d x}+n y(x)=0\text{ is Laguerre differential equation.}
	\end{align*}
	So the correct answer is \textbf{Option (a)}
\end{answer}
\item The generating function $F(x, t)=\sum_{n=0}^{\infty} P_{n}(x) t^{n}$ for the Legendre polynomials $P_{n}(x)$ is $F(x, t)=\left(1-2 x t+t^{2}\right)^{-1 / 2}$. The value of $P_{2}(-1)$ is
 \begin{tasks}(4)
	\task[\textbf{a.}]$5 / 2$
	\task[\textbf{b.}]$3 / 2$
	\task[\textbf{c.}] $+1$
	\task[\textbf{d.}] $-1$
\end{tasks}
\begin{answer}
	\begin{align*}
	\text{The generating function for Legendre polynomial is }F(x, t)&=\left(1-2 x t+t^{2}\right)^{-1 / 2}.\text{ Thus}\\P_{2}(x)=\frac{1}{2}\left(3 x^{2}-1\right) \Rightarrow P_{2}(-1)=\frac{1}{2}(3-1)=1
	\end{align*}
		So the correct answer is \textbf{Option (c)}
\end{answer}
\item If we observe plot of Bessel functions $J_{0}(x), J_{1}(x)$, and $J_{2}(x)$ we find their maxima at $x_{0}, x_{1}$ and $x_{2}$ respectively. Then which of the following is true
 \begin{tasks}(2)
	\task[\textbf{a.}]$x_{0}<x_{1}<x_{2}$
	\task[\textbf{b.}]$x_{0}>x_{1}>x_{2}$
	\task[\textbf{c.}]$x_{0}<x_{1}=x_{2}$
	\task[\textbf{d.}] $x_{0}=x_{1}<x_{2}$
\end{tasks}
\begin{answer}
	So the correct answer is \textbf{Option (a)}
\end{answer}
\item Which one of the following is correctly matched?\\
\begin{figure}[H]
	\centering
	\includegraphics[height=3.5cm,width=6.5cm]{SF-01}
\end{figure}
 \begin{tasks}(2)
	\task[\textbf{a.}](1) $J_{0}$,
	(2) $J_{2}$, (3) $J_{1}$
	\task[\textbf{b.}]$(1) J_{0}$,
	(2) $J_{1}, \quad(3) J_{2}$
	\task[\textbf{c.}](1) $J_{2}$,
	(2) $J_{1}$,
	(3) $J_{0}$
	\task[\textbf{d.}] None of the above
\end{tasks}
\begin{answer}
	So the correct answer is \textbf{Option (b)}
\end{answer}
\item If the generating function of Legendre polynomial is $\frac{1}{\sqrt{1-6 t+t^{2}}}$, then coefficient of $t^{2}$ is
 \begin{tasks}(4)
	\task[\textbf{a.}] 11
	\task[\textbf{b.}]$-11$
	\task[\textbf{c.}]13
	\task[\textbf{d.}] $-13$
\end{tasks}
\begin{answer}
	\begin{align*}
	\intertext{The generating function for the polynomial solutions of the Legendre ODE is given by}
	g(x, t)&=\frac{1}{\sqrt{1-2 x t+t^{2}}}=\sum_{n=0}^{\infty} P_{n}(x) t^{n}\\
	\text{Thus }x&=3\text{ and }n=2.\\
	P_{2}(x)&=\frac{1}{2}\left(3 x^{2}-1\right) \Rightarrow P_{2}(3)=\frac{1}{2}\left(3 \times 3^{2}-1\right)=13
	\end{align*}
		So the correct answer is \textbf{Option (c)}
\end{answer}
\item Which of the following relation is true for Bessel's differential equation?
 \begin{tasks}(2)
	\task[\textbf{a.}]$J_{0}^{\prime}(x)=J_{1}(x)$
	\task[\textbf{b.}]$J_{0}^{\prime}(x)=-J_{2}(x)$
	\task[\textbf{c.}]$J_{0}^{\prime}(x)=J_{2}(x)$
	\task[\textbf{d.}] $J_{0}^{\prime}(x)=-J_{1}(x)$
\end{tasks}
\begin{answer}
	So the correct answer is \textbf{Option (d)}
\end{answer}
\item Given that $\sum_{n=0}^{\infty} H_{n}(x) \frac{t^{n}}{n !}=e^{-t^{2}+2 x x}$ the value of $H_{6}(0)$ is
 \begin{tasks}(4)
	\task[\textbf{a.}]$-120$
	\task[\textbf{b.}]$+120$
	\task[\textbf{c.}]12
	\task[\textbf{d.}]  $-12$
\end{tasks}
\begin{answer}
	\begin{align*}
	\sum_{n=0}^{\infty} I_{n}(x) \frac{t^{\prime \prime}}{n !}&=e^{-t^{2}+2 t x} \Rightarrow \sum_{n=0}^{\infty} H_{n}(0) \frac{t^{n}}{n !}=e^{-t^{2}}=1-t^{2}+\frac{t^{4}}{2 !}-\frac{t^{6}}{3 !}\\
	\Rightarrow \frac{H_{6}(0)}{6 !} t^{6}&=-\frac{1}{3 !} t^{6} \Rightarrow H_{6}(0)=-\frac{6 !}{3 !}=-120
	\end{align*}
	So the correct answer is \textbf{Option (a)}
\end{answer}
\item Given that $\sum_{n=0}^{\infty} H_{n}(x) \frac{t^{n}}{n !}=e^{-t^{2}+2 e x}$ the value of $H_{4}(0)$ is
 \begin{tasks}(4)
	\task[\textbf{a.}]12
	\task[\textbf{b.}] 6
	\task[\textbf{c.}]24
	\task[\textbf{d.}] $-6$
\end{tasks}
\begin{answer}
	\begin{align*}
	\sum_{n=0}^{\infty} H_{n}(x) \frac{t^{n}}{n !}&=e^{-t^{2}+2 t x} \Rightarrow \sum_{n=0}^{\infty} H_{n}(0) \frac{t^{n}}{n !}=e^{-t^{2}}=1-t^{2}+\frac{t^{4}}{2 !}-\frac{t^{6}}{3 !}\\
	\Rightarrow \frac{H_{4}(0)}{4 !} t^{4}&=\frac{t^{4}}{2 !} \Rightarrow H_{4}(0)=\frac{4 !}{2 !}=12
	\end{align*}
	So the correct answer is \textbf{Option (a)}
\end{answer}
\item If Hermite polynomial of order 2 is given by $H_{2}(x)=a x^{2}-2 ; a>0$, then the value of $a$ is
 \begin{tasks}(4)
	\task[\textbf{a.}]3
	\task[\textbf{b.}]4
	\task[\textbf{c.}]5
	\task[\textbf{d.}] 6
\end{tasks}
\begin{answer}
	\begin{align*}
	\intertext{Orthonormality condition,}
	\int_{-\infty}^{+\infty}\left[H_{n}(x)\right]^{2} e^{-x^{2}} d x&=2^{\prime \prime} n ! \sqrt{\pi}\\
	\text{For, }n&=2, \int_{-\infty}^{+\infty}\left(a x^{2}-2\right)^{2} e^{-x^{2}} d x=8 \sqrt{\pi}\\
\text{	Now}
	\int_{-\infty}^{+\infty}\left[H_{2}(x)\right]^{2} e^{-x^{2}} d x&=\int_{-\infty}^{+\infty}\left(a x^{2}-2\right)^{2} e^{-x^{2}} d x=\left\{a^{2} \times \frac{3}{4}+4-2 a\right\} \sqrt{\pi}
	\intertext{Thus, we have}
	\frac{3 a^{2}}{4}+4-2 a&=8 \Rightarrow 3 a^{2}-8 a-16=0 \Rightarrow 3 a^{2}-12 a+4 a-16=0\\
	\Rightarrow 3 a(a-4)+4(a-4)&=0 \Rightarrow(3 a+4)(a-4)=0\\
	\text{Thus, }a&=4
	\end{align*}
		So the correct answer is \textbf{Option (b)}
\end{answer}
\item The value of Legendre polynomial $p_{n}(x)$ for odd $n$ and $x=0$. i.e., $p_{n}(0)$ is
 \begin{tasks}(4)
	\task[\textbf{a.}]1
	\task[\textbf{b.}]0
	\task[\textbf{c.}]$-1$
	\task[\textbf{d.}]  $0.5$
\end{tasks}
\begin{answer}
	\begin{align*}
	\intertext{The generating function for Legendre polynomial is}
	\left(1-2 x t+t^{2}\right)^{-1 / 2}&=\sum_{n=0}^{\infty} p_{n}(x) t^{n}\\
	\text{Put, $x=0$, we get, }&\left(1+t^{2}\right)^{-1 / 2}=\sum p_{n}(\theta) t^{n}
	\end{align*}
		So the correct answer is \textbf{Option (b)}
\end{answer}
\item For the Legendre's polynomial $P_{n}(x)$, given below are two statements. Study these carefully and pick out the correct option.\\
Statement I: $\quad \int_{-1}^{1} x\left[P_{n}(x)\right]^{2} d x=0$\\
Statement I: $\lim _{n \rightarrow \infty}\left[\int_{-1}^{1} x P_{n}(x) P_{n+1}(x) d x\right]=0$
 \begin{tasks}(1)
	\task[\textbf{a.}]Only statement (I) is correct
	\task[\textbf{b.}]Only statement (II) is correct
	\task[\textbf{c.}]Both (I) and (II) are correct
	\task[\textbf{d.}]Neither (I) nor (II) is correet
\end{tasks}
\begin{answer}
	\begin{align*}
	\intertext{From recurrence relation we have}
	(n+1) P_{n+1}(x)&=(2 n+1) x p_{n}(x)-n p_{n-1}(x)\\
	x p_{n}(x)&=\frac{1}{(2 n+1)}\left\{(n+1) p_{n+1}(x)+n p_{n-1}(x)\right\}\\
	x\left[p_{n}(x)\right]^{2}&=\frac{1}{(2 n+1)}\left\{(n+1) p_{n}(x) p_{n+1}(x)+n p_{n}(x) p_{n-1}(x)\right\}\\
	\therefore \int_{-1}^{+1} x\left[p_{n}(x)\right]^{2} d x&=0\left\{\because \int_{-1}^{+1} p_{m}(x) p_{n}(x)=0\right.\text{ if }\left.m \neq n\right\}\\
	\therefore &\int_{-1}^{+1} x\left[p_{n}(x)\right]^{2} d x=0
	\intertext{From recurrence relation, we have}
	(n+1) p_{n+1}(x)&=(2 n+1) x p_{n}(x)-n p_{n-1}(x)\\
	(2 n+1) x p_{n}(x)&=(n+1) p_{n+1}(x)+n p_{n-1}(x)\\
	\int_{-1}^{+1}(2 n+1) x p_{n}(x) p_{n+1}(x) d x&=\int_{-1}^{+1}\left[(n+1)\left\{p_{n+1}(x)\right\}^{2}+n p_{n-1}(x) p_{n+1}(x)\right] d x\\
	=\int_{-1}^{+1}(n+1)\left\{p_{n+1}(x)\right\}^{2} d x+n \int_{-1}^{+1}& p_{n-1}(x) p_{n+1}(x) d x=(n+1) \frac{2}{2(n+1)+1}+0=\frac{2 n+2}{2 n+3}\\
	\therefore \int_{-1}^{+1} x p_{n}(x) p_{n+1}(x) d x&=\frac{2 n+2}{(2 n+1)(2 n+3)}\\
	\lim _{n \rightarrow \infty} \frac{n\left(2+\frac{2}{n}\right)}{n^{2}\left(2+\frac{1}{n}\right)\left(2+\frac{3}{n}\right)}&=\lim _{n \rightarrow \infty} \frac{\left(2+\frac{2}{n}\right)}{n\left(2+\frac{1}{n}\right)\left(2+\frac{3}{n}\right)}=0
	\end{align*}
		So the correct answer is \textbf{Option (c)}
\end{answer}
\item Which of the following statements is Incorrect about the Hermite polynomials $H_{n}(x)$ ?
 \begin{tasks}(1)
	\task[\textbf{a.}] The value of integral $\frac{1}{\sqrt{\pi}} \int_{-\infty}^{\infty} e^{-x^{2}}\left[H_{4}(x)\right]^{2} d x$ is 384
	\task[\textbf{b.}] Hermite polynomial of order $3, H_{3}(x)$, satisfies the differential equation $y^{\prime \prime}-2 x y^{\prime}+6 y=0$
	\task[\textbf{c.}] The value of $\mathrm{H}_{4}(\mathrm{l})$ is $-20$
	\task[\textbf{d.}] $H_{n}(x)=\frac{H_{n+1}(x)+2 n H_{n-1}(x)}{x}$
\end{tasks}
\begin{answer}
	\begin{align*}
	\intertext{When integrated with respect to weight function $e^{-x^{2}}$, the Hermite polynomials satisfy}
	\int_{-\infty}^{\infty} e^{-x^{2}} H_{n}(x) H_{m}(x) d x&= \begin{cases}0, & n \neq m \\ \sqrt{\pi} 2^{n} n !, & n=m\end{cases}
	\intertext{In our case $n=m=4$, hence}
	\frac{1}{\sqrt{\pi}} \int_{-\infty}^{\infty} e^{-x^{2}}\left[H_{4}(x)\right]^{2} d x&=\frac{\sqrt{\pi} 2^{4}(4 !)}{\sqrt{\pi}}=384
	\intertext{Hermite polynomial of order $n$, satisfies the differential equation}
	y^{\prime \prime}-2 x y^{\prime}+2 n y=0\\
\text{	when }n=3, y^{\prime \prime}-2 x y^{\prime}+6 y=0\\
	\text{We have }H_{4}(x)&=16 x^{4}-48 x^{2}+12\\
	\text{Therefore, }H_{4}(1)&=-48+28=-20
	\intertext{The recursion relation for Hermite polynomials is}
	H_{n+1}(x)&=2 x H_{n}(x)-2 n H_{n-1}(x) \Rightarrow H_{n}(x)=\frac{H_{n+1}(x)+2 n H_{n-1}(x)}{2 x}
	\end{align*}
		So the correct answer is \textbf{Option (d)}
\end{answer}
\item If $P_{n}(x)$ denotes the Legendre polynomials of order $n$, then which of the following statements is incorrect?
 \begin{tasks}(1)
	\task[\textbf{a.}]$P_{n}(x)=\frac{1}{2^{n} n !} \frac{d^{n}}{d x^{n}}\left[\left(x^{2}-1\right)^{n}\right]$ where $n=0,1,2 \ldots$
	\task[\textbf{b.}]The Legendre polynomials satisfy the differential equation\\$
	\left(1-x^{2}\right) \frac{d^{2} y}{d x^{2}}-2 x \frac{d y}{d x}+n(n+1) y=0
	$
	\task[\textbf{c.}] For each value of $n$ the Legendre polynomials satisfy the relation $P_{n}(1)=1$.
	\task[\textbf{d.}] The value of integral $\int_{-1}^{1}\left[P_{4}(x)\right]^{2} d x$ is $\frac{2}{7}$.
\end{tasks}
\begin{answer}
	\begin{align*}
	\intertext{Option (a) is the correct definition of Legendre polynomial. Legendre polynoimials satisfy the differential equation given in option (b). For each value of $n$ Legendre polynomials satisfy $P_{n}(1)=1$.}
	\text{Since, }\int_{-1}^{1}\left[P_{n}(x)\right]^{2} d x&=\frac{2}{2 n+1}\\
	\text{Hence, }\int_{-1}^{1}\left[P_{4}(x)\right]^{2} d x&=\frac{2}{2 \cdot 4+1}=\frac{2}{9}\\
	\text{Hence option }&(d)\text{ is incorrect.}
	\end{align*}
		So the correct answer is \textbf{Option (d)}
\end{answer}
\end{enumerate}